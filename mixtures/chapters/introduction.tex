\documentclass[../notes.tex]{subfiles}
\begin{document}
\chapter{Introduction}
\label{chap:intro}

Surprisingly, a \emph{mixture} does not have a good universal definition. It can be \emph{informally} said, that a mixture is a \emph{more sophisticated} material.

    \begin{itemize}
	    \item from geophysics: thermohaline circulation in oceans\sidenote{Convection flows due to differences in densities and concetratitons.}, \emph{porous media flows}\sidenote{This branch is having is a reneissance period; a lot of many has been dumped into this.}, avalanches, pollution spreading, phase transitions and changes (partial melting, glaceology), liquefaction of sediments
	    \item from astrophysics: plasmas, gaseous mixtures, stellar mixtures
	    \item from biophysics: flow of blood through tissue, membrane processess
	    \item from chemistry: stechiometry, chemical equilibrium, chemical kinetics
	    \item from material science: composite materials
    \end{itemize}


There can be in fact more approaches to the theory of mixtures.

\begin{lemma}[Approach I: Molecular mixing]
    We imagine that the mixing of the constituents happens at a molecular level; we are thus discussing \emph{multicomponent continuum theory}, or the so called \emph{mixture theory.} At each point of the continuum, all the constituents are present
\end{lemma}

\begin{lemma}[Approach II: Multi-phase theory]
   The mixture is "less mixed": there exists an ansamble of simpler continua separated by interfaces. It is interesting that the multi-component continuum theory can be obtained from multi-phase theory through some kind of averaging. 
\end{lemma}

\end{document}
