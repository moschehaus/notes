\documentclass[../notes.tex]{subfiles}
\begin{document}

\chapter{Classes of mixtures}
\label{chap:classes}

\section{Class I mixtures}
\label{sec:classI}
The evolution equations are

\begin{equations}[columns,*]
	\partial_t \rho_{\alpha} + \divergence{\rho_{\alpha}\bm{v}_{\alpha}} &= m_{\alpha}, \\
	\partial_t\qty(\rho \bm{v}_{\alpha}) + \divergence{\rho \bm{v} \otimes \bm{v}} &= \divergence{\cstress} + \rho \bm{b}, \\
	\partial_t\qty(\rho\qty(e+ \frac{1}{2}\abs{\bm{v}}^{2})) + \divergence{\qty(\rho\qty(e + \frac{1}{2}\abs{\bm{v}}^{2})\bm{v})} &= \divergence{\qty(\cstress \bm{v} - \bm{q})} + \rho \bm{b}\vdot \bm{v} + \rho r.
\end{equations}

But the problem is that the balance of mass still \emph{contains the velocities $\bm{v}_{\alpha}$.} Let us deal with it by using the \emph{diffusive velocity} and then use some constitutive relations:

\begin{equations}[single,*]
	\partial_t \rho_{\alpha} + \divergence{\qty(\rho_{\alpha}\overbrace{\qty(\bm{v}_{\alpha} - \bm{v})}^{= \bm{u}_{\alpha}})} + \divergence{\qty(\rho_{\alpha \bm{v}})} = m_{\alpha},
\end{equations}
where we set 
\begin{equations}[single,!,numberline=all]
	\label{eq:diffusive_mass_flux}
	\bm{j}_{\alpha} \coloneq \rho_{\alpha}\bm{u}_{\alpha},
\end{equations}

and so our equation reads as

\begin{equations}[single,*]
	\partial_t \rho_{\alpha} + \divergence{\qty(\rho_{\alpha}\bm{v})} = m_{\alpha} - \divergence{\bm{j}_{\alpha}}, \sum_{\alpha=1}^N m_{\alpha} = 0, \sum_{\alpha=1}^N \bm{j}_{\alpha} = \bm{0}.
\end{equations}

Let us sum the equations up, so then we obtain: (\textcolor{gray}{recall $\rho = \sum_{\alpha=1}^N \rho_{\alpha}$})

\begin{equations}[single,*]
	\partial_t \rho + \divergence{\qty(\rho \bm{v})} = 0,
\end{equations}

because of the constraints.

It is convenient to rewrite the equations in a single form, with the constraints involved. 

Recall:
\begin{equations}[single,*]
	c_{\alpha} = \frac{\rho_{\alpha}}{\rho},
\end{equations}

and the quality 

\begin{equations}[lines,*]
	\mdv{\rho c_{\alpha}} = \mdv{\rho}c_{\alpha} + \rho \mdv{c_{\alpha}} = \qty(\partial_t \rho + \qty(\bm{v} \vdot \grad)\rho)c_{\alpha} + \rho\qty(\partial_t c_{\alpha} + \qty(\bm{v} \vdot \grad)c_{\alpha}),
\end{equations}

using

\begin{equations}[single,*]
	0 = \partial_t \rho + \divergence{\qty(\rho \bm{v})} = \partial_t \rho + \grad \rho \vdot \bm{v} + \rho\qty(\divergence{\bm{v}}),
\end{equations}

we obtain 

\begin{equations}[lines,*]
	- \rho\qty(\divergence{\bm{v}})c_{\alpha} + \rho \mdv{c_{\alpha}},
\end{equations}

on the other hand 

\begin{equations}[single,*]
	\mdv{\qty(\rho c_{\alpha})} = \mdv{\rho_{\alpha}} = \partial_t \rho_{\alpha} + \bm{v} \vdot \grad \rho_{\alpha} = - \rho_{\alpha}\qty(\divergence{\bm{v}}) + m_{\alpha} - \divergence{\bm{j}_{\alpha}},
\end{equations}

so in fact

\begin{equations}[single,*]
	\rho \mdv{c_{\alpha}} = \qty(\divergence{\bm{v}})\qty(c_{\alpha} \rho - \rho_{\alpha}) + m_{\alpha} - \divergence{\bm{j}_{\alpha}} = m_{\alpha} - \divergence{\bm{j}_{\alpha}}.
\end{equations}

Our final equations reads as

\begin{equations}[single,!,numberline=all]
	\label{eq:concetration_transport}
	\rho \mdv{c_{\alpha}} = m_{\alpha} - \divergence{\bm{j}_{\alpha}}, \sum_{\alpha=1}^N m_{\alpha} = 0, \sum_{\alpha=1}^N \bm{j}_{\alpha} = 0, \sum_{\alpha=1}^N c_{\alpha} = 1,
\end{equations}

which is the typical advection-reaction-diffusion system. In total

\begin{equations}[columns,!, numberline=all]
	\label{eq:classI}
	\partial_t \rho + \divergence{\qty(\rho \bm{v})} &= 0, \\
	\rho \mdv{c_{\alpha}} &= m_{\alpha} - \divergence{\bm{j}_{\alpha}},\\
	\sum_{\alpha=1}^N m_{\alpha} &= 0, \sum_{\alpha=1}^N \bm{j}_{\alpha} = 0, \sum_{\alpha=1}^N c_{\alpha}=1 \\
	\partial_t\qty(\rho \bm{v}_{\alpha}) + \divergence{\qty(\rho \bm{v} \otimes \bm{v})} &= \divergence{\cstress} + \rho \bm{b}, \\
	\partial_t\qty(\rho\qty(e+ \frac{1}{2}\abs{\bm{v}}^{2})) + \divergence{\qty(\rho\qty(e + \frac{1}{2}\abs{\bm{v}}^{2})\bm{v})} &= \divergence{\qty(\cstress \bm{v} - \bm{q})} + \rho \bm{b}\vdot \bm{v} + \rho r.
\end{equations}

Our goal is to complete the system with \emph{thermodynamic closures}, \textit{i.e.}, to provide a \emph{constitutive theory}.

\subsection{Constitutive theory}
\label{sec:constitutive_theory}

Our thermodynamical potential of choice is the \emph{Helmholtz free energy}

\begin{equations}[single,*]
	\psi = e - \theta \eta, \psi = \hat{\psi}\qty(\theta, \frac{1}{\rho}, c_1, \dots, c_N).
\end{equations}

And our goal is to obtain

\begin{equations}[single,*]
	\rho \mdv{\eta} = \qty(\, \text{something} \,) \geq 0.
\end{equations}

For simplicity, let us first do this in a binary setting: $N =2, c_1 = c, c_2 = 1-c.$
Let us begin \sidenote{Recall that from the relation $$e = \hat{e}\qty(\eta, \frac{1}{\rho}, c_1, \dots, c_N)$$ one can define $$\psi = \inf_{\eta>0}\qty(\hat{e}\qty(\eta, \frac{1}{\rho}, c_1, \dots, c_N)- \theta \eta),$$ and then the infimum is at the point where the derivative vanishes, so $$0 = \pdv{\hat{e}}{\eta}\qty(\eta, \frac{1}{\rho}, c_1, \dots, c_N) - \eta)$$. The rest of the relation is just chain rule...}

\begin{equations}[lines,*]
	\mdv{\psi} = \mdv{e} - \mdv{\theta}\eta - \theta \mdv{\eta} = \pdv{\hat{\psi}}{\theta} \mdv{\theta} + \pdv{\hat{\psi}}{\qty(\frac{1}{\rho})}\mdv{\qty(\frac{1}{\rho})} + \pdv{\hat{\psi}}{c} \mdv{c} = \\ = - \eta \mdv{\theta} + p \frac{1}{\rho^{2}}\mdv{\rho} + \mu \mdv{c},
\end{equations}

so in fact

\begin{equations}[lines,*]
	\theta \rho \mdv{\eta} = \rho \mdv{e}  -\frac{p}{\rho}\underbrace{\mdv{\rho}}_{=-\rho\qty(\divergence{\bm{v}})} \mu \rho \mdv{\gamma} = \\ = \cstress : \symvgrad - \divergence{\bm{q}} = \rho r + p\qty(\divergence{\bm{v}}) - \mu\qty(m - \divergence{\bm{j}}),
\end{equations}
using the decomposition

\begin{equations}[single,*]
	\cstress = \frac{1}{3}\trace \identityM + \cstress^{\, \text{d} \,} = M \identityM + \cstress^{\, \text{d} \,},
\end{equations}
one obtains

\begin{equations}[single,*]
	\theta \rho \mdv{\eta} = \qty(M+p)\qty(\divergence{\bm{v}}) + \cstress^{\, \text{d} \,}:\symvgrad^{\, \text{d} \,} - \divergence{\bm{q}} - \mu m + \underbrace{\mu \divergence{j}}_{=\divergence{\qty(\mu \bm{j})}- \bm{j} \vdot \grad \mu}.
\end{equations}

Recall the general balance of entropy:

\begin{equations}[single,*]
	\rho \mdv{\eta} = - \divergence{\bm{q}_{\eta}} + r_{\eta} + \zeta.
\end{equations}

So if we divide our equation by entropy, we obtain

\begin{equations}[lines,*]
	\rho \mdv{\eta} = \frac{1}{\theta}\qty(\qty(M+P)\qty(\divergence{\bm{v}}) + \cstress^{\, \text{d} \,}:\symvgrad^{\, \text{d} \,}- \mu m) - \frac{1}{\theta}\divergence{\qty(\bm{q}- \mu_{j})} - \frac{1}{\theta}\bm{j} \vdot \grad \mu = \frac{\rho r}{\theta} = \\ = \frac{1}{\theta}\qty(\qty(M+p)\qty(\divergence{\bm{v}})+ \cstress^{\, \text{d} \,}: \symvgrad^{\, \text{d} \,} - \mu m) + \frac{\rho r}{\theta} - \divergence{\qty(\frac{\bm{q}-\mu \bm{j}}{\theta})} + \qty(\bm{q} - \mu \bm{j})\vdot \grad\qty(\frac{1}{\theta}) - \frac{1}{\theta} \bm{j} \vdot \grad \mu.
\end{equations}

So one can \emph{interpret}

\begin{equations}[columns,!,numberline=all]
	\label{eq:entropy_balances}
	\bm{q}_{\eta} &\coloneq \frac{\bm{q} - \mu \bm{j}}{\theta}, \\
	\rho r_{\eta} &\coloneq \frac{\rho r}{\rho}, \\
	\zeta &\coloneq \frac{1}{\theta}\qty(\qty(M+p)\qty(\divergence{\bm{v}})+\cstress^{\, \text{d} \,}:\symvgrad^{\, \text{d} \,} - \mu m) - \frac{1}{\theta}\bm{j} \vdot \grad \mu + \qty(\bm{q}-\mu \bm{j}) \vdot \grad\qty(\frac{1}{\theta}).
\end{equations}

How to make the entropy production positive? inspired by Navier-Stokes-Fourier one can take

\begin{equations}[columns,*]
	M+P &= \frac{3 \lambda + 2 \nu}{3}\qty(\divergence{\bm{v}}), 3 \lambda + 2 \nu \geq 0, \\
	\cstress^{\, \text{d} \,} &= 2 \nu \symvgrad^{\, \text{d} \,}, \nu \geq 0 \\
	m &= - b \mu, \beta \geq 0 \\
	\bm{j} &= - \alpha \grad \mu, \bm{q} -\mu \bm{j} =  \kappa \grad\qty(\frac{1}{\theta}) \\
	\bm{j} &= - \alpha \grad\qty(\frac{\mu}{\theta}), \bm{q} = \kappa \grad\qty(\frac{1}{\theta}) (\, \text{this one is in fact preferred by statistical physics} \,).
\end{equations}

This implies

\begin{equations}[columns,!,numberline=all]
	\label{eq:constitutive_classI}
	\cstress &= - p\qty(\theta, \frac{1}{\rho}, c) \identityM + \lambda\qty(\divergence{\bm{v}})\identityM + 2 \nu \symvgrad, 3 \lambda + 2 \nu > 0, \nu >0, \\
	\bm{q} &= \kappa \grad\qty(\frac{1}{\theta}) = - \frac{\kappa}{\theta^{2}}\grad \theta, \\
	m &= - \beta \mu, \beta >0\\
	\bm{j} &= - \alpha \grad\qty(\frac{\mu}{\theta}), \alpha >0
\end{equations}
One can see the last relation is very similliar to the Fourier law.

Suppose for a second now that 

\begin{equations}[single,*]
	\hat{\psi}\qty(\theta, \frac{1}{\rho}, c) = \hat{\psi}_0\qty(\theta, \frac{1}{\rho}) + \hat{\psi}_1(c),
\end{equations}
than

\begin{equations}[single,*]
	\mu = \pdv{\hat{\psi}}{c} = \psi_1'(c), \grad \mu = \psi_1''(c) \grad c.
\end{equations}

In the isothermal case, our constitutive relation reads

\begin{equations}[single,*]
	\bm{j} = - \alpha \grad\qty(\frac{\mu}{\theta}) = - \frac{\alpha}{\theta}\grad \mu = - \frac{\alpha}{\theta}\psi_1''(c) \grad c = - \tilde{k} \grad c, \tilde{k} > 0.
\end{equations} 

Which is a law in the form

\begin{equations}[single,*]
	\bm{j} = - \tilde{k}\grad c, k>0,
\end{equations}
exactly similiar to $\bm{q} = - \tilde{\kappa} \grad \theta.$ This law is known as the \emph{Fick law}.

The governing equations are

\begin{equations}[single,*]
	\rho \mdv{c} = - \divergence{\bm{j}} +m = \divergence{\qty(k\qty(c) \grad c)} - 3 \psi'_1(c)
\end{equations}

\begin{equations}[single,*]
	\rho\qty(\partial_t c + \bm{v} \vdot \grad c) - \divergence{\qty(k\qty(c)\grad c)} + 3 \psi'_1(c) = 0,
\end{equations}

which is again an advection-diffusion-reaction system.

\subsection{Multi-component diffusion}
\label{sec:multi-component}

The previous result have been done in a binary setting - let us repeat it in more generality: we again have

\begin{equations}[single,*]
	\psi = \hat{\psi}\qty(\theta, \frac{1}{\rho}, c_1, \dots, c_N),
\end{equations}

so we obtain

\begin{equations}[single,*]
	\psi \rho \mdv{\eta} = \qty(M+p)\qty(\divergence{\bm{v}}) + \cstress^{\, \text{d} \,}:\symvgrad^{\, \text{d} \,} - \divergence{\bm{q}} + \rho r - \sum_{\alpha=1}^{N}\mu_{\alpha}m_{\alpha} + \sum_{\alpha=1}^{N}\mu_{\alpha}\qty(\divergence{\bm{j}_{\alpha}}),
\end{equations}

where the last term can in fact be written

\begin{equations}[single,*]
	\sum_{\alpha=1}^{N-1}\qty(\mu_{\alpha} - \mu_N)m_{\alpha} + \sum_{\alpha=1}^{N-1}\qty(\mu_{\alpha} - \mu_N)\qty(\divergence{\bm{j}_{\alpha}}).
\end{equations}

Dividing by temperature yields

\begin{equations}[lines,*]
	\rho \mdv{\eta} = - \divergence{\qty(\frac{\bm{q}- \sum_{\alpha=1}^{N}\mu_{\alpha}\bm{j}_{\alpha}}{\theta})} + \frac{\rho r}{\theta} + \frac{1}{\theta}\bigg( \qty(M+p)\divergence{\bm{v}} + \\ 
	+ \cstress^{\, \text{d} \,}:\symvgrad^{\, \text{d} \,} - \sum_{\alpha=1}^{N-1}\qty(\mu_{\alpha} - \mu_N)m_{\alpha} + \bm{q} \vdot \grad\qty(\frac{1}{\theta}) - \sum_{\alpha=1}^{N-1} \bm{j}_{\alpha}\vdot \grad\qty(\frac{\mu_{\alpha} - \mu_N}{\theta}) \bigg).
\end{equations}

Linear irreversible thermodynamics proposes to define 

\begin{equations}[single,*]
	\begin{bmatrix}
		- \bm{j}_1 \\	
		\dots \\
		- \bm{j}_{N-1} \\
		\bm{q}
	\end{bmatrix}
	= \tensorq{L} 
	\begin{bmatrix}
		\grad\qty(\frac{\mu_1 - \mu_N}{\theta}) \\
		\dots \\
		\grad\qty(\frac{\mu_{N-1}- \mu_N}{\theta}) \\
		\grad\qty(\frac{1}{\theta}) 
	\end{bmatrix},
\end{equations}
where in general $\tensorq{L}$ is a \emph{full matrix}. If $\tensorq{L}$ is diagonal, we call the closures \emph{diagonal.} We see that if $\tensorq{L}$ is positive \textcolor{gray}{semi-}definite, the entropy production is nonnegative.

The matrix $\tensorq{L}$ has some other qualities, called the \emph{Onsager-Casimir relations}:\sidenote{For $N=2$ we have 
	$$\begin{bmatrix}
		- \bm{j} \\
		\bm{q}
	\end{bmatrix}
	= \begin{bmatrix}
		L_{11} & L_{12} \\
		L_{12} & L_{22}
	\end{bmatrix}
	\begin{bmatrix}
		\grad\qty(\frac{\mu}{\theta}) \\
		\grad\qty(\frac{1}{\theta})
	\end{bmatrix},$$
	and so we see 
	$$\bm{j} = - L_{11} \grad\qty(\frac{\mu}{\theta}) - L_{12} \grad\qty(\frac{1}{\theta})$$
	.
	The diffusive flux $\bm{j}$ is thus proportional to $\grad\qty(\frac{1}{\theta})$; this effect is called the \emph{Soret effect.}

	Also 
	$$\bm{q} = L_{12} \grad\qty(\frac{\mu}{\theta}) + L_{22} \grad\qty(\frac{1}{\theta}),$$
called the \emph{Dufour effect.}}


\begin{itemize}
	\item $\tensorq{L} = \tensorq{L}^{\transpose}.$
\end{itemize}


The problem is that the matrix $\tensorq{L}$ is in general \emph{not known;} for example, \emph{the choice of a constant matrix is \textbf{totally wrong.}}

Let us show a different approach: the \emph{Maxwell-Stefan} closure.

\subsubsection{Maxwell-Stefan closure}
\label{sec:maxwell_stefan}

\begin{equations}[lines,*]
	\rho \mdv{\eta} + \divergence{\qty(\frac{\bm{q} - \sum_{\alpha=1}^{N}\mu_{\alpha}\bm{j}_{\alpha}}{\theta})} =  \frac{\rho r}{\theta} + \bm{q} \vdot \grad\qty(\frac{1}{\theta}) - \sum_{\alpha=1}^N \bm{j}_{\alpha} \vdot \grad\qty(\frac{\mu_{\alpha}}{\theta}) +\\+ \frac{1}{\theta}\qty(\qty(M + p)\qty(\divergence{\bm{v}}) +  \cstress^{\, \text{d} \,} : \symvgrad^{\, \text{d} \,} - \sum_{\alpha=1}^N \mu_{\alpha}m_{\alpha})
\end{equations}
recall that

\begin{equations}[single,*]
	\rho r = \sum_{\alpha=1}^N\qty(\rho_{\alpha}r_{\alpha} + \bm{j}_{\alpha} \vdot \bm{b}_{\alpha},)
\end{equations}
so if we borrow this we obtain

\begin{equations}[lines,*]
	\rho \mdv{\eta} + \divergence{\qty(\frac{\bm{q} - \sum_{\alpha=1}^{N}\mu_{\alpha}\bm{j}_{\alpha}}{\theta})} = \\ = \frac{1}{\theta}\qty(\qty(M + p)\qty(\divergence{\bm{v}}) + \cstress^{\, \text{d} \,} : \symvgrad^{\, \text{d} \,} - \sum_{\alpha=1}^N \mu_{\alpha}m_{\alpha}) + \frac{\rho r}{\theta} + \bm{q} \vdot \grad\qty(\frac{1}{\theta}) - \sum_{\alpha=1}^N \bm{j}_{\alpha} \vdot \qty(\grad\qty(\frac{\mu_{\alpha}}{\theta} - \frac{\bm{b}_{\alpha}}{\theta})).
\end{equations}

Let us now work only with the "diffusive component of the entropy production"

\begin{equations}[single,*]
	- \sum_{\alpha=1}^N \bm{j}_{\alpha} \vdot\qty(\grad\qty(\frac{\mu_{\alpha}}{\theta})- \frac{\bm{b}_{\alpha}}{\theta}) = - \sum_{\alpha=1}^N \bm{u}_{\alpha} \vdot \qty(\rho_{\alpha}\grad\qty(\frac{\mu_{\alpha}}{\theta}) - \frac{\bm{b}_{\alpha}}{\theta}\rho_{\alpha} - \rho_{\alpha} \Lambda),
\end{equations}

which is legit, since $\sum_{\alpha=1}^N \bm{u}_{\alpha} = 0.$ Denote

\begin{equations}[single,*]
	\bm{d}_{\alpha} = \rho_{\alpha}\qty(\grad\qty(\frac{\mu_{\alpha}}{\theta}) - \frac{\bm{b}_{\alpha}}{\theta}\rho_{\alpha} - \Lambda),
\end{equations}

and thus we can write

\begin{equations}[single,*]
	- \sum_{\alpha=1}^N \bm{j}_{\alpha} \vdot\qty(\grad \qty(\frac{\mu_{\alpha}}{\theta}) - \frac{\bm{b}_{\alpha}}{\theta}) = - \sum_{\alpha=1}^N \bm{u}_{\alpha} \vdot \bm{d}_{\alpha}.
\end{equations}


We require $\Lambda$ to be such that

\begin{equations}[single,*]
	\sum_{\alpha=1}^N \bm{d}_{\alpha} = 0,
\end{equations} 
\textit{i.e.},

\begin{equations}[single,*]
	\rho \Lambda = \sum_{\alpha=1}^N \rho_{\alpha}\qty(\grad \qty(\frac{\mu_{\alpha}}{\theta}) - \frac{\bm{b}_{\alpha}}{\theta}),
\end{equations}
from which it follows

\begin{equations}[single,*]
	\Lambda = \sum_{\alpha=1}^N c_{\alpha}\qty(\grad\qty(\frac{\mu_{\alpha}}{\theta}) - \frac{\bm{b}_{\alpha}}{\theta}),
\end{equations}

and so $\bm{d}_{\alpha}$ have the form

\begin{equations}[lines,*]
	\bm{d}_{\alpha}  = \rho_{\alpha}\qty(\grad\qty(\frac{\mu_{\alpha}}{\theta})) = \frac{\bm{b}_{\alpha}}{\theta}) - \rho_{\alpha} \sum_{\beta=1}^N c_{\beta}\qty(\grad\qty(\frac{\mu_{\beta}}{\theta}) - \frac{\bm{b}_{\beta}}{\theta}) =\\= \rho_{\alpha}\qty(\grad \qty(\frac{\mu_{\alpha}}{\theta}) - \frac{\bm{b}_{\alpha}}{\theta} -  \frac{\rho_{\alpha}}{\rho} \sum_{\beta=1}^N \rho_{\beta} \grad\qty(\frac{\mu_{\beta}}{\theta}) + \frac{\rho_{\alpha}}{\rho}\sum_{\beta=1}^N \frac{\rho_{\beta} \bm{b}_{\beta}}{\rho}) = \\ = \rho_{\alpha}\qty(\grad\qty(\frac{\mu_{\alpha}}{\theta}) - \sum_{\beta=1}^N c_{\beta} \grad\qty(\frac{\mu_{\beta}}{\theta}) - \frac{\bm{b}_{\alpha} - \bm{b}}{\theta}),
\end{equations}
where we have used the definition of the body forces

\begin{equations}[single,*]
	\rho \bm{b} = \sum_{\alpha=1}^N \rho_{\alpha} \bm{b}_{\alpha}.
\end{equations}

This means that the "diffusive entropy production is"

\begin{equations}[single,*]
	\Pi_{\eta}^{\, \text{diff} \,} = \sum_{\alpha=1}^N \bm{u}_{\alpha} \vdot \bm{d}_{\alpha},
\end{equations}

with 

\begin{equations}[single,*]
	\bm{d}_{\alpha} = \rho_{\alpha}\qty(\grad \qty(\frac{\mu_{\alpha}}{\theta}) - \sum_{\beta=1}^N c_{\beta} \grad\qty(\frac{\mu_{\beta}}{\theta}) = \frac{\bm{b}_{\alpha} - \bm{b}}{\theta}).
\end{equations}

To proceed, we need some thermodynamical relations

\begin{itemize}
	\item the \emph{Gibbs relation} $$\theta \dd{\eta} = \dd{e} + p \dd{\qty(\frac{1}{\rho})} - \sum_{\alpha=1}^N \mu_{\alpha} \dd{c}_{\alpha},$$
	\item the \emph{Euler relation} $$\rho \eta = e + p \frac{1}{\rho} - \sum_{\alpha=1}^N \mu_{\alpha} c_{\alpha}$$,
	\item the \emph{Gibbs-Duhem relation} $$\eta \dd{\theta} = \frac{1}{\rho} \dd{p} - \sum_{\alpha=1}^N c_{\alpha} \dd{\mu}_{\alpha}.$$
\end{itemize}


So we can in fact write  (\textcolor{gray}{using Gibbs-Duhem\sidenote{Gibbs-Duhem works for $\grad$ also, since that is almost the same like $\dd{d}$.} in the first step and Euler in the second})

\begin{equations}[single,*]
	\sum_{\beta=1}^N c_{\beta} \grad \qty(\frac{\mu_{\beta}}{\theta}) = \underbrace{\frac{1}{\theta} \sum_{\beta=1}^N c_{\beta} \grad \mu_{\beta}}_{= \frac{\grad p}{\rho} - \eta \grad \theta} + \underbrace{\sum_{\beta=1}^N c_{\beta} \mu_{\beta}}_{= e - \theta \eta + \frac{p}{\rho}} \grad\qty(\frac{1}{\theta}) = \frac{1}{\theta} \frac{\grad p}{\rho} - \eta \frac{1}{\theta} \grad \theta + \qty(e - \theta \eta + \frac{p}{\rho})(\frac{- \grad \theta}{\theta^{2}} = h \grad\qty(\frac{1}{\theta}) + \frac{\grad p}{\rho \theta},
\end{equations}

where we have denoted

\begin{equations}[single,!,numberline=all]
	\label{eq:enthalpy}
	h = e + \frac{p}{\rho},
\end{equations}
as the \emph{(specific) enthalpy.} We can thus write 

\begin{equations}[single,!,numberline=all]
	\label{eq:d_alpha}
	\bm{d}_{\alpha} = \rho_{\alpha} \grad \qty(\frac{\mu_{\alpha}}{\theta}) - \rho_{\alpha} h \grad\qty(\frac{1}{\theta} ) - \frac{c_{\alpha}}{\theta} \grad p - \rho_{\alpha} \frac{\bm{b}_{\alpha}-\bm{b}}{\theta},
\end{equations}

and the diffusive entropy production is 

\begin{equations}[single,*]
	\Pi_{\eta}^{\, \text{diff} \,} = - \sum_{\alpha=1}^N \bm{u}_{\alpha} \vdot \bm{d}_{\alpha} = - \sum_{\alpha=1}^N\qty(\bm{u}_{\alpha} - \bm{u}_{n})\vdot \bm{d}_{\alpha}.
\end{equations}

We now \emph{postulate}

\begin{equations}[single,!,numberline=all]
	\label{eq:inverse_closure}
	\bm{d}_{\alpha} \coloneq - \sum_{\beta=1}^{N-1}\tau_{\alpha \beta}\qty(\bm{u}_{\beta} - \bm{u}_N),
\end{equations}
where the matrix $\bbtau = \bbtau^{\transpose} \in \R^{N-1 \times N-1}$ is some symmetric positive \textcolor{gray}{semi-}definite matrix to make sure $\Pi_{\eta}^{\, \text{diff} \,} \geq 0.$

But the true beauty of the \emph{Maxwell-Stefan closure} lies in the construction of a \emph{different matrix} $\tilde{\bbtau} \in \R^{N \times N},$ such that we require

\begin{equations}[columns,*]
	\sum_{\alpha=1}^N \tilde{\tau}_{\alpha \beta} = 0 \Leftrightarrow \tilde{\tau}_{N \beta} &= - \sum_{\alpha=1}^{N-1}\tau_{\alpha \beta},\\
	\sum_{\beta=1}^N \tilde{\tau}_{\alpha \beta} = 0 \Leftrightarrow \tilde{\tau}_{\alpha N} &= - \sum_{\beta=1}^N \tau_{\alpha \beta}, \\
	\tilde{\tau}_{NN} &= - \sum_{\alpha=1}^{N-1} \tilde{\tau}_{\alpha N} = -\sum_{\beta=1}^{N-1} \tilde{\tau}_{N \beta}, \\
	\alpha, \beta &= 1, \dots, N-1.
\end{equations}

This is in particular handy, because we omit the \emph{privileged} position of $\bm{u}_N$ and obtain

\begin{equations}[single,!,numberline=all]
	\label{eq:inverse_closure_better}
	\bm{d}_{\alpha} =  - \sum_{\beta=1}^N \tilde{\tau}_{\alpha \beta}\qty(\bm{u}_{\beta} - \bm{u}_{\alpha}),
\end{equations}

which is in fact \emph{very nice,} because $\bm{u}_{\beta} - \bm{u}_{\alpha} = \bm{v}_{\beta} - \bm{v}_{\alpha}.$ Some popular choices are

\begin{equations}[single,!,numberline=all]
	\label{eq:choice_ms}
	\tilde{\tau}_{\alpha \beta} = \rho_{\alpha} \rho_{\beta} f_{\alpha \beta},
\end{equations}
for some\sidenote{The structure of $\tensorq{f}$ can be obtaned from statistical physics, whereas the structure of $\tensorq{L}$ not. In fact, one can obtain a different matrix $\tensorq{d}$ related to $\tensorq{f}$ which \emph{can be taken as constant.}} $\tensorq{f} = \tensorq{f}^{\transpose} \in \R^{N \times N}$ \emph{ symmetric positive \textcolor{gray}{semi-}definite matrix}. This ansatz gives 

\begin{equations}[single,*]
	\bm{d}_{\alpha} = - \sum_{\beta=1}^N f_{\alpha \beta}\qty(\rho_{\beta} \bm{j}_{\alpha} - \rho_{\alpha}\bm{j}_{\beta}),
\end{equations}

and so the full \emph{Maxwell-Stefan} closure reads as

\begin{equations}[single,!,numberline=all]
	\label{eq:ms_closure}
	- \sum_{\beta=1}^N f_{\alpha \beta}\qty(\rho_{\beta}\bm{j}_{\alpha} - \rho_{\alpha} \bm{j}_{\beta}) = \rho_{\alpha} \grad\qty(\frac{\mu_{\alpha}}{\theta}) - \frac{c_{\alpha}}{\theta} \grad p - \rho_{\alpha} h \grad\qty(\frac{1}{\theta}) - \rho_{\alpha} \frac{\bm{b}_{\alpha} - \bm{b}}{\theta},
\end{equations}


\subsection{Constraints}
\label{sec:constraints}

How to add some constraints to the theory? For example, we would like to model the fluids in the mixture as \emph{incompressible.} We will discuss two constraints

\begin{itemize}
	\item incompressibility,
	\item quasi-compressibility
\end{itemize}

\subsubsection{Compressibility}
\label{sec:compressibility}

The constraint reads as

\begin{equations}[single,!,numberline=all]
	\label{eq:compress}
	\divergence{\bm{v}} = 0.
\end{equations}

The entropy balance is

\begin{equations}[single,*]
	\rho \psi \mdv{\eta} = \cstress^{\, \text{d} \,}:\symvgrad^{\, \text{d} \,} + \qty(M+p)\qty(\divergence{ \bm{v}}) - \divergence{\bm{q}} - \mu m + \mu \divergence{\bm{j}} = \cstress^{\, \text{d} \,}:\symvgrad^{\, \text{d} \,} - \divergence{\bm{q}} - \mu m + \mu \divergence{\bm{j}},
\end{equations} 

and so 

\begin{equations}[columns,!,numberline=all]
	\label{eq:compress_closure}
	\cstress^{\, \text{d} \,} &= 2 \nu \symvgrad^{\, \text{d} \,} =\sidenotemark 2 \nu \symvgrad, \\
	\bm{q} &= \kappa \grad\qty(\frac{1}{\theta}), \\
	m &= - \beta \mu, \\
	\bm{j} &= - \alpha \grad\qty(\frac{\mu}{\theta})
\end{equations}

\sidenotetext{$$\trace \symvgrad = \divergence{\bm{v}} = 0.$$}

\subsubsection{Quasi-compressibility}
\label{sec:quasi_compress}

Our assumption is 
\begin{equations}[single,*]
	\rho = \hat{\rho}\qty(c),
\end{equations}

\textcolor{gray}{in general $\rho = \hat{\rho}\qty(c_1, \dots, c_N).$} Compute the material time derivative:

\begin{equations}[single,*]
	\mdv{\rho} = \pdv{\hat{\rho}}{c} \mdv{c},
\end{equations}

mulitply by $\rho$ and realize $\mdv{\rho} = - \rho \qty(\divergence{\bm{v}}),$ and $\rho \mdv{c} = m - \divergence{\bm{j}}$, so

\begin{equations}[single,*]
	\rho \mdv{\rho} = - \rho^{2}\qty(\divergence{\bm{v}}) = \pdv{\hat{\rho}}{c} \rho \mdv{c} = \pdv{\hat{\rho}}{c}\qty(m - \divergence{\bm{j}}),
\end{equations}
and so 

\begin{equations}[single,*]
	\divergence{\bm{v}} =- \frac{1}{\rho^{2}} \pdv{\hat{\rho}}{c}\qty(m - \divergence{\bm{j}}) = \qty(\dv{c}\qty(\frac{1}{\hat{\rho}\qty(c)}))\qty(m - \divergence{\bm{j}})
\end{equations}

\begin{example}[Porous media setting]
	Is the above setting even sensible? Suppose we have $N=2$ and are in the \emph{volume-additivity} setting
	\begin{equations}[single,*]
		\sum_{\alpha=1}\varphi_{\alpha} = 1, \varphi_{\alpha} = \frac{\rho_{\alpha}}{\rho_{\alpha}^m},
	\end{equations}
	and each of the components are incompressible
	\begin{equations}[single,*]
		\rho_{\alpha}^m = \, \text{const} \,.
	\end{equations}

	Let us compute
	\begin{equations}[single,*]
		1 = \sum_{\alpha=1}^N \varphi_{\alpha} = \sum_{\alpha=1}^N \frac{\rho_{\alpha}}{\rho_{\alpha}^m} = \sum_{\alpha=1}^N \frac{\rho c_{\alpha}}{\rho_{\alpha}^m} ,
	\end{equations}
	and so 
	\begin{equations}[single,*]
		\frac{1}{\rho} = \sum_{\alpha=1}^N \frac{c_{\alpha}}{\rho_{\alpha}^m},
	\end{equations}
	and in a binary mixture $N=2$ we have

	\begin{equations}[single,*]
		\frac{1}{\rho} = \frac{c}{\rho_1^m} + \frac{1-c}{\rho_2^m},
	\end{equations}
	and so

	\begin{equations}[single,*]
		\dv{c}\qty(\frac{1}{\hat{\rho}(c)}) = \frac{1}{\rho_1^m} - \frac{1}{\rho_2^m} = \frac{\rho_2^m - \rho_1^m}{\rho_1^m \rho_2^m},
	\end{equations}
	so if we set

	\begin{equations}[single,*]
		r_{*} \coloneq \frac{\rho_2^m - \rho_1^m}{\rho_1^m \rho_2^m},
	\end{equations}

	we obtain 

	\begin{equations}[single,*]
		\divergence{\bm{v}} = r_* \qty(m - \divergence{\bm{j}}).
	\end{equations}
	Recall now the balance of entropy

	\begin{equations}[single,*]
		\rho \theta \mdv{\eta} = \cstress^{\, \text{d} \,}: \symvgrad^{\, \text{d} \,} + \qty(M+p)\qty(\divergence{\bm{v}}) - \mu\qty(m - \divergence{\bm{j}}),
	\end{equations}
	we see we have two options: 
	\begin{enumerate}
		\item substitute for $\divergence{\bm{v}}$ and suppose the chemical reactions are the \emph{main ones},
		\item substitute for $m-\divergence{\bm{j}}$ and suppose the mechanilcal interactions are the \emph{main ones}.
	\end{enumerate}

	First approach:

	\begin{equations}[single,*]
		\rho \theta \mdv{\eta} = \cstress^{\, \text{d} \,}: \symvgrad^{\, \text{d} \,} + \qty(M+p)\qty(r_*\qty(m- \divergence{\bm{j}})) - \mu\qty(m- \divergence{\bm{j}}) - \divergence{q} = \cstress^{\, \text{d} \,}:\symvgrad^{\, \text{d} \,} -\qty(m-\divergence{j})\qty(\mu - r_*\qty(M+p)) - \divergence{\bm{q}},
	\end{equations}
	so if we set 
	\begin{equations}[single,*]
		\mu_* \coloneq \mu - r_*\qty(M+P),
	\end{equations}
	we are exactly in the "incompressible" Fick-Navier-Stokes-Fourier setting.
	\begin{equations}[single,*]
		\rho \theta \mdv{\eta} = \cstress^{\, \text{d} \,}: \symvgrad^{\, \text{d} \,}-\qty(m- \divergence{\bm{j}})\mu_*,
	\end{equations}
	and so

	\begin{equations}[columns,*]
		\cstress &= M \identityM + 2 \nu \symvgrad^{\, \text{d} \,}, \\
		\bm{q} &= \kappa \grad\qty(\frac{1}{\theta}), \\
		\bm{j} &= - \alpha \grad m_{\alpha},
		m &= - \beta \mu_{*}.
	\end{equations}

	Plugging this into the balance equations we have

	\begin{equations}[columns,*]
		\rho \mdv{\bm{v}} &= \grad M + \divergence{\qty(2 \nu \symvgrad^{\, \text{d} \,})} + \rho \bm{b}, \\
		\divergence{\bm{v}} &= r_*\qty(-\beta \mu_* + \divergence{\qty(\alpha \grad \mu_{\alpha})}) = r_*\qty(-\beta + \alpha \laplace\qty(\mu 0 r_*\qty(M+p))) =\\
				    &=r_*\qty(-\beta + \alpha \laplace)\qty(\mu\qty(c, \theta) - r_* p\qty(\theta,c)) + r_*^{2}\qty(\beta - \alpha \laplace)M = \\
				    &=\gamma\qty(\theta,c) + r^{2}_*\qty(\beta - \alpha \laplace)M, \lambda>0, \beta>0, \\
				    &\, \text{concetration equation} \,.
	\end{equations}
	This is quite interesting - we see that the divergence of the velocity is proportional to some elliptical operator on $M$. This provides some mathematical rooting of the quasi-incompressibilty concept; suppose the system

	\begin{equations}[columns,*]
		\divergence{\bm{v}_{\varepsilon}} &= - \varepsilon \laplace M, \\
		\partial_t \bm{v}_{\varepsilon} + \bm{v}_{\varepsilon} \vdot \grad \bm{v}_{\varepsilon} &= \grad M_{\varepsilon} + \nu \laplace \bm{v}_{\varepsilon} - \frac{1}{2} \qty(\divergence{\bm{v}_{\varepsilon}})\bm{v}_{\varepsilon},
	\end{equations}
	the apriori estimates are

\end{example}

\subsection{Cahn-Hilliard}
\label{sec:cahn_hilliard}

Let us begin with a more general \emph{ansatz} for the free energy

\begin{equations}[single,!,numberline=all]
	\label{eq:ansatz}
	\psi = \hat{\psi}\qty(\theta, \frac{1}{\rho}, c, \abs{\grad c}).
\end{equations}

One can then derive

\begin{comment}
	\begin{equations}[lines,!,numberline=all]
		\label{eq:ch_entropy}
		\rho \dot{\eta} + \divergence{\qty(\frac{\bm{q} + \qty(\frac{m - \divergence{\bm{j}}}{\rho})\bm{\mu}_c - \mu_{c}\bm{j}}{\theta})} = \\ = \frac{1}{\theta}\qty(\qty(\cstress^{\, \text{d} \,} + \tensorq{K}^{\, \text{d} \,}):\symvgrad^{\, \text{d} \,} + \qty(M + p \frac{1}{3} \tr \tensorq{K})\qty(\divergence{\bm{v}}) =\\= - \bm{\mu}_c - \qty(\bm{q} + \frac{m- \divergence{\bm{j}}}{\rho} \bm{\mu}_c)\vdot \frac{\grad \theta}{\theta}) + \\ - \bm{j}\theta \grad\qty(\frac{\mu_c}{\theta}),
	\end{equations}
\end{comment}

where 

\begin{equations}[single,!,numberline=all]
	\label{eq:vector_chemical_potentital}
	\bm{\mu}_c = \rho \pdv{\hat{\psi}}{\grad c},
\end{equations}
is the \emph{vector chemical potential},

\begin{equations}[single,!,numberline=all]
	\label{eq:chem}
	\mu_c = \mu - \frac{\divergence{\bm{\mu}_c}}{\rho}, \mu = \pdv{\hat{\psi}}{c},
\end{equations}

is the \emph{concetration chemical potential} and

\begin{equations}[single,!,numberline=all]
	\label{eq:korteweg}
	\tensorq{K} = \bm{m}_{c} \otimes \grad c,
\end{equations}
is the \emph{Korteweg stress tensor.}

\subsection{Allen-Cahn-Fourier}
\label{sec:ACF}
This is the case $\bm{j} = \bm{0}, m \neq 0.$ With the following ansatz for the free energy

\begin{equations}[single,!,numberline=all]
	\label{eq:free_ansatz}
	\psi = \psi_0\qty(\theta, \frac{1}{\rho}) + \psi_1\qty(\theta, c) + \frac{s}{2 \rho} \abs{\grad c}^{2}.
\end{equations}

Let us also, for simplicity, $\bm{v} = , \rho = \, \text{konst} \,.$ The evolution equations then are

\begin{equations}[single,*]
	\rho \mdv{c} = m - \divergence{\bm{j}} = m,
\end{equations}
and so

\begin{equations}[single,*]
	\rho \partial_t c - \frac{\beta}{\rho}\qty(\divergence{\qty(s \grad c)}) + \frac{\beta}{\rho} \pdv{\psi_1}{c} = 0.
\end{equations}

If one chooses the energetic representation

\begin{equations}[single,*]
	e = \overline{e}\qty(\psi, \frac{1}{\rho}, c, \grad c) = \hat{\psi} - \theta \pdv{\hat{\psi}}{\theta},
\end{equations}
and so it holds

\begin{equations}[lines,*]
	\rho\qty(\pdv{\overline{e}}{\theta} \pdv{\theta}{t} + \pdv{\overline{e}}{c} \pdv{c}{t} + \pdv{\overline{e}}{\grad c} \pdv{c}{t} + \pdv{\overline{e}}{\grad c} \pdv{\qty(\grad c)}{t}) = \\ = \divergence{\qty(\kappa \frac{\grad \theta}{\theta} - \qty(\frac{\beta s}{\rho^{2}})\qty(\pdv{\psi_1}{c} - \divergence{\qty(s \grad c)})\grad c)},
\end{equations}

realize now 

\begin{equations}[single,*]
	\pdv{\qty(\gamma \psi + \lambda c)}{t} = \divergence{\qty(\kappa \grad \psi)},
\end{equations}
and one also has

\begin{equations}[single,*]
	\pdv{c}{t} = \frac{b}{\rho^{2}}\qty(\divergence{\qty(s \grad c)} - \pdv{\hat{\psi}_1}{c}),
\end{equations}
where

\begin{equations}[single,*]
	\divergence{\qty(s \grad c \pdv{c}{t})} = \pdv{c}{t}\qty(\divergence{\qty(s \grad c)}) + s \grad c \pdv{\grad c}{t},
\end{equations}
and so

\begin{equations}[single,*]
	\rho\qty(\pdv{\overline{e}}{\theta} \pdv{\theta}{t} + \qty(\pdv{\overline{e}}{c} - \frac{\qty(\divergence{ s \grad c})}{\rho} + \qty(\pdv{\overline{e}}{\grad c}) - \frac{s \grad c}{\rho})\vdot \grad)\pdv{c}{t})
\end{equations}

This is terrible\sidenote{The dependency is on $c, \grad c, \laplace c$}. Are there some cases when it is not?

\begin{equations}[lines,*]
	\pdv{\overline{e}}{\grad c} - \frac{s \grad c}{\rho} = \\ = \pdv{\grad c}\qty(\hat{\psi} - \theta \pdv{\hat{\psi}}{\theta}) - s \frac{\grad c}{\rho} = \\= \frac{s \grad c}{\rho} - \theta \pdv{\theta}\qty(\frac{s \grad c}{\rho}) - \frac{s \grad c}{\rho} = \\ = - \frac{\theta}{\rho} \grad c \pdv{s}{\theta} = 0 \Leftrightarrow \sigma \neq \sigma\qty(\theta).
\end{equations}

So when $\sigma$ the surface stress is constant, one gets

\begin{equations}[single,*]
	\rho\qty(\pdv{\overline{e}}{\theta} \pdv{\theta}{t} + \qty(\pdv{\overline{e}}{c} - \frac{\qty(\divergence{\qty(s \grad c)})}{\rho}) \pdv{c}{t}) = \divergence{\qty(\kappa \frac{\grad \theta}{\theta})}
\end{equations}

But $\sigma \neq \sigma\qty(\theta)$ is really \emph{nonphysical}. We see that there is \emph{a fundamental difference} in the case when the surface density depends on temperature - \emph{the equation has a completely different structure!}

\begin{remark}[Development of theories]
	Many times, one develops a model through mechanical assumptions and then try to "extend it thermodynamically", \textit{e.g.} take the parameters (viscosity, moduli) to be temperature-dependent. 

	But this is wrong.

	If one starts within thermodynamic, he ends up with a \emph{totally different} set of equations.
\end{remark}

\subsection{Cahn-Hilliard-Navier-Stokes-Fourier}
\label{sec:CHNSF}

This is the case $\bm{j} \neq 0, m = 0.$ The equations are

\begin{equations}[columns,*]
	\cstress &= - p \identityM + \lambda \qty(\divergence{\bm{v}})\identityM + 2 \nu \symvgrad - \tensorq{K}, \\
	\bm{q} &= -\kappa \frac{\grad \theta}{\theta} + \bm{\mu}_{c} \frac{\qty(\divergence{ \bm{j}})}{\rho},\\
	\bm{j} &= - \alpha \theta \grad\qty(\frac{\mu_{c}}{\theta}),\\
	\mu_c &= \mu - \frac{\qty(\divergence{ \bm{\mu}_c})}{\theta}.
\end{equations}
From these full equations one can obtain several \emph{reductions}

\subsection{Cahn-Hilliard}
\label{sec:CH}

The Cahn-Hilliard model can be derived as follows: for simplicity, let us take $\rho = \, \text{const} \,, \bm{v} = 0, \theta = \, \text{const} \,.$ \emph{Mainly, we are not assuming temperature dependence and no advection.} 

\begin{equations}[lines,*]
	\rho \mdv{c} = \rho \pdv{c}{t} = - \divergence{\bm{j}} = \divergence{\qty(\alpha \grad \mu_c)} = \divergence{\qty(\alpha \grad\qty(\mu - \frac{\divergence{\bm{\mu}_c}}{\rho}))} = \divergence{\qty(\alpha \grad\qty(\pdv{\psi_1}{\grad c}) - \divergence{\qty(s \grad c)})},
\end{equations}
so if we take $\alpha = \, \text{const} \,, s = \, \text{const} \,$ we obtain

\begin{equations}[single,!,numberline=all]
	\label{eq:CH}
	\rho \pdv{c}{t} = \alpha \laplace\qty(\pdv{\psi_1}{c} - s \laplace c).
\end{equations}

For comparison, the Allen-Cahn model looks like

\begin{equations}[single,*]
	\rho \pdv{c}{t} = -\beta \qty(\pdv{\psi_1}{c} - s \laplace c),
\end{equations}

where $\alpha , \beta >0.$

\subsection{Cahn-Hilliard-Navier-Stokes}
\label{sec:CHNS}

This is the case wehn we omit temperature dependence. Also, we suppose the fluid is incompressible, \textit{i.e.} $\theta = \, \text{const} \,, \divergence{\bm{v}} = 0, \rho = \, \text{const} \,, s = \, \text{const} \,.$ The system is

\begin{equations}[columns,!, numberline=all]
	\label{eq:CHNS_system}
	\rho \mdv{\bm{v}} &= \divergence{\qty(2 \nu \symvgrad)} - \grad p + \divergence{\qty(s \grad c \otimes \grad c)}, \\
	\rho \mdv{c} &= \alpha \laplace\qty(\pdv{\psi_1}{c} - s \laplace c).
\end{equations}

\section{Allen-Cahn and Cahn-Hilliard as gradient flows}
\label{sec:gradient_flows}

Let us have a Hilbert space $H$, a function $c\qty(t, \bm{x}) \in H$ and a functional $F: \to \R,$ given as

\begin{equations}[single,*]
	F(c) = \int_{\R^3}\qty(f(c) + \frac{s}{2}\abs{\grad c}^{2})\dd{\bm{x}}.
\end{equations}

The evolution is called \emph{a gradient flow} if it holds

\begin{equations}[single,!,numberline=all]
	\label{eq:grad_flow}
	\partial_t c = - k \grad F.
\end{equations}

where $\grad F \in H$ is the \emph{functional derivative} defined as

\begin{equations}[single,!,numberline=all]
	\label{eq:fdr}
	<\fdv{F\qty(c)}{c}, v>_{H^{*}, H} = \dv{s}\eval{\qty(F\qty(c+ sv))}_{s = 0} = \qty(\grad F, v)_{H, H}
\end{equations}
and so $\fdv{F}{c}$ is the distribution representing the Gateux derivative of $F$ and $\grad F$ is the Riesz representation of $\fdv{F}{c}.$

Whis is this any useful? Let us compute

\begin{equations}[single,*]
	\dv{t} F(c)(t) = <\fdv{F}{c}, \partial_t c> = \qty(\grad F, \partial_t c) = - k \norm{\grad F}^{2}_H \leq 0, 
\end{equations}

and so one sees 

\begin{equations}[single,*]
	\dv{t} F(c) \leq 0,
\end{equations}
\textit{i.e.} the evolution of a gradient flow minimizes \emph{the free energy functional}.

\subsection{A-C as a gradient flow}
\label{sec:A-C_gr}

We have

\begin{equations}[single,*]
	< \delta F(c), v> = \int_{\R^3}\qty(f'(c) v + s \grad c \vdot \grad v)\dd{x} = \int_{\R^3}\qty(f'(c) - \divergence{\qty(s \grad c)}) v\dd{x} = \qty(\grad F, v)_{L_2},
\end{equations}
and so one sees 

\begin{equations}[single,!,numberline=all]
	\label{eq:ac_gr}
	\partial_t c = -k \qty(\pdv{\psi_1}{c} - \divergence{\qty(s \grad c)}),
\end{equations}
\textit{i.e.} A-C is a \emph{gradient flow} in $L_2.$

\subsection{C-H as a gradient flow}
\label{sec:C-H_gr}

\begin{equations}[single,*]
	<\fdv{F}{c}, v> = \int_{\R^3}\qty(\pdv{\psi_1}{c} + s \grad c \vdot \grad v)\dd{x},
\end{equations}

let us set $H = H^{-1}\qty(\R^3),$ with 

\begin{equations}[single,*]
	\qty(u,v)_{H^{-1}} = \qty(\grad u^{*}, \grad v^{*})_{L_2},
\end{equations}

with $u^{*}$ being the solution to $\laplace u^{*} = u.$ Now

\begin{equations}[lines,*]
	<\fdv{F}{c},v> = \int_{\R^3}\laplace\qty(\pdv{\psi_1}{c})^{*} \laplace v^{*} + \laplace \grad \laplace c^{*} \vdot \grad \laplace v^{*}\dd{x} =\\= \int_{\R^3}\qty(\laplace\qty(\pdv{\psi_1}{c})^{*} - s \laplace \laplace c^{*})\laplace v^{*} \dd{x} = \\= -\int_{\R^3}\grad\qty(\laplace\qty(\pdv{\psi_1}{c})^{*} - s \laplace \laplace c^{*})\vdot \grad v^{*}\dd{x},
\end{equations}
and so (with our dot product) the Riesz representation is

\begin{equations}[single,*]
	<\fdv{F}{c}, v> = \qty(- \laplace\qty(\pdv{\psi_1}{c}), v)_{H^{-1}},
\end{equations}
meaning

\begin{equations}[single,*]
	\partial_tc = +k \laplace\qty(\pdv{\psi_1}{c} - s \laplace c),
\end{equations} 

which is \emph{exactly the Cahn-Hilliard model.}
\end{document}
