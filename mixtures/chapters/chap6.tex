\documentclass[../notes.tex]{subfiles}
\begin{document}
\chapter{Multi-phase continuum thermodynamics}
\label{chap:multi_phase}

This is the approach where we assume that the region is occupied by regions of \emph{different species}, that are not \emph{mixed on a molecular level}. This is a more physically sound approach:

\begin{itemize}
	\item one can formulate balance laws on the interfaces
	\item the \emph{multicomponent} theory can be obtained by some kind of averaging
\end{itemize}

Our starting point must be a generalisation of the \emph{classical } continuum mechanics to a setting with \emph{active interfaces}, or discontinuities (\emph{passive interfaces})

\section{General balance laws with interfaces}
\label{sec:interface}

Assume the presence of (active) interfaces $\Gamma$ separating $\Omega_+, \Omega_-$. Then one can take \emph{e.g} density and define its \emph{excess} as

\begin{equations}[single,!,numberline=all]
\label{eq:excess}
\rho_{\Gamma}^{E} = \begin{cases}
	\rho - \rho^+, \, &\text{in} \, \Omega_{+}, \\
  \rho - \rho^-, \, &\text{in} \, \Omega_-
\end{cases}.
\end{equations}

We see that $\, \text{supp} \, \rho_{\Gamma}^{E} \subset \Omega_{\varepsilon}$, where $\Omega_{\varepsilon}$ is the interfacial region \sidenote{ A strip of (varying) thickness $\varepsilon$ around the interface.}

It can be thein integrated over the whole region $\Omega_{\varepsilon}$ defined

\begin{equations}[single,*]
	\rho_{\Gamma} = \int_{-\varepsilon}^{\varepsilon} \int_{S\qty(\xi)} \rho^E_{\Gamma}\dd{S}\qty(\xi) \dd{\xi},
\end{equations}

where $S\qty(\xi)$ is the "elevatin surface" that is changing with $\xi.$
and it can be shown that this is equal to

\begin{equations}[single,!,numberline=all]
\label{eq:curv_excess}
\rho_{\Gamma} = \int_{\Gamma} \int_{-\varepsilon}^{\varepsilon}\qty(1 - \kappa_1)\qty(1- \kappa_2)\rho_{\Gamma}^E\dd{\xi} \dd{S},
\end{equations}

where $\kappa_i\qty(\xi)$ are the principal curvatures of the surface. 

In this manner one cane generalise also the fluxes:

\begin{equations}[single,!,numberline=all]
\label{eq:inter_flux}
\mathcal{F}^{\psi} = \int_{S_{+} \cup S_{i}}\Phi_{\Omega}^{\psi}\dd{S} + \int_{\partial \Gamma}\bm{\Phi}_{\Gamma}^{\psi}\vdot \bm{n}_{\gamma} \dd{l},
\end{equations}

and the production

\begin{equations}[single,!,numberline=all]
\label{eq:inter_production}
\mathcal{P}^{\psi} \int_{\Omega_+ \cup \Omega_i}\xi_{\Omega}^{\psi}\dd{x} + \int_{\Gamma}\xi_{\Gamma}^{\psi}\dd{S},
\end{equations}

and the supply

\begin{equations}[single,!,numberline=all]
\label{eq:inter_supply}
\mathcal{S}^{\psi} = \int_{\Omega_{+} \cup \Omega_i} \Sigma_{\Omega}^{\psi}\dd{x} + \int_{\Gamma}\Sigma_{\Gamma}^{\psi}\dd{S}.
\end{equations}

We then postulate

\begin{equations}[single,!,numberline=all]
\label{eq:inter_balance_law}
\dv{\psi}{t} = - \mathcal{F}^{\psi} + \mathcal{P}^{\psi} + \mathcal{S}^{\psi}, \forall \Omega \, \text{control volumes.} \,
\end{equations}

To manipulate further, we require some advanced tools

\subsection{Generalization of some theorems}
\label{sec:generalization}

\subsubsection{Reynolds transport theorem}
\label{sec:gen_reynolds}

One has

\begin{equations}[single,!,numberline=all]
\label{eq:gen_reynolds_volume}
\dv{t} \int_{\Omega_{+} \cup \Omega_{i}}\psi_{\Omega}\dd{x} = \int_{\Omega_{+} \cup \Omega_i}\qty(\pdv{\Psi_{\Omega}}{t} + \divergence{\qty(\psi_{\Omega} \otimes \bm{v})})\dd{x} + \int_{\Gamma}\qty[\psi_{\Omega} \otimes \qty(\bm{v} - \bm{v}_{\Gamma}^\perp \bm{n}_{\Gamma})]\dd{S},
\end{equations}

where $\qty[\varphi] = \varphi^+ - \varphi^-$ is the jump operator and $\bm{n}_{\Gamma}$ is an unit normal to $\Gamma$ pointing from $\Omega_{-}$ to $\Omega_+$ and $\bm{v}_{\Gamma}^{\perp}$ is the normal velocity of $\Gamma$. The intepretation of $\qty[\psi_{\Omega} \otimes\qty(\bm{v} - \bm{v}_{\Gamma}^{\perp} \bm{n}_{\Gamma})] \vdot \bm{n}_{\Gamma}$ is thus the \emph{relative flux of $\psi_{\Omega}$ through $\Gamma$.}


\begin{remark}[Material surfaces]
Recall that the surfaces in general \emph{need not be material!} Meaning they do not have to evolve (deform) with the material - there can be nontrivial relative movement of the bulk and the surface.
\end{remark}


\subsubsection{Gauss theorem}
\label{sec:gen_gauss}

That is

\begin{equations}[single,!,numberline=all]
\label{eq:gen_gauss_formula}
\int_{S_+ \cup S_-}\bm{\Phi}_{\Omega}^{\psi} \vdot \bm{n}\dd{S} = \int_{\Omega_{+} \cup \Omega_i}\divergence{\bm{\Phi}_{\Omega}^{\psi}}\dd{x} + \int_{\Gamma}\qty[\bm{\Phi}_{\Omega}^{\varphi}] \bm{n}_{\Gamma}\dd{S},
\end{equations}

again, we see that there is a jump operator.

\subsubsection{Surface Reynolds transport theorem}
\label{sec:surf_reynolds}

Finally, one requires the surface analog for the Reynolds transport theorem:

\begin{equations}[single,!,numberline=all]
\label{eq:surf_reynolds_formula}
\dv{t} \int_{\Gamma}\psi_{\Gamma}\dd{S} = \int_{\Gamma}\qty(\mdv[\Gamma]{\psi_{\Gamma}} + \psi_{\Gamma} \, \text{div}_{\Gamma} \bm{v}_{\Gamma} - 2 K_{\Gamma} \psi_{\Gamma} \bm{v}_{\Gamma}^\perp)\dd{S},
\end{equations}

where $K$ is the mean curvature\sidenote{$K = \frac{1}{2}g^{\beta \alpha}b_{\beta \alpha}$ where $g^{\beta \alpha}$ is the \textcolor{gray}{dual} metric surface tensor and $b_{\beta \alpha} = \Gamma^3_{\beta \alpha}$ is the Christoffel symbol.}

\subsubsection{Surface Gauss theorem}
\label{sec:surface_gaus}

That is the equation

\begin{equations}[single,!,numberline=all]
\label{eq:surf_gauss_formula}
\int_{\Gamma}\, \text{div}_{\Gamma} \bm{t}_{\Gamma}\dd{S} = \int_{\partial \Gamma}\bm{t}_{\Gamma} \vdot \bm{\nu}_{\Gamma}\vdot \dd{\bm{l}}.
\end{equations}

\subsection{Localized form of balance laws}
\label{sec:localization}

Finally, using the localization arguments, one can obtain the form inthe nonsingular points

\begin{equations}[single,!,numberline=all]
\label{eq:localized_balance_nonsingular}
\pdv{\Psi_{\Omega}}{t} + \divergence{\qty(\psi_{\Omega} \otimes \bm{v} + \bm{\Phi}_{\Omega}^{p.})} - \xi^{\psi} - \Sigma^{\psi} = 0,
\end{equations}

\begin{equations}[single,!,numberline=all]
\label{eq:localize_balance_singular}
\mdv[\Gamma]{\psi_{\Gamma}} + \psi_{\Gamma} \, \text{div}_{\Gamma} \bm{v}_{\Gamma}^{\parallel} - 2 K_{\Gamma} \psi_{\Gamma} \bm{v}_{\Gamma}^{\perp} + \, \text{div}_{\Gamma} \bm{\Phi}_{\Gamma}^{\psi} - \xi_{\Gamma}^{\psi} - \Sigma_{\Gamma}^{\psi} = - \qty[\bm{\Phi}_{\Omega}^{\psi} + \bm{\Psi}_{\Omega}^{\psi} \otimes \qty(\bm{v} - \bm{v}_{\Gamma}^{\perp} \bm{n}_{\Gamma})] \bm{n}_{\Gamma}
\end{equations}

\subsubsection{Surface mass balance}
\label{sec:surface_mass}

\begin{equations}[single,!,numberline=all]
\label{eq:surf_mass_balance_form}
\mdv[\Gamma]{\rho}_{\Gamma} + \rho_{\Gamma}\qty(\, \text{div}_{\Gamma} \bm{v}_{\Gamma}^{\parallel} - 2 K_{\Gamma} \bm{v}_{\Gamma}^\perp) = -\qty[\rho\qty(\bm{v} - v_{\Gamma}^{\perp} \bm{n}_{\Gamma})]\vdot \bm{n}_{\Gamma}
\end{equations}

\begin{example}[Baloon inflation]
	Assume the case $\bm{v}_{\Gamma}^{\parallel} = 0, \bm{n} \vdot \bm{v} = v_{\Gamma}^{\perp}$. Then the  evolution becomes

	\begin{equations}[single,*]
	\pdv{\rho_{\Gamma}}{t} = 2 K_{\Gamma} \bm{v}_{\Gamma}^{\perp} = - \frac{2}{R} \bm{v}_{\Gamma}^\perp.
	\end{equations}

	Remember than when a baloon inflates, you can see the walls thinning.
\end{example}

\subsubsection{Surface momentum balance}
\label{sec:surf_momentum}
\begin{equations}[single,!,numberline=all]
\label{eq:surf_momentum_formula}
\mdv[\Gamma]{\qty(\rho_{\Gamma} \bm{v}_{\Gamma})} + \rho_{\Gamma} \bm{v}_{\Gamma}\qty(\, \text{div}_{\Gamma} \bm{v}_{\Gamma}^{\perp} - 2 K_{\Gamma} v_{\Gamma}^{\perp}) - \, \text{div}_{\Gamma} \cstress_{\Gamma} - \rho_{\Gamma} \bm{b}_{\Gamma} = - \qty[\rho \bm{v} \otimes \qty(\bm{v} - v_{\Gamma}^\perp \bm{n}_{\Gamma}) - \cstress] \vdot \bm{n}_{\Gamma}.
\end{equations}

\subsubsection{Jump conditions}
\label{sec:jump_conditions}

Assume the special case when $\psi_{\Gamma} = \Phi_{\Gamma}^{\psi} = \xi^{\psi}_{\Gamma} = \Sigma_{\Gamma}^{\psi} = 0.$ On $\Gamma$ one has

\begin{equations}[single,*]
	\qty[\Phi_{\Omega}^{\psi} + \psi_{\Omega} \otimes\qty(\bm{v} - b_{\Gamma}^{\perp} \bm{n}_{\Gamma})] \vdot \bm{n}_{\Gamma} = 0,
\end{equations}

so for mass in particular one has

\begin{equations}[single,!,numberline=all]
\label{eq:mass_jump_conds}
\qty[\rho\qty(\bm{v} - v_{\Gamma}^{\perp} \bm{n}_{\Gamma})] \vdot \bm{n}_{\Gamma} = 0,
\end{equations}

and for the momentum

\begin{equations}[single,!,numberline=all]
\label{eq:momentum_jump_conds}
\qty[\cstress - \rho \bm{v} \otimes \qty(\bm{v} - v_{\Gamma}^{\perp} \bm{n}_{\Gamma})] \vdot \bm{n}_{\Gamma} = 0,
\end{equations}

which means

\begin{equations}[columns,!,numberline=all]
\label{eq:traction_jump}
\qty[\cstress] \vdot \bm{n}_{\Gamma} &= \qty[\rho \bm{v} \otimes \qty(\bm{v - v_{\Gamma}^\perp} \bm{n}_{\Gamma})] \vdot \bm{n}_{\Gamma}\\ &= \qty[\bm{v}]\rho\qty(\bm{v} - v_{\Gamma}^{\perp}\bm{n}_{\Gamma}) \vdot \bm{n}_{\Gamma},
\end{equations}
\textit{i.e.} \emph{the traction vector need not be continuous!} in general. However,in the case of \emph{material discontinuities,} one has $\bm{v} \vdot \bm{n}_{\Gamma} = v_{\Gamma}^{\perp}$ and the \emph{tracton vector is continous.}

\subsubsection{Surface tension}
\label{sec:surf_tension}
There however exists a special case: \emph{interface with surface tension}. Then one has

\begin{equations}[single,*]
\rho_{\Gamma} = 0, \cstress_{\Gamma} = \sigma \identityM_{\Gamma} = \sigma\qty(\identityM - \bm{n}_{\Gamma} \otimes \bm{n}_{\Gamma}),
\end{equations}

where $\sigma$ is the surface tension. Then one has

\begin{equations}[single,*]
\, \text{div}_{\Gamma} \cstress_{\Gamma} = \, \text{div}_{\Gamma}\qty(\sigma \identityM_{\Gamma}) = \grad_{\Gamma} \sigma - \sigma \, \text{div}_{\Gamma}\qty(\bm{n}_{\Gamma} \otimes \bm{n}_{\Gamma}) =\sidenotemark \grad_{\Gamma} \sigma + 2 K_{\Gamma} \sigma \bm{n}_{\Gamma}
\end{equations}

\sidenotetext{$$\, \text{div}_{\Gamma}\qty(\bm{n}_{\Gamma} \otimes \bm{n}_{\Gamma} ) = \qty(\grad_{\Gamma} \bm{n}_{\Gamma} \vdot \bm{n}_{\Gamma}) + \bm{n}_{\gamma} \qty(\, \text{div}_{\Gamma} \bm{n}_{\Gamma}),$$ and we have $$\grad_{\Gamma} \bm{n}_{\Gamma} \vdot \bm{n}_{\Gamma} = 0$$, because the vectors are orthogonal and $$\bm{n}_{\Gamma} \qty(\, \text{div}_{\Gamma} \bm{n}_{\Gamma}) = - 2 K_{\Gamma}$$ from our definition of the mean curvature.}

Let us furthermore consider only material interfaces, \textit{i.e.} $\bm{v} \vdot \bm{n}_{\Gamma} = v_{\Gamma}^{\perp}.$ Then

\begin{equations}[single,*]
	\qty[\cstress] \vdot \bm{n}_{\Gamma} - \, \text{div}_{\Gamma} \cstress_{\Gamma} = - \grad_{\Gamma} \sigma - 2 K_{\Gamma} \sigma \bm{n}_{\Gamma}.
\end{equations}

If we take the nnormal component of this, we obtain

\begin{equations}[single,*]
\qty[\bm{n}_{\Gamma} \vdot \cstress \bm{n}_{\Gamma}] = - 2 K_{\Gamma} \sigma,
\end{equations}
but since $\cstress = - p \identityM + \tensorq{S},$ we have

\begin{equations}[single,*]
	\qty[p] = 2 K_{\Gamma} \sigma  = \frac{-2 \sigma}{R},
\end{equations}

\textit{i.e.} we have recovered

\begin{equations}[single,!,numberline=all]
\label{eq:lpl_young}
p^- - p^+ = \frac{2 \sigma}{R},
\end{equations}

the \emph{Laplace Young condition.}

What about the tangent component? One has

\begin{equations}[single,*]
	\qty[\qty(\bm{t})_{\tau}] = - \grad_{\Gamma} \sigma.
\end{equations}

Suppose we are dealing with the mixture of a liquid and its gas; then the gas tangent traction is \emph{negligible} and one has

\begin{equations}[single,*]
	\qty(\bm{t})_{\tau}^{\, \text{liquid} \,} = \grad_{\Gamma} \sigma,
\end{equations}
and since

\begin{equations}[single,*]
	\qty(\bm{t}_{\tau})^{\, \text{liquid} \,} = \qty(2 \eta \symvgrad \bm{n}_{\Gamma})_{\tau},
\end{equations}

we have obtained


\begin{equations}[single,*]
	\qty(2 \eta \symvgrad)\bm{n}_{\Gamma})_{\tau} = \grad_{\Gamma} \sigma.
\end{equations}

This is called the \emph{Marangoni effect:} the liquid is forced to flow by the gradient of surface tension. \sidenote{This can be examined, for example the \emph{wine tears}.}


\subsubsection{Stefan condition}
\label{sec:stefan_conds}

Let us derive the jump condition for the total energy in the case\sidenote{Th e notation is of course totally misleading.}

\begin{equations}[single,*]
\rho_{\Gamma} = \bm{q}_{\Gamma} = \cstress_{\Gamma} = 0.
\end{equations}

From our general law, we have

\begin{equations}[single,*]
	\qty[\rho\qty(e + \frac{1}{2} \abs{\bm{v}}^{2}) \qty(\bm{v} - v_{\Gamma}^{\perp} \bm{n}_{\Gamma}) - \bm{v} \vdot \cstress + \bm{q}] \vdot \bm{n}_{\Gamma} = 0.
\end{equations}

Decompose

\begin{equations}[single,*]
\cstress = - p_{th}\identityM + \tensorq{S},
\end{equations}

and let us make some assumptions:

\begin{itemize}
	\item $\qty[\rho\qty(\bm{v} - v_{\Gamma}^{\perp}\bm{n}_{\Gamma})] \vdot \bm{n}_{\Gamma} = 0,$
	\item $\qty[\cstress]\bm{n}_{\Gamma} = \qty[\bm{v}] \rho\qty(\bm{v} - v_{\Gamma}^{\perp} \bm{n}_{\Gamma}) \vdot \bm{n}_{\Gamma} - \grad_{\Gamma} \sigma - 2K_{\Gamma} \sigma \bm{n}_{\Gamma}],$
	\item $\qty[p_{th}] = 0.$
\end{itemize}

We than have

\begin{equations}[lines,*]
	0 = \qty[\bm{q}] \vdot \bm{n}_{\Gamma} = - \qty[\bm{v} \vdot \qty(p_{th}\identityM + \tensorq{S}) \vdot \bm{n}_{\Gamma}] + \qty[\rho\qty(e + \frac{1}{2} \abs{\bm{v}}^{2})\qty(\bm{v} - v_{\Gamma}^{\perp} \bm{n}_{\Gamma})] \vdot \bm{n}_{\Gamma},
\end{equations}

and with our assumption on the continuity of the pressure we obtain

\begin{equations}[lines,*]
	\qty[p_{th} \bm{v}] \vdot \bm{n}_{\Gamma} = \qty[p_{th} \qty(\bm{v} - v^{\perp}_{\Gamma}\bm{n}_{\Gamma}) ] \vdot \bm{n}_{\Gamma} = \qty[\frac{p_{th}}{\rho} \rho\qty(\bm{v} - v^{\perp}_{\Gamma} \bm{n}_{\Gamma})] \vdot \bm{n}_{\Gamma} = \\ = \qty[\frac{p_{th}}{\rho}] \rho\qty(\bm{v} -v^{\perp}_{\Gamma} \bm{n}_{\Gamma}) \vdot \bm{n}_{\Gamma}.
\end{equations}

It is customary to define
\begin{equations}[single,*]
	\dot{m} = \rho\qty(\bm{v} - v^{\perp}_{\Gamma} \bm{n}_{\Gamma}) \vdot \bm{n}_{\Gamma}
\end{equations}
as \textcolor{gray}{some kind of a } mass flux. In total, we have

\begin{equations}[single,*]
	0 = \qty[\bm{q}] \vdot \bm{n}_{\Gamma} + \qty[\qty(e + \frac{p_{th}}{\rho})]\dot{m} = - \qty[\bm{v} \vdot \tensorq{S}] \bm{n}_{\Gamma} + \qty[\frac{1}{2} \abs{\bm{v}}^{2}] \dot{m},
\end{equations}

and if we now assume the transport velocity is negligible \sidenote{Which often is - think of the solidifying of a water body.} and manipulate 

\begin{equations}[single,*]
	\qty[\bm{v} \vdot \tensorq{S}] \bm{n}_{\Gamma} = \qty[\bm{v}]_{\tau}\qty(\tensorq{S} \bm{n}_{\Gamma})_{\tau} + \qty[\qty(\tensorq{S} \bm{n})_{n} \bm{v}].
\end{equations}

Most often the normal jump is negligible and the tangential components aswell \textcolor{gray}{(these are for example friction forces)}

\begin{equations}[single,*]
	\qty[\bm{q}] \vdot \bm{n}_{\Gamma} + [h] \dot{m} \approx 0,
\end{equations}

where $h = e + \frac{p}{\rho}$ is the specific enthalpy. Realize that the jump of enthalpy is the latent heat $L$, and so one obtains

\begin{equations}[single,!,numberline=all]
\label{eq:stefan_conds}
\qty[\bm{q}] \vdot \bm{n}_{\Gamma} + L \dot{m} \approx 0.
\end{equations}

\subsection{"Blurring"}
\label{sec:blur}

We are able to describe the continuum precisely on a detailed level; but is this description compatible with the previous \emph{molecular mixing } theory?

Suppose $\Omega = \Omega_1 \cup \Omega_2, \Gamma = \overline{\Omega_1} \cap \overline{\Omega_2}, \Omega_1 \cap \Omega_2 = \emptyset.$ In each $\Omega_{\alpha}, \alpha = 1, 2$ we have some balance equations:

\begin{equations}[single,*]
\partial_t \psi_{\alpha} + \divergence{\qty(\psi_{\alpha} \bm{v}_{\alpha})} + \divergence{\bm{\Phi}_{\Omega}^{\psi_{\alpha}}} - \Pi_{\Omega}^{\psi_{\alpha}}- \xi_{\Omega}^{\psi_{\alpha}} = 0,
\end{equations}

and at the interface $\Gamma_{\alpha \beta}$ we have

\begin{equations}[single,*]
	\mdv[\Gamma]{\psi_{\alpha \beta}^{\Gamma}} + \psi_{\alpha \beta}^{\Gamma}\qty(\, \text{div}_{\Gamma} \bm{v}_{\Gamma}^\parallel - 2 K_{\Gamma} \bm{v}_{\Gamma}^\perp ) + \, \text{div}_{\Gamma}^{\psi} \bm{\Phi}_{\Gamma}^{\alpha \beta} - \Pi_{\Gamma}^{\alpha \beta \psi} - \xi_{\Gamma}^{\alpha \beta \psi} = - \qty(\bm{\Phi}^{\psi}_{\Omega} + \bm{\Psi}_{\Omega} \otimes \qty(\bm{v} - \bm{v}_{\Gamma}^\perp \bm{n}_{\Gamma})) \vdot \bm{n}_{\Gamma},
\end{equations}

where the meaning of the terms is either \emph{known} or can be \emph{guessed}.

Define now the \emph{characteristic function} of each phase

\begin{equations}[single,*]
	\chi_{\alpha}\qty(t, \bm{x}) = \begin{cases}
	  1, \bm{x} \in \Omega_{\alpha}(t), \\
	  0, \bm{x} \notin \Omega_{\alpha}(t).
	\end{cases}
\end{equations}

Or, to make the math more sensible. we can take some \emph{mollification} of the characteristic function, \textit{i.e.} $\chi_{\alpha}^{\varepsilon}(t, \bm{x}).$ The goal is now to write the \emph{local, detailed} evolution equations as $\chi_{\alpha}^{\varepsilon}(t, \bm{x})$ times the \emph{global, less detailed} evolution.

Define the average 

\begin{equations}[single,!,numberline=all]
\label{eq:avg}
\expval{\varphi}_V = \frac{1}{\lambda(B\qty(\bm{x}, \delta)} \int_{B\qty(\bm{x}, \delta)} \varphi\qty(\bm{x} + \bm{z}) \dd{\bm{z}},
\end{equations}
and realize that some averages are in fact known\sidenote{$$\expval{\chi_{\alpha}} = \frac{1}{\lambda\qty(B\qty(\bm{x}, \delta))} \int_{B\qty(\bm{x}, \delta)}\chi_{\alpha}(t, \bm{x} + \bm{z})\dd{\bm{z}}$$ }

\begin{equations}[single,*]
	\expval{\chi_{\alpha}} = \Phi_{\alpha},
\end{equations}

\begin{equations}[single,*]
	\expval{\chi_{\alpha} \rho} = \frac{m_{\alpha}}{V} = \rho_{\alpha}.
\end{equations}

This procedue will be very interesting\sidenote{We will \textit{e.g.} obtain the \emph{exact interaction forces.}} We must however give meaning to\sidenote{This is the problem of \emph{multiplication} of \textcolor{gray}{general} distributons.}

\begin{equations}[lines,*]
	\int_{\R^3 \times \R}\partial_t \xi_{\alpha}) \varphi\qty(t, \bm{x}) \dd{\bm{x}} \dd{t} = - \int_{\R^3 \times \R}\chi_{\alpha} \partial_t \varphi\dd{\bm{x}}\dd{t} = \\ - \int_{\R}\int_{\Omega_{\alpha}(t)}\partial_t \varphi\dd{\bm{x}}\dd{t} = - \int_{\R}\dv{t}\int_{\Omega_{\alpha}(t)}\varphi\dd{\bm{x}}\dd{t} = \\ = -\int_\R \dv{t}\qty(\int_{\Omega_{\alpha(t)}}\dd{\bm{x}} + \int_{\Omega_{\alpha}}\varphi \bm{v}_{\partial \Omega_{\alpha}} \vdot \bm{n}\dd{S} + \int_{\R}\int_{\partial \Omega_{\alpha}}\varphi \bm{v}_{\Gamma} \vdot \bm{n}\dd{S}\dd{t}) = \\ = \int_{\R}\int_{\partial \Omega_{\alpha}(t)}\varphi \bm{v}_{\Gamma} \vdot \bm{n}\dd{S}\dd{t},
\end{equations}
where $\varphi\qty(t, \bm{x})\in \mathcal{D}\qty(\R^3 \times \R).$ Next, we need

\begin{equations}[lines,*]
	\int_{\R^3 \times \R}\qty(\nabla_i \chi_{\alpha})\dd{\bm{x}} \dd{t} = - \int_{\R^3 \times \R}\chi_{\alpha}\chi_{\alpha} \pdv{\varphi}{x_i}\dd{\bm{x}}\dd{t} - \int_{\R}\int_{\Omega_{\alpha}}\pdv{\varphi}{x_i}\dd{\bm{x}}\dd{t} = \\ = - \int_{\R}\int_{\partial \Omega_{\alpha}}\varphi \bm{n}_i \dd{S} \dd{t} = \\= \int_{\R^3 \times \R}\varphi\qty(- n_{i \Gamma} \delta_{\Gamma})\dd{\bm{x}}\dd{t},
\end{equations}

where $\delta_{\Gamma}$ is the Dirac delta with $\, \text{supp} \,\delta_{\Gamma} = \Gamma \subset \R^2.$ Next we use

\begin{equations}[single,*]
\grad \chi_{\alpha} = - \bm{n}_{\Gamma} \delta_{\partial \Omega_{\alpha}},
\end{equations}
where $\bm{n}_{\Gamma}$ is the \emph{outter unit normal to}\sidenote{This means that the characteristic function rises in the direction to $\Omega_{\alpha}$} $\partial \Omega_{\alpha}.$ We will also need the \emph{topological equation}

\begin{equations}[single,!,numberline=all]
\label{eq:topological_equation}
\partial_t \chi_{\alpha} + \tilde{\bm{v}}_{\Gamma} \vdot \grad \chi_{\alpha} = 0,
\end{equations}

where

\begin{equations}[lines,*]
	\int_{\R^3 \times \R}\qty(\tilde{\bm{v}}_{\Gamma} \vdot \grad \chi_{\alpha}) \varphi\dd{\bm{x}} \dd{t} = - \int_{\R^3 \R}\qty(\divergence{\qty(\tilde{ \bm{v}}_{\Gamma}) \varphi})\dd{\bm{x}} \dd{t} = - \int_{\R}\int_{\Omega_{\alpha}}\divergence{\qty(\tilde{\bm{v}}_{\Gamma} \varphi)}\dd{\bm{x}}\dd{t} = \\ = - \int_{\R}\int_{\partial \Omega_{\alpha}}\bm{v}_{\Gamma} \vdot \bm{n}_{\Gamma}\varphi\dd{S}\dd{t}.
\end{equations}

Next next, we need to \emph{give meaning} to 

\begin{equations}[single,*]
\nabla_i\qty(\chi_{\alpha} f), \partial_t \qty(\chi_{\alpha} f).
\end{equations}

This is \textbf{a problem}: both $f$ and $\chi_{\alpha}$ are not continuous, \textit{i.e.} their derivatives are distributions; \emph{we cannot use a simple Leibniz formula.}

\begin{equations}[lines,*]
	\int_{\R^3 \times \R}\nabla_i\qty(\chi_{\alpha} f) \varphi\dd{\bm{x}} \dd{t} = - \int_{\R}\int_{\R^3}\chi_{\alpha}f \qty(\nabla_i \varphi)\dd{\bm{x}}\dd{t} = - \int_{\R}\int_{\Omega_{\alpha}}f \pdv{\varphi}{x_i}\dd{\bm{x}}\dd{t} = \\ = \int_{\R}\int_{\Omega_{\alpha}}\pdv{f}{x_i}\dd{\bm{x}}\dd{t} = - \int_{\R}\int_{\partial \Omega_{\alpha}}f_{\alpha}^{\Gamma} n_i \varphi\dd{S}\dd{t} = \int_{\R^3 \times \R}\qty(\chi_{\alpha} \nabla_i f)\varphi\dd{\bm{x}} \dd{t} + \int_{\R^3 \times \R}\qty(f_{\Gamma}^{\alpha} \grad \chi_{\alpha})\varphi\dd{\bm{x}} \dd{t}
\end{equations}

where $f^{\alpha}_{\Gamma}$ is the value from "inside of $\Omega_{\alpha}$" \textcolor{gray}{the function is discontinuous.} Meaning that

\begin{equations}[single,*]
\nabla_i\qty(f \chi_{\alpha}) = \chi_{\alpha} \nabla_i f + f_{\Gamma}^{\alpha} \grad \chi_{\alpha}.
\end{equations}

And for the time derivative we obtain

\begin{equations}[lines,*]
	\int_{\R^3 \times \R}\partial_t\qty(\chi_{\alpha} f)\varphi\dd{\bm{x}} \dd{t} = - \int_{\R^3 \times |R}\chi_{\alpha} f \partial_t \varphi\dd{\bm{x}}\dd{t} = \\ = - \int_{\R}\int_{\Omega_{\alpha}}f \partial_t \varphi\dd{\bm{x}}\dd{t} = - \int_{\R}\int_{\Omega_{\alpha}}\partial_t\qty(f \varphi)\dd{\bm{x}}\dd{t} + \int_{\R}\int_{\Omega_{\alpha}}\qty(\partial_t \varphi) \varphi\dd{\bm{x}}\dd{t} = \\ = - \int_{\R}\dv{t}\qty(\int_{\Omega_{\alpha}}f \varphi\dd{\bm{x}})\dd{t} + \int_{\R}\int_{\partial \Omega_{\alpha}}f^{\alpha}_{\Gamma} f \bm{v}_{\Gamma} \vdot \bm{n}\dd{S}\dd{t}\int_{\R^3 \times \R}\qty(\chi_{\alpha} \partial_t f)\varphi\dd{\bm{x}}\dd{t} = \\ = 0 + \int_{\R^3 \times \R}\qty(\partial_t \chi_{a} f_{\Gamma}^{\alpha} + \chi_{\alpha} \partial_t f)\varphi\dd{\bm{x}} \dd{t},
\end{equations}
and so in total

\begin{equations}[single,*]
	\partial_t\qty(\chi_{\alpha} f) = \chi_{\alpha} \partial_t f + \qty(\partial_t \chi_{\alpha}) f_{\Gamma}^{\alpha}.
\end{equations} 

Note that we are in fact working with the \emph{smooth extension} of $f_{\Gamma}^{\alpha}$ rather than the function itself \textcolor{gray}{(which is defined only on the boundary)}, so we in fact are in the case of \emph{Dirac times a smooth function.}
	
Finally, the evolution equation can be then rewritten as

\begin{equations}[single,!,numberline=all]
\label{eq:evol}
\chi_{\alpha}\qty(\partial_t \varphi + \divergence{\qty(\varphi \bm{v})} + \divergence{\bm{\Phi}_{\Omega}^{\psi}} - \Pi_{\Omega}^{\psi} - \xi_{\Omega}^{\psi}) = 0
\end{equations}

or equivalently

\begin{equations}[lines,*]
	\partial_t\qty(\chi_{\alpha} \psi) - \qty(\tilde{\psi}_{\Gamma}^{\varphi} \partial_t \chi_{\alpha}) + \divergence{\qty(\chi_{\alpha} \psi_{\Omega} \bm{v})} - \qty(\tilde{\qty(\psi_{\Omega} \bm{v})^{\alpha}_{\gamma}} \vdot \grad \chi_{\alpha})+ \\ + \divergence{\qty(\chi_{\alpha} \bm{\Phi}^{\psi}_{\Omega})} - \tilde{\bm{\Phi}_{\Omega, \Gamma}^{\alpha}} \vdot \grad \chi_{\alpha} - \chi_{\alpha} \Pi_{\Omega}^{\psi} - \chi_{\alpha} \xi_{\Omega}^{\psi} = 0.
\end{equations}

As for averaging, we have multiple options

\begin{itemize}
	\item ensemble averaging (over configurations in the phase space)
	\item time averaging (throughout evolution)
	\item volume averaging (our introduced)
\end{itemize}

What properties do we require?
\begin{itemize}
	\item \emph{linearity}: $\expval{c_1 f_1 + c_2 f_2} = c_1 \expval{f_1} + c_2 \expval{f_2},$
	\item \emph{perfect filter property}: $\expval{f_1 \expval{f_2}} = \expval{f_1} \expval{f_2}.$
\end{itemize}

We see that linearity holds for all of them, but \textit{e.g.} the \emph{perfect filter property does not hold for volume averaging.} And there is a bigger problem also, we would want

\begin{equations}[columns,*]
	\expval{\partial_t f} &= \partial_t \expval{f}, \\
	\expval{\partial_{i} f} &= \partial_{i} \expval{f}.
\end{equations}

Note that \textit{e.g.} the second requirement \textbf{does not hold in general}: i can have lots if discontinuities on the LHS, which produce diracs in the differentation, but the RHS is nicely smooth.

Not everything is lost. It can be shown that for \emph{volume averaging}, it holds

\begin{equations}[single,!,numberline=all]
\label{eq:averaging_property}
\expval{\chi_{\alpha} \nabla_i f}_V = \nabla_i \expval{\chi_{\alpha} f}_V + \expval{f^{\alpha}_{\Gamma} \nabla_i \chi_{\alpha}}_{V}.
\end{equations}

This allows us to do the volume averaging of the balance equations to obtain

\begin{equations}[lines,*]
	\partial_t \expval{\psi_{\Omega} \chi_{\alpha}} + \divergence{\qty(\expval{\psi_{\Omega} \chi_{\alpha} \bm{v}})} + \divergence{\qty(\chi_{\alpha} \bm{\Phi}^{\psi}_{\Omega})} + \\ - \expval{\chi_{\alpha}^{\psi} \Pi_{\Omega}^{\psi}} - \expval{\chi_{\alpha} \xi_{\Omega}^{\psi}} = \expval{\qty(\psi_{\Omega})^{\alpha}_{\Gamma} \partial_t \chi_{\alpha} + \tilde{\qty(\psi_{\Omega} \bm{v}_{\Omega})}_{\Gamma}^{\alpha} \vdot \grad \chi_{\alpha}} + \tilde{\bm{\Phi}_{\Omega}}_{\Gamma}^{\alpha} \vdot \grad \chi_{\alpha},
\end{equations} 

where the last term can be manipulated as

\begin{equations}[single,*]
	\expval{\tilde{\qty(\varphi_{\Omega})}_{\Gamma}^{\alpha}\qty(\tilde{\bm{v}})_{\Gamma}^{\alpha} - \tilde{\bm{v}_{\Gamma}} \vdot \grad \chi_{\alpha} + \tilde{\qty(\bm{\Phi})_{\Omega}}_{\Gamma}^{\alpha} \vdot \grad \chi_{\alpha}},
\end{equations}

where we have used the topological equation. Recognizing the volume fraction $\Phi_{\alpha} = \expval{\chi_{\alpha}}$ we can recover the known expressions as \emph{phasic averages}

\begin{equations}[single,*]
	\overline{\varphi}_{\alpha} \Phi_{\alpha} = \expval{\chi_{\alpha} \varphi_{\Omega}}, 
\end{equations}

so for $\rho$ we obtain

\begin{equations}[single,*]
\frac{1}{\lambda(B)} \int_{B_{\alpha}}\rho\dd{x} = \qty(\frac{1}{\lambda(B)} \int_{B_{\alpha}}\rho\dd{x}) \qty(\frac{\lambda\qty(B_{\alpha})}{\lambda\qty(B)}) = \rho_{\alpha}^m \Phi_{\alpha}.
\end{equations}
We see a second option would be to use the mass-weighted average

\begin{equations}[single,*]
\hat{\varphi}_{\alpha} \overline{\rho}_{\alpha} \Phi_{\alpha} = \expval{\chi_{\alpha} \rho \varphi_{\Omega}}.
\end{equations}

Let us recover some known equations.

\subsubsection{Mass balance}
\label{sec:mass_recovery}

\begin{equations}[single,*]
	\partial_t \expval{\chi_{\alpha} \rho} + \divergence{\qty(\expval{\chi_{\alpha} \rho \bm{v}})} = \expval{\tilde{\qty(\rho_{\Omega})}^{\alpha}_{\Gamma}\qty(\tilde{\qty(\bm{v}_{\Omega})}_{\Gamma}^{\alpha} - \tilde{\bm{v}}_{\Gamma}) \vdot \grad \chi_{\alpha}}.
\end{equations}

Realize now that the LHS is $\partial_t\qty(\Phi_{\alpha} \overline{\rho_{\alpha}}) + \divergence{\qty(\Phi_{\alpha} \overline{\rho}_{\alpha}\hat{\bm{v}_{\alpha}})} = m_{\alpha},$ so \emph{also the RHS} must be $m_{\alpha}.$ We have obtained new expressions for $m_{\alpha}.$

\subsubsection{Momentum balance}
\label{sec:mom_recovery}

\begin{equations}[lines,*]
	\partial_t\expval{\chi_{\alpha} \rho \bm{v}} + \divergence{\qty(\expval{\chi_{\alpha} \rho \bm{v} \otimes \bm{v}})} - \divergence{\qty(\expval{\chi_{\alpha}\cstress_{\Omega}})} - \expval{\chi_{\alpha} \rho \bm{b}} = \\ = \expval{\qty(\tilde{\qty(\rho_{\Omega})}^{\alpha}_{\Gamma} \tilde{\qty(\bm{v}_{\Omega})}^{\alpha}_{\Gamma} \otimes \qty(\tilde{\qty(\bm{v}_{\Omega})}_{\Gamma}^{\alpha} - \tilde{\bm{v}_{\Gamma}}) - \qty(\tilde{\cstress_{\Omega}})_{\Gamma}^{\alpha}) \vdot \grad \chi_{\alpha}},
\end{equations}

and so again taking the time derivative and manipulating we obtain \emph{something}. It is interesting that the RHS will be

\begin{equations}[single,*]
- p_{\Gamma}^{\alpha} \grad \Phi_{\alpha} + \bm{I}_{\alpha}^{\, \text{drag} \,},
\end{equations}

which is exatly the expression we obtained in the chapter abut \emph{Darcy law}.

\end{document}
