\documentclass[reqno, a4paper]{article}
\usepackage{amsmath}
\usepackage{amssymb}
\usepackage{amsthm}

% PAGE DIMENSION

% BIBLIOGRAPHY
\usepackage{natbib}
\usepackage{bibentry} % inline refereces

% ENCODING, LANGUAGE
\usepackage[english]{babel}
\usepackage[utf8]{inputenc}

% GRAPHICS
\usepackage{subfig}
\usepackage{graphicx}
\usepackage{tikz}
\usetikzlibrary{intersections,calc}
\usepackage{tcolorbox}

% HYPERTEXT, SOURCE CODE SPECIALS
\usepackage[unicode]{hyperref}
\usepackage[active]{srcltx} % use TeX-souce-specials-mode

% SYMBOLS, FONTS
\usepackage{mathbbol}
\usepackage{bm} % sophisticated \boldsymbol
\usepackage{physics}
%\usepackage{stmaryrd}
\usepackage{MnSymbol} % \lsem, \rsem, tensor product :
\usepackage{gensymb}
\usepackage{eurosym}

% UNITS, TYPESETTING TENSORS
\usepackage{units}
\usepackage{tensor}
\usepackage{accents}

% COMPACT LIST ENVIRONMENT
\usepackage{enumitem}

% LINE NUMBERS
\usepackage{lineno}

\usepackage{multicol}

% SELECTIVELY INCLUDE/EXCLUDE PARTS OF TEXT
\usepackage{comment}

% FLOAT BARRIER
\usepackage{placeins}

%\makeatletter
% \@ifpackageloaded{tensor}% tensor is a package for a better typesetting of tensors
% {
% \renewcommand{\tnsr@Aux}[3][]{%
% \mathpalette{\tnsr@Plt{#1}{#3}}{\mathrm #2}%
% \tnsr@Wrn
% }%\tnsr@Aux
% }{%
% \relax%
% }
% \makeatother


\theoremstyle{definition}
\newtheorem{definition}{Definition}
\newtheorem*{example}{Example}

\theoremstyle{plain}
\newtheorem{lemma}{Lemma}
\newtheorem{theorem}{Theorem}

\theoremstyle{remark}
\newtheorem*{remark}{Remark}

% operators
\DeclareMathOperator{\Sym}{Sym}
\DeclareMathOperator{\signum}{sign}
\DeclareMathOperator{\supp}{supp}
\DeclareMathOperator{\diam}{diam}
\DeclareMathOperator{\cof}{cof} % cofactor
\DeclareMathOperator{\residue}{res}
\DeclareMathOperator{\ad}{ad} % adjoint ad_X (Y) = [X,Y]  
\DeclareMathOperator{\dist}{dist} % distance in a metric space

% Load xparse (if not already loaded)
\usepackage{xparse}

% Continuous functions


\newcommand{\CkSet}[2]{%
	\ensuremath{\text{C}^{#1}\!\,\left(#2 \right)}}%

\newcommand{\CinfSet}[1]{%
	\ensuremath{\text{C}^{\infty}\!\,\left(#1 \right)}}%

\newcommand{\DSet}[1]{%
	\ensuremath{\mathcal{D}\!\,\left(#1 \right)}}%

\newcommand{\CklSet}[3]{%
	\ensuremath{\text{C}^{\!\, \,#1,#2}\!\,\left(#3 \right)}}%

\newcommand{\Ckl}[2]{%
\ensuremath{\text{C}^{\!\,\, #1,#2}}}%



%%%%%%%%%%%%%%%%%%%%%%%%%%%%%%%%%%%%%%%%%%%%%%%
% Lebesgue Spaces and Their Norms
%%%%%%%%%%%%%%%%%%%%%%%%%%%%%%%%%%%%%%%%%%%%%%%

% Generic Lebesgue space on a set.
\DeclareDocumentCommand{\LpSet}{ o m }{%
	\ensuremath{\text{L}_{\IfNoValueTF{#1}{\text{p}}{#1}}\!\left( #2 \right)}%
}


\newcommand{\LinfSet}[1]{%
	\ensuremath{\text{L}_{\infty}\!\,\left(#1 \right)}}%


% Norm in a Lebesgue space on a set.
\DeclareDocumentCommand{\NormLpSet}{ O{p} m m }{%
	\ensuremath{\norm{#2}_{\text{L}_{\IfNoValueTF{#1}{\text{p}}{#1}}\!\left( #3 \right)}}%
}

%%%%%%%%%%%%%%%%%%%%%%%%%%%%%%%%%%%%%%%%%%%%%%%
% Lebesgue-Bochner Spaces and Their Norms
%%%%%%%%%%%%%%%%%%%%%%%%%%%%%%%%%%%%%%%%%%%%%%%

%Generic Lebesgue - Bochner space on a set.
\newcommand{\LpIntX}[4]{%
	\ensuremath{\text{L}_{\text{#1}}\!\,\Bigl( (#2,#3);#4 \Bigr)}%
}

\newcommand{\LinfIntX}[3]{%
	\ensuremath{\text{L}_{\infty}\!\,\Bigl( (#1,#2);#3 \Bigr)}%
}

% Norm in a Lebesgue space on a set.
\newcommand{\NormLpIntX}[5]{%
	\ensuremath{\norm{#1}_{\text{L}_{\text{#2}}\!\,\left( (#3,#4);#5 \right)}}%
}

\newcommand{\NormLinfIntX}[4]{%
	\ensuremath{\norm{#1}_{\text{L}_{\infty}\!\,\left( (#2,#3);#4 \right)}}%
}


%%%%%%%%%%%%%%%%%%%%%%%%%%%%%%%%%%%%%%%%%%%%%%%
% Sobolev Spaces and Their Norms
%%%%%%%%%%%%%%%%%%%%%%%%%%%%%%%%%%%%%%%%%%%%%%%

% Generic Sobolev space on a set.
\DeclareDocumentCommand{\WkpSet}{ o o m }{%
	\ensuremath{\text{W}^{\IfNoValueTF{#1}{\text{k}}{#1},\IfNoValueTF{#2}{\text{p}}{#2}}\!\left( #3 \right)}%
}


% Sobolev space with zero boundary conditions on a set.
\DeclareDocumentCommand{\WkpzeroSet}{ o o m }{%
	\ensuremath{\text{W}^{\IfNoValueTF{#1}{\text{k}}{#1},\IfNoValueTF{#2}{\text{p}}{#2}}_0\!\left( #3 \right)}%
}

% Norm in a Sobolev space on a set.
\DeclareDocumentCommand{\NormWkpSet}{ O{k} O{p} m m }{%
	\ensuremath{\norm{#3}_{W^{\IfNoValueTF{#1}{\text{k}}{#1},\IfNoValueTF{#2}{\text{p}}{#2}}\!\left( #4 \right)}}%
}


\newcommand{\WminfSet}[2]{%
	\ensuremath{\text{W}^{#1, \infty}\!\,\left(#2 \right)}}%
% Norm in a Sobolev space with zero boundary conditions on a set.
\DeclareDocumentCommand{\NormWkpzeroSet}{ O{k} O{p} m m }{%
	\ensuremath{\norm{#3}_{W^{\IfNoValueTF{#1}{\text{k}}{#1},\IfNoValueTF{#2}{\text{p}}{#2}}_0\!\left( #4 \right)}}%
}

% Differential operators
\DeclareMathOperator{\laplace}{\bigtriangleup}
% Kernel, range, rank
\DeclareMathOperator{\kernelop}{{\mathcal N}}
\DeclareMathOperator{\rangeop}{{\mathcal R}}
\DeclareMathOperator{\rankop}{rank}
% jump
\newcommand{\jumpdis}[1]{\ensuremath{\left\lsem #1 \right\rsem}} % difference between function values at the point of jump discontinuity

% hyperbolic functions
\DeclareMathOperator{\arcsinh}{arcsinh}
\DeclareMathOperator{\arccosh}{arccosh}
\DeclareMathOperator{\arctanh}{arctanh}
\DeclareMathOperator{\arccoth}{arccoth}

% sinc function
\DeclareMathOperator{\sinc}{sinc}

% invariants of second order tensor
\DeclareMathOperator{\invariantI}{I_1}
\DeclareMathOperator{\invariantII}{I_2}
\DeclareMathOperator{\invariantIII}{I_3}

% big o
\newcommand{\bigo}[1]{\ensuremath{O\left(#1 \right)}}
\newcommand{\smallo}[1]{\ensuremath{o\left(#1 \right)}}


% imaginary unit
\newcommand{\iunit}{\ensuremath{\mathrm{i}}}


% real and imaginary part
\newcommand{\realp}{\mathrm{real}}
\newcommand{\imagp}{\mathrm{imag}}

%\newcommand{\Real}{\Re}
%\newcommand{\Imag}{\Im}
\providecommand{\Real}{\Re}
\providecommand{\Imag}{\Im}

% predicates
\newcommand{\charac}{\ensuremath{\mathrm{char}}} % characteristic quantity such as length scale, etc.
\newcommand{\reference}{\mathrm{ref}}
\newcommand{\boundary}{\mathrm{bdr}}
\newcommand{\initial}{\mathrm{init}}
\newcommand{\crit}{\mathrm{crit}}
\newcommand{\bydefinition}{\mathrm{def}}
\newcommand{\traceless}[1]{{#1}_{\delta}}

% dimensionless variables and functions
\newcommand{\dimless}[1]{#1^\star}

% derivatives
\newcommand{\diff}{\mathrm{d}}
\newcommand{\Diff}[1][]{\mathrm{D}_{#1}} % For Frechet and Gateaux derivative
\newcommand{\hDiff}[2][]{\mathrm{D}^{#1}_{#2}} % Higher order Frechet and Gateaux derivative

% inexact differential
\newcommand{\dbar}{{\mathchar'26\mkern-12mu \diff}}
\newcommand{\idiff}{\dbar}

% body
\newcommand{\body}{{\mathcal B}}

% vectors and tensors
\renewcommand{\vec}[1]{\ensuremath{\mathbf{#1}}}
\newcommand{\greekvec}[1]{\ensuremath{\boldsymbol{#1}}}
\makeatletter
\@ifpackageloaded{bm}% 
{\renewcommand{\vec}[1]{\ensuremath{\bm{#1}}}%
\renewcommand{\greekvec}[1]{\ensuremath{\bm{#1}}}%
}{%
\relax% do nothing
}
\makeatother

\newcommand{\tensorq}[1]{\ensuremath{\mathbb{#1}}}      % tensorial quantity
\newcommand{\tensorc}[1]{\ensuremath{\mathrm{#1}}}      % tensorial quantity components  

\newcommand{\conjugate}[1]{#1^\star}
\newcommand{\transpose}[1]{#1^\top}
\newcommand{\transposei}[1]{#1^{-\top}}
\newcommand{\inverse}[1]{#1^{-1}}

% Identity matrix and zero matrix
\newcommand{\identity}{\ensuremath{\tensorq{I}}} % identity
\newcommand{\tensorzero}{{\mathbb{O}}} % zero tensor

% Cauchy stress
\newcommand{\cstress}{\tensorq{T}}
\newcommand{\cstressc}{\tensorc{T}}

% Cauchy stress, thermodynamically determined part
%\DeclareMathSymbol{\robustrho}{\mathord}{letters}{"1A} % If I want to write \fid{\thcstressrho} it sometimes happens that the greek letters in subscripts get crippled, this happens especially in MDPI class. This trick protects \rho. It would work also for other greek letters; the codes are given in fontdef.dtx
% Sometimes it also helps to swith of the bm package.
\newcommand{\thcstress}{\ensuremath{\cstress_{\mathrm{th}}}} 
%\newcommand{\thcstressrho}{\ensuremath{\cstress_{\mathrm{th},\, \robustrho}}} % thermodynamically determined part divided by rho
\newcommand{\thcstressrho}{\ensuremath{\cstress_{\mathrm{th},\, \mathnormal{\rho}}}} % thermodynamically determined part divided by rho
\newcommand{\tracelessthcstress}{\traceless{\left(\thcstress\right)}} % traceless part
\newcommand{\tracelessthcstressrho}{\traceless{\left(\cstress_{\mathrm{th},\, \rho}\right)}} % traceless part divided by rho

% Extra stress tensor
\newcommand{\ecstress}{\tensorq{S}}
\newcommand{\ecstressc}{\tensorc{S}}

% First Piola stress tensor
\newcommand{\fpstress}{\tensorq{T}_{\mathrm{R}}}
\newcommand{\fpstressc}{\tensorc{T}_{\mathrm R}}

% Second Piola--Kirchhoff stress tensor
\newcommand{\spstress}{\tensorq{S}_{\mathrm{R}}}
\newcommand{\spstressc}{\tensorc{S}_{\mathrm{R}}}

% Couple stress tensor
\newcommand{\couplestress}{\tensorq{M}}
\newcommand{\couplestressc}{\tensorc{M}}

% deformation, deformation gradient
\newcommand{\deformation}{\greekvec{\chi}}
\newcommand{\deformationc}{\tensorc{\chi}}

\newcommand{\fg}{\tensorq{F}}
\newcommand{\detf}{\det\, \fg}
\newcommand{\fgradc}{\tensorc{F}}
\newcommand{\fgradrel}[3][]{\fgrad^{#1}_{#2}\left(#3\right)}

% determinant of deformation gradient, Jacobian
\newcommand{\detfgrad}{J}

% displacement
\newcommand{\displacement}{\vec{U}}
\newcommand{\displacementc}{\tensorc{U}}

% right Cauchy-Green tensor
\newcommand{\rcg}{\tensorq{C}}
\newcommand{\rcgc}{\tensorc{C}}        
\newcommand{\rcgrel}[3][]{\rcg^{#1}_{#2}\left(#3\right)}

\newcommand{\rcgb}{\overline{\rcg}} % rescaled right Cauchy--Green tensor, theory of slightly compressible materials
\newcommand{\rcgbc}{\overline{\rcgc}} % rescaled right Cauchy--Green tensor, theory of slightly compressible materials, components

% left Cauchy-Green tensor
\newcommand{\lcg}{\tensorq{B}}
\newcommand{\lcgc}{\tensorc{B}}        
\newcommand{\lcgrel}[3][]{\lcg^{#1}_{#2}\left(#3\right)}

\newcommand{\lcgb}{\overline{\lcg}} % rescaled left Cauchy--Green tensor, theory of slightly compressible materials
\newcommand{\lcgbc}{\overline{\lcgc}} % rescaled left Cauchy--Green tensor, theory of slightly compressible materials, components


%\newcommand{\piolastrain}{\tensorq{b}} % Piola deformation tensor (inverse of right Cauchy--Green)
%\newcommand{\fingerstrain}{\tensorq{c}} % Finger deformation tensor (inverse of left Cauchy--Green)

% rotation
\newcommand{\rotation}{\tensorq{R}}
\newcommand{\rotationrel}[3][]{\rotation^{#1}_{#2}\left(#3\right)}

% stretch
\newcommand{\stretchu}{\tensorq{U}}
\newcommand{\stretchurel}[3][]{\stretchu^{#1}_{#2}\left(#3\right)}
\newcommand{\stretchv}{\tensorq{V}}
\newcommand{\stretchvrel}[3][]{\stretchv^{#1}_{#2}\left(#3\right)}

% linearized strain (symmetric part of displacement gradient), skew-symmetric part of displacement gradient
% THIS MUST BE FIXED
\makeatletter
\@ifpackageloaded{bm}% 
{%
\newcommand{\linstrain}{\bbespilon} %requires \usepackage[bbgreekl]{mathbbol}
% YES, the spelling is wrong, but this is how it is coded in the package
}{%
\newcommand{\linstrain}{\bbespilon} %requires \usepackage[bbgreekl]{mathbbol}
%\newcommand{\linstrain}{\tensorq{\varepsilon}}
}

\@ifpackageloaded{bm}%
{%
\newcommand{\skewdgradient}{\bbomega} 
}{%
\newcommand{\skewdgradient}{\tensorq{\omega}}
}

\@ifpackageloaded{bm}%
{%
\newcommand{\linstress}{\bbtau} % stress in linearised elasticity
}{%
\newcommand{\linstress}{\bbtau} % stress in linearised elasticity
%\newcommand{\linstress}{\tensorq{\tau}}
}
\makeatother

\newcommand{\linstrainc}{\mathrm{\varepsilon}}
\newcommand{\linstressc}{\mathrm{\tau}}
\newcommand{\skewdgradientc}{\mathrm{\omega}}

% Lagrangean and Eulerian strain
\newcommand{\lstrain}{\tensorq{E}} % Green--Saint-Venant strain
\newcommand{\lstrainc}{\tensorc{E}} % Green--Saint-Venant strain, components
\newcommand{\estrain}{\tensorq{e}} % Euler--Almansi strain, components
\newcommand{\estrainc}{\tensorc{e}} % Euler--Almansi strain, components

% Hencky strain
\newcommand{\henckystrain}{\tensorq{H}} % Hencky strain
\newcommand{\henckystrainc}{\tensorc{H}} % Hencky strain, components

\newcommand{\henckystrainb}{\overline{\tensorq{H}}} % Hencky strain for rescaled left Cauchy--Green tensor
\newcommand{\henckystrainbc}{\overline{\tensorc{H}}} % Hencky strain for rescaled left Cauchy--Green tensor, components

\newcommand{\devhencky}{\overline{\tensorq{H}}} % Hencky strain, deviatoric part via deviatoric deformation
\newcommand{\devhenckyc}{\overline{\tensorc{H}}} % Hencky strain, deviatoric part via deviatoric deformation, components

% Hencky strain, Lagrangian
\newcommand{\henckystrainr}{\tensorq{G}} % Hencky strain 
\newcommand{\henckystrainrc}{\tensorc{G}} % Hencky strain, components

\newcommand{\henckystrainrb}{\overline{\tensorq{G}}} % Hencky strain for rescaled right Cauchy--Green tensor
\newcommand{\henckystrainrbc}{\overline{\tensorc{G}}} % Hencky strain for rescaled right Cauchy--Green tensor, components

\newcommand{\devhenckyr}{\overline{\tensorq{G}}} % Hencky strain, deviatoric part via deviatoric deformation
\newcommand{\devhenckyrc}{\overline{\tensorc{G}}} % Hencky strain, deviatoric part via deviatoric deformation, components

% Rivlin-Ericksen tensor
\newcommand{\rivlin}{{\tensorq{A}}}

% generic tensor quantity
\newcommand{\generictensor}{{\tensorq{A}}}
\newcommand{\generictensorc}{\tensorc{A}} % component of the tensor

% deviatoric part of Cauchy stress
\newcommand{\dcstress}{\cstress - \left( \frac{1}{3}\Tr \cstress \right) \identity}
\newcommand{\dcstresssymb}{\traceless{\cstress}}

% mean normal stress
\newcommand{\cstressnorm}{\frac{1}{3}\Tr \cstress}

% velocity and velocity gradient, (skew)symmetric part of velocity gradient
\newcommand{\vecv}{\ensuremath{\vec{v}}}
\newcommand{\gradv}{\ensuremath{\nabla \vecv}}
\newcommand{\gradasym}{\ensuremath{\tensorq{W}}}
\newcommand{\gradsym}{\ensuremath{\tensorq{D}}}
\newcommand{\dgradsymsymb}{\ensuremath{\gradsym_{\delta}}}
\newcommand{\gradvl}{\ensuremath{\tensorq{L}}}

% logarithmic spin
\newcommand{\logspin}{\ensuremath{\tensorq{\Omega}}^{\mathrm{log}}}

% surface velocity
\newcommand{\unders}[1]{\ensuremath{\underaccent{\mathrm{s}}{#1}}}

\newcommand{\gradsymop}{\nabla_{\mathrm{sym}}}
\newcommand{\gradasymop}{\nabla_{\mathrm{asym}}}

\newcommand{\vecvc}{\tensorc{v}}

% velocity and velocity gradient, (skew)symmetric part of velocity gradient, COMPONENTS
\newcommand{\gradsymc}{\tensorc{D}}
\newcommand{\gradasymc}{\tensorc{W}}

% functionals
\newcommand{\functional}[1]{{\mathfrak #1}}
\newcommand{\fhistory}[3]{{\functional{#1}_{#2}^{#3}}}

% base vectors
\newcommand{\bvec}[1]{\vec{e}_{#1}} % current configuration
\newcommand{\Bvec}[1]{\vec{E}_{#1}} % reference configuration

% dual base vectors
\newcommand{\bvecd}[1]{\vec{e}^{#1}} % current configuration
\newcommand{\Bvecd}[1]{\vec{E}^{#1}} % reference configuration

% Cartesian basis, current configuration
\newcommand{\bvecx}{\bvec{\hat{x}}}
\newcommand{\bvecy}{\bvec{\hat{y}}}
\newcommand{\bvecz}{\bvec{\hat{z}}}

% Cartesian basis, reference configuration
\newcommand{\BvecX}{\Bvec{\hat{X}}}
\newcommand{\BvecY}{\Bvec{\hat{Y}}}
\newcommand{\BvecZ}{\Bvec{\hat{Z}}}

% Cartesian dual basis, reference configuration
\newcommand{\BvecdX}{\Bvecd{\hat{X}}}
\newcommand{\BvecdY}{\Bvecd{\hat{Y}}}
\newcommand{\BvecdZ}{\Bvecd{\hat{Z}}}

% Cartesian dual basis, current configuration
\newcommand{\bvecdx}{\bvecd{\hat{x}}}
\newcommand{\bvecdy}{\bvecd{\hat{y}}}
\newcommand{\bvecdz}{\bvecd{\hat{z}}}

% same as above but now in cylindrical coordinates
\newcommand{\bvecr}{\bvec{\hat{r}}}
\newcommand{\bvect}{\bvec{\hat{\theta}}}
\newcommand{\bvecp}{\bvec{\hat{\varphi}}}
%\newcommand{\bvecz}{\bvec{\hat{z}}}

\newcommand{\bvecdr}{\bvecd{\hat{r}}}
\newcommand{\bvecdt}{\bvecd{\hat{\theta}}}
\newcommand{\bvecdp}{\bvecd{\hat{\varphi}}}

\newcommand{\BvecR}{\Bvec{\hat{R}}}
\newcommand{\BvecP}{\Bvec{\hat{\Phi}}}
%\newcommand{\BvecZ}{\Bvec{\hat{Z}}}

\newcommand{\BvecdR}{\Bvecd{\hat{R}}}
\newcommand{\BvecdP}{\Bvecd{\hat{\Phi}}}
%\newcommand{\BvecdZ}{\Bvecd{\hat{Z}}}

% components
\newcommand{\vhatx}[1][\vecvc]{{#1}^{\hat{x}}}
\newcommand{\vhaty}[1][\vecvc]{{#1}^{\hat{y}}}
%\newcommand{\bvhatz}{\vhat{e}_{\hat{z}}}

\newcommand{\vhatr}[1][\vecvc]{{#1}^{\hat{r}}}
\newcommand{\vhatt}[1][\vecvc]{{#1}^{\hat{\theta}}}
\newcommand{\vhatp}[1][\vecvc]{{#1}^{\hat{\varphi}}}
\newcommand{\vhatz}[1][\vecvc]{{#1}^{\hat{z}}}

% indices
\newcommand{\hatx}{\hat{x}}
\newcommand{\haty}{\hat{y}}
\newcommand{\hatz}{\hat{z}}
\newcommand{\hatr}{\hat{r}}
\newcommand{\hatp}{\hat{\varphi}}
\newcommand{\hatt}{\hat{\theta}}
\newcommand{\hatX}{\hat{X}}
\newcommand{\hatY}{\hat{Y}}
\newcommand{\hatZ}{\hat{Z}}

% inner and outer radius (for some calculations)
\newcommand{\Rin}{R_{\mathrm{in}}}
\newcommand{\Rout}{R_{\mathrm{out}}}
\newcommand{\rin}{r_{\mathrm{in}}}
\newcommand{\rout}{r_{\mathrm{out}}}
 
% base vectors, abstract covariant and contravariant basis, current configuration
\newcommand{\cobvec}[1]{\vec{g}_{#1}} % covariant base vector
\newcommand{\conbvec}[1]{\vec{g}^{#1}} % contravariant base vector
\newcommand{\cobvecn}[1]{\vec{g}_{\hat{#1}}} % covariant base vector, normalised
\newcommand{\conbvecn}[1]{\vec{g}^{\hat{#1}}} % contravariant base vector, normalised

% base vectors, abstract covariant and contravariant basis, reference configuration
\newcommand{\coBvec}[1]{\vec{G}_{#1}} % covariant base vector
\newcommand{\conBvec}[1]{\vec{G}^{#1}} % contravariant base vector
\newcommand{\coBvecn}[1]{\vec{G}_{\hat{#1}}} % covariant base vector, normalised
\newcommand{\conBvecn}[1]{\vec{G}^{\hat{#1}}} % contravariant base vector, normalised

% current configuration
\newcommand{\mtensor}{\tensorq{g}}  % metric tensor
\newcommand{\mtensorc}{{\mathrm g}} % metric tensor, components

% reference configuration
\newcommand{\mTensor}{\tensorq{G}}  % metric tensor
\newcommand{\mTensorc}{{\mathrm G}} % metric tensor, components

% Christoffel symbols
\newcommand{\christoffel}[2]{\tensor{\Gamma}{^{#1}_{#2}}}

% mean curvature
\newcommand{\meancurvature}{\mathrm{K}} % mean curvature

\newcommand{\mtensorref}{\tensorq{G}}  %metric tensor in reference configuration
\newcommand{\mtensorrefc}{{\mathrm G}} %metric tensor in reference configuration, components

% Kronecker delta, Levi--Civitta symbol
\newcommand{\kdelta}[1]{\tensor{\delta}{#1}}
\newcommand{\lcepsilon}[1]{\tensor{\epsilon}{#1}}

% distributions
\newcommand{\diracdelta}{\delta}
\newcommand{\Heaviside}{H}
\newcommand{\UnitTriangle}{U_{\mathrm{Triangle}}}

% hypergeometric function
\newcommand{\hypergeom}[4]{\ensuremath{ \mathrm{F}\left( \left[#1, #2 \right]; \left[ #3 \right]; #4\right)}}

% sets
\newcommand{\R}{\ensuremath{{\mathbb R}}}
%\@ifpackageloaded{hyperref}% \C is defined in hyperref package
%{\renewcommand{\C}{\ensuremath{{\mathbb C}}}%
%}{%
%\newcommand{\C}{\ensuremath{{\mathbb C}}}%
%}
%\renewcommand{\C}{\ensuremath{{\mathbb C}}}% The lines above are no longer needed?
\newcommand{\Q}{\ensuremath{{\mathbb Q}}}
\newcommand{\N}{\ensuremath{{\mathbb N}}}
\newcommand{\Z}{\ensuremath{{\mathbb Z}}}


% Reynolds, Womersley number, etc.
\newcommand{\Reynolds}{\mathrm{Re}}
\newcommand{\Womersley}{\mathrm{Wo}}
\newcommand{\Rayleigh}{\mathrm{Ra}}
\newcommand{\RayleighSqrt}{\mathrm{R}}
\newcommand{\Prandtl}{\mathrm{Pr}}
\newcommand{\Grashof}{\mathrm{Gr}}
\newcommand{\Mach}{\mathrm{Ma}}
\newcommand{\Froude}{\mathrm{Fr}}
\newcommand{\Peclet}{\mathrm{Pe}}
\newcommand{\Eckert}{\mathrm{Ec}}
\newcommand{\Brinkman}{\mathrm{Br}}
\newcommand{\Nusselt}{\mathrm{Nu}}

% Young modulus, Poisson ratio
\newcommand{\Young}{\mathrm{E}}
\newcommand{\Poisson}{\mathrm{\nu}}

% bulk modulus, shear modulus
\newcommand{\bulkm}{\mathrm{K}}
\newcommand{\shearm}{\mathrm{G}}

% Symetric and antisymetric tensors
\newcommand{\asym}[1]{\ensuremath{\Asym \left( #1 \right)}}
\newcommand{\sym}[1]{\ensuremath{\Sym \left( #1 \right)}}

% Energy, free energy, entropy, temperature
\newcommand{\tenergy}{\ensuremath{e}_{\mathrm{tot}}} % specific total energy (energy per unit mass), sum of specific internal energy and the specific kinetic energy
\newcommand{\ienergy}{\ensuremath{e}} % specific internal energy (energy per unit mass)
\newcommand{\menergy}{\ensuremath{e}_{\mathrm{mech}}} % specific mechanical energy (energy per unit mass), kinetic energy plus internal energy minus thermal contribution
\newcommand{\kenergy}{\ensuremath{e_{\mathrm{kin}}}} % specific kinetic energy (kinetic energy per unit mass)
\newcommand{\fenergy}{\ensuremath{\psi}} % specific free energy
\newcommand{\entropy}{\ensuremath{\eta}} % specific entropy
\newcommand{\entalphy}{\ensuremath{h}} % specific enthalpy
\newcommand{\gibbs}{\ensuremath{g}} % specific Gibbs free energy

% Decomposition of Helmholtz free energy to thermal and mechancial part
\newcommand{\fenergyth}{\fenergy^{\mathrm{thermal}}} % purely thermal part of Helmholtz free energy
\newcommand{\fenergymech}{\fenergy^{\mathrm{mech}}} % deformation-dependent part of Helmholtz free energy

\newcommand{\temp}{\ensuremath{\theta}} % temperature, Eulerian description
\newcommand{\Temp}{\ensuremath{\Theta}} % temperature, Lagrangian description
\newcommand{\thpressure}{\ensuremath{p_{\mathrm{th}}}} % thermodynamic pressure

\newcommand{\pressure}{\ensuremath{p}} % pressure -- incompressible fluids

\newcommand{\mns}{\ensuremath{m}} % mean normal stress
\newcommand{\temptoref}{\ensuremath{\vartheta}} % (temperature - reference temperature)/(reference temperature)

% Net energy, free energy, entropy, ...
\newcommand{\nettenergy}{\ensuremath{E}_{\mathrm{tot}}} % net total energy
\newcommand{\netmenergy}{\ensuremath{E}_{\mathrm{mech}}} % net mechanical energy
\newcommand{\netthenergy}{\ensuremath{E}_{\mathrm{therm}}} % net thermal energy
\newcommand{\netienergy}{\ensuremath{E}} % net internal energy
\newcommand{\netkenergy}{\ensuremath{E_{\mathrm{kin}}}} % net kinetic energy
\newcommand{\netentropy}{\ensuremath{S}} % net entropy
\newcommand{\netheat}{\ensuremath{Q}} % net heat

% Specific molar gas constant
\newcommand{\Rspecific}{\ensuremath{R_{\mathrm{s}}}}
\newcommand{\Rmol}{\ensuremath{R_{\mathrm{m}}}}
 
% Specific heat at constant volume 
\newcommand{\cheatvol}{\ensuremath{c_{\mathrm{V}}}}
\newcommand{\cheatvolref}{\ensuremath{c_{\mathrm{V},\, \reference}}} % reference value

% Specific heat at constant pressure 
\newcommand{\cheatpressure}{\ensuremath{c_{\mathrm{P}}}}
\newcommand{\cheatpressureref}{\ensuremath{c_{\mathrm{P},\, \reference}}} % reference value

% Density in reference configuration
\newcommand{\rhor}{\rho_{\mathrm{R}}}

% Energy flux, heat flux, entropy flux
\newcommand{\efluxc}{\vec{j}_{e}} % energy flux, current configuration
\newcommand{\eflux}{\vec{J}_{e}} % energy flux, reference configuration

\newcommand{\hfluxc}{\vec{j}_{q}}     % heat flux, current configuration
\newcommand{\hfluxcc}{\tensorc{j}_{q}}     % heat flux, current configuration, components
\newcommand{\hflux}{\vec{J}_{q}}     % heat flux, reference configuration

\newcommand{\entfluxc}{\vec{j}_{\entropy}} % entropy flux, current configurtion 
\newcommand{\entflux}{\vec{J}_{\entropy}} % entropy flux, reference configuration

% Energy source, entropy source
\newcommand{\esourcec}{\ensuremath{q_{e}}} % energy source, current configuration
\newcommand{\hsourcec}{\ensuremath{q}} % heat source, current configuration
\newcommand{\entsourcec}{\ensuremath{q_{\entropy}}} % entropy source, current configuration

% Thermodynamical fluxes and affinities
\newcommand{\thfluxc}[1]{\vec{j}_{#1}} % thermodynamic flux, current configuration
\newcommand{\thaffinityc}[1]{\vec{a}_{#1}} % thermodynamic affinity, current configuration

% Entropy production
\newcommand{\entprodc}{\xi} % entropy production, current configuration
%  The entropy evolution equation is written as \rho \dd{\entropy}{t} + \divx \entfluxc = \entprodc
\newcommand{\entprodctemp}{\zeta} % entropy production times temperature, current configuration

% Upper convected (Oldroyd) derivative
\newcommand{\fid}[1]{\ensuremath{\accentset{\triangledown}{#1}}}
\newcommand{\fidd}[1]{\ensuremath{\accentset{\triangledown \! \triangledown}{#1}}}

% Lower convected derivative
\newcommand{\lfid}[1]{\ensuremath{\accentset{\vartriangle}{#1}}}
\newcommand{\lfidd}[1]{\ensuremath{\accentset{\vartriangle \! \vartriangle}{#1}}}

% Jaumann derivative
\newcommand{\jfid}[1]{\ensuremath{\accentset{\medcircle}{#1}}}
\newcommand{\jfidd}[1]{\ensuremath{\accentset{\medcircle \! \medcircle}{#1}}}

% Logarithmic corrotational derivative
\newcommand{\logfid}[1]{\ensuremath{\accentset{\medcircle_{\mathrm{log}}}{#1}}}
\newcommand{\logfidd}[1]{\ensuremath{\accentset{\medcircle_{\mathrm{log}} \! \medcircle _{\mathrm{log}}}{#1}}}

% Green--Naghdi derivative
\newcommand{\gfid}[1]{\ensuremath{\accentset{\medsquare}{#1}}}
\newcommand{\gfidd}[1]{\ensuremath{\accentset{\medsquare \! \medsquare}{#1}}}

% Truesdell derivative
\newcommand{\tfid}[1]{\ensuremath{\accentset{\meddiamond}{#1}}}
\newcommand{\tfidd}[1]{\ensuremath{\accentset{\meddiamond \! \meddiamond}{#1}}}

% Generic objective derivative
\newcommand{\genericfid}[1]{\ensuremath{\accentset{\star}{#1}}}

% Material derivative (\dot with \overline)
\newcommand{\mdif}[1]{\ensuremath{\dot{\overline{#1}}}}

\makeatletter
\@ifpackageloaded{tensor}% tensor is a package for a better typesetting of tensors
{
\newcommand{\codev}[2]{\ensuremath{\left. {#1} \right|\indices{_{#2}}}}
}{%
\newcommand{\codev}[2]{\ensuremath{\left. {#1} \right|_{#2}}}
}
\makeatother

\makeatletter
\@ifpackageloaded{tensor}% tensor is a package for a better typesetting of tensors
{
\newcommand{\contradev}[2]{\ensuremath{\left. {#1} \right|\indices{^{#2}}}}
}{%
\newcommand{\contradev}[2]{\ensuremath{\left. {#1} \right|^{#2}}}
}
\makeatother


% Bessel and Kelvin functions

\newcommand{\BesselI}[2]{\ensuremath{{\mathrm I}_{#1}\left(#2\right)}} 
\newcommand{\BesselK}[2]{\ensuremath{{\mathrm K}_{#1}\left(#2\right)}}
\newcommand{\BesselJ}[2]{\ensuremath{{\mathrm J}_{#1}\left(#2\right)}}
\newcommand{\BesselY}[2]{\ensuremath{{\mathrm Y}_{#1}\left(#2\right)}}

\newcommand\BesselRoot[2]{\ensuremath{{\rm j}_{#1,#2}}}

\newcommand{\KelvinBer}[2]{\ensuremath{{\mathrm{ber}}_{#1}\left(#2\right)}} 
\newcommand{\KelvinBei}[2]{\ensuremath{{\mathrm{bei}}_{#1}\left(#2\right)}}
\newcommand{\KelvinKer}[2]{\ensuremath{{\mathrm{ker}}_{#1}\left(#2\right)}}
\newcommand{\KelvinKei}[2]{\ensuremath{{\mathrm{kei}}_{#1}\left(#2\right)}}

% Chebyshev polynominals
\newcommand{\Chebyshevp}[3]{\ensuremath{{\mathrm T}_{#1}^{#2}\left(#3\right)}} 
\newcommand{\Chebyshev}[2]{\Chebyshevp{#1}{}{#2}} 


% distance
\newcommand{\distance}[3][]{\distanceop_{#1}\left(#2, #3\right)} % distance in a metric space

% volume
\makeatletter
\@ifundefined{volume}{%
\newcommand{\volume}[1][\Omega]{\ensuremath{#1}}}%
{%
\renewcommand{\volume}[1][\Omega]{\ensuremath{#1}}}
\makeatother

% surface and volume elements (reference configuration)
\newcommand{\svolume}[1][\Omega]{\ensuremath{\partial #1}}
\newcommand{\volumee}{\diff \mathrm{V}}
\newcommand{\surfacee}{\diff \vec{S}}
\newcommand{\surfacees}{\diff \mathrm{S}}
\newcommand{\linee}{\diff \vec{X}}

% surface and volume elements (current configuration)
\newcommand{\cvolumee}{\diff \mathrm{v}}
\newcommand{\csurfacee}{{\diff \vec{s}}}
\newcommand{\csurfacees}{\diff \mathrm{s}}
\newcommand{\clinee}{{\diff \vec{x}}}

% volume and surface integral
\newcommand{\intvolume}[2][\volume]{\int_{#1} #2\; \volumee} % volume integral, reference configuration
\newcommand{\intcvolume}[2][\volume]{\int_{#1} #2\; \cvolumee} % volume integral, current configuration
\newcommand{\intsvolume}[2][\svolume]{\int_{#1} #2\; \surfacee} % surface integral, reference configuration
\newcommand{\intcsvolume}[2][\svolume]{\int_{#1} #2\; \csurfacee} % surface integral, current configuration
\newcommand{\intcsvolumes}[2][\svolume]{\int_{#1} #2\; \csurfacees} % surface integral, current configuration, scalar
\newcommand{\intsvolumes}[2][\svolume]{\int_{#1} #2\; \surfacees} % surface integral, reference configuration, scalar

% surface Jacobian
\newcommand{\surfacej}{\mathrm{j}}

% products
\newcommand{\tensortensor}[2]{\ensuremath{#1 \otimes #2}}

\makeatletter

\@ifpackageloaded{MnSymbol} % : as binary operator needs MnSymbol package
{
\newcommand{\tensordot}[2]{\ensuremath{#1 \vdotdot #2}} 
}{%
\newcommand{\tensordot}[2]{\ensuremath{#1 : #2}} 
}

\@ifpackageloaded{MnSymbol} % : as binary operator needs MnSymbol package
{
  \newcommand{\tensorddot}[2]{\ensuremath{#1 \vdots #2}} 
}{%
  \newcommand{\tensorddot}[2]{\ensuremath{#1 \vdots #2}} 
}
\makeatother

\newcommand{\tensortensorbox}[2]{\ensuremath{#1 \boxtimes #2}}
\newcommand{\vectordot}[2]{\ensuremath{#1 \bullet #2}}
\newcommand{\vectorcross}[2]{\ensuremath{#1 \times #2}}
\newcommand{\tensorschur}[2]{\ensuremath{#1 \circ #2}} % Schur/Hadamard product

\newcommand{\vectordotalt}[3]{\ensuremath{#1 \bullet_{#3} #2}} % alternative scalar product

\newcommand{\liebracket}[2]{\ensuremath{\left[#1, #2\right]}}

% function spaces
\newcommand{\scont}[2][\Omega]{\ensuremath{{\mathcal C}^{#2} \left(#1 \right)}} % space of continuous functions
\newcommand{\sdist}[1][\Omega]{\ensuremath{{\mathcal D} \left(#1 \right)}} % space of smooth functions with compact support
\newcommand{\sdistd}[1][\Omega]{\ensuremath{{\mathcal D}^\prime \left(#1 \right)}} % dual to the space of smooth functions with compact support

\newcommand{\schwartzd}[1][\Omega]{\ensuremath{{\mathcal S}^\prime \left(#1 \right)}}   % Schwartz space
\newcommand{\schwartz}[1][\Omega]{\ensuremath{{\mathcal S} \left(#1 \right)}}           % dual to Schwartz space           

\newcommand{\scdiv}[1][\Omega]{\ensuremath{{\mathcal V} \left(#1 \right)}}

\newcommand{\loc}{\mathrm{loc}}

\newcommand{\slebl}[2]{\ensuremath{L}^{#1}_{\loc} \left(#2 \right)}     % Lebesgue space, locally
\newcommand{\sleb}[2]{\ensuremath{L}^{#1} \left(#2 \right)}             % Lebesgue space


\newcommand{\ssob}[3]{\ensuremath{W}^{#1, #2} \left(#3 \right)}         % Sobolev space
\newcommand{\ssobzero}[3]{\ensuremath{W}_{0}^{#1, #2} \left(#3 \right)} % Sobolev space, functions with zero trace


% dualities, scalar products
\newcommand{\fadual}[4]{\left\langle #1, #2\right\rangle_{#3, #4}}
\newcommand{\fascal}[4]{\left\langle #1, #2\right\rangle_{#3, #4}}
\newcommand{\fascalalt}[2]{\left\langle #1, #2 \right\rangle} % alternative scalar product
\newcommand{\ddual}[2]{\left\langle #1, #2\right\rangle} % duality in distribution theory


% dual space
\newcommand{\dspace}[1]{#1^{\star}}

% tensorial function
\newcommand{\tensorf}[1]{{\mathfrak{#1}}}

% normal stress differences
\newcommand{\firstnsd}{N_1}
\newcommand{\secondnsd}{N_2}

% Laplace and Fourier transform
\newcommand{\laplacetransforms}{{\mathcal L}}
\newcommand{\laplacetransform}[2]{\laplacetransforms \left[#1\right]\left(#2\right)}
\newcommand{\inverselaplacetransform}[2]{\inverse{\laplacetransforms} \left[#1\right]\left(#2\right)}

\newcommand{\fouriertransforms}{{\mathcal F}}
\newcommand{\fouriertransform}[2]{\fouriertransforms \left[#1\right]\left(#2\right)}
\newcommand{\inversefouriertransform}[2]{\inverse{\fouriertransforms} \left[#1\right]\left(#2\right)}

% Radon transformation
\newcommand{\radontransforms}{{\mathcal R}}
\newcommand{\radontransform}[2]{\radontransforms \left[#1\right]\left(#2\right)}
\newcommand{\inverseradontransform}[2]{\inverse{\radontransforms} \left[#1\right]\left(#2\right)}

% Hilbert transformation
\newcommand{\hilberttransforms}{{\mathcal H}}
\newcommand{\hilberttransform}[2]{\hilberttransforms \left[#1\right]\left(#2\right)}

% Convolution
\newcommand{\convolution}[2]{#1 \ast #2}

% Lagrangian
\newcommand{\lagrangian}{{\mathcal L}}
\newcommand{\lpotential}{V}
\newcommand{\lkinetic}{T}


\begin{document}

\title{Classical problems in continuum mechanics}

\date{\today}

\author{Kamil Belan}



\begin{comment}
\begin{abstract}
  % \input{article-abstract}
  Here comes the abstract.
\end{abstract}
\end{comment}

\maketitle

\section{Curvilinear coordinates, tensor \& vector calculus}
\label{sec:curvilinear_coords}
How to write $\divergence{\vb{u}}, \curl{\vb{u}}$ etc. in polar, cylindrical and other coordinates? Notice that there are similiarities between change of coordinats $\vb{x} = \vb{x}(\vb*{\gamma})$ and deformation $\vb{x}=\vb*{\chi}(\vb{x})$, the $\lcg$ tensor and the metric tensor $\tensorq{g}$.

\subsection{Curvilinear coordinates}
\label{sec:curvi}
Let us have $x^1,x^2,\dots,x^n$ cartesian coordinates and a different set $\vb{x}=\vb{x}(\vb{y})$, for example $x=r \cos \varphi, y=r \sin \varphi, [x,y]=[x^1,x^2], [r,\varphi] = [y^1,y^2]$. That means every point on a plane can be described by using $[x^1,x^2]$ or $[r,\varphi]$. We are used to analysis in cartesian coordinates - how can i do it in a more general setting?

\begin{remark}
	The name curvilinear coordinates come from the fact that the lines $y^k=\text{const}$ are not "straight lines"
\end{remark}

\begin{definition}[Coordinate lines]
	Coordinate lines/curves are the curves
	\[
		\vb*{\gamma}^j(y^j) = \vb{x}(y^1,\dots,y^j,\dots,y^n).
	\]
	
\end{definition}

\begin{figure}
	\label{fig:coordinate_systems}
\begin{tikzpicture}
	\draw[<->] (-5,0) -- (5,0) node[anchor=north east]{$x^1$};
	\draw[<->] (0,-5) -- (0,5) node[anchor=north east]{$x^2$};
	\draw[thin,blue] (2,-5) -- (2,5) node[anchor=north east]{$x^1 =\, \text{const} \,$};
	\path [name path=vert] (2,-5) -- (2,5);

	\draw[thin,blue] (-5,-3) -- (5,-3) node[anchor=north east]{$x^2 =\, \text{const} \,$};
	\path [name path=horiz] (-5,-3) -- (5,-3);

	\path [name intersections={of=vert and horiz, by=A}];
	\draw[thick,blue,->] (A) --++ (3,0) node[anchor=north west]{$\vb{e}_2$};

	\draw[thick,blue,->] (A) --++ (0,3) node[anchor=north west]{$\vb{e}_1$};

	\draw[thin,purple] (0,0) circle (3.5cm) node[anchor=south east]{$y^1 = r = \, \text{const} \,$};
	\path [name path=circlePath] (0,0) circle (3.5cm);

	\draw[thin,purple] (0,0) -- (5,5) node[anchor=north east]{$y^2=\varphi_1=\, \text{const} \,$};
	\path [name path=radialPath] (0,0) -- (5,5);

	\path [name intersections={of=circlePath and radialPath, by=B}];

	\draw[thick,purple,->] (B) --++ (-1,1) node[anchor=south east]{$\vb{g}_1\qty(r,\varphi_1)$};
	\draw[thick,purple,->] (B) --++ (1,1) node[anchor=south east]{$\vb{g}_2\qty(r,\varphi_1)$};

	\draw[thin,purple] (0,0) -- (-5,5) node[anchor=south east]{$y^2 = \varphi_2 = \, \text{const} \,$};
	\path [name path=radialPath2] (0,0) -- (-5,5);
	\path [name intersections={of=circlePath and radialPath2, by=C}];

	\draw[thick,purple,->] (C) --++ (-1,1) node[anchor=south west]{$\vb{g}_2\qty(r,\varphi_2)$};
	\draw[thick,purple,->] (C) --++ (-1,-1) node[anchor=south east]{$\vb{g}_1\qty(r,\varphi_2)$}; 

\end{tikzpicture}
\caption{Coordinate lines and basis vectors in cartesian and polar coordinates (\textit{the length of the vectors is the same...})}
\end{figure}

\subsubsection{Basis of a vector space}
\label{sec:basis}
In cartesian coordinates: $\{\vb{e}_1,\vb{e}_2,\dots,\vb{e}_n \}$, where the vectors are \textit{tangent to the coordinate lines}, that is

\begin{equation}
	\vb{e}_i = \pdv{\vb*{\gamma}^i}{x^i}.
\end{equation}

In a curvilinear coordinate system, we can repeat the same construction. We can \textit{define a vector tangent to the coordinate line}

\begin{equation}
	\vb{g}_i \qty(\vb{y})=\dv{\vb*{\gamma}}{y^i}\qty(y^i) = \pdv{\vb{x}}{y^i}\qty(y^1, \dots, y^i, \dots y^n)
\end{equation}

The problem is that the vectors $\vb{g}_i$ are not constant in space! It is a vector field!.

\subsubsection{Vector fields}
\label{sec:vector_fields}

A vector $\vb{v}$ is independent of a basis; i can write $\vb{v}=v^i \vb{e}_i = \nu^i \vb{g}_i$. (Note that in general $v^i \neq \nu^i$.) What about its derivatives?

\[
	\pdv{\vb{v}}{x^i}=\pdv{\qty(v^j \vb{e}_j)}{x^i} = \pdv{v^j}{x^i}\vb{e}_j,
\]
works perfectly fine in cartesian coordinates, as $\vb{e}_j = \text{const}$. In curvilinear setting

\begin{equation}
	\label{eq:dvvectorfield}
	\pdv{\vb{v}}{y^i}=\pdv{\qty(v^j \vb{g}_j)}{y^i}= \pdv{v^j}{y^i}\vb{g}_j + v^j \pdv{\vb{g}_j}{y^i},
\end{equation}
as generally $\pdv{\vb{g}_j}{y^i} \neq \vb{0}$. We can identify the last term, as it must be a vector:
\[
	\pdv{\vb{g}_j}{y^i} = \Gamma^k_{j i}\vb{g}_k,
\]
where $\Gamma^k_{ji}$ are the coefficients of the linear combinations. Thanks to the \textit{commutation of the partial derivatives} \footnote{We are still in flat $\R^d$, i.e. euclidian space. No curvature, torsion, that would obstruct the comutation properties.}, it holds
\begin{equation}
	\label{eq:symmetry_christoffel}
	\Gamma^k_{ji} = \Gamma^k_{ij},
\end{equation}
i.e., $\Gamma^k_{ij}$ is symmetric in $ij$. Well, that did not help \textit{very} much, as we don't know $\Gamma^k_{ji}$, but at least we have the symmetry property. Going back to \ref{eq:dvvectorfield}:

\begin{equation}
	\pdv{\vb{v}}{y^i}=\pdv{v^j \vb{g}_j}{y^i}= \pdv{v^j}{y^i}\vb{g}_j + v^j \pdv{\vb{g}_j}{y^i} = \pdv{v^k}{y^i}\vb{g}_k + v^j \Gamma^k_{ij}\vb{g}_k = \Big( \pdv{v^k}{y^i}+\Gamma^k_{ij}v^j \Big) \vb{g}_k.
\end{equation}
In short $\pdv{\vb{v}}{y^i} = \Big(\pdv{v^k}{y^i}+\Gamma^k_{ij}v^j\Big) \vb{g}_k$. Compare it to $\pdv{\vb{v}}{x^i}=\pdv{v^k}{x^i}\vb{e}_k$. This leads us to the definition

\begin{definition}[Covariant derivative of a vector field]
    The quantity:
\begin{equation}
	v^k|_i = \pdv{v^k}{y^i}+\Gamma^k_{ij}v^j,
\end{equation}
is called the \textbf{covariant derivative of the vector field }$\vb{v}$
\end{definition}

\subsubsection{Dot product}
\label{sec:dot_product}

The number $\vb{v}\vdot \vb{u}$ is obtained in a special manner:
\[
	\vb{v} \vdot \vb{u}= v^i \vb{e}_i \vdot u^j \vb{e}_j = (\vb{e}_i \vdot \vb{e}_j) v^i u^j = \delta_{ij}v^i u^j.
\]
I can of course write the vectors in a different basis:

\[
	\vb{v}\vdot \vb{u} = v^i \vb{g}_i \vdot u^j \vb{g}_j = (\vb{g}_i \vdot \vb{g}_j) v^i u^j = g_{ij}v^i u^j.
\]

\begin{definition}[Metric tensor]
    The tensor $\tensorq{g}$ such that $\forall \vb{v} = v^i \vb{g}_i, \vb{u}= u^j \vb{g}_j$ it holds:
    \[
	    \vb{v} \vdot \vb{u}=g_{ij} v^i u^j, g_{ij}= \vb{g}_i \vdot \vb{g}_j,
    \]
    is called the \textbf{metric tensor}.
\end{definition}


\subsubsection{Dual space}
\label{sec:dual_space}

The (vector) dual space is the space of all linear forms on the underlying vector space. In particular it is a vector basis itselfs, so $\forall \vb{l} \in V^{*}: \vb{l} = l_i \vb{e}^i$, where $\vb{e}^i$ is the i-th basis vector. The action of the forms can be described as
\[
	\vb{l}(\vb{v}) = l_i \vb{e}^i(v^j \vb{e}_j) = l_i v^j \vb{e}^i(\vb{e}_j), \forall \vb{v} \in V.
\]
If it holds $\vb{e^i}(\vb{e}_j) = \delta^i_j$, we call the basis $\vb{e}^i$ dual to $\vb{e}_j$. What about curvilinear setting? We can adopt the same definition
\begin{definition}
    We call the basis $\vb{g}^j$ of $V^{*}$ the dual basis to $\vb{g}_i$ iff

\[
	\vb{g}^j(\vb{g}_i) = \delta^j_i.
\]
\end{definition}

For the original basis we had $\vb{g}_i = \pdv{\vb{\gamma}}{y^i} = \pdv{x^j}{y^i}\vb{e}_j$, in the dual case (using the chain rule):

\[
	\delta^i_j = \pdv{y^i}{y^j}= \pdv{y^i}{x^k}\pdv{x^k}{y^j},
\]
so i can conclude
\[
	\vb{g}^j = \pdv{y^j}{x^k}\vb{e}^k.
\]

Recall that we have the \textit{Riesz representation theorem}:
\[
	\forall \vb{l} \in V^{*}, \exists \quad{unique} \vb{u} \in V: \vb{l}(\vb{v}) = \vb{u} \vdot \vb{u}, \forall \vb{v} \in V.
\]

This implies

\[
	l_i \vb{g}^i (v^j \vb{g}_j) = u^m u^n g_{mn}, \quad{i.e.} l_i v^j \delta^i_j = u^m v^n g_{mn}, \quad{i.e.} l^i v_i = u^m v^i g_{mi}, \quad{i.e.} l_i = g_{im} u^m.
\]

So $l_i = g_{im}u^m$, where $\vb{u}$ represents $\vb{l}$. It is common to write

\[
	l_i = g_{im}l^m.
\]

\subsubsection{Covector fields}
\label{sec:covector_fields}

How to compute $\pdv{\vb{l}}{y^i}$? Just change the location of the index :)
\[
	\pdv{\vb{l}}{y^i} = \pdv{(l_j \vb{g}^j)}{y^i} = \pdv{l^j}{y^i}\vb{g}^j + \pdv{\vb{g}^j}{y^i}l_j.
\]
Again, the last term must be expressable in the dual basis, so 

\begin{equation}
	\pdv{\vb{l}}{y^i}= \pdv{l_j}{y^i}\vb{g}^j+ \tilde{\Gamma}^j_{im}l_j \vb{g}^m,
\end{equation}


where again $\tilde{\Gamma^j_{im}} = \tilde{\Gamma^j_{mi}}$ are the coefficients of the linear combinations, "that are symmetric".

What is the relation between $\Gamma^k_{im}$ and $\tilde{\Gamma^k_{im}}$? Recall $\delta^i_j = \vb{g}^i(\vb{g}_j)$, so differentiating can lead us to

\begin{equation}
	\label{eq:relationGammas}
    \Gamma^j_{im} = - \Gamma^j_{im}.
\end{equation}

\begin{definition}
    Let $\vb{l}$ be a covector field. The quantity
    \begin{equation}
        \vb{l}_m|_j = \pdv{\vb{l}}{y^j}-\Gamma \indices{^l_{jm}}v_l,
    \end{equation}
    is called \textbf{the covariant derivative of the covector field} $\vb{l}$.
\end{definition}

TODO $\tensorq{A} = A_{mn}\vb{g}^m \otimes \vb{g}^n.$

\subsubsection{Direct expression of the Christoffel symbols}
\label{sec:christoffel_expression}
With the above relation, we can express $g_{mn}|_j$. Moreover, we can directly differentiate.
\[
	\pdv{g_{mn}}{y^j}=\pdv{\qty(\vb{g}_m \vdot \vb{g}_n)}{y^j} = \pdv{\vb{g}_m}{y^j} \vdot \vb{g}_n + \vb{g}_m \pdv{\vb{g}_n}{y^j}=\Gamma \indices{^k_{mj}}\vb{g}_k \vdot \vb{g}_n + \vb{g}_m \vdot \Gamma \indices{^k_{nj}}\vb{g}_k = \Gamma \indices{^k_{mj}}g_{kn} + \Gamma \indices{^k_{nj}}g_{mk}
\]
,
\[
	g_{mn}|_j = \pdv{g_{mn}}{y^j}-g_{kn}\Gamma \indices{^k_{jm}}-g_{mk} \Gamma \indices{^k_{jn}}.
\]
From this, it follows
\begin{equation}
    g_{mn}|_j = 0.
\end{equation}

This property is particularly useful, as it allows us to express the Christoffel symbols. Using cyclic permuation, we can write

\begin{align*}
A=	\pdv{g_{mn}}{y^j} &= \Gamma \indices{^k_{mj}}g_{kn} + \Gamma \indices{^k_{nj}}g_{mk}, \\
B=	\pdv{g_{jm}}{y^j} &= \Gamma \indices{^k_{jn}} g_{kn} + \Gamma \indices{^k_{mn}}g_{jk}, \\
C=	\pdv{g_{nj}}{y^n} &= \Gamma \indices{^k_{nm}}g_{kj} + \Gamma \indices{^k_{jm}} g_{nk}.
\end{align*}
Taking $A-B-C$ yields
\[
	\pdv{g_{mn}}{y^j}-\pdv{g_{jm}}{y^n}-\pdv{g_{nj}}{y^m} = -2 \Gamma \indices{^k_{nm}}g_{jk},
\]
multiplying by $g^{jl}$ gives
\[
	-2 \Gamma \indices{^k_{nm}}\delta^l_k = g^{jl}\qty(\pdv{g_{mn}}{y^j}-\pdv{g_{jm}}{y^n}-\pdv{g_{nj}}{y^m}),
\]
from which it follows
\begin{equation}
	\label{eq:christoffel}
	\Gamma \indices{^l_{nj}} = \frac{1}{2}g^{lm}\qty(\pdv{g_{mn}}{y^j}+\pdv{g_{jm}}{y^n}-\pdv{g_{nj}}{y^m}).
\end{equation}


\subsubsection{Interchangability of the derivatives}
\label{sec:interchangability}
In euclidian space:
\[
	\pdv[2]{\vb{v}}{y^j}{y^i} = \pdv[2]{\qty(v^k \vb{e}_k)}{y^j}{y^i}= \qty(\pdv[2]{v^k}{x^j}{x^i}) \vb{e}_k,
\]
when $\vb{e}_k$ are basis vectors of \textit{cartesian coordinate system.} Will it hold even in curvilinear coordinate systems?
\begin{align*}
	0 &= \pdv[2]{\vb{v}}{x^j}{x^k} - \pdv[2]{\vb{v}}{x^j}{x^k} = \, \text{apply the covariant derivative two times}= \, \\&=
   \qty(v^k|_{ij} - v^k|_{ji})\vb{g}_k = \qty(\pdv{\Gamma \indices{^i_{jm}}}{y^k} - \pdv{\Gamma \indices{^i_{km}}}{y^i} + \Gamma \indices{^i_{lk}} \Gamma \indices{^l_{jm}}- \Gamma \indices{^i_{lj}} \Gamma \indices{^l_{km}})v^m \vb{g}_i.
\end{align*}

	\begin{definition}[Riemann curvature tensor]
    The \textit{tensor}
    \begin{equation}
R \indices{^i_{jkm}}=\pdv{\Gamma \indices{^i_{jm}}}{y^k} - \pdv{\Gamma \indices{^i_{km}}}{y^i} + \Gamma \indices{^i_{lk}} \Gamma \indices{^l_{jm}}- \Gamma \indices{^i_{lj}} \Gamma \indices{^l_{km}},
    \end{equation}
    is called the \textbf{Riemann curvature tensor}
\end{definition}
We see that if the Riemann curvature tensor is zero, then effectively, we are in the case of a flat euclidian space, as the derivatives commute (?).
In other words, in flat euclidian space, the Riemann curvature tensor is always zero. If we flip this, we see that if we have a space with zero Riemann curvature tensor, \textit{we have a chance} that the derivatives commute, i.e. that the structure is euclidian.

\begin{tcolorbox}
\begin{example}[Interpretation in continuum mechanics]
	\begin{align*}
		\vb{x} = \vb{x}\qty(\vb{y}) \, &\text{vs} \, \vb{x}=\vb*{\chi}\qty(\vb{X}), \\
		\vb{g}_i = \pdv{\vb{x}}{y^i} \, &\text{vs} \, \vb{g}_i = \pdv{\vb*{\chi}}{X^i}, \, \text{i.e.} \, (\vb{g}_m)^i = F \indices{^i_m} = \pdv{\chi^i}{X^m}, \\
												  &	g_{ij}=\vb{g}_i \vdot \vb{g}_j = (\transpose{\fgrad} \fgrad)_{ij}=(\rcg)_{ij}, \\
		\end{align*}
		So
		\begin{equation}
			\label{eq:cauchy_metric}
			\tensorq{g}=\rcg.
		\end{equation}
		We can calculate:
		\[
			\pdv{\vb{g}_m}{X^j} = \Gamma \indices{^k_{mj}}\vb{g}_k,
		\]
		which is a system of equations for the basis vectors. There are some solvability conditions
		\begin{equation*}
			\partial_{JI}\vb{g}_m = \partial_{IJ}\vb{g}_m = \pdv{\qty(\Gamma \indices{^k_{im}} \vb{g}_k)}{X^j} = \pdv{\qty(\Gamma \indices{^k_{mj}} \vb{g}_k)}{X^i} ,
		\end{equation*}
		this is equivalent to 
		\[
			\partial_{JI}\vb{g}_m - \partial_{IJ} \vb{g}_n =0 \Leftrightarrow \dots R \indices{^i_{jkm}} = 0.
		\]
	\end{example}
	\textit{So this implies that all physically admissable deformations produce a deformed space with zero Riemann curvature.}
\end{tcolorbox}

\subsection{Calculus}
\label{sec:calculus}

\subsubsection{Gradient}
\label{sec:gradient}

Remember $(\grad \varphi)_i = \pdv{\varphi}{x^i}$, so that means
\[
	\grad \varphi = \pdv{\varphi}{x^i} \vb{e}^i.
\]
\textbf{The gradient is a covector.}

\[
	\grad \varphi = \pdv{\varphi}{x^i}\vb{e}^i = \pdv{\varphi}{\xi^j}\underbrace{\pdv{\xi^j}{x^i}\vb{e}^i}_{=\vb{g}^j} = \pdv{\varphi}{\xi^j}\vb{g}^j,
\]

What about the gradient of a vector field? In cartesian coordinate system:
\[
	\grad \vb{v} = \grad(v^i \vb{e}_i) = \pdv{v^i}{x^j}\vb{e}_i \otimes \vb{e}^j = \vb{v} \otimes \grad.
\]
In curvilinear coordinates:
\begin{align*}
	\grad(v^i \vb{g}_i)& = \grad \qty(v^i \pdv{x^m}{\xi^i}\vb{e}_m) = \pdv{x^j}\qty(v^i \pdv{x^m}{\xi^i})\vb{e}_m \otimes \vb{e}^j = \qty(\pdv{v^i}{x^j} \pdv{x^m}{\xi^i}+v^i \pdv[2]{x^m}{x^j}{\xi^i})\vb{e}_m \otimes \vb{e}^j = \\
			   & = \qty(\pdv{v^i}{\xi^n}\pdv{\xi^n}{x^j}\pdv{x^m}{\xi^i}+v^i \pdv[2]{x^m}{x^j}{\xi^i})\vb{e}_m \otimes \vb{e}^j = \pdv{v^i}{\xi^n} \qty(\pdv{x^m}{\xi^u}\vb{e}_m)\otimes \qty(\pdv{\xi^n}{x^j}\vb{e}^j)+v^i \pdv{\xi^l}\qty(\pdv{x^m}{\xi^i})\pdv{\xi^l}{x^j}\vb{e}_m \otimes \vb{e}^j = \\
			   & = \pdv{v^i}{\xi^n} \qty(\pdv{x^m}{\xi^i}\vb{e}_m)\otimes \qty(\pdv{\xi^n}{x^j}\vb{e}^j)+ v^i \pdv{\xi^l}\qty(\pdv{x^m}{\xi^i}\vb{e}_m)\otimes \qty(\pdv{\xi^l}{x^j}\vb{e}^j) = \pdv{v^i}{\xi^n}\vb{g}_i \otimes \vb{g}^n+ v^i \pdv{\vb{g}^i}{\xi^l} \otimes \vb{g}^l = \\
			   &= \pdv{v^i}{\xi^n}\vb{g}_i \otimes \vb{g}^n + v^i \Gamma \indices{^s_{il}}\vb{g}_s \otimes \vb{g}^l = \qty(\pdv{v^s}{\xi^l}+\Gamma \indices{^s_{il}}v^i)\vb{g}_s \otimes \vb{g}^l = \\
			    & = v^s|_l \vb{g}_s \otimes \vb{g}^l.
\end{align*}

Until now, we have not discussed the fact $|\vb{g}_i| \neq 1,$ which is a kind of a problem. Let us define
\[
	\vb{v}=v^i \vb{g}_i = v^i |\vb{g}_i| \frac{\vb{g}_i}{|\vb{g}_i|} = v^{\hat{i}} \vb{g}_{\hat{i}},
\]
where we have defined
\[
	v^{\hat{i}} = |\vb{g}^i| v^i, \vb{g}_{\hat{i}} = \frac{\vb{g}_i}{|\vb{g}_i|}.
\]
But! the differential formulas work for $v^i, \vb{g}_i$, \textbf{not for} $v^{\hat{i}}, \vb{g}_{\hat{i}}$!

\begin{align}
	\grad \varphi &= \pdv{\varphi}{\xi^j}\vb{g}^j = |\vb{g}^i|\pdv{\varphi}{\xi^i} \vb{g}^{\hat{i}},\\
	\grad \vb{v} &= v^s|_l \vb{g}_s \otimes \vb{g}^l = |\vb{g}_s| |\vb{g}^l| v^s|_l \vb{g}_{\hat{s}} \otimes \vb{g}^{\hat{l}}, 
\end{align}
For the divergence of a vector field, we know: $\tr\qty(\vb{u} \otimes \vb{v})$, so
\[
	\divergence{\vb{v}} = \tr\qty(\grad \vb{v}) = \tr\qty(v^s|_l \vb{g}_s \otimes \vb{g}^l) = v^s|_s.
\]
The divergence of a tensor field a can be tricky, but be guided by the summation convention; for the tensor $\tensorq{A} = A^{is}\vb{g}_i \otimes \vb{g}_s$ we can define
\[
	\divergence{\tensorq{A}} = A^{is}|_s \vb{g}_i.
\]
\textit{For the tensors of a different type, we need to change the position of the indices to obtain a bivector.}

\subsubsection{Laplace-Beltrami operator}
\label{sec:laplace_beltrami}

\[
	\laplace \varphi = \frac{1}{\sqrt{\det \tensorq{g}}}\pdv{\xi^i}\qty(\sqrt{\det \tensorq{g}} g^{ij} \pdv{\varphi}{\xi^j}),
\]

on one hand:
\begin{equation*}
	\laplace \varphi = \divergence{\grad \varphi}= \qty(\grad \varphi)^i|_i = \eval{\qty(g^{ij}\pdv{\varphi}{\xi^j})}_i = 
\end{equation*}
where we have rised the index $ \qty(\grad \varphi)^i = g^{ij}\qty(\grad \varphi)_j = g^{ij} \pdv{\varphi}{\xi^j},$ so using the covariant derivative definition
\[
	\divergence{\grad \varphi} = \pdv{\xi^i}\qty(g^{ij}\pdv{\varphi}{\xi^j})+ \Gamma \indices{^i_{il}} g^{lj} \pdv{\varphi}{\xi^j},
\]
on the other

\begin{align*}
\laplace \varphi &= \frac{1}{\sqrt{\det \tensorq{g}}}\qty(\pdv{\xi^i}\qty(\sqrt{\det \tensorq{g}})g^{ij}\pdv{\varphi}{\xi^j}+ \sqrt{\det \tensorq{g}} \pdv{g^{ij}}{\xi^i}\pdv{\varphi}{\xi^j}+ \sqrt{\det \tensorq{g}} g^{ij} \pdv[2]{\varphi}{\xi^i}{\xi^j}) = \\
			 &= \frac{1}{\sqrt{\det \tensorq{g}}} \qty(\frac{1}{2\sqrt{\det \tensorq{g}}}\pdv{\xi^i} \qty(\det \tensorq{g})g^{ij}\pdv{\varphi}{\xi^j}+\sqrt{\det \tensorq{g}}\pdv{g^{ij}}{\xi^i}+ \sqrt{\det \tensorq{g}} g^{ij}\pdv[2]{\varphi}{\xi^i}{\xi^j}) = \\
			 & = \frac{1}{\sqrt{\det \tensorq{g}}}\qty(\frac{1}{2}\tr\qty(\inverse{\tensorq{g}} \pdv{\tensorq{g}}{\xi^i})g^{ij}\pdv{\varphi}{\xi^j}-\sqrt{\det \tensorq{g}} \qty(\Gamma \indices{^j_{kn}}g^{in}- \Gamma \indices{^i_{km}}g^{mj})+ \sqrt{\det \tensorq{g}}g^{ij}\pdv[2]{\varphi}{\xi^i}{\xi^j}) = \\
			 & = \frac{1}{\sqrt{\det \tensorq{g}}}\qty(\frac{1}{2}\qty(g^{mn} \pdv{g_{mn}}{\xi^i})g^{ij}\pdv{\varphi}{\xi^j}-)
\end{align*}

\subsubsection{Bipolar coordinates}
\label{sec:bipolar_coordinates}

Define $\vb*{\xi} = [\alpha,\beta],$ where
\[
	\alpha+i \beta = \log \frac{y+i(x+a)}{y+i(x-a)}.
\]
This can be inversed and write
\begin{align*}
	x&=\frac{a \sinh \alpha}{\cosh \alpha - \cos \beta},\\
	y&= \frac{a \sin \beta}{\cosh \alpha -\cos \beta},
\end{align*}
moreover,

\begin{align*}
	(x-a \coth \alpha)^{2}+y^{2} &= \frac{a^{2}}{\sinh^{2}\alpha}, \\
	x^{2}+(y-a \cot \beta)^{2} &= \frac{a^2}{\sin^{2}\beta}.
\end{align*}

Calculate \textit{everything} for this coordinate system.

In general $\vb{g}_i=\pdv{x^j}{\xi^i}\vb{e}_j,$ so in our case
\begin{align*}
	\vb{g}_{\alpha}&=\pdv{x}{\alpha}\vb{e}_x+\pdv{y}{\alpha}\vb{e}_y\\
		       &=\qty(\frac{(a \cosh \alpha)(\cosh \alpha-\cos \beta)-a \sinh \alpha\sinh \alpha}{\qty(\cosh \alpha - \cos \beta)^{2}})\vb{e}_x+\qty(\frac{a \cos \beta(\cosh \alpha-\cos \beta)-a \sin \beta \sinh \alpha}{\qty(\cosh \alpha - \cos \beta)^{2}})\vb{e}_y = \\
		       & = \frac{a}{\qty(\cosh \alpha - \cos \beta)^{2}}\qty((1-\cosh \alpha \cos \beta)\vb{e}_x-(\sin \beta \sinh \alpha) \vb{e}_y),\\
	\vb{g}_{\beta}& = \pdv{x}{\beta}\vb{e}_x+\pdv{y}{\beta}\vb{e}_y\\
		      &= \dots =\\
		      &= \frac{a}{\qty(\cosh \alpha-\cos \beta)^{2}}\qty(-\qty(\sin \beta \sinh \alpha)\vb{e}_x + \qty(-1+\cosh \alpha \cos \beta)\vb{e}_y).
\end{align*}
We can see that $\vb{g}_{\alpha} \vdot \vb{g}_{\beta}=0$ and so
\[
	\tensorq{g}=\qty(\frac{a}{\cosh \alpha - \cos \beta})^{2}\identity, \inverse{\tensorq{g}} = \qty(\frac{\cosh \alpha-\cos \beta}{a})^{2} \identity.
\]

Coming back to Laplace-Beltrami operator, we can calculate
\[
	\qty(\sqrt{\det \tensorq{g}}\inverse{\tensorq{g}})=\qty(\frac{a}{\cosh \alpha-\cos \beta})^{2} \qty(\frac{\cosh \alpha-\cos \beta}{\alpha})^{2} \identity = \dots = \identity,
\]
and calculating a bit more yields
\[
	\laplace \varphi \to \qty(\frac{\cosh \alpha-\cos \beta}{a})^{2} \laplace_{\alpha \beta}\varphi.
\]
\begin{remark}[Relation to complex analysis]
    This can be seen as a conformal transformation
    \[
	    \gamma = f(z),
    \]
    where
    \begin{align*}
	    \gamma &= \alpha + i \beta, \\
	    z &= x+iy,
    \end{align*}

Let us write
\[
	f(z) = f^x(x,y) +i f^y(x,y) \Leftrightarrow \vb{f}(\vb{x})=[f^x\qty(\vb{x}),f^y\qty(\vb{x})], z = x+iy,
\]
and compute
\[
	\pdv{\vb{f}}{\vb{x}} = \begin{bmatrix}
		\pdv{f^x}{x} & \pdv{f^x}{y}\\
		\pdv{f^y}{x} & \pdv{f^y}{y}
	\end{bmatrix}.
\]
Recall Cauchy-Riemmann conditions:
\begin{align*}
	\pdv{f^x}{x}&=\pdv{f^y}{y}, \\
	\pdv{f^x}{y}&=-\pdv{f^y}{x},
\end{align*}
using which the gradient becomes:
\[
	\pdv{\vb{f}}{\vb{x}} = \begin{bmatrix}
		\pdv{f^x}{x} & \pdv{f^x}{y}\\
		-\pdv{f^x}{y} & \pdv{f^y}{y}
	\end{bmatrix},
\]
which is an \textbf{orthogonal matrix:}
\[
	\qty(\pdv{\vb{f}}{\vb{x}}) \transpose{\qty(\pdv{\vb{f}}{\vb{x}})} = \qty(\qty(\pdv{f^x}{x})^{2}+\qty(\pdv{f^y}{y})^{2})\identity.
\]
Realize that all this structure comes just from the fact that the transformation is given through a holomorfic function.
\end{remark}

\subsubsection{Compatibility conditions in linearised elasticity}
\label{sec:compt_cond}

\[
	R \indices{^i_{jkm}}= \pdv{\Gamma \indices{^i_{jm}}}{\xi^k}-\pdv{\Gamma \indices{^i_{km}}}{\xi^j}+\Gamma \indices{^i_{lk}}\Gamma \indices{^l_{jm}}-\Gamma \indices{^i_{lj}}\Gamma \indices{^l_{km}},
\]
and we know
\[
	R \indices{^i_{jkm}} = 0 \Leftrightarrow \rcg = \transpose{\fgrad} \fgrad, \fgrad = \pdv{\vb*{\chi}}{\vb{X}}.
\]
All this works in fully \textit{nonlinear setting!}. In the classical lecture, we have been able to obtain compatibility condition in \textit{linearised elasticity}: $\curl{\tensorq{\varepsilon}} = \tensorq{0}, \tensorq{e}=\frac{1}{2}\qty(\grad \vb{u}+\transpose{\qty(\grad \vb{u})}).$

Consider the following setting:
\begin{align*}
	\vb{x}&=\vb*{\chi}\qty(\vb{X}),\\
	\vb{u}&= \vb{\chi}\qty(\vb{X})-\vb{X},\\
	\grad \vb{u} &= \fgrad - \identity, \\
	\fgrad &= \identity + \grad \vb{u},\\
\end{align*}
then
\[
	\rcg = \transpose{\fgrad}\fgrad=\qty(\identity+\transpose{\qty(\grad \vb{u})})\qty(\identity+\grad \vb{u}) = \identity + 2 \tensorq{\varepsilon} + \, \text{h.o.t.} \,,
\]
and so
\[
	\inverse{\tensorq{g}} = \identity - 2 \tensorq{\varepsilon}.
\]

The Christoffel symbols are

\begin{align*}
	\Gamma \indices{^l_{nj}}&= \frac{1}{2}g^{lm}\qty(\pdv{g_{mn}}{X^j}+\pdv{g_{jm}}{X^n}-\pdv{g_{nj}}{X^m}) \\
				&\approx \frac{1}{2}\qty(\identity-2 \tensorq{\varepsilon})^{lm}\qty(\pdv{X^j}\qty(\identity+2 \tensorq{\varepsilon})_{mn}+\pdv{X^n}\qty(\identity+2 \varepsilon)_{jm}-\pdv{X^m}\qty(\identity+2 \varepsilon)_{nj}), \\
				&\approx \delta^{lm}\qty(\pdv{\varepsilon_{mn}}{X^j}+\pdv{\varepsilon_{jm}}{X^n}-\pdv{\varepsilon_{nj}}{X^m}) = \pdv{\varepsilon_{n}^l}{X^j}+\pdv{\varepsilon_{j}^l}{X^n}-\pdv{\varepsilon_{nj}}{X^m},
\end{align*}

the Riemann curvature tensor is (linear approximation)

\begin{align*}
0 = R \indices{^i_{jkm}}& \approx \pdv{\Gamma \indices{^i_{jm}}}{X^k}-\pdv{\Gamma \indices{^i_{km}}}{X^j}\\
			& = \pdv{X^k}\qty(\pdv{\varepsilon^i_j}{X^m}+\pdv{\varepsilon^i_m}{X^j}-\pdv{\varepsilon_{mj}}{X^i})-\pdv{X^j}\qty(\pdv{\varepsilon^i_k}{X^m}+\pdv{\varepsilon^i_m}{X^k}-\pdv{\varepsilon_{km}}{X^i}) =\\
			 =& \pdv[2]{\varepsilon_{ij}}{X^k}{X^m}-\pdv[2]{\varepsilon_{mj}}{X^k}{X^i}-\pdv[2]{\varepsilon_{ik}}{X^j}{X^m}+\pdv[2]{\varepsilon_{km}}{X^j}{X^i},
\end{align*}
so the compatibility conditions are
\[
	\pdv[2]{\varepsilon_{ij}}{X^k}{X^m}-\pdv[2]{\varepsilon_{mj}}{X^k}{X^i}-\pdv[2]{\varepsilon_{ik}}{X^j}{X^m}+\pdv[2]{\varepsilon_{km}}{X^j}{X^i} = 0.
\]

\subsection{Surface geometry}
\label{sec:surface_geometry}

In this part, we will work with surfaces embedded in $\R^3$.

Let $G = \{\vb{u}\} \subset \R^2 $ be the parametrization space and $\vb*{\Phi}:G \subset \R^2 \to \R^3$ is the parametrization, so the points of the surface are
\[
	\vb{x}=\vb*{\Phi}\qty(\vb{u}), \vb{x} \in \R^3.
\]

\begin{definition}
	The indices $i,j,k, \dots \in \{1,2,3\}$ will denote objects from $\R^3$ and indices $\alpha, \beta, \gamma, \dots \in \{1,2\}$ will denote indices of objects from $\R^2$.
\end{definition}

\subsubsection{Tangent and normal vectors}
\label{sec:tangent_normal_vectors}
As in the previous story, we can define (basis) tangent vectors:
\[
	\vb{t}_1 = \pdv{\vb*{\Phi}}{u^1}, \vb{t}_2 = \pdv{\vb*{\Phi}}{u^2},
\]
and on surfaces, of importance is also the normal vector
\[
	\vb{n}=\frac{\vb{t}_1 \cross \vb{t}_2}{|\vb{t}_1 \cross \vb{t}_2|}.
\]

\subsubsection{Distances and angles}
\label{sec:distances_angles}

The metric tensor on the surface is given by
\[
	\tensorq{g}_s = \begin{bmatrix} \vb{t}_1 \vdot \vb{t}_1 & \vb{t}_1 \vdot \vb{t}_2 \\ \vb{t}_1 \vdot \vb{t}_2 & \vb{t}_2 \vdot \vb{t}_2 \end{bmatrix},
\]
or in context of diff. geo. it is called \textbf{the first fundamental form}.

\subsubsection{Derivatives}
\label{sec:derivatives_on_surfaces}
In $\R^3$, we know how to differentiate tangent vectors (using Christoffel symbols).
The metric tensor in $\R^3$ is given by 

\[
	\tensorq{g} = 
	\begin{bmatrix}
		\tensorq{g}_s & 0 \\
		0 & 1 
	\end{bmatrix}
\]
Realize that, veiwed in $\R^3$ the relation can be written as
\[
	\pdv{\vb{t}_{\alpha}}{u^{\beta}} = \Gamma \indices{^\gamma_{\alpha \beta}} \vb{t}_{\gamma}+b_{\alpha \beta} \vb{n},
\]
because viewed from $\R^3$, the basis vectors are $\vb{t}_1, \vb{t}_2, \vb{n}$ and we have just denoted $b_{\alpha \beta} = \Gamma \indices{^3 _{\alpha \beta}}.$

What about the derivative of the normal vector? From the length of $\vb{n}$ we know
\[
	\vb{n}\vdot \vb{n} = 1 \Rightarrow \pdv{\vb{n}}{u^{\alpha}}\vdot \vb{n} = 0,
\]
so that means the derivative is perpendicular to the normal direction, so
\[
	\pdv{\vb{n}}{u^\alpha}= A \indices{^\gamma_{\alpha}} \vb{t}_{\gamma}.
\]

Next trick is to realize
\[
	0 = \pdv{u^{\alpha}}\qty(\vb{n}\vdot \vb{t}_{\beta}) = \pdv{\vb{n}}{u^{\alpha}} \vdot \vb{t}_{\beta}+ \vb{n}\vdot \pdv{\vb{t}_{\beta}}{u^{\alpha}} = A \indices{^{\gamma}_{\alpha}}\vb{t}_{\gamma} \vdot \vb{t}_{\beta}+ \vb{n}\vdot \qty(\Gamma \indices{^{\delta}_{\alpha \beta}} \vb{t}_{\delta}+b_{\alpha \beta} \vb{n}) = A \indices{^\gamma _{\alpha}} g_{s, \gamma \beta}+ b_{\alpha \beta},
\]
from which it follows

\[
	A \indices{^{\gamma}_{\alpha}} = -g \indices{^{\gamma \beta}}b_{\beta \alpha}.
\]

\subsubsection{Commutation of derivatives}
\label{sec:com_der}

What are the \textit{implications} of
\[
	\pdv[2]{\vb{t}_{\alpha}}{u^{\beta}}{u^{\gamma}} = \pdv[2]{\vb{t}_{\alpha}}{u^{\beta}}{u^{\alpha}}?
	\]
Write
\[
	0 = \pdv[2]{\vb{t}_{\alpha}}{u^{\beta}}{u^{\gamma}} - \pdv[2]{\vb{t}_{\alpha}}{u^{\beta}}{u^{\alpha}} = (\, \text{something} \,) \vb{t}_{\delta}+(\, \text{something different} \,) \vb{n},
\]
so we see the whole thing splits into two parts. It can be shown
\begin{theorem}[Gauss relation]
	\[
		R \indices{_{\psi \beta \delta \alpha}} = b_{\alpha \beta}b_{\psi \delta} - b_{\alpha \delta} b_{\psi \beta}.
	\]
\end{theorem}

\begin{theorem}[Codazzi-Mainardi relation]
	\[
		b_{\alpha \beta}|_{\delta}- b_{\alpha \delta}|_{\beta}=0
	\]
\end{theorem}

\subsubsection{Surfaces evolving in time}
\label{sec:time_evolving_surfaces}

Now the points of the surface are given by
\[
	\vb{x}=\vb*{\Phi}\qty(t,\vb{u}), \, \text{where} \,\vb*{\Phi}: \R \times G \to \R^3.
\]

We can define the \textbf{velocity of the surface}:
\begin{equation*}
	\vb{v}_s = \pdv{\vb*{\Phi}}{t}\qty(t,\vb{u}).
\end{equation*}

The basis of everything has always been Gauss theorem; we will be interested in the quantity of the type

\[
	\dv{t} \int_{S(t)}\psi\qty(t,\vb{x})\dd{S},
\]
where $S(t)$ is a time-dependent surface. Let us try the approach from Reynolds:
\[
	\dv{t} \int_{S(t)}\psi\qty(t,\vb{x})\dd{S} = \dv{t} \int_{\inverse{\vb*{\Phi}\qty(t,\vb{x}(t))}}\psi(t,\vb*{\Phi}(t,\vb{u})) \sqrt{\det \tensorq{g}_s}\dd{u^1} \dd{u^2} = ,
\]
and now we need to calculate the derivatives. Start slow:

\begin{align*}
	\dv{\vb{t}_{\alpha}}{t} &= \pdv{t}\qty(\pdv{\vb{\Phi}}{u^\alpha}\qty(t,\vb{u})) = \pdv{u^{\alpha}}\underbrace{\qty(\pdv{\vb{\Phi}}{t}\qty(t,\vb{u}))}_{= \vb{v}_s(t,\vb{u})} = \pdv{u^{\alpha}}\qty(\vb{v}_{\parallel}+v_{\perp}\vb{n}) = \\
				&= \pdv{\vb{v}_{\parallel}}{u^{\alpha}}+\pdv{\qty(v_{\perp} \vb{n})}{u^{\alpha}} = \pdv{\qty(v^{\beta}_{\parallel}\vb{t}_{\beta})}{u^{\alpha}}+\pdv{v_{\perp}}{u^{\alpha}}\vb{n}+ v_{\perp} \pdv{\vb{n}}{u^{\alpha}} = \\
				&=\pdv{v_{\perp}^{\beta}}{u^{\alpha}}\vb{t}_{\beta}+\pdv{\vb{t}_{\beta}}{u^{\alpha}}v^{\beta}_{\parallel}+\pdv{v_{per}}{u^{\alpha}}\vb{n} - v_{\perp} g^{\gamma \beta}b_{\beta \alpha}\vb{t}_{\gamma} = \\
				& = \pdv{v^{\beta}_{\parallel}}{u^{\alpha}}\vb{t}_{\beta}+v^{\beta}_{\parallel}\Gamma \indices{^{\gamma}_{\alpha \beta}} \vb{t}_{\gamma}+ v_{\parallel}^{\beta}b_{\alpha \beta} \vb{n} + \pdv{v_{\perp}}{u^{\alpha}}\vb{n}- v_{\perp}g^{\gamma \beta}b_{\alpha \beta}\vb{t}_{\gamma} =\\
				& = v^{\beta}_{\parallel}|_{\alpha} \vb{t}_{\beta}- v_{\perp}g^{\gamma \beta} b_{\alpha \beta}\vb{t}_{\gamma}+\qty(v^{\beta}_{\parallel}b_{\alpha \beta}+ \pdv{v_{\perp}}{u^{\alpha}})\vb{n} = \\
				& = \qty(v_{\parallel}^{\beta}|_{\alpha}-v_{\perp}g^{\beta \gamma} b_{\alpha \gamma})\vb{t}_{\beta}+ \qty(v_{\parallel}^{\beta}b_{\alpha \beta} + \pdv{v_{\perp}}{u^{\alpha}})\vb{n}.
\end{align*}
So all in all
\[
	\dv{\vb{t}_{\alpha}}{t}= \qty(v_{\parallel}^{\beta}|_{\alpha}-v_{\perp}g^{\beta \gamma} b_{\alpha \gamma})\vb{t}_{\beta}+ \qty(v_{\parallel}^{\beta}b_{\alpha \beta} + \pdv{v_{\perp}}{u^{\alpha}})\vb{n}.
\]

Next ingredient is the quantity $\dv{t} \tensorq{g}_s$, so in components:
\[
	\dv{g_{\alpha \beta}}{t} = \dv{t}\qty(\vb{t}_{\alpha} \vdot \vb{g}_{\beta}) = \dots = v_{\parallel}^\delta|_{\alpha}g_{\delta \beta}+ v^{\delta}_{\parallel}|_{\beta}g_{\delta \alpha}-2v_{\perp}b_{\alpha \beta}.
\]
After some further manipulation, the final formula becomes

\begin{equation}
	\dv{t} \int_{S(t)}\psi(t,\vb{x})\dd{S} = \int_{S(t)}\dv{\psi}{t}\qty(t,\vb{x})+\psi(t,\vb{x})\qty(\divergence{\vb{v}_{\parallel S}} - 2 v_{\perp}\qty(t,\vb{x})K\qty(t,\vb{x}))\dd{S},
\end{equation}
where
\[
	\divergence{\vb{v}_{\parallel S}} - 2 v_{\perp}K \coloneq \eval{v^{\beta}(t,\vb{u})_{\parallel}|_{\beta}-2v_{\perp}(t,\vb{u})K(t,\vb{u})}_{\vb{u}=\inverse{\vb*{\Phi}\qty(t,\vb{x})}},
	\]

	\[
		K = \frac{1}{2}g^{\beta \alpha} b_{\alpha \beta}
	\]
	is the mean curvature.
\bibliographystyle{chicago} %\bibliography

\section{Linearised elasticity}
\label{sec:linearised_elasticity}

The static setting of the linearised elasticity theory is

\begin{align}
\label{eq:static_momentum_balance}
\divergence{\tensorq{\tau}} + \vb{f} = \vb{0},
\end{align}
and for now we will want to solve for the stress, that is

\[
	\tensorq{\tau} = \begin{bmatrix}
		\tau_{xx} & \tau_{xy} & \tau_{xz} \\
			  & \tau_{yy} & \tau_{yz} \\
			  & & \tau_{zz}
	\end{bmatrix},
\]
since $\tensorq{\tau}$ is symmetric. Recall the compatibility conditions

\begin{align}
  \label{eq:compatibility}
  \tensorq{\tau} &= \lambda \qty(\tr \tensorq{\varepsilon})\identity + 2 \mu \tensorq{\varepsilon}, \\
  \tensorq{\varepsilon} &= \frac{1}{2}\qty(\grad \vb{u}+ \transpose{\qty(\grad \vb{u})}),\\
  \curl{\qty(\transpose{\qty(\curl{\tensorq{ \varepsilon}})})} &= \tensorq{0}, \\
  \laplace \tensorq{\tau}+\frac{1}{1+\nu} \grad \grad \tr \tensorq{\tau} &= -\qty(\grad \vb{f} \transpose{\qty(\grad \vb{f})}) - \frac{\nu}{1-\nu} \qty(\divergence{\vb{f}})\identity
\end{align}

\subsection{Plane stress/strain problems}
\label{sec:plane_problems}

In each of the cases, the stress/strain tensors have a \textit{special structure}:
\begin{equation}
	\label{eq:plane_structure}
	\tensorq{\tau}\qty(x,y)=\begin{bmatrix}
		\tau_{xx}\qty(x,y) & \tau_{xy}\qty(x,y) & 0\\
		\tau_{xy}\qty(x,y) & \tau_{yy}\qty(x,y) & 0\\
		0 & 0 & 0
	\end{bmatrix},
\end{equation}
and the same for the strain tensor. Inverting the stress-strain relation yields

\[
	\tensorq{\varepsilon} = \frac{1}{2 \mu}\qty(\tensorq{\tau}-\frac{\lambda}{3 \lambda+2 \mu}\qty(\tr \tensorq{\tau})\identity),
\]
but since $I_{zz} = 1$, in general we have
\[
	\tensorq{\varepsilon}\qty(x,y)=\begin{bmatrix}
		\varepsilon_{xx}\qty(x,y) & \varepsilon_{xy}\qty(x,y) & 0 \\
		\varepsilon_{xy}\qty(x,y) & \varepsilon_{yy}\qty(x,y) & 0 \\
		0 & 0 & \varepsilon_{zz}(x,y)
	\end{bmatrix},
\]
for $\varepsilon_{zz}(x,y) \neq 0$.

\begin{remark}[Notation]
Note that in the following, operators acting on tensors will always respect the dimensionality of the tensor (so i will write $\tr \tensorq{\tau}_{2D}$ instead of $\tr_{2D} \tensorq{\tau}_{2D}.$ And the same for the laplacian, divergence and so on
\end{remark}

\subsection{Plane stress problem}
\label{sec:plane_stress_problem}

The stress is given as
\[
	\tensorq{\tau} = 2 \mu \tensorq{\varepsilon} + \lambda \qty(\tr \tensorq{\varepsilon})\identity,
\]
where $\tensorq{\tau}$ has the structure \ref{eq:plane_structure}. It must hold
\[
	0 = \tau_{zz} = 2 \mu \varepsilon_{zz}+\lambda\qty(\varepsilon_{xx}+\varepsilon_{yy}+\varepsilon_{zz}),
\]
so 
\[
	0 = \lambda\qty(\varepsilon_{xx}+ \varepsilon_{yy})+\qty(2 \mu + \lambda)\varepsilon_{zz},
\]
and that yields a condition on $\varepsilon_{zz}$:
\[
	\varepsilon_{zz} = -\frac{\lambda}{2 \mu+ \lambda} \tr_{2D}\tensorq{\varepsilon}_{2D}.
\]
The constitutive relation can than be rewritten as
\[
	\tensorq{\tau}_{2D} = 2 \mu \tensorq{\varepsilon}_{2D}+ \lambda \qty(\tr \tensorq{\varepsilon}_{2D}+\varepsilon_{zz})\identity_{2D} = 2 \mu\qty(\tensorq{\varepsilon}_{2D}+\frac{\lambda}{2 \mu+\lambda}\qty(\tr \tensorq{\varepsilon}_{2D})\identity_{2D}).
\]
The "2D Beltrami-Michel equations" can be derived from:
\[
	\laplace \tensorq{\tau}+\frac{1}{1+\nu} \grad \grad \tr \tensorq{\tau} = -\qty(\grad \vb{f} +\transpose{\qty(\grad \vb{f})}) - \frac{\nu}{1-\nu} \qty(\divergence{\vb{f}})\identity,
\]
but there is a problem: the $zz$ equation yields: 
\[
	0 + \frac{1}{1+\nu}\pdv[2]{z}\qty(\tr \tensorq{\tau}_{2D} = 0 - \frac{\nu}{1- \nu}\qty(\divergence{\vb{f}_{2D}})),
\]
but in our case $\tensorq{\tau}_{2D}$ is not a function of $z$, so of course we would have
\[
	\divergence{\vb{f}_{2D}} = 0,
\]
\textit{which is not generally true!} The forces are given to us. Try something different: take the trace of the Beltrami-Michell equation and obtain (after some calculation)

\[
	\laplace \tr \tensorq{\tau} = - \frac{1+\nu}{1-\nu} \divergence{\vb{f}},
\]
so rewritting in "2D" view:

\[
	\qty(\laplace_{2D}+\pdv[2]{z})\tr \tensorq{\tau}_{2D} = -\frac{1+\nu}{1-\nu}\divergence{\vb{f}_{2D}}.
\]
That maybe did not help much, because the $z$ derivative is still zero, but here comes the time for some handwaving: what about we use the above equation to replace the troublemaking term? We would obtain

\[
	\laplace_{2D} \tr \tensorq{\tau}_{2D} - \nu \frac{1+ \nu}{1- \nu} \divergence{\vb{f}_{2D}} = - \frac{1+\nu}{1- \nu}\qty(\divergence{ \vb{f}_{2D}}),
\]
so after some manipulation

\[
	\laplace \tr \tensorq{\tau}_{2D} = -\qty(1+\nu)\divergence{\vb{f}_{2D}}.
\]

In total, the problem is described as

\begin{align}
  \label{eq:plane_stress_formulation}
  \vb{0}_{2D} &= \divergence{\tensorq{\tau}_{2D}} + \vb{f}_{2D},\\
  \tensorq{\tau}_{2D} &= 2 \mu\qty(\tensorq{\varepsilon}_{2D}+\frac{\lambda}{2 \mu+\lambda}\qty(\tr \tensorq{\varepsilon}_{2D})\identity_{2D}) \\
  \laplace \tr \tensorq{\tau}_{2D} &= -\qty(1+\nu)\divergence{\vb{f}_{2D}}.
\end{align}

\subsection{Plain strain problem}
\label{sec:plain_strain_problem}
This time, the structure of the stress and strain are:
\[
	\tensorq{\varepsilon}\qty(x,y)=\begin{bmatrix}
		\varepsilon_{xx}\qty(x,y) & \varepsilon_{xy}\qty(x,y) & 0 \\
		\varepsilon_{xy}\qty(x,y) & \varepsilon_{yy}\qty(x,y) & 0 \\
		0 & 0 & 0
	\end{bmatrix}, 	
	\tensorq{\tau}\qty(x,y)=\begin{bmatrix}
		\tau_{xx}\qty(x,y) & \tau_{xy}\qty(x,y) & 0 \\
		\tau_{xy}\qty(x,y) & \tau_{yy}\qty(x,y) & 0 \\
		0 & 0 & \tau_{zz}\qty(x,y)
	\end{bmatrix}.
\]
Using a similiar approach, we can calculate, using $\tensorq{\varepsilon} = \frac{1}{2 \mu}\qty(\tensorq{\tau}-\frac{\lambda}{3 \lambda + 2 \mu}\qty(\tr \tensorq{\tau})\identity)$,
\[
	\tau_{zz} = \lambda \tr \tensorq{\varepsilon}_{2D},
\]
so the constituive relation is
\[
	\tensorq{\varepsilon}_{2D} = \frac{1}{2 \mu}\qty(\tensorq{\tau}_{2D}- \frac{\lambda}{3 \lambda+ 2 \mu}\qty(\tr \tensorq{\tau}_{2D}+\tau_{zz})\identity_{2D}) = \frac{1}{2 \mu}\qty(\tensorq{\tau}_{2D}-\frac{\lambda}{3 \lambda+ 2 \mu}\qty(\tr \tensorq{\tau}_{2D})\identity_{2D}-\frac{\lambda}{3 \lambda+ 2 \mu}\lambda \qty(\tr \tensorq{\varepsilon}_{2D})\identity_{2D}),
\]
so taking the trace we can obtain: $\tau_{zz} = \frac{\lambda}{\lambda\qty(\lambda+\mu)} \tr \tensorq{\tau}_{2D}$ and plugging it into the original equation yields

\[
	\tensorq{\varepsilon}_{2D} = \frac{1}{2 \mu}\qty(\tensorq{\tau}_{2D}-\frac{\lambda}{2\qty(\lambda+\mu)}\qty(\tr \tensorq{\tau}_{2D})\identity_{2D}).
\]
As for the Beltami-Michell equations, taking the trace gives us again
\[
	\laplace \tr \tensorq{\tau} = - \frac{1+\nu}{1-\nu}\divergence{\vb{f}},
\]
and in \textit{plain strain}, we are able to simply do

\[
	\laplace \tr \tensorq{\tau}_{2D} = - \frac{1}{1-\nu}\divergence{\vb{f}_{2D}},
\]
without any magic. In total, the equations we are solving are

\begin{align}
  \label{eq:plain_strain_formulation}
  \vb{0}_{2D} &= \divergence{\tensorq{\tau}_{2D}}+ \vb{f}, \\
  \tensorq{\varepsilon}_{2D} &= \frac{1}{2 \mu}\qty(\tensorq{\tau}_{2D}-\frac{\lambda}{2\qty(\lambda+\mu)}\qty(\tr \tensorq{\tau}_{2D})\identity_{2D}),\\
  \laplace \tr \tensorq{\tau}_{2D} &= - \frac{1}{1-\nu}\divergence{\vb{f}_{2D}}.
\end{align}

\subsection{Airy stress function}
\label{sec:airy_stress_function}

Let us assume that the force is given as
\[
	\vb{f}_{2D} = - \grad \varphi,
\]
i.e. the force is conservative. Moreover, let us use the following ansatz:
\[
	\tensorq{\tau}_{2D}= \begin{bmatrix}
		\pdv[2]{\Phi}{y} + \varphi & \pdv[2]{\Phi}{x}{y} \\
		\pdv[2]{\Phi}{x}{y} & \pdv[2]{\Phi}{x} + \varphi
	\end{bmatrix},
\]
for some function $\Phi\qty(x,y)$ called the \textit{Airy stress function}. Why that would be useful? Calculate the divergence of the stress:
\[
	\divergence{\tensorq{\tau}_{2D}} = \begin{bmatrix}
		\pdv{x}\qty(\pdv[2]{\Phi}{y}+\varphi) - \pdv{y}\qty(\pdv[2]{\Phi}{x}{y}) \\
		-\pdv{x}\qty(\pdv[2]{\Phi}{x}{y}) + \pdv{y}\qty(\pdv[2]{\Phi}{x}+\varphi)
	\end{bmatrix}
	= \grad \varphi,
\]
so identically we have
\[
	\vb{0}_{2D} = \divergence{\tensorq{\tau}_{2D}} - \grad \varphi,
\]
and one of our equations is solved. What about the remaining ones? Beltrami-Michell:
\[
	\laplace \tr \tensorq{\tau}_{2D} = \laplace\qty(\laplace \Phi+2 \varphi) = \laplace \laplace \Phi + \laplace \varphi.
\]

Using this in plain strain case:

\[
	\laplace \laplace \Phi + \frac{1-2 \nu}{1-\nu}\laplace \varphi = 0,
\]
and in plain stress case:

\[
	\laplace \laplace \Phi + \qty(1- \nu)\laplace \varphi = 0.
\]

Let us take a glimpse at the biharmonic equation.
\subsection{Bending of a narrow rectangular beam by uniform load}
\label{sec:beam_uniform}
Assume we have a narrow rectangular beam of length $L$, height $h$ and width $b$, subjected to the load $q \vb{e}_y$, which is constant in the x-direction. $[q] = \frac{\, \text{N} \,}{\, \text{m} \,}$.

\begin{tikzpicture}
  % long beam, ex to the right, ey downwards, ez through the monitor, center is in the middle.
\end{tikzpicture}

Boundary conditions are \textit{essential}: they specify the problem. In our case, the \textbf{front/back face} is traction free:
\[
	\pm\tensorq{\tau} \vb{e}_z = \vb{0}, \, \text{on} \, \{z = \pm \frac{b}{2} \},
\]
the \textbf{bottom face} is also stress free:
\[
	\tensorq{\tau}\vb{e}_{y} = \vb{0}, \, \text{on} \, \{y=\frac{h}{2}\},
\]
the \textbf{top face} is subjected to the load
\[
	\tensorq{\tau}\vb{e}_y = \frac{-q}{b}\vb{e}_y, \, \text{on} \, \{y=-\frac{h}{2}\}.
\]

On the lateral faces, we would like \textit{something like}
\[
	\pm\tensorq{\tau} \vb{e}_x = \vb{f}\qty(y,z), \, \text{on} \, \{x=\pm \frac{L}{2}\},
\]
however, in our analysis, we are only interested in the fact whether the force can support the beam - but we dont care about the exact distribution of it. Thus, we require the \textit{balance of forces:}
\begin{equation}
\label{eq:bal_forces}
    \int_{-\frac{h}{2}}^{\frac{h}{2}}\int_{-\frac{b}{2}}^{\frac{b}{2}}\vb{f}\qty(y,z)\dd{y}\dd{z} = \frac{qL}{2}\vb{e}_y,
\end{equation}
and moreover we require the \textit{balance of torques:}

\begin{equation}
    \label{eq:bal_torques}
    \int_{-\frac{h}{2}}^{\frac{h}{2}}\int_{-\frac{b}{2}}^{\frac{b}{2}}\vb{r}\cross\vb{f}\qty(y,z)\dd{y}\dd{z} = \vb{0},
\end{equation}

So the roadplan is to find the stress $\tensorq{\tau}$ and check whether \ref{eq:bal_forces} and \ref{eq:bal_torques} are satisfied.

From the symmetry of the load, we assume that
\[
	\tau^{zz} = 0,
\]
so our problem is essentialy a \textit{plane stress problem}. Let us sum up our analysis (this takes some work)

\begin{align*}
	t^{xy}\qty(x,y=\frac{h}{2}) &= 0\\
	t^{yy}\qty(x,y=\frac{h}{2}) &= 0\\
	t^{xy}\qty(x,y=-\frac{h}{2}) &= 0\\
	t^{yy}\qty(x,y=-\frac{h}{2}) &= - \frac{q}{b}\\
	b\int_{-\frac{h}{2}}^{\frac{h}{2}}\tau^{xy}\qty(x=\pm \frac{L}{2},y)\dd{y} &= \mp \frac{qL}{2},\\
	b\int_{-\frac{h}{2}}^{\frac{h}{2}}\tau^{xx}\qty(x=\pm \frac{L}{2},y)\dd{y} &= 0, \\
	b\int_{-\frac{h}{2}}^{\frac{h}{2}}y \tau^{xx}\qty(x= \pm \frac{L}{2},y) \dd{y} &=0\footnote{The other equations are identically met from the oddness of the integrand and the properties of the cross product. Also, it is not a bright idea to use footnotes in mathematical equations.} 
\end{align*}
Remember, that on the lateral sides, $x$ is fixed, so \textit{the coordinates are y and z}; some manipulation with the cross product and stuff is needed, for example:
\[
	\vb{r}\cross \tensorq{\tau} \vb{e}_x = \pm\qty(z \tau^{xx}\vb{e}_z \cross \vb{e}_x+ z \tau^{xy}\vb{e}_z \cross \vb{e}_y + y \tau^{xx}\vb{e}_{y} \cross \vb{e}_x + y \tau^{xy} \vb{e}_y \cross \vb{e}_y)
\]

Evidently, the system is complicated enough. We thus make the following assumptions:

\begin{itemize}
	\item the material of interest is a homogenous isotropc elastic solid
	\item the beam is massless $\Leftrightarrow$ the predominant force is the external load (not the body force)
\end{itemize}

From our work on the plain-stress problem, we know the Airy-stress function will be helpful for us. It will be convenient to find $\Phi$ in the form
\[
	\Phi = \Phi\qty(x,y) = A y^{3} + b y^5 + C yx^{2}+ D x^{2}y^{3}+Ex^{2},
\]
where $A,B,C,D,E$ are some constants fitted so that $\Phi$ solves the homogenous biharmonic equation: 
\[
	\laplace \laplace \Phi = 0,
\]
(recall that since we have no body forces, $\varphi = 0.$) Once we solve for the stress field, we can obtain the strain field using the constitutive relation and then solve for the displacement (solve a linear PDE) using the definition of the linearised strain tensor; see \ref{sec:plane_stress_problem}.

It can be shown the deflection of the middle point is 
\[
	\delta = \frac{5}{384}\frac{qL^4}{EI_{zz}}\qty(1+\frac{12}{5}\frac{h^{2}}{L^{2}}\qty(\frac{4}{5}+\frac{\nu}{2})),
\]
where $I_{zz}$ is a component of the inertia tensor:
\[
	I_{zz} = \frac{bh^{3}}{12}
\]


\subsection{Biharmonic equation in $\R^2$}
\label{sec:biharmonic}

Let $\Phi\qty(x,y)$ be the Airy stress function. In the previous, we have come up to the problem of solving
\[
	\begin{cases}\laplace \laplace \Phi = 0, \, &\text{in} \, \Omega \subset \R^2,\\
	\, \text{some boundary conditions} \,, &\, \text{on} \, \partial  \Omega.\end{cases}
\]
We are in $\R^2$, so we immedietaly use complex analysis: $z = x+iy, x = \frac{z+\overline{z}}{2}, y = \frac{z-\overline{z}}{2i},$ and for a function $f: \R^2 \to \R$ we make the following identification
\[
	f(x,y) \Leftrightarrow f(z,\overline{z}),
\]
and the derivatives are
\[
	\pdv{f}{x}\qty(x,y) = \pdv{f}{z}\qty(z,\overline{z})\pdv{z}{x}+\pdv{f}{z}\qty(z,\overline{z})\pdv{\overline{z}}{x} = \pdv{f}{z}\qty(z,\overline{z})+\pdv{f}{z}\qty(z,\overline{z}),
\]
\[
	\pdv{f}{y}\qty(x,y) = i\qty(\pdv{f}{z}\qty(z,\overline{z})-\pdv{f}{\overline{z}}\qty(z,\overline{z})),
\]
from which it follows
\begin{align*}
	\pdv{f}{z}\qty(z,\overline{z}) &= \frac{1}{2}\qty(\pdv{f}{x}\qty(x,y)-i \pdv{f}{y}\qty(x,y)), \\
	\pdv{f}{\overline{z}}\qty(z,\overline{z}) &= \frac{1}{2}\qty(\pdv{f}{x}\qty(x,y)+i \pdv{f}{y}\qty(x,y)).
\end{align*}

If we now take a look at the laplacian of a function $f(x,y)$, we can formally manipulate:
\[
	\laplace f\qty(x,y) = \qty(\pdv[2]{x}+\pdv[2]{y})f\qty(x,y) = \qty(\pdv{x}+i\pdv{y})\qty(\pdv{x}-i\pdv{y})f\qty(x,y) = 4 \pdv[2]{f\qty(z,\overline{z})}{\overline{z}}{z},
\]
so in total
\[
	\laplace f\qty(x,y) = 4 \pdv[2]{f}{z}{\overline{z}}\qty(z,\overline{z}).
\]
Using this, we can rewrite the \textit{Laplace equation} to the form
\[
	\pdv[2]{g\qty(z,\overline{z})}{z}{\overline{z}} = 0.
\]
Let us solve it. It must be:
\[
	\pdv{g\qty(z,\overline{z})}{\overline{z}} = C_1\qty(\overline{z}), g\qty(z,\overline{z}) = \underbrace{\int C_1\qty(\overline{z})\dd{\overline{z}}}_{\coloneq d_1\qty(\overline{z})} + d_2\qty(z)
\]
so
\[
	g\qty(z,\overline{z}) = d_1(\overline{z}) + d_2\qty(z).
\]
Now for the biharmonic equation, we need to solve
\[
	\pdv[2]{\Phi}{z}{\overline{z}} = d_1\qty(\overline{z})+d_2\qty(z),
\]
so that gives $\pdv{\Phi}{\overline{z}} = z d_1\qty(\overline{z})+D_2(z) + e_1(\overline{z}),$ and
\[
	\Phi\qty(z,\overline{z}) = z D_1\qty(\overline{z})+\overline{z}D_2(z) + E_1(\overline{z}) + E_2\qty(z).
\]
In total, we have been able to derive:
\[
	\Phi\qty(x,y) = \eval{\Re\qty(\qty(\overline{z}\gamma(z)+\chi(z)))}_{z=x+iy} = \Re\qty(\overline{\qty(x+iy)}\gamma\qty(x+iy)+\chi\qty(x+iy)).
\]

\subsection{Elliptic hole in uniformly stressed infinite plane}
\label{sec:elliptic_hole}

Suppose an infinite plane with a elliptic hole $\Omega$ with the standard paramateres $a,b$. The boundary conditions are
\begin{align*}
	\tensorq{\tau} \vb{n} &= \vb{0}, \, \text{on} \, \partial \Omega,\\
	\lim_{x^{2}+y^{2}\to \infty}\tensorq{\tau}\qty(x,y) &= S \identity,
\end{align*}
where $S \in \R$ is given. The problem can be reformulated as
\begin{equation}
	\laplace \laplace \Phi\qty(x,y) = 0, \tensorq{\tau} = \begin{bmatrix}
		\pdv[2]{\Phi}{y} & -\pdv[2]{\Phi}{x}{y} \\
		- \pdv[2]{\Phi}{x}{y} & \pdv[2]{\Phi}{x}
	\end{bmatrix},
\end{equation}
plus the boundary conditions. The general representation of the solution is $ \Phi = \Re\qty(\overline{z}\psi(z)+\chi(z))$, moreover, we adopt a sensible coordinate system: \textbf{elliptical coordinates}
\begin{align*}
	z &= c \cosh \zeta,\\
	z &= x + i y.
\end{align*}
Equivalently
\begin{align*}
	x &= c \cosh \xi \cos \eta, \\
	y &= c \sinh \xi \sin \eta, \\
	\zeta &= \xi + i \eta.
\end{align*}
It follows immedietaly:
\begin{align*}
	\qty(\frac{x}{c\cosh \xi})^{2} + \qty(\frac{y}{c\sinh \xi})^{2} &= 1, \\ 
	\qty(\frac{x}{c\cos \eta})^{2}-\qty(\frac{y}{c \sin \eta})^{2} &= 1,
\end{align*}
so the lines $\xi = \, \text{const} \,$ are \textit{ellipses} and the lines $\eta = \, \text{const} \,$ are hyperbolas. This will be useful, as we can represent the boundary of the ellipse $\partial \Omega$ as some coordinate line $\xi = \, \text{const} \,.$

Through some simple calculation, we can show
\begin{align*}
	\vb{g}_{\eta} &= \sqrt{J}\qty(-\sin \alpha \vb{e}_1+ + \cos \alpha\vb{e}_2), \\
	\vb{g}_{\zeta} &= \sqrt{J}\qty(\cos \alpha \vb{e}_1 + \sin \alpha \vb{e}_2),
\end{align*}
where $\alpha$ is the angle between the $x$ axis and $\vb{g}_{\xi}\qty(\xi,\eta)$ and
\[
	J = c^{2}\qty(\sinh^{2}\zeta \cos^{2}\eta + \cosh^{2}\zeta \sin^{2}\eta).
\]
We are interested in the quantity \footnote{That describes the rotation of the coordinate lines, which could mean "the ripping of the ellipse" when pulling}
\[
	\exp\qty(i 2 \alpha) = \frac{\sinh \zeta}{\sinh \overline{\zeta}},
\]
Formally, for the normalised vectors, we can write something like
\[
	\begin{bmatrix}
		\vb{g}_{\hat{\xi}} \\ \vb{g}_{\hat{\eta}} 
	\end{bmatrix}
	= \begin{bmatrix}
		\cos \alpha & \sin \alpha \\
		- \sin \alpha & \cos \alpha 
	\end{bmatrix}
	\begin{bmatrix}
		\vb{e}_x \\
		\vb{e}_y
	\end{bmatrix} =
\]
To solve for the stress, we need to express the stress in the elliptical coordinates: 
\begin{equation*}
  \tensorq{\tau} = \begin{bmatrix}
	  \tau^{\xi \xi} & \tau^{\xi \eta} \\
	  \tau^{\xi \eta} & \tau^{\eta \eta}
  \end{bmatrix} = \tau^{xx}\vb{e}_x \otimes \vb{e}_x + \dots = t^{\hat{\xi}\hat{\xi}}\vb{g}_{\hat{\xi}}\otimes \vb{g}_{\hat{\xi}} + \dots.
\end{equation*}
Thats just some similiarity transformation, we are representing the matrix in a different basis. The traces must be preserved:

\[
	\tau^{xx}+\tau^{yy}=\tau^{\hat{\xi}\hat{\xi}}+\tau^{\hat{\eta}\hat{\eta}}
\]
and similiarly, it can be shown
\[
	\tau^{\hat{\eta}\hat{\eta}}-\tau^{\hat{\xi}\hat{\xi}}+2i \tau^{\hat{\xi}\hat{\eta}} = \exp\qty(i 2 \alpha)\qty(\tau^{yy}-t^{xx}+2i \tau^{xy}).
\]
Combining \textbf{all of this} we obtain for the Airy stress function the following relations 
\begin{align*}
	\tau^{xx} + \tau^{yy} &= 4 \Re \dv{\psi}{z},\\
	\tau^{yy} - \tau^{xx} + 2 i \tau^{xy} &= 2\qty(z \overline{\dv[2]{\psi}{z}} + \overline{\dv[2]{\chi}{z}}).
\end{align*}
\textit{Solving this system} (heh) gives 
\begin{align*}
	\psi &= \frac{1}{2}S \sinh \zeta, \\
	\chi & = \frac{1}{2}S c^{2} \zeta \cosh\qty(2 \xi_0) \\
	t^{\hat{\eta}\hat{\eta}} &= \frac{2 S \sinh \qty(2\xi_0)}{\cosh\qty(2 \xi_0) - \cos \qty(2 \eta)}, \\
	\max_{\eta \in (0,2 \pi)} \tau^{\hat{\eta}\hat{\eta}} &= 2 S \frac{a}{b}, \\
	\min_{\eta \in (0,2 \pi)} \tau^{\hat{\eta}\hat{\eta}} &= 2 S \frac{b}{a}.
\end{align*}
We see that if $b$ is small (the ellipse is very flat), the maximum explodes; the quantity $2\frac{a}{b}$ is called the stress coefficient factor. Just note that even if the stress at infinity is controlled, the stress at the tips can be enormous.

\section{Stability of fluid flows}
\label{sec:stability}

Let us investigate the following PDE:
\[
	\begin{cases}
		\partial_t u = \pdv[2]{u}{x} + au, & \, \text{in} \, \Omega = (0,1) \\
		u(t,x) = 0, & \, \text{on} \, x=0, x=1
	\end{cases}.
\]
Clearly, $\hat{u}(t,x) = 0$ is a solution, moreover it is a \textit{steady solution}. Our interest is whether, given some $u(t=0,x) = u_0(x)$ initial condition, the solution converges to the steady one; in other words, whether
\[
	" \lim_{t \to \infty}u(t,x) = 0",
\]
in some sense of convergence. We will be interested in certain questions:

\begin{enumerate}
	\item In what sense is the convergence?
\end{enumerate}

\subsection{Energy theory}
\label{sec:energy_theory}
As opposed to the ODE stability theory, we do not want to linearize anything. Let use measure the convergence in the $\LpSet[2]{\Omega}$ norm, "the energy norm", \textit{i.e.}
\[
	\norm{u}_{\LpSet[2]{\Omega}} = \qty(\int_{\Omega}|u|^{2}\dd{x})^{\frac{1}{2}}.
\]
Meaning we are interested in the conditions under which
\[
	u \to 0 \, \text{in} \,\LpSet[2]{\Omega} \Leftrightarrow \norm{u}_{\LpSet[2]{\Omega}} \to 0.
\]
Let us investigate the following quantity:
\begin{align*}
	\frac{1}{2}\dv{t} \norm{u}_{\LpSet[2]{\Omega}}^{2} &= \frac{1}{2} \dv{t} \int_{\Omega}u\qty(t,x) u\qty(t,x)\dd{x} = \int_{\Omega}\partial_t u\qty(t,x) u\qty(t,x)\dd{x} =\\
							   &=\int_{\Omega}\qty( \partial_{xx}u+ a u^{2})\dd{x} = -\int_{\Omega}\qty(\partial_x u)^{2}\dd{x} + a \int_{\Omega}u^{2}\dd{x} = \\
							   & = -\norm{\partial_x u}_{\LpSet[2]{\Omega}}^{2} + a \norm{u}_{\LpSet[2]{\Omega}}^{2} \leq - \frac{1}{C_p^{2}}\norm{u}_{\LpSet[2]{\Omega}}^{2}+a \norm{u}_{\LpSet[2]{\Omega}}^{2} =\\
							   &=-\qty(\frac{1}{C_p^{2}}-a)\norm{u}_{\LpSet[2]{\Omega}}^{2},
\end{align*}
where we used the Poincare inequality in the form $\norm{u}_{\LpSet[2]{\Omega}} \leq C_p \norm{\partial_x u}_{\LpSet[2]{\Omega}}$ (we have zero trace). So we have :
\[
	\dv{t} \norm{u}_{\LpSet[2]{\Omega}}^{2} \leq -2\qty(\frac{1}{C_p^{2}}-a)\norm{u}_{\LpSet[2]{\Omega}}^{2},
\]
so if
\[
	\frac{1}{C_p^{2}}-a>0,
\]
the system has the following solution: \textbf{ADD IT}, and so
\[
	\norm{u(t,x)}_{\LpSet[2]{\Omega}} \to 0, \, \text{as} \, t\to \infty.
\]
We are not happy yet. The Poincare constant is undetermined, so let us get an estimate for it. The equation has the form
\begin{align*}
	\frac{1}{2}\dv{t} \norm{u}_{\LpSet[2]{\Omega}}^{2} &= - \norm{\partial_x u}_{\LpSet[2]{\Omega}}^{2}+a \norm{u}_{\LpSet[2]{\Omega}}^{2} = -a \norm{\partial_x}_{\LpSet[2]{\Omega}}^{2}\qty(\frac{1}{a}-\frac{\norm{u}_{\LpSet[2]{\Omega}}^{2}}{\norm{\partial_x}_{\LpSet[2]{\Omega}}^{2}}) \leq \\
	&\leq -a \norm{\partial_x u}_{\LpSet[2]{\Omega}}^{2}\qty(\frac{1}{a}-\max_{u \in \WkpzeroSet[1][2]{\Omega}}\frac{\norm{u}_{\LpSet[2]{\Omega}}^{2}}{\norm{\partial_x u}_{\LpSet[2]{\Omega}}}),
\end{align*}
let us define
\[
	\frac{1}{a_{\, \text{crit} \,}} = \max_{u \in \WkpzeroSet[1][2]{\Omega}}\frac{\norm{u}_{\LpSet[2]{\Omega}}^{2}}{\norm{\partial_x u}_{\LpSet[2]{\Omega}}^{2}},
\]
and so we have
\[
	\frac{1}{2}\dv{t}\norm{u}_{\LpSet[2]{\Omega}}^{2} \leq -a \norm{\partial_x u}_{\LpSet[2]{\Omega}}^{2}\qty(\frac{1}{a}-\frac{1}{a_{\, \text{crit} \,}}) = -a \norm{\partial_x u}_{\LpSet[2]{\Omega}}^{2} \frac{a_{\, \text{crit} \,}-a}{a a_{\, \text{crit} \,}}.
\]
We see that if $a < a_{\, \text{crit} \,} \Leftrightarrow \frac{1}{a} > \frac{1}{a_{\, \text{crit} \,}}$ the $\LpSet[2]{\Omega}$ norm vanishes exponentially.
But \textit{how much is it?} That depends on $a_{\, \text{crit} \,},$ so let us define the functional
\[
	F: \LpSet[2]{\Omega} \to \R^+, F: u \mapsto \frac{\norm{u}_{\LpSet[2]{\Omega}}^{2}}{\norm{\partial_x u}_{\LpSet[2]{\Omega}}^{2}},
\]
and find its extrema. The Gateux derivative at the extrema is
\begin{align*}
	0 = &\delta F\qty(u_{\, \text{ext} \,})[v] = \dv{t} \eval{F\qty(u_{\, \text{ext} \,} + t v)}_{t=0} = \dv{t} \eval{\frac{\int_{\Omega}\qty(u_{\, \text{ext} \,}+tv)\qty(u_{\, \text{ext} \,}+tv)\dd{x}}{\int_{\Omega}\qty(\partial_x u_{\, \text{ext} \,}+t \partial_x v)\qty(\partial_x u_{\, \text{ext} \,}+t \partial_xv)\dd{x}}}_{t=0} =\\
	    & = 2\frac{\int_{\Omega}u_{\, \text{ext} \,}v\dd{x}\int_{\Omega}\qty(\partial_x u_{\, \text{ext} \,})^{2}\dd{x} - \int_{\Omega}u_{\, \text{ext} \,}^{2}\dd{x}\int_{\Omega}\partial_x u_{\, \text{ext} \,} \partial_x v\dd{x}}{\qty(\int_{\Omega}\qty(\partial_x u_{\text{ext}})^{2}\dd{x})^{2}} = \\
	    & = \frac{1}{\qty(\int_{\Omega}\qty(\partial_x u_{\, \text{ext} \,})^{2}\dd{x})^{2}}\qty(\int_{\Omega}u_{\, \text{ext} \,}v\dd{x}-\frac{\int_{\Omega}u_{\, \text{ext} \,}^{2}\dd{x}}{\int_{\Omega}\qty(\partial_x u_{\text{ext}})^{2}\dd{x}} \int_{\Omega}\partial_x u_{\, \text{ext} \,} \partial_x v\dd{x}) = \frac{1}{\int_{\Omega} (\partial_x u_{\, \text{ext} \,})^{2}\dd{x}} \frac{1}{a_{\, \text{crit} \,}}\qty(\int_{\Omega}\qty(a_{\, \text{crit} \,}u_{\, \text{ext} \,}v - \partial_x u_{\, \text{ext} \,}\partial_x v)\dd{x}),
\end{align*}
and so we see
\[
	\delta F\qty(u_{\, \text{ext} \,})[v] = 0 \Leftrightarrow \int_{\Omega}\qty(a_{\, \text{crit} \,} u_{\, \text{ext} \,} v - \partial_x u_{\, \text{ext} \,} \partial_x v)\dd{x}, \forall v \in \WkpzeroSet[1][2]{\Omega}.
\]
This is a weak formulation of the problem
\[
	\int_{\Omega}\qty(a_{\, \text{crit} \,} u_{\, \text{ext} \,}+ \partial_{xx}u_{\, \text{ext} \,})v\dd{x}, \forall v \in \WkpzeroSet[1][2]{\Omega} \Leftrightarrow \begin{cases}\partial_{xx}u_{\, \text{ext} \,} = - a_{\, \text{crit} \,}u_{\, \text{ext} \,}, & \, \text{in} \,\Omega \\
		u_{\, \text{ext} \,} = 0,& \, \text{on} \, \partial \Omega 
	\end{cases}.
	\]
	This is an \textit{eigenproblem for the elliptic operator}. The solution is the following:
	\begin{align*}
		u_{\, \text{ext} \,}^n &= C \sin\qty(\sqrt{a_{\, \text{crit} \,}^n}x), \\
		a^n_{\, \text{crit} \,} &= n^{2}\pi^{2}, n \in \N
	\end{align*}
The \textit{smallest eigenvalue} is
\[
	a_{\, \text{crit} \,} = \pi^{2}.
\]
This means that $\forall a < a_{\, \text{crit} \,} = \pi^{2}$ the perturbations in the initial condition decay exponentially.

\subsection{Rayleigh-Bénard convection}
\label{sec:convection}
Let us use the developed theory on the problem of Rayleigh-Bénard convection. 
\begin{tikzpicture}
  % two parallel plates, fluid in between, spaced d away, ez to the sky, ex to me, ey to the right
\end{tikzpicture}
There are two plates, the top with the temperature $T_t$ and the bottom one with the temperature $T_b$ in the gravitational field $\vb{g}= - g \vb{e}_z.$ The governing equations are
\[
	\begin{cases}
		\dv{\rho}{t} + \rho \qty(\divergence{\vb{v}}) = 0, \\
		\rho \dv{\vb{v}}{t} = \rho \vb{g} + \divergence{\qty(-\pth\qty(\rho, \theta)\identity + \lambda \qty(\divergence{\vb{v}})\identity + 2 \mu \symvgrad)} \\
		\rho c_{V}\dv{\theta}{t} = \divergence{\qty(\kappa \grad \theta)}+\qty(\pth\qty(\theta, \rho)-\pdv{\pth\qty(\theta, \rho)}{\theta})\qty(\divergence{\vb{v}}) + \cstress : \symvgrad,
	\end{cases}
\]
Why something happens?
\begin{itemize}
	\item gravitational field is crucial, as without it, buyoancy oscillations won't work
	\item dependence of density on temperature is also essential
\end{itemize}

\subsubsection{Boussinesq approximation}
\label{sec:boussinesq_approximation}
Doing a stability analysis of a system of nonlinear PDEs is difficult. Make the following assumptions: Oberbeck-Boussinesq approximation

\begin{itemize}
	\item the density depends only on the temperature, not on pressure, and only linearly: $\rho(\theta) = \rho_{\, \text{ref} \,}\qty(1- \alpha\qty(\theta - \theta_{\, \text{ref} \,}))$
	\item working with compressible fluids is a nightmare, lets make it incompressible: $\divergence{\vb{v}} = 0.$
	\item the density in the momentum equation is constant in the first term and the same in the temperature equation
	\item ignore all the nonlinear terms in the thermal equation
\end{itemize}
Doing all this produces the following system of equations:
\begin{align*}
	\divergence{\vb{v}} &= 0, \\
	\rho_{\, \text{ref} \,} \dv{\vb{v}}{t} &= \rho_{\, \text{ref} \,}\qty(1- \alpha\qty(\theta-\theta_{\, \text{ref} \,})) \vb{g} + \divergence{\qty(-\pth\qty(\rho, \theta)\identity +  2 \mu \symvgrad)} \\
	\rho_{\, \text{ref} \,} c_{V}\dv{\theta}{t} &= \divergence{\qty(\kappa \grad \theta)},
	\end{align*}
with the boundary conditions
\[
	\begin{cases}
		\theta = T_{top}&, \, \text{on} \, z=d \\
		\theta = T_{bot}&, \, \, \text{on} \,z=0
	\end{cases}
	\]

Note that ,physically, this makes no sense. Since $\cstress : \symvgrad = 0$, there is no viscous dissipation, but since $\cstress \neq \tensorq{0}$, we are just losing energy but the temperature does not increase.

\subsubsection{Steady state, pure conduction $\qty(\vb{v} = \vb{0})$}
\label{sec:steady_state}
In the case $\vb{v} = \vb{0}$ the equation for the temperature $\theta$ becomes $0 = \divergence{\qty(\kappa \grad \theta)}$ that has the solution
\[
	\hat{\theta} = - \frac{T_{bot}-T_{top}}{d}z + T_{bot} = -\beta z + T_{bot},
\]
and the equation for the pressure reads as
\[
	0 = - \grad p - \rho_{\, \text{ref} \,}\qty(1-\alpha\qty(\theta - \theta_{\, \text{ref} \,}))g \vb{e}_{z}.
\]
\subsubsection{Perturbation of the steady}
\label{sec:perturbation}
What happens if now perturb the steady state? We are solving the following system
\[
	\begin{cases}
		\divergence{\vb{v}} = 0, \\
		\rho_{\, \text{ref} \,}\qty(\partial_t \vb{v}+ \qty(\vb{v} \vdot \grad)\vb{v}) = -\rho_{\, \text{ref} \,}\qty(1- \alpha\qty(\theta-\theta_{\, \text{ref} \,}))g \vb{e}_z - \grad p + \mu \laplace \vb{v}\\ 
		\rho_{\, \text{ref} \,} c_{V}\qty(\partial_t \theta + \qty(\vb{v} \vdot \grad)\theta) = \kappa \laplace\theta
	\end{cases},
\]
with the initial condiitons $\vb{v} = \hat{\vb{v}} + \tilde{\vb{v}} = \tilde{\vb{v}}, \theta = \hat{\theta}+ \tilde{\theta}, p = \hat{p}+ \tilde{p}$ all at $t =0$ and the hatted variables being the steady state \ref{sec:steady_state}. The equations for the perturbations (after some manipulation) become 
\[
	\begin{cases}
		\divergence{\tilde{\vb{v}}} = 0, \\
		\rho_{\, \text{ref} \,}\qty(\partial_t \tilde{\vb{v}}+ \qty(\tilde{\vb{v}} \vdot \grad)\tilde{\vb{v}}) = +\rho_{\, \text{ref} \,} \alpha\tilde{\theta}g \vb{e}_z - \grad \tilde{p} + \mu \laplace \tilde{\vb{v}}\\ 
		\rho_{\, \text{ref} \,} c_{V}\qty(\partial_t \tilde{\theta} + \qty(\tilde{\vb{v}} \vdot \grad)\qty(\hat{\theta + \tilde{\theta}})) = \kappa \laplace\tilde{\theta}
	\end{cases},
\]
\subsubsection{Non-dimensonalisation}
\label{sec:dimless}
Next, we need to non-dimensionalize the equations. For that, we need to chose 
\begin{itemize}
	\item a characteristic length: $l_{char} \coloneq d$
	\item a characteristic density $\rho_{char} \coloneq  \rho_{\, \text{ref} \,}$
	\item a characteristic temperature $\theta_{char} = T_{bot}-T_{top}$
	\item a characteristic time $t_{char} = ?$
\end{itemize}
There is however a problem: how to choose the characteristic time? We have no characteristic velocity \footnote{If we would, that would suffice, as we have characteristic length}, because $\hat{\vb{v}} = \vb{0}.$ There are some candidates whose units include seconds: $[\mu] = \, \text{Pa s} \,, [g] = \frac{\, \text{m} \,}{\, \text{s}^{2}}, [\kappa] = \frac{\, \text{W} \,}{\, \text{m K} \,}.$

Whatever, let us continue:
\[
	\tilde{\vb{v}} = v_{char}\vb{v}^{*}, \tilde{\theta} = \theta_{char}\theta^{*}, \vb{x} = l_{char}\vb{x}^{*},
\]
where the starred variables denote dimensionless ones. Plugging all this into the equations yields
\begin{align*}
		\grad^{*} \vdot \vb{v}^{*} &= 0 \\
		\frac{\rho_{ref}d^{2}}{\mu t_{char}}\qty(\partial_{t^{*}} \vb{v}^{*} + \qty(\vb{v}^{*} \vdot \grad^{*})\vb{v}^{*}) &= - \grad^{*} p^{*} + \laplace^{*} \vb{v}^{*} + \frac{\alpha g \theta_{char}d \rho_{ref} t_{char}}{\mu}\theta^{*} \vb{e}_z\\
		\partial_{t^{*}}\theta^{*} + \qty(\vb{v}^{*} \vdot \grad^{*})\theta^{*}&= \divergence{\qty(\frac{\kappa t_{char}}{\rho_{ref}c_V d^{2}}\grad^{*} \theta^{*})}+ \underbrace{v^{*}_z}_{\rho c_V \tilde{\vb{v}}\vdot \grad \hat{\theta}}, 
\end{align*}

And now we see how the choice of $t_{char}$ influences the equations. I can require one of the following
\begin{align*}
		\frac{\rho_{ref}d^{2}}{\mu t_{char}} &= 1\\
		\frac{\alpha g \theta_{char}d \rho_{ref}t_{char}}{\mu} &= 1\\
		\frac{\kappa t_{char}}{\rho_{ref}c_Vd^{2}} &=1
	.
\end{align*}
Each of these choices are sensible. In our case, we are inrested in the thermal conduction mainly, so let us choose
\[
	t_{char} = \frac{\rho_{ref}c_V d^{2}}{\kappa}.
\]
Finally, we arrive to the following system of equations (we omit the stars and tildas)
\begin{align*}
	\divergence{\vb{v}} &= 0, \\
	\frac{1}{\, \text{Pr} \,}\qty(\partial_t \vb{v}+\qty(\vb{v} \vdot \grad)\vb{v}) &= - \grad p + \laplace \theta + \, \text{Ra} \,\theta \vb{e}_z, \\
	\partial_t \theta + \qty(\vb{v} \vdot \grad)\theta &=  \laplace \theta + v_z,
\end{align*}
where
\begin{equation}
\label{eq:prandtl_number}
	\, \text{Pr} \, = \frac{\nu}{k} = \frac{\rho_{ref} d^{2}}{\mu t_{char}},
\end{equation}
is the Prandtl number and
\begin{equation}
\label{eq:rayleigh_number}
	\, \text{Ra} \, = \frac{\alpha g \theta_{char}d^{3}}{\nu k}, \nu = \frac{\mu}{\rho_{ref}}, k = \frac{\kappa}{\rho_{ref}c_V}.
\end{equation}
is the Rayleigh number.


Another form of the equations can be derived when rescaling the temperateure (choosing a different characteristic temperature)\footnote{Of course $\theta^{*}$ is totally different than the previous one}
\[
	\theta = \frac{\, \text{Pr} \,}{\sqrt{\, \text{Ra} \,}}\theta^{*},
\]
and this leads (of course, other quantities will have to be rescaled as well)

\begin{align*}
	 \grad^{*} \vdot \vb{v}^{*}&= 0 \\
	\partial_{t^{*}} \vb{v}^{*} + \qty(\vb{v}^{*} \vdot \grad^{*})\vb{v}^{*} &= - \grad^{*} p^{*} + \laplace^{*} \vb{v}^{*} + \sqrt{\, \text{Ra} \,}\theta^{*} \vb{e}_z \\
	\, \text{Pr} \,\qty(\partial_{t^{*}}\theta^{*} + \qty(\vb{v}^{*} \vdot \grad^{*})\theta^{*}) &= \laplace^{*} \theta^{*} + \sqrt{\, \text{Ra} \,}v^{*}_z.
\end{align*}
This scaling is popular in mathematical literature and \textit{we will stick to it.} It is also common to denote
\[
	R \coloneq \sqrt{\, \text{Ra} \,}.
\]
So finally finally, we are solving

\begin{align*}
	\divergence{\vb{v}}&= 0, \\
	\partial_t \vb{v}+ \qty(\vb{v} \vdot \grad)\vb{v} &= - \grad p + \laplace \vb{v} + \, \text{R} \, \theta \vb{e}_z \\
	\, \text{Pr} \,\qty(\partial_t \theta + \qty(\vb{v} \vdot \grad)\theta) & = \laplace \theta + \, \text{R} \,v_z.
\end{align*}
To add another issue, realise that we are working on unbounded domains, so integrals over the domain are problematic. This can be solved using periodic boundary conditions on lateral faces. 

Let us take now the velocity equation, multiply $\vdot \vb{v}$ and integrate $\int_{\Omega}\dd{x}$.This yields:
\[
	\int_{\Omega}\qty(\, \text{equation} \,) \vdot \vb{v}\dd{x} = \dv{t}\qty(\frac{1}{2}\int_{\Omega}\vb{v} \vdot \vb{v}\dd{x} + \int_{\Omega}\qty(\vb{v}\vdot \grad)\vdot \vb{v}\dd{x}) = - \int_{\Omega}\grad p \vdot \vb{v}\dd{x} + \int_{\Omega}\laplace \vb{v} \vdot \vb{v}\dd{x} + \int_{\Omega}\, \text{R} \,\theta \vb{e}_z \vb{v}\dd{x},
\]
realize that
\[
	\int_{\Omega}\qty(\vb{v}\vdot \grad) \vdot\vb{v}\dd{x} = \int_{\Omega}\vb{v}\vdot \grad\qty(\frac{\vb{v}\vdot \vb{v}}{2})\dd{x} = \int_{\Omega}\qty(\divergence{\vb{v}})\frac{\vb{v}\vdot \vb{v}}{2}\dd{x} = 0,
\]
and
\[
	\int_{\Omega}\grad p \vdot \vb{v}\dd{x} = - \int_{\Omega}p \qty(\divergence{\vb{v}})\dd{x} = 0,
\]
and also
\[
	\int_{\Omega}\laplace \vb{v}\vdot \vb{v}\dd{x} = \int_{\Omega}\grad \vb{v} : \grad\vb{v}\dd{x},
\]
where we have used the periodicity of the boundary conditions and incompressibility.
This means
\[
	\frac{1}{2} \dv{t}\qty(\int_{\Omega}\vb{v}\vdot \vb{v}\dd{x}) = \frac{1}{2} \dv{t} \norm{\vb{v}}_{\LpSet[2]{\Omega}}^{2} = - \norm{\grad \vb{v}}_{\LpSet[2]{\Omega}}^{2} + \int_{\Omega}\, \text{R} \, \theta \vb{e}_z \vdot \vb{v}\dd{x},
\]
which is exactly the similiar expression to the one derived at the beginning of our studies of the stability analysis. \footnote{We could again use Poincare to obtain the estimate for $\norm{\grad \vb{v}}_{\LpSet[2]{\Omega}}^{2} \leq \frac{1}{C_p} \norm{\vb{v}}_{\LpSet[2]{\Omega}}^{2}$ and stuff.}. It is evident that when
\[
	\, \text{R} \, = 0 = \, \text{Re} \,,
\]
the norm decays exponentially. 

Let us repeat the previous maniupulation. Define
\[
	\psi = \frac{1}{2}\int_{\Omega}\vb{v}\vdot \vb{v}\dd{x} + \, \text{Pr} \, \frac{1}{2} \int_{\Omega}\theta^{2}\dd{x},
\]
and investigate (we are using the equations extensively)
\[
	\dv{\psi}{t} = -\int_{\Omega}\grad \vb{v} : \grad \vb{v}\dd{x} - \int_{\Omega}\grad \theta \vdot \grad \theta\dd{x} + 2\int_{\Omega}\, \text{R} \, \theta v^z\dd{x}.
\]
Introduce yet a different notation:
\[
	\mathcal{D} \coloneq \int_{\Omega} \grad \vb{v}: \grad \vb{v}\dd{x} + \int_{\Omega}\grad \theta \vdot \grad \theta\dd{x}, \mathcal{I} = 2 \int_{\Omega}\theta v^z\dd{x},
\]
so we have
\[
	\dv{\psi}{t} = - \mathcal{D} + \sqrt{\Rayleigh}\mathcal{J} = -\mathcal{D} \sqrt{\Rayleigh}\qty(\frac{1}{\sqrt{\Rayleigh}}-\frac{\mathcal{J}}{\mathcal{D}}).
\]

Denote
\[
	\sqrt{\Rayleigh_{\, \text{crit} \,}} \coloneq \max_{\theta \in \WkpzeroSet[1][2]{\Omega}, \vb{v} \in \WkpzeroSet[1][2]{\Omega}_{\, \text{xiv} \,}} \frac{\mathcal{J}}{\mathcal{D}},
\]
and we are interested when
\[
	\frac{1}{\sqrt{\Rayleigh}}- \frac{1}{\sqrt{\Rayleigh_{\, \text{crit} \,}}} = \frac{\sqrt{\Rayleigh_{\, \text{crit} \,}}- \sqrt{\Rayleigh}}{\sqrt{\Rayleigh_{\, \text{crit} \,} \sqrt{\Rayleigh}}} >0.
\]
Let us define the functional
\[
	\mathcal{F}\qty(\vb{v}, \theta) \coloneq \frac{\mathcal{I}\qty(\vb{v}, \theta)}{\mathcal{D}\qty(\vb{v}, \theta)} - \int_{\Omega}\lambda\qty(\vb{x}) \qty(\divergence{\vb{v}})\dd{x},
\]
for some Lagrange multiplier $\lambda(\vb{x})$. We now seek:
\[
	\max_{\vb{v}\in \WkpzeroSet[1][2]{\Omega}, \theta \in \WkpzeroSet[1][2]{\Omega}}\mathcal{F}\qty(\vb{v}, \theta),
\]
so evaluate the Gateaux derivative


\[
	\delta \mathcal{F}(\vb{v}^{*}, \theta^{*})[\vb{v}, \theta] =\eval{\dv{t} (\frac{\mathcal{I}(\vb{v}^{*}+ t \vb{v}, \theta^{*} + t \theta)}{\mathcal{D}(\vb{v}^{*}+t \vb{v}, \theta^{*}+t \theta)} - \int_{\Omega}\lambda \divergence{(\vb{v}^{*}+t \vb{v})}\dd{x})}_{t=0},
\]

it is easy to realize the numerator is
\[
	2 \int_{\Omega}\theta v^{*z} + \theta v^z\dd{x},
\]
and the denominator is
\[
	2 \int_{\Omega}\grad \vb{v}^{*} : \grad \vb{v} + \grad \theta^{*} \vdot \grad \theta \dd{x},
\]
so using the Leibniz rule we obtain

\begin{align*}
	0 &= \frac{1}{\mathcal{D}\qty(\vb{v}^{*}, \theta^{*})}\qty(\dv{\mathcal{I}\qty(\vb{v}^{*}, \theta^{*})}{t} - \frac{\mathcal{I}}{\mathcal{D}}\qty(\vb{v}^{*}, \theta^{*})\dv{\mathcal{D}}{t}\qty(\vb{v}^{*}, \theta^{*})- \int_{\Omega}\lambda \qty(\divergence{\vb{v}^{*}})\dd{x}) \\
	  &\equiv \frac{1}{\mathcal{D}^{*}}\qty(\dv{\mathcal{I}^{*}}{t}-\sqrt{\Rayleigh_{\, \text{crit} \,}}\dv{\mathcal{D}^{*}}{t}-\int_{\Omega}\lambda \qty(\divergence{\vb{v}^{*}})\dd{x}),
\end{align*}
where we have just rescaled the Lagrange multiplier. This becomes
\[
	= \frac{1}{\mathcal{D}^{*}}\qty(\int_{\Omega} \theta v^{*z} + \theta^{*} v^z\dd{x} - \sqrt{\Rayleigh_{\, \text{crit} \,}} \qty(\int_{\Omega}\grad \vb{v}^{*} : \vb{v}\dd{x} + \int_{\Omega}\grad \theta^{*} \vdot \grad \theta\dd{x} - \int_{\Omega}\lambda \qty(\divergence{\vb{v}^{*}})\dd{x})).
\]
Hmm, this is familiar - it is a weak formulation of some problem! (If we take $\vb{v}, \theta$ to be test functions...). The problem then becomes

\begin{align*}
	- \int_{\Omega}\qty(-\sqrt{\Rayleigh_{\, \text{crot} \,}}\laplace \theta^{*} + v^{*z}) \theta\dd{x} = 0 &\Leftrightarrow -\frac{1}{\sqrt{\Rayleigh_{\, \text{crit} \,}}}\laplace \theta^{*} + v^{*z} = 0, \\
	\int_{\Omega}\qty(\theta^{*} \vb{e}_z - \sqrt{\Rayleigh_{\, \text{crit} \,}}\laplace \vb{v}^{*} - \grad \lambda)\vdot \vb{v}\dd{x} = 0 &\Leftrightarrow - \frac{1}{\sqrt{\Rayleigh_{\, \text{crit} \,}}}\laplace \vb{v}^{*} + \theta^{*} \vb{e}_z - \grad \lambda =0.
\end{align*}
So the pair $\qty(\vb{v}^{*}, \theta^{*})$ maximizes $\mathcal{F}\qty(\vb{v}, \theta)$ iff it solves the system above. We must thus solve now

\begin{align*}
	- \laplace \vb{v}^{*} - \grad \lambda + \sqrt{\Rayleigh_{\, \text{crit} \,}}\theta^{*} \vb{e}_z &= 0, \\
	\laplace \theta^{*} + \sqrt{\Rayleigh_{\, \text{crit} \,}}v^{*z} &= 0,, \\
	\divergence{\vb{v}^{*}}&= 0,
\end{align*}
which is \textit{exactly the system for the perturbtion, only linearized nd stationary.} It can be rewritten in this "suggestive notation":

\[
	\begin{bmatrix}
		\laplace & - \grad & 0 \\
		\grad \vdot & 0 & 0 \\
		0 & 0 & \laplace
	\end{bmatrix}
	\begin{bmatrix}
		\vb{v}^{*}\\
		\lambda \\
		\theta^{*}
	\end{bmatrix}
	= -\sqrt{\Rayleigh_{\, \text{crit} \,}}\begin{bmatrix}
		0 & 0 & \vb{e}_z \\
		0 & 0 & 0 \\
		\vb{e}_z \vdot & 0 & 0
	\end{bmatrix}
	\begin{bmatrix}
		\vb{v}^{*} \\
		\lambda \\
		\theta^{*}
	\end{bmatrix}
\]
which is a generalized eigenvalue problem
\[
	\tensorq{A} \vb{x} = \mu \tensorq{B} \vb{x},
\]
and so we see that we are interested in the (generalized) spectrum of some operators. Coming back to our conditions for maximizing, taking the divergence of the first equation:
\[
	0 = \underbrace{- \divergence{\laplace \vb{v}^{*}}}_{= \laplace \qty(\divergence{\vb{v}}) = 0}- \laplace \lambda + \sqrt{\Rayleigh_{\, \text{crit} \,}}\qty(\divergence{\qty(\theta \vb{e}_z)}),
\]
taking the laplacian of the first one yields
\[
	\laplace \laplace \vb{v}^{*} - \underbrace{\laplace \grad \lambda}_{=\grad \laplace \lambda} + \sqrt{\Rayleigh_{\, \text{crit} \,}}\laplace\qty(\theta^{*} \vb{e}_{z}),
\]
so combining the two yields (eliminating $\lambda$):

\begin{align*}
	\laplace \laplace \vb{v}^{*} + \sqrt{\Rayleigh_{\, \text{crit} \,}}\grad \qty(\divergence{\qty(\theta^{*} \vb{e}_z)}) + \sqrt{\Rayleigh_{\, \text{crit} \,}}\laplace\qty(\theta^{*} \vb{e}_z) &= 0, \\
	\laplace \theta^{*} + \sqrt{\Rayleigh_{\, \text{crit} \,}} v^{*z} &= 0.
\end{align*}

Now let us take the $z$ component of the first equation, so we are solving
\begin{align*}
	\laplace \laplace \vb{v}^{*z} - \sqrt{\Rayleigh_{\, \text{crit} \,}}\pdv[2]{\theta^{*}}{z}  + \sqrt{\Rayleigh_{\, \text{crit} \,}}\laplace\qty\theta^{*}  &= 0, \\
	\laplace \theta^{*} + \sqrt{\Rayleigh_{\, \text{crit} \,}} v^{*z} &= 0,
\end{align*}
and Fourier transform it:
\[
	\vb{v}^{*} = \hat{\vb{v}}(z) \exp\qty(i(k_x x + k_y y)), \theta^{*} = \hat{\theta}(z) \exp\qty(i(k_x x+ k_y y)),
\]
so formally
\[
	\laplace \to \dv[2]{z} - k^{2},
\]
and the equations become \textbf{(now we are solving for the hatted variables, the amplitudes, without renaming anything.)}

\begin{align}
\label{eq:eqn_transformed}
0 &= \qty(\dv[2]{z}-k^{2})^{2} v^{*z} - \sqrt{\Rayleigh_{\, \text{crit} \,}} \\
		0 &= \qty(\dv[2]{z}-k^{2})\theta^{*} + \sqrt{\Rayleigh_{\, \text{crit} \,}} v^{*z}.
\end{align}

Now we apply the Fourier transformed laplacian to the first one and write

\[
	0 = \qty(\dv[2]{z}-k^{2})^{3} v^{*z}- \sqrt{\Rayleigh_{\, \text{crit} \,}} k^{2}\qty(\dv[2]{z}-k^{2})\theta^{*},
\]
and plug this into the remaining equation gives

\[
	\qty(\dv[2]{z}-k^{2})^{3} v^{*z} = - \sqrt{\Rayleigh_{\, \text{crit} \,}}k^{2} v^{*z}, z \in [0,1].
\]
We have thus derived a sixth order ODE for the velocity, which is in fact an \textit{eigenvalue problem for the (linear unbounded) operator}. The problem is about the boundary condition: it makes \textit{some} sense to assume:

\begin{align*}
	v^{*z} &= 0, \\
	\dv[2]{v^{*z}}{z} &= 0,\\
	\dv[4]{v^{*z}}{z} &= 0,
\end{align*}
all on $\{z = 0,1\}.$ Really, it can be shown
\[
	v^{*z} = \sum_{n=1}^{\infty}v_n^{*z}\sin\qty(n \pi z),
\]
for some numbers $v_n^{*z}.$ This representation really \textit{splnuje} the above boundary conditions. Plug this an write:

\[
	\qty(-n^{2} \pi^{2} - k^{2})^{3}v^{*z}_n = -\sqrt{\Rayleigh_{\, \text{crit} \,}}v^{*z}_n,
\]
and so 
\[
	\sqrt{\Rayleigh^n_{\, \text{crit} \,}} = \frac{\qty(\pi^{2}n^{2}+k^{2})^{3}}{k^{2}}, n \in \N.
\]
As we are looking for the smallest one, our value is:

\[
	\sqrt{\Rayleigh_{\, \text{crit} \,}} = \frac{\qty(\pi^{2}+k^{2})^{3}}{k^{2}}.
\]
Notice that this still depends on $k_n = \frac{2 \pi}{L}n$ the choice of $L$, \textit{i.e.}, the choice of the periodicity of the boundary. So in fact we want to minimize this

\[
	\pdv{\sqrt{\Rayleigh_{\, \text{crit} \,}}}{k} = \frac{\qty(k^{2}+\pi^{2})^{2}\qty(3k^{2}-\qty(k^{2}+\pi^{2}))}{k^{2}} = 0 \Rightarrow k_{\, \text{crit} \,} = \frac{\pi^{2}}{2},
\]
plugging this yields

\begin{equation}
    \label{eq:Rayileigh_crit}
    \sqrt{\Rayleigh_{\, \text{crit} \,}} = \frac{27}{4}\pi^4 \Leftrightarrow \Rayleigh_{\, \text{crit} \,} = \frac{729}{16}\pi^8,
\end{equation}

\subsubsection{Free-free boundary conditions}
\label{sec:free_free}

But hold on, how did we get the boundary conditions? Those are called the \textit{free-free} boundary conditions.

\[
	v^z = 0 \, \text{on} \,\{z = 0,1 \},\cstress \vb{n} = - p_{\, \text{ambient} \,} \vb{n},\, \text{on} \, \{z = 0,1\},
\]
and
\[
	\vb{v} = \vb{0} + \tilde{\vb{v}}, \cstress = -\qty(\hat{p}+\tilde{p})\identity + \dots.
\]
The pressure in the steady case also satisfies the boundary conditions, namely
\[
	-\hat{p}\identity \vb{n} = - p_{\, \text{ambient} \,}\vb{n}.
\]
This means
\[
	\tilde{\cstress}= - \tilde{p}\identity + 2 \mu\qty(\frac{1}{2}\qty(\grad \tilde{\vb{v}})+\transpose{\qty(\grad \tilde{\vb{v}})}),
\]
with the following boundary conditions
\[
	\tilde{\cstress}\vb{n} = \vb{0} , \tilde{v}^z = 0 \, \text{on} \, \{z = 0,1\}.
\]
This translates to\footnote{on $\{z = 0,1\}$ the outer unit normal $\vb{n}$ equals to $\vb{e}_z$.}
\[
	\transpose{[T_{zx}, T_{yz}, T_{zz}]} = \vb{0},
\]
which implies
\[
	\pdv{\tilde{v}^x}{z} + \pdv{\tilde{v}^z}{x} = 0, \pdv{\tilde{v}^y}{z}+ \pdv{\tilde{v}^z}{y} = 0,
\]
again on $\{z = 0,1\}.$ Since $\tilde{v}^z = 0$ there, also its derivative (assuming continuity...) is zero there, and so those conditions really mean
\[
	\pdv{\tilde{v}^x}{z} = 0, \pdv{\tilde{v}^y}{z} = 0.
\]

Recall that
\[
	\divergence{\vb{v}} = \pdv{\tilde{v}^x}{x}+ \pdv{\tilde{v}^y}{y} + \pdv{\tilde{v}^z}{z} = 0,
\]
\textit{inside of $\Omega$.} Let us however suppose that it holds also \textit{on the boundary\footnote{We are on $\{z = 0,1\}$.}}. Differentiate w.r.t $z$, swap the derivatives and obtain

\[
	\pdv{x}\qty(\pdv{\tilde{v}^x}{z})+\pdv{y}\qty(\pdv{v^y}{z})+\pdv[2]{v^z}{z} = 0,
\]
on $\{z = 0,1\}.$ Since the first two terms are zero, we read
\[
	\pdv[2]{\tilde{v}^z}{z} = 0 \, \text{on} \, \{z = 0,1\}.
\]

Finally, let us deal with the BC for the forth derivative. For that, recall that we have not yet discussed the boundary conditions for the temperature, which are: $\tilde{\theta} = 0 \, \text{on} \, \{z = 0,1\}.$ Take a look at \ref{eq:eqn_transformed} now, using the boundary conditions and the definition of $\laplace$, we obtain \footnote{Again, we are also assuming the equations hold on the boundary aswell.}

\[
	\qty(\dv[2]{z}-k^{2})^{2}v^{*z} = 0 \, \text{on} \, \{z = 0,1\},
\]
so in particular
\[
	\dv[4]{v^{*z}}{z} = 0 \, \text{on} \, \{z = 0,1\}.
\]

\subsubsection{The case $\Rayleigh > \Rayleigh_{\, \text{crit} \,}$ "slightly"}
\label{sec:r>r_crit}

What happens when we perturb the system with
\[
	\Rayleigh > \Rayleigh_{\, \text{crit} \,},
\]
meaning \textit{slightly larger}? That would mean

\[
	\frac{\qty(\pi^{2}+k^{2})^{3}}{k^{2}}>\frac{\qty(\pi^{2}+k_{\, \text{crit} \,}^{2})^{3}}{k_{\, \text{crit} \,}^{2}},
\]

%picture of a almost parabola, nonsymmetric, with a horizontal line cutting the parabola. At the minimum of the parabola, the k_crit takes place

As a toy problem, let us suppose the following ODE

\[
	\dv{t} \begin{bmatrix}
		q_1 \\
		q_2
	\end{bmatrix}
	= - \begin{bmatrix}
		\gamma_1\qty(\Rayleigh) & 0 \\
		0 & \gamma_2\qty(\Rayleigh)
	\end{bmatrix}
	\begin{bmatrix}
		q_1 \\
		q_2
	\end{bmatrix}
	+
	\begin{bmatrix}
		-a q_1 q_2 \\
		b q_1^{2}
	\end{bmatrix},
\]
where $\gamma_1\qty(\Rayleigh), \gamma_2\qty(\Rayleigh)$ are some functions of the Rayleigh number. THere are some regimes:

\begin{itemize}
	\item $\gamma_1>0, \Rayleigh < \Rayleigh_{\, \text{crit} \,}:$ then $q_1$ is \textit{damped exponentially} and the nonlinearity does not play a role,
	\item $\gamma_1<0, \Rayleigh > \Rayleigh_{\, \text{crit} \,}:$ then $q_1$ \textit{grows exponentially} and therefore the nonlinearity cannot be ignored.
	\item $\gamma_2 >>1$ means that the second equation is (almost) only a algebraic one, which we can solve, substitue back into the first one and obtain
		\[
			\dv{q_1}{t} = -\gamma_1 q_1 - \frac{ab}{\gamma_2}q_1^{3} = -\gamma_1 q_1\qty(1+\frac{ab}{\gamma_1\gamma_2}q_1^{2}),
		\]
		which is really interesting; it is only a cubic correction to a linear system (\textit{i.e.}, a \textit{quadratic} nonlinearity.) This model might serve as a precursor to the \textit{Ginzburg - Landau} equations.
\end{itemize}

\end{document} 
%% Local Variables: 
%%% mode: latex
%%% TeX-master: t
%%% End: 
