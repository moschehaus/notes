% !TEX root = main.tex

\documentclass[../main.tex]{subfiles}
\begin{document}
\section{Differential geometry, tensor calculus}
\label{sec:curvilinear_coords}
How to write $\divergence{\vb{u}}, \curl{\vb{u}}$ etc. in polar, cylindrical and other coordinates? Can the deformation $\vb*{\gamma}$ be viewed as a change of \textit{something} rather than the body itself?  Notice that there are similiarities between change of coordinates $\vb{x} = \vb{x}(\vb*{\gamma})$ and the deformation $\vb{x}=\vb*{\chi}(\vb{x})$, the Left Cauchy-Green tensor $\lcg$ and the metric tensor $\tensorq{g}$.


\subsection{Curvilinear coordinates}
\label{sec:coordinates}

Let us for now \textit{pretend} we are still in our flat Euclidian space $\mathbb{E}^{n},$ and we are just changing our coordinates... consider $\vb{x}=\qty(x^1,x^2,\dots,x^n)$ as the cartesian coordinates of the point $\vb{x}$ and a different set $\vb{x}=\vb{x}(\vb*{\xi})$, \textit{e.g.}, the polar coordinates
\[
	x^1(r,\varphi)=r \cos \varphi, x^2\qty(r,\varphi)=r \sin \varphi, [x,y] = [x^1, x^2], [r,\varphi] = [\xi^1,\xi^2].
\]
That means every point $\vb{x}$ in a plane can be described by using $[x^1,x^2]$ or $[r,\varphi]$. We are used to analysis in cartesian coordinates - how can i do it in a more general setting?

Viewed from a different perspective, we do not need to pretend everything takes place in $\mathbb{E}^{n}.$ Let $G \subset \R^{n}$ be an open subset of $\R^{n}$ ("space of coordinates") and let $\vb*{\Theta}: G \to \mathcal{M}$ be a $C^1$ diffemorphism between $G$ and a $\mathcal{M} \subset \mathbb{E}^n$ an open subset of the Euclidian space (it can in fact be a manifold). In this view, all points $\vb{x} \in \mathcal{M}$ can be uniquely represented as $\vb{x} = \vb*{\Theta}\qty(\vb*{\xi}),$ for some $\vb*{\xi} \in G.$ Consider some examples

\begin{example}[Polar, spherical and cylindrical coordinates]
	\begin{itemize}
		\item polar coordinates: $G = \qty(0, \infty) \times \qty(0, 2 \psi) \subset \R^2, \mathcal{M} = \mathcal{S}^2, \vb*{\xi} = (r, \varphi), \vb*{\Theta}: \qty(r, \varphi) \mapsto (r \cos \varphi, r \sin \varphi), \vb{x} = \qty(r \cos \varphi, r \sin \varphi)$,
		\item spherical coordinates: $G = \qty(0, \infty) \times \qty(0, \psi) \times (0, 2 \psi) \subset \R^{3}, \mathcal{M} = \mathcal{S}^3, \vb*{\xi} = \qty(r, \theta, \varphi), \vb*{\Theta}: \qty(r, \theta, \varphi) \mapsto \qty(r \sin \theta \cos \varphi, r \sin \theta \sin \varphi, r \cos \theta), \vb{x} = \qty(r \sin \theta \cos \varphi, r \sin \theta \sin \varphi, r \cos \theta)$,
		\item cylindrical coordinates: $G = \qty(0, \infty) \times \qty(0, 2 \pi) \times \R \subset \R^{3}, \mathcal{M} = \mathcal{S}^2 \times \R, \vb*{\xi} = \qty(r, \varphi, z), \vb*{\Theta}: \qty(r, \varphi, z) \mapsto \qty(r \cos \varphi, r \sin \varphi, z), \vb{x} = \qty(r \cos \varphi, r \sin \varphi, z).$
	\end{itemize}
\end{example}
It is customary, however imprecise, to write $\vb{x} = \vb{x}\qty(\vb*{\xi})$ instead of $\vb{x} = \vb*{\Theta}\qty(\vb*{\xi}).$ By the first one we mean $\vb{x}$ can be obtained through some function of the argument $\vb*{\xi},$ but it it is not true that $\vb{x}$ itself is a function. In fact, $\vb{x}$ is a point in $\mathcal{M},$ that is the image of $\vb*{\xi}$ under $\vb*{\Theta}.$

This approach is elegant, as from the very start we are connecting the properties of $\vb{x} \in \mathcal{M}$, \textit{i.e.}, the properties of (the manifold) $\mathcal{M} \subset \tensorq{E}^n$ with the propertis of $\vb*{\xi} \in G,$ \textit{i.e.}, with the properties of $G \subset \R^{n}.$ Also, it immedietaly allows for the comparison with continuum mechanics: instead of $\vb{x} = \vb*{\chi}\qty(\vb{X})$ we are just writing $\vb{x} = \vb*{\Theta}\qty(\vb*{\xi}).$

\begin{remark}
	The name curvilinear coordinates come from the fact that the image $\vb*{\Theta}\qty(\qty{\xi^k = \, \text{const} \,})$ are not "straight lines"
\end{remark}

\begin{definition}[Coordinate lines]
	For $j \in \qty{1, \dots, n}$ and $\xi^j \in \R$ such that $\qty(\xi^1, \dots, \xi^j, \dots, \xi^n) \in G$ we define the $j-$th coordinate lines/curve $\vb*{\gamma}_j$ as the curve
	\[
		\vb*{\gamma}_j(\xi^j) = \vb{x}(\xi^1,\dots,\xi^j,\dots,\xi^n), \, \text{\textit{i.e.}} \,, \vb*{\gamma}_j\qty(\xi^j) = \vb*{\Theta}\qty(\xi^1, \dots, \xi^j, \dots, \xi^n),
	\]
	and the rest of $\xi^i$ remain arbitrary.
\end{definition}

\begin{figure}
	\label{fig:coordinate_systems}
\begin{tikzpicture}
	\draw[<->] (-5,0) -- (5,0) node[anchor=north east]{$x^1$};
	\draw[<->] (0,-5) -- (0,5) node[anchor=north east]{$x^2$};
	\draw[thin,blue] (2,-5) -- (2,5) node[anchor=north east]{$x^1 =\, \text{const} \,$};
	\path [name path=vert] (2,-5) -- (2,5);

	\draw[thin,blue] (-5,-3) -- (5,-3) node[anchor=north east]{$x^2 =\, \text{const} \,$};
	\path [name path=horiz] (-5,-3) -- (5,-3);

	\path [name intersections={of=vert and horiz, by=A}];
	\draw[thick,blue,->] (A) --++ (3,0) node[anchor=north west]{$\vb{e}_2$};

	\draw[thick,blue,->] (A) --++ (0,3) node[anchor=north west]{$\vb{e}_1$};

	\draw[thin,purple] (0,0) circle (3.5cm) node[anchor=south east]{$y^1 = r = \, \text{const} \,$};
	\path [name path=circlePath] (0,0) circle (3.5cm);

	\draw[thin,purple] (0,0) -- (5,5) node[anchor=north east]{$y^2=\varphi_1=\, \text{const} \,$};
	\path [name path=radialPath] (0,0) -- (5,5);

	\path [name intersections={of=circlePath and radialPath, by=B}];

	\draw[thick,purple,->] (B) --++ (-1,1) node[anchor=south east]{$\vb{g}_1\qty(r,\varphi_1)$};
	\draw[thick,purple,->] (B) --++ (1,1) node[anchor=south east]{$\vb{g}_2\qty(r,\varphi_1)$};

	\draw[thin,purple] (0,0) -- (-5,5) node[anchor=south east]{$y^2 = \varphi_2 = \, \text{const} \,$};
	\path [name path=radialPath2] (0,0) -- (-5,5);
	\path [name intersections={of=circlePath and radialPath2, by=C}];

	\draw[thick,purple,->] (C) --++ (-1,1) node[anchor=south west]{$\vb{g}_2\qty(r,\varphi_2)$};
	\draw[thick,purple,->] (C) --++ (-1,-1) node[anchor=south east]{$\vb{g}_1\qty(r,\varphi_2)$}; 

\end{tikzpicture}
\caption{Coordinate lines and basis vectors in cartesian and polar coordinates (\textit{the length of the vectors is the same...})}
\end{figure}

\subsubsection{Basis of a vector space}
\label{sec:basis}
In cartesian coordinates: $\{\vb{e}_1,\vb{e}_2,\dots,\vb{e}_n \}$, where the vectors are \textit{tangent to the coordinate lines}, that is

\begin{equation}
	\vb{e}_i = \dv{\vb*{\gamma}^i}{x^i},
\end{equation}

In a curvilinear coordinate system, we can repeat the same construction. We can \textit{define a vector tangent to the coordinate line}

\begin{equation}
	\vb{g}_i \qty(\vb*{\xi})=\dv{\vb*{\gamma}_i}{\xi^i}\qty(\xi^i) = \pdv{\vb*{\Theta}}{\xi^i}\qty(\vb*{\xi}).
\end{equation}

The problem is that the vectors $\vb{g}_i$ are not constant in space! It is a vector field!.

\begin{remark}
	If it is evident what variable we are differentiating with respect to (which is not always), we write
	\[
		\pdv{\xi^i} = \partial_{i},
	\]
	so
	\[
		\vb{g}_i\qty(\vb*{\xi}) = \partial_{i} \vb*{\Theta}\qty(\vb*{\xi}).
	\]
	This makes the manipulation with indices a bit easier, but sometimes covers the true meaning...
\end{remark}

\subsubsection{Vector fields}
\label{sec:vector_fields}

A vector $\vb{v}$ is independent of a basis; i can express it w.r.t $\qty{\vb{e}_j}$ and $\qty{\vb{g}_j}$ also:
\[
	\vb{v}=v^i \vb{e}_i = \nu^i\qty(\vb*{\xi}) \vb{g}_i\qty(\vb*{\xi}).
\]
(Note that in general $v^i \neq \nu^i$.) What about its derivatives? We already sense trouble, as the "curvilinear basis" is not constant!

\[
	\pdv{\vb{v}}{x^i}=\pdv{\qty(v^j \vb{e}_j)}{x^i} = \pdv{v^j}{x^i}\vb{e}_j,
\]
works perfectly fine in cartesian coordinates, as $\vb{e}_j = \text{const}$. In curvilinear setting

\begin{equation}
	\label{eq:dvvectorfield}
	\pdv{\vb{v}}{\xi^i}=\pdv{\qty(v^j \vb{g}_j)}{\xi^i}= \pdv{v^j}{\xi^i}\vb{g}_j + v^j \pdv{\vb{g}_j}{\xi^i},
\end{equation}
as generally
\[
	\pdv{\vb{g}_j}{\xi^i} \neq \vb{0}.
\]
We can identify the last term, as differentiating a vector \textit{should give} a vector, so in particular it can be expressed w.r.t to the basis $\qty{\vb{g}_k}$:
\[
	\pdv{\vb{g}_j}{\xi^i} = \Gamma \indices{^k_{ji}} \vb{g}_k,
\]
where $\Gamma \indices{^k_{ji}} $ are the coefficients of the linear combinations. Thanks to the \textit{commutation of the partial derivatives} \footnote{We are still in flat $\R^d$, i.e. euclidian space. No curvature, torsion, that would obstruct the commutation properties.}, it holds
\begin{equation}
	\label{eq:symmetry_christoffel}
	\Gamma \indices{^k_{ji}} = \Gamma \indices{^k_{ij}},
\end{equation}
i.e., $\Gamma \indices{^k_{ij}}$ is symmetric in $ij$. Well, that did not help \textit{very} much, as we don't know $\Gamma \indices{^k_{ij}}$, but at least we have the symmetry property. Going back to \ref{eq:dvvectorfield}:

\begin{equation}
	\pdv{\vb{v}}{\xi^i}=\pdv{\qty(v^j \vb{g}_j)}{\xi^i}= \pdv{v^j}{\xi^i}\vb{g}_j + v^j \pdv{\vb{g}_j}{\xi^i} = \pdv{v^k}{\xi^i}\vb{g}_k + v^j \Gamma \indices{^k_{ij}} \vb{g}_k = \qty(\pdv{v^k}{\xi^i}+\Gamma^k_{ij}v^j) \vb{g}_k.
\end{equation}
In short
\[
	\pdv{\vb{v}}{\xi^i} = \qty(\pdv{v^k}{\xi^i}+\Gamma^k_{ij}v^j) \vb{g}_k.
\]
Compare it to
\[
	\pdv{\vb{v}}{x^i}=\pdv{v^k}{x^i}\vb{e}_k.
\]
This leads us to the definition

\begin{definition}[Covariant derivative of a vector field]
    The quantity:
\begin{equation}
	\nabla_i v^j = \pdv{v^j}{\xi^i} + \Gamma \indices{^j_{ik}}v^k = \partial_{i}v^j + \Gamma \indices{^j_{ik}}v^k
\end{equation}
is called the \textbf{covariant derivative of the components vector field }$\vb{v}$. We have also shown
\[
	\partial_{i} \vb{v} = \nabla_i v^j \vb{g}_j.
\]
\end{definition}

\subsubsection{Dot product}
\label{sec:dot_product}

The number $\vb{v}\vdot \vb{u}$ is obtained in a special manner:
\[
	\vb{v} \vdot \vb{u}= v^i \vb{e}_i \vdot u^j \vb{e}_j = (\vb{e}_i \vdot \vb{e}_j) v^i u^j = \delta_{ij}v^i u^j.
\]
I can of course write the vectors in a different basis:

\[
	\vb{v}\vdot \vb{u} = v^i \vb{g}_i \vdot u^j \vb{g}_j = (\vb{g}_i \vdot \vb{g}_j) v^i u^j = g_{ij}v^i u^j.
\]

\begin{definition}[Metric tensor]
    The tensor $\tensorq{g}$ such that $\forall \vb{v} = v^i \vb{g}_i, \vb{u}= u^j \vb{g}_j$ it holds:
    \[
	    \vb{v} \vdot \vb{u}=g_{ij} v^i u^j, g_{ij}= \vb{g}_i \vdot \vb{g}_j,
    \]
    is called the \textbf{metric tensor}.
\end{definition}


\subsubsection{Dual space}
\label{sec:dual_space}

The (vector) dual space is the space of all linear forms on the underlying vector space. In particular it is a vector space itself, so
\[
	\forall \vb{l} \in V^{*}: \vb{l} = l_i \vb{e}^i,
\]
where $\vb{e}^i$ is the i-th basis vector. The action of the forms can be described as
\[
	\vb{l}(\vb{v}) = l_i \vb{e}^i(v^j \vb{e}_j) = l_i v^j \vb{e}^i(\vb{e}_j), \forall \vb{v} \in V.
\]
If it holds $\vb{e^i}(\vb{e}_j) = \delta^i_j$, we call the basis $\vb{e}^i$ dual to $\vb{e}_j$. What about curvilinear setting? We can adopt the same definition
\begin{definition}
    We call the basis $\vb{g}^j$ of $V^{*}$ the dual basis to $\vb{g}_i$ iff

\[
	\vb{g}^j(\vb{g}_i) = \delta^j_i.
\]
\end{definition}

For the original basis we had $\vb{g}_i = \pdv{\vb*{\Theta}}{\xi^i} = \pdv{\Theta^j}{\xi^i}\vb{e}_j$, in the dual case (using the chain rule):

\[
	\vb{g}^i\qty(\vb{g}_j) = \delta^i_j = \pdv{\xi^i}{\xi^j}= \pdv{\xi^i}{\Theta^k}\pdv{\Theta^k}{\xi^j} \vb{e}^k(\vb{e}_k) = \qty(\pdv{\xi^i}{\Theta^k}\vb{e}^k)\qty(\pdv{\Theta^k}{\xi^j}\vb{e}_k) = \qty(\pdv{\xi^i}{\Theta^k}\vb{e}^k)\qty(\vb{g}_j)
\]
so i can conclude
\[
	\vb{g}^i = \pdv{\xi^i}{\Theta^k}\vb{e}^k.
\]

Recall that we have the \textit{Riesz representation theorem}:
\[
	\forall \vb{l} \in V^{*} \exists ! \vb{u} \in V: \vb{l}(\vb{v}) = \vb{v} \vdot \vb{u}, \forall \vb{v} \in V.
\]

This implies

\[
	l_i \vb{g}^i (v^j \vb{g}_j) = v^m u^n g_{mn}, \, \text{\textit{i.e.}}, \, l_i v^j \delta^i_j = u^m v^n g_{mn}, \, \text{\textit{i.e.}}, \, l^i v_i = u^m v^i g_{mi}, \, \text{\textit{i.e.}}, \,l_i = g_{im} u^m.
\]

So $l_i = g_{im}u^m$, where $\vb{u}$ represents $\vb{l}$. It is common to write

\[
	l_i = g_{im}l^m.
\]

\subsubsection{Covector fields}
\label{sec:covector_fields}

How to compute $\pdv{\vb{l}}{y^i}$? Just change the location of the index :)
\[
	\pdv{\vb{l}}{\xi^i} = \pdv{(l_j \vb{g}^j)}{\xi^i} = \pdv{l^j}{\xi^i}\vb{g}^j + \pdv{\vb{g}^j}{\xi^i}l_j.
\]
Again, the last term must be expressable in the dual basis, so 

\begin{equation}
	\pdv{\vb{l}}{\xi^i}= \pdv{l_j}{\xi^i}\vb{g}^j+ \tilde{\Gamma}\indices{^j_{im}} l_j \vb{g}^m,
\end{equation}


where again $\tilde{\Gamma}\indices{^j_{im}} = \tilde{\Gamma}\indices{^j_{mi}}$ are the coefficients of the linear combinations, "that are symmetric".

What is the relation between $\Gamma \indices{^k_{im}}$ and $\tilde{\Gamma}\indices{^k_{im}}$? Recall $\delta^i_j = \vb{g}^i(\vb{g}_j)$, so differentiating can lead us to

\begin{equation}
	\label{eq:relationGammas}
    \Gamma^j_{im} = - \Gamma^j_{im}.
\end{equation}

\begin{definition}
    Let $\vb{l}$ be a covector field. The quantity
    \begin{equation}
	    \nabla_i l_j = \partial_{i}l_j - \Gamma \indices{^k_{ij}}l_k
    \end{equation}
    is called \textbf{the covariant derivative of the components covector field} $\vb{l}$. We have shown
    \[
	    \partial_{i}\vb{l} = \nabla_i l_j \vb{g}^j.
    \]
\end{definition}

Finally, for the tensor field of type $(0,2)$, \textit{i.e.}, for bilinear forms, one can obtain
\begin{align*}
	\pdv{\tensorq{A}}{\xi^i} &= \pdv{\xi^i}\qty(A_{mn} \vb{g}^m \otimes \vb{g}^n) = \pdv{A_{mn}}{\xi^i}\vb{g}^m \otimes \vb{g}^n + A_{mn}\pdv{\vb{g}^m}{\xi^i} \otimes \vb{g}^n + A_{mn}\vb{g}^m \otimes \pdv{\vb{g}^n}{\xi^i}=\\
				 &=\pdv{A_{mn}}{\xi^i}\vb{g}^m \otimes \vb{g}^n - A_{mn} \Gamma \indices{^m_{ik}}\vb{g}^k \otimes \vb{g}^n - A_{mn}\vb{g}^m \otimes \Gamma \indices{^n_{ik}} \vb{g}^k = \\
				 &= \pdv{A_{mn}}{\xi^i}\vb{g}^m \otimes \vb{g}^n - \Gamma \indices{^m_{ik}} A_{mn} \vb{g}^k \otimes \vb{g}^n - \Gamma \indices{^n_{ik}}	A_{mn}\vb{g}^m \otimes \vb{g}^k = \\
				 &= \pdv{A_{mn}}{\xi^i}\vb{g}^m \otimes \vb{g}^n - \Gamma \indices{^l_{im}}A_{ln} \vb{g}^m \otimes \vb{g}^n - \Gamma \indices{^l_{in}}A_{ml} \vb{g}^m \otimes \vb{g}^n = \\
				 &= \qty(\pdv{A_{mn}}{\xi^i} -\Gamma \indices{^l_{im}}A_{ln} - \Gamma \indices{^l_{in}}A_{ml})\vb{g}^m \otimes \vb{g}^n,
\end{align*}
so if we denote
\[
	\nabla_i A_{mn} = \partial_{i}A_{mn} - \Gamma \indices{^l_{im}}A_{ln} - \Gamma \indices{^l_{in}}A_{ml},
\]
we can write
\[
	\partial_{i}\tensorq{A} = \nabla_i A_{mn} \vb{g}^m \otimes \vb{g}^n.
\]

\subsubsection{Direct expression of the Christoffel symbols}
\label{sec:christoffel_expression}
All our formulas depend on the Christoffel symbols. How to compute them? With the above relation, we can express $\nabla_j g_{mn}$:

\[
	\nabla_j g_{mn} = \partial_{j} g_{mn} - \Gamma \indices{^k_{jm}}g_{kn} - \Gamma^{k_{jn}} g_{mk}.
\]


Moreover, we can directly differentiate.
\[
	\pdv{g_{mn}}{\xi^j}=\pdv{\qty(\vb{g}_m \vdot \vb{g}_n)}{\xi^j} = \pdv{\vb{g}_m}{\xi^j} \vdot \vb{g}_n + \vb{g}_m \pdv{\vb{g}_n}{\xi^j}=\Gamma \indices{^k_{mj}}\vb{g}_k \vdot \vb{g}_n + \vb{g}_m \vdot \Gamma \indices{^k_{nj}}\vb{g}_k = \Gamma \indices{^k_{mj}}g_{kn} + \Gamma \indices{^k_{nj}}g_{mk},
\]
but realize that if we rearrange, we exactly obtain the covariant derivative $\nabla_j g_{mn}$ on the LHS. We have shown a remarkable identity

\begin{equation}
    \nabla_j g_{mn} = 0,
\end{equation}
which also implies 
\begin{equation}
  \partial_{i} \tensorq{g} = \tensorq{0}.
\end{equation}

This property is particularly useful, as it allows us to express the Christoffel symbols. Using cyclic permuation, we can write

\begin{align*}
A=	\pdv{g_{mn}}{\xi^j} &= \Gamma \indices{^k_{mj}}g_{kn} + \Gamma \indices{^k_{nj}}g_{mk}, \\
B=	\pdv{g_{jm}}{\xi^j} &= \Gamma \indices{^k_{jn}} g_{kn} + \Gamma \indices{^k_{mn}}g_{jk}, \\
C=	\pdv{g_{nj}}{\xi^n} &= \Gamma \indices{^k_{nm}}g_{kj} + \Gamma \indices{^k_{jm}} g_{nk}.
\end{align*}
Taking $A-B-C$ yields
\[
	\pdv{g_{mn}}{y^j}-\pdv{g_{jm}}{y^n}-\pdv{g_{nj}}{\xi^m} = -2 \Gamma \indices{^k_{nm}}g_{jk},
\]
multiplying by $g^{jl}$ gives
\[
	-2 \Gamma \indices{^k_{nm}}\delta^l_k = g^{jl}\qty(\pdv{g_{mn}}{\xi^j}-\pdv{g_{jm}}{\xi^n}-\pdv{g_{nj}}{\xi^m}),
\]
from which it follows
\begin{equation}
	\label{eq:christoffel}
	\Gamma \indices{^l_{nj}} = \frac{1}{2}g^{lm}\qty(\pdv{g_{mn}}{\xi^j}+\pdv{g_{jm}}{\xi^n}-\pdv{g_{nj}}{\xi^m}).
\end{equation}


\subsubsection{Interchangability of the derivatives}
\label{sec:interchangability}
In euclidian space:
\[
	\pdv[2]{\vb{v}}{x^j}{x^i} = \pdv[2]{\qty(v^k \vb{e}_k)}{x^j}{x^i}= \qty(\pdv[2]{v^k}{x^j}{x^i}) \vb{e}_k = \qty(\pdv[2]{v^k}{x^i}{x^j})\vb{e}_k = \pdv[2]{\vb{v}}{x^i}{x^j},
\]
where $\vb{e}_k$ are the \textit{cartesian basis vectors}. Will it hold even in curvilinear coordinate systems?
\begin{align*}
	\pdv[2]{\vb{v}}{\xi^j}{\xi^k} - \pdv[2]{\vb{v}}{\xi^j}{\xi^k} &= \partial_{j} \partial_{k}\vb{v} - \partial_{k} \partial_{j} \vb{v}=  \partial_{j} \qty(\nabla_k v^i)\vb{g}_i - \partial_{k}\qty(\nabla_j v^i)\vb{g}_i = \qty(\nabla_j \nabla_k v^i) \vb{g}_i - \qty(\nabla_k \nabla_j v^i)\vb{g}_i =\\
								    &= \qty(\pdv{\Gamma \indices{^i_{jm}}}{\xi^k} - \pdv{\Gamma \indices{^i_{km}}}{\xi^i} + \Gamma \indices{^i_{lk}} \Gamma \indices{^l_{jm}}- \Gamma \indices{^i_{lj}} \Gamma \indices{^l_{km}})v^m \vb{g}_i.
\end{align*}

We have skipped the calculation, although it it not trivial whatsoever. Let us just state the quantity $\nabla_k v^i$ are coordinates of a tensor field of type $(1,1),$ and the covariant derivative of such coordinates would be
\[
	\nabla_j A \indices{_k^i} = \partial_{j}A \indices{_k^i} - \Gamma \indices{^l_{jk}} A \indices{_l^i} + \Gamma \indices{^i_{jl}} A \indices{_k^l},
\]
so one would have to manipulate 
\[
	\nabla_j \nabla_k v^i = \partial_{j} \nabla_k v^i - \Gamma \indices{^l_{jk}} \nabla_l v^i + \Gamma \indices{^i_{jl}} \nabla_k v^l.
\]

\begin{definition}[Riemann curvature tensor]
	The \textit{tensor}\footnote{This truly is a tensor, even though the Christoffel symbols are not.}
    \begin{equation}
R \indices{^i_{jkm}}=\pdv{\Gamma \indices{^i_{jm}}}{y^k} - \pdv{\Gamma \indices{^i_{km}}}{y^i} + \Gamma \indices{^i_{lk}} \Gamma \indices{^l_{jm}}- \Gamma \indices{^i_{lj}} \Gamma \indices{^l_{km}},
    \end{equation}
    is called the \textbf{Riemann curvature tensor}. We have shown that it holds
    \[
	    \nabla_i \nabla_j v^k - \nabla_j \nabla_i v^k = R \indices{^k_{ijl}}v^l.
    \]
\end{definition}
We see that if the Riemann curvature tensor is zero, then effectively we are in the case of a flat euclidian space, as the derivatives commute.
In other words, in flat euclidian space, the Riemann curvature tensor is always zero. If we flip this, we see that if we have a space with zero Riemann curvature tensor, \textit{we have a chance} that the derivatives commute, i.e. that the structure is euclidian.

\begin{tcolorbox}
\begin{example}[Interpretation in continuum mechanics]
	\begin{align*}
		\vb{x} = \vb*{\Theta}\qty(\vb*{\xi}) \, &\text{vs} \, \vb{x}=\vb*{\chi}\qty(\vb{X}), \\
		\vb{g}_i = \pdv{\vb{x}}{\xi^i} \, &\text{vs} \, \vb{g}_I = \pdv{\vb*{\chi}}{X^I}, \, \text{i.e.} \, (\vb{g}_M)^i = F \indices{^i_M} = \pdv{\chi^i}{X^M}, \\
						  &	g_{IJ}=\vb{g}_I \vdot \vb{g}_J = (\transpose{\fgrad} \fgrad)_{IJ}=(\rcg)_{IJ}, \\
	\end{align*}
	So
	\begin{equation}
		\label{eq:cauchy_metric}
		\tensorq{g}=\rcg = \transpose{\fgrad}\fgrad = \transpose{\qty(\grad \vb*{\chi})} \vb*{\chi}.
	\end{equation}

	The way our metric tensor $\tensorq{g}$ is not insignificant, not all positive definite symmetric bilinear forms can be factorized as in our case. In particular, our vectors $\vb{g}_M = \pdv{\vb*{\chi}}{X^M}$ solve the following system of PDE's:
	\[
		\pdv{\vb{g}_M}{X^J} = \Gamma \indices{^K_{MJ}}\vb{g}_K.
	\]
	It turns out \cite{ciarletIntroductionDifferentialGeometry2005} this system has \textit{an unique solution} if and only if
	\[
		\pdv{\vb{g}_m}{X^I}{X^J}  = \pdv{\vb{g}_m}{X^J}{X^I},
	\]
	\textit{i.e.}, when the second partial derivatives commute. But we have shown in the previous that this means
	\begin{equation*}
		\partial_{JI}\vb{g}_M = \partial_{IJ}\vb{g}_M = \pdv{\qty(\Gamma \indices{^K_{IM}} \vb{g}_K)}{X^J} = \pdv{\qty(\Gamma \indices{^K_{MJ}} \vb{g}_K)}{X^I} ,
	\end{equation*}
	this is equivalent to 
	\[
		\partial_{JI}\vb{g}_m - \partial_{IJ} \vb{g}_n =0 \Leftrightarrow \dots R \indices{^I_{JKM}} = 0.
	\]
\end{example}
\textbf{So this implies that all physically admissable deformations produce a deformed space with zero Riemann curvature tensor, \textit{i.e.}, a flat space.}
\end{tcolorbox}

\subsection{Calculus}
\label{sec:calculus}

\subsubsection{Gradient}
\label{sec:gradient}

For a scalar function, we set
\[
	\qty(\grad \varphi)_i = \pdv{\varphi}{x^i},
\]
so the summation convention forces us to write
\[
	\grad \varphi = \pdv{\varphi}{x^i} \vb{e}^i,
\]
meaning the gradient is in fact a covector. This causes a lot of misconseption. In fact, the functions $\pdv{\varphi}{x^i}$ are coordinates of the differential covector
\[
	\textbf{d}\varphi = \pdv{\varphi}{x^i}\vb{e}^i = \pdv{\varphi}{x^i}\textbf{d}{x}^i,
\]
but in physics, it is common to put the coordinates of a \textit{covector} into a \textit{vector}, and so $\grad \varphi$ is born.

Expressed in curvilinear coordinates
\[
	\grad \varphi = \pdv{\varphi}{x^i}\vb{e}^i = \pdv{\varphi}{\xi^j}\underbrace{\pdv{\xi^j}{x^i}\vb{e}^i}_{=\vb{g}^j} = \pdv{\varphi}{\xi^j}\vb{g}^j.
\]
Realize that for a scalar field, it holds\footnote{We can formally suppose $\varphi$ is a vector field with all components being $\varphi$, then $\nabla_j \varphi = \partial_{j} \varphi + \Gamma \indices{^k_{jl}}\varphi^l$, but $\Gamma \indices{^k_{jl}} \varphi^l = \varphi \Gamma \indices{^k_{jk}}=0.$} $\partial_{j} = \nabla_j,$ so in fact
\[
	\grad \varphi = \nabla_j f \vb{g}^j.
\]

What about the gradient of a vector field? In the cartesian coordinate system:
\[
	\grad \vb{v} = \grad(v^i \vb{e}_i) = \pdv{v^i}{x^j}\vb{e}_i \otimes \vb{e}^j,
\]
and in curvilinear coordinates:
\begin{align*}
	\grad(v^i \vb{g}_i)& = \grad \qty(v^i \pdv{\Theta^m}{\xi^i}\vb{e}_m) = \pdv{\Theta^j}\qty(v^i \pdv{\Theta^m}{\xi^i})\vb{e}_m \otimes \vb{e}^j = \qty(\pdv{v^i}{\Theta^j} \pdv{\Theta^m}{\xi^i}+v^i \pdv[2]{\Theta^m}{\Theta^j}{\xi^i})\vb{e}_m \otimes \vb{e}^j = \\
			   & = \qty(\pdv{v^i}{\xi^n}\pdv{\xi^n}{\Theta^j}\pdv{\Theta^m}{\xi^i}+v^i \pdv[2]{\Theta^m}{\Theta^j}{\xi^i})\vb{e}_m \otimes \vb{e}^j = \pdv{v^i}{\xi^n} \qty(\pdv{\Theta^m}{\xi^u}\vb{e}_m)\otimes \qty(\pdv{\xi^n}{\Theta^j}\vb{e}^j)+v^i \pdv{\xi^l}\qty(\pdv{\Theta^m}{\xi^i})\pdv{\xi^l}{\Theta^j}\vb{e}_m \otimes \vb{e}^j = \\
			   & = \pdv{v^i}{\xi^n} \qty(\pdv{\Theta^m}{\xi^i}\vb{e}_m)\otimes \qty(\pdv{\xi^n}{\Theta^j}\vb{e}^j)+ v^i \pdv{\xi^l}\qty(\pdv{\Theta^m}{\xi^i}\vb{e}_m)\otimes \qty(\pdv{\xi^l}{\Theta^j}\vb{e}^j) = \pdv{v^i}{\xi^n}\vb{g}_i \otimes \vb{g}^n+ v^i \pdv{\vb{g}^i}{\xi^l} \otimes \vb{g}^l = \\
			   &= \pdv{v^i}{\xi^n}\vb{g}_i \otimes \vb{g}^n + v^i \Gamma \indices{^s_{il}}\vb{g}_s \otimes \vb{g}^l = \qty(\pdv{v^s}{\xi^l}+\Gamma \indices{^s_{il}}v^i)\vb{g}_s \otimes \vb{g}^l = \\
			    & = \nabla_l v^s \vb{g}_s \otimes \vb{g}^l,
\end{align*}
and so
\[
	\grad \vb{v} = \nabla_i v^j \vb{g}_j \otimes \vb{g}^i.
\]

Until now, we have not discussed the fact $|\vb{g}_i| \neq 1,$ which is a kind of a problem. Let us define
\[
	\vb{v}=v^i \vb{g}_i = v^i |\vb{g}_i| \frac{\vb{g}_i}{|\vb{g}_i|} = v^{\hat{i}} \vb{g}_{\hat{i}},
\]
where we have defined
\[
	v^{\hat{i}} = |\vb{g}^i| v^i, \vb{g}_{\hat{i}} = \frac{\vb{g}_i}{|\vb{g}_i|}.
\]
But! the differential formulas work for $v^i, \vb{g}_i$, \textbf{not for} $v^{\hat{i}}, \vb{g}_{\hat{i}}$! We have to be careful
\begin{align}
	\grad \varphi &= \nabla_j \varphi \vb{g}^j = |\vb{g}^i|\nabla_j \varphi \vb{g}^{\hat{i}},\\
	\grad \vb{v} &= \nabla_i v^j \vb{g}_j \otimes \vb{g}^i = \abs{\vb{g}_j} \abs{\vb{g}^i} \nabla_i v^j \vb{g}_{\hat{j}} \otimes \vb{g}^{\hat{i}}.
\end{align}

For the divergence of a vector field, we know: $\divergence{\vb{v}} = \trace\qty(\grad \vb{v}),$ so 
\[
	\divergence{\vb{v}} = \tr\qty(\grad \vb{v}) = \tr\qty(\nabla_i v^j \vb{g}_j \otimes \vb{g}^i) = \nabla_j v^j.
\]
The divergence of a tensor field a can be tricky, but be guided by the summation convention; for the tensor of type $(2,0)$ (a bivector)
\[
	\tensorq{A} = A^{ij}\vb{g}_i \otimes \vb{g}_j
\]
we can define
\[
	\divergence{\tensorq{A}} = \nabla_j A^{ij} \vb{g}_i.
\]
\textit{For the tensors of a different type, we need to change the position of the indices to obtain a bivector.} Notice that the result is a vector, not a bivector.

\subsubsection{Laplace-Beltrami operator}
\label{sec:laplace_beltrami}

\[
	\laplace \varphi = \frac{1}{\sqrt{\det \tensorq{g}}}\pdv{\xi^i}\qty(\sqrt{\det \tensorq{g}} g^{ij} \pdv{\varphi}{\xi^j}),
\]

on one hand:
\begin{equation*}
	\laplace \varphi = \divergence{\grad \varphi}= \nabla_i \qty(\grad \varphi)^i = \nabla_i \qty(g^{ij}\pdv{\varphi}{\xi^j}) = \nabla_i g^{ij}\partial_{j}\varphi,
\end{equation*}
where we have rised the index
\[
	\qty(\grad \varphi)^i = g^{ij}\qty(\grad \varphi)_j = g^{ij} \pdv{\varphi}{\xi^j},
\]
so using the covariant derivative definition
\[
	\divergence{\grad \varphi} = \pdv{\xi^i}\qty(g^{ij}\pdv{\varphi}{\xi^j})+ \Gamma \indices{^i_{il}} g^{lj} \pdv{\varphi}{\xi^j},
\]
on the other

\begin{align*}
\laplace \varphi &= \frac{1}{\sqrt{\det \tensorq{g}}}\qty(\pdv{\xi^i}\qty(\sqrt{\det \tensorq{g}})g^{ij}\pdv{\varphi}{\xi^j}+ \sqrt{\det \tensorq{g}} \pdv{g^{ij}}{\xi^i}\pdv{\varphi}{\xi^j}+ \sqrt{\det \tensorq{g}} g^{ij} \pdv[2]{\varphi}{\xi^i}{\xi^j}) = \\
			 &= \frac{1}{\sqrt{\det \tensorq{g}}} \qty(\frac{1}{2\sqrt{\det \tensorq{g}}}\pdv{\xi^i} \qty(\det \tensorq{g})g^{ij}\pdv{\varphi}{\xi^j}+\sqrt{\det \tensorq{g}}\pdv{g^{ij}}{\xi^i}+ \sqrt{\det \tensorq{g}} g^{ij}\pdv[2]{\varphi}{\xi^i}{\xi^j}) = \\
			 & = \frac{1}{\sqrt{\det \tensorq{g}}}\qty(\frac{1}{2}\tr\qty(\inverse{\tensorq{g}} \pdv{\tensorq{g}}{\xi^i})g^{ij}\pdv{\varphi}{\xi^j}-\sqrt{\det \tensorq{g}} \qty(\Gamma \indices{^j_{kn}}g^{in}- \Gamma \indices{^i_{km}}g^{mj})+ \sqrt{\det \tensorq{g}}g^{ij}\pdv[2]{\varphi}{\xi^i}{\xi^j}) = \\
			 & = \frac{1}{\sqrt{\det \tensorq{g}}}\qty(\frac{1}{2}\qty(g^{mn} \pdv{g_{mn}}{\xi^i})g^{ij}\pdv{\varphi}{\xi^j}-)
\end{align*}

\subsubsection{Bipolar coordinates}
\label{sec:bipolar_coordinates}

Define $\vb*{\xi} = [\alpha,\beta],$ where
\[
	\alpha+i \beta = \log \frac{y+i(x+a)}{y+i(x-a)}.
\]
This can be inversed and write
\begin{align*}
	x&=\frac{a \sinh \alpha}{\cosh \alpha - \cos \beta},\\
	y&= \frac{a \sin \beta}{\cosh \alpha -\cos \beta},
\end{align*}
moreover,

\begin{align*}
	(x-a \coth \alpha)^{2}+y^{2} &= \frac{a^{2}}{\sinh^{2}\alpha}, \\
	x^{2}+(y-a \cot \beta)^{2} &= \frac{a^2}{\sin^{2}\beta}.
\end{align*}

Calculate \textit{everything} for this coordinate system.

In general $\vb{g}_i=\pdv{x^j}{\xi^i}\vb{e}_j,$ so in our case
\begin{align*}
	\vb{g}_{\alpha}&=\pdv{x}{\alpha}\vb{e}_x+\pdv{y}{\alpha}\vb{e}_y\\
		       &=\qty(\frac{(a \cosh \alpha)(\cosh \alpha-\cos \beta)-a \sinh \alpha\sinh \alpha}{\qty(\cosh \alpha - \cos \beta)^{2}})\vb{e}_x+\qty(\frac{a \cos \beta(\cosh \alpha-\cos \beta)-a \sin \beta \sinh \alpha}{\qty(\cosh \alpha - \cos \beta)^{2}})\vb{e}_y = \\
		       & = \frac{a}{\qty(\cosh \alpha - \cos \beta)^{2}}\qty((1-\cosh \alpha \cos \beta)\vb{e}_x-(\sin \beta \sinh \alpha) \vb{e}_y),\\
	\vb{g}_{\beta}& = \pdv{x}{\beta}\vb{e}_x+\pdv{y}{\beta}\vb{e}_y\\
		      &= \dots =\\
		      &= \frac{a}{\qty(\cosh \alpha-\cos \beta)^{2}}\qty(-\qty(\sin \beta \sinh \alpha)\vb{e}_x + \qty(-1+\cosh \alpha \cos \beta)\vb{e}_y).
\end{align*}
We can see that $\vb{g}_{\alpha} \vdot \vb{g}_{\beta}=0$ and so
\[
	\tensorq{g}=\qty(\frac{a}{\cosh \alpha - \cos \beta})^{2}\identity, \inverse{\tensorq{g}} = \qty(\frac{\cosh \alpha-\cos \beta}{a})^{2} \identity.
\]

Coming back to Laplace-Beltrami operator, we can calculate
\[
	\qty(\sqrt{\det \tensorq{g}}\inverse{\tensorq{g}})=\qty(\frac{a}{\cosh \alpha-\cos \beta})^{2} \qty(\frac{\cosh \alpha-\cos \beta}{\alpha})^{2} \identity = \dots = \identity,
\]
and calculating a bit more yields
\[
	\laplace \varphi \to \qty(\frac{\cosh \alpha-\cos \beta}{a})^{2} \laplace_{\alpha \beta}\varphi.
\]
\begin{remark}[Relation to complex analysis]
    This can be seen as a conformal transformation
    \[
	    \gamma = f(z),
    \]
    where
    \begin{align*}
	    \gamma &= \alpha + i \beta, \\
	    z &= x+iy,
    \end{align*}

Let us write
\[
	f(z) = f^x(x,y) +i f^y(x,y) \Leftrightarrow \vb{f}(\vb{x})=[f^x\qty(\vb{x}),f^y\qty(\vb{x})], z = x+iy,
\]
and compute
\[
	\pdv{\vb{f}}{\vb{x}} = \begin{bmatrix}
		\pdv{f^x}{x} & \pdv{f^x}{y}\\
		\pdv{f^y}{x} & \pdv{f^y}{y}
	\end{bmatrix}.
\]
Recall Cauchy-Riemmann conditions:
\begin{align*}
	\pdv{f^x}{x}&=\pdv{f^y}{y}, \\
	\pdv{f^x}{y}&=-\pdv{f^y}{x},
\end{align*}
using which the gradient becomes:
\[
	\pdv{\vb{f}}{\vb{x}} = \begin{bmatrix}
		\pdv{f^x}{x} & \pdv{f^x}{y}\\
		-\pdv{f^x}{y} & \pdv{f^y}{y}
	\end{bmatrix},
\]
which is an \textbf{orthogonal matrix:}
\[
	\qty(\pdv{\vb{f}}{\vb{x}}) \transpose{\qty(\pdv{\vb{f}}{\vb{x}})} = \qty(\qty(\pdv{f^x}{x})^{2}+\qty(\pdv{f^y}{y})^{2})\identity.
\]
Realize that all this structure comes just from the fact that the transformation is given through a holomorfic function.
\end{remark}

\subsubsection{Compatibility conditions in linearised elasticity}
\label{sec:compt_cond}

\[
	R \indices{^i_{jkm}}= \pdv{\Gamma \indices{^i_{jm}}}{\xi^k}-\pdv{\Gamma \indices{^i_{km}}}{\xi^j}+\Gamma \indices{^i_{lk}}\Gamma \indices{^l_{jm}}-\Gamma \indices{^i_{lj}}\Gamma \indices{^l_{km}},
\]
and we know
\[
	R \indices{^i_{jkm}} = 0 \Leftrightarrow \rcg = \transpose{\fgrad} \fgrad, \fgrad = \pdv{\vb*{\chi}}{\vb{X}}.
\]
All this works in fully \textit{nonlinear setting!}. In the classical lecture, we have been able to obtain compatibility condition in \textit{linearised elasticity}: $\curl{\bbespilon} = \tensorq{0}, \bbespilon=\frac{1}{2}\qty(\grad \vb{u}+\transpose{\qty(\grad \vb{u})}).$

Consider the following setting:
\begin{align*}
	\vb{x}&=\vb*{\chi}\qty(\vb{X}),\\
	\vb{u}&= \vb{\chi}\qty(\vb{X})-\vb{X},\\
	\grad \vb{u} &= \fgrad - \identity, \\
	\fgrad &= \identity + \grad \vb{u},\\
\end{align*}
then
\[
	\rcg = \transpose{\fgrad}\fgrad=\qty(\identity+\transpose{\qty(\grad \vb{u})})\qty(\identity+\grad \vb{u}) = \identity + 2 \bbespilon + \, \text{h.o.t.} \,,
\]
and so
\[
	\inverse{\tensorq{g}} = \identity - 2 \bbespilon.
\]

The Christoffel symbols are

\begin{align*}
	\Gamma \indices{^l_{nj}}&= \frac{1}{2}g^{lm}\qty(\pdv{g_{mn}}{X^j}+\pdv{g_{jm}}{X^n}-\pdv{g_{nj}}{X^m}) \\
				&\approx \frac{1}{2}\qty(\identity-2 \bbespilon)^{lm}\qty(\pdv{X^j}\qty(\identity+2 \bbespilon)_{mn}+\pdv{X^n}\qty(\identity+2 \bbespilon)_{jm}-\pdv{X^m}\qty(\identity+2 \bbespilon)_{nj}), \\
				&\approx \delta^{lm}\qty(\pdv{\varepsilon_{mn}}{X^j}+\pdv{\varepsilon_{jm}}{X^n}-\pdv{\varepsilon_{nj}}{X^m}) = \pdv{\varepsilon_{n}^l}{X^j}+\pdv{\varepsilon_{j}^l}{X^n}-\pdv{\varepsilon_{nj}}{X^m},
\end{align*}

the Riemann curvature tensor is (linear approximation)

\begin{align*}
0 = R \indices{^i_{jkm}}& \approx \pdv{\Gamma \indices{^i_{jm}}}{X^k}-\pdv{\Gamma \indices{^i_{km}}}{X^j}\\
			& = \pdv{X^k}\qty(\pdv{\varepsilon^i_j}{X^m}+\pdv{\varepsilon^i_m}{X^j}-\pdv{\varepsilon_{mj}}{X^i})-\pdv{X^j}\qty(\pdv{\varepsilon^i_k}{X^m}+\pdv{\varepsilon^i_m}{X^k}-\pdv{\varepsilon_{km}}{X^i}) =\\
			 =& \pdv[2]{\varepsilon_{ij}}{X^k}{X^m}-\pdv[2]{\varepsilon_{mj}}{X^k}{X^i}-\pdv[2]{\varepsilon_{ik}}{X^j}{X^m}+\pdv[2]{\varepsilon_{km}}{X^j}{X^i},
\end{align*}
so the compatibility conditions are
\[
	\pdv[2]{\varepsilon_{ij}}{X^k}{X^m}-\pdv[2]{\varepsilon_{mj}}{X^k}{X^i}-\pdv[2]{\varepsilon_{ik}}{X^j}{X^m}+\pdv[2]{\varepsilon_{km}}{X^j}{X^i} = 0.
\]

\subsection{Surface geometry}
\label{sec:surface_geometry}

In this part, we will work with surfaces embedded in $\R^3$.

Let $G = \{\vb{u}\} \subset \R^2 $ be the parametrization space and $\vb*{\Phi}:G \subset \R^2 \to \R^3$ is the parametrization, so the points of the surface are
\[
	\vb{x}=\vb*{\Phi}\qty(\vb{u}), \vb{x} \in \R^3.
\]

\begin{definition}
	The indices $i,j,k, \dots \in \{1,2,3\}$ will denote objects from $\R^3$ and indices $\alpha, \beta, \gamma, \dots \in \{1,2\}$ will denote indices of objects from $\R^2$.
\end{definition}

\subsubsection{Tangent and normal vectors}
\label{sec:tangent_normal_vectors}
As in the previous story, we can define (basis) tangent vectors:
\[
	\vb{t}_1 = \pdv{\vb*{\Phi}}{u^1} = \partial_{1} \vb*{\Phi}, \vb{t}_2 = \pdv{\vb*{\Phi}}{u^2} = \partial_{2} \vb*{\Phi},
\]
and on surfaces, of importance is also the normal vector
\[
	\vb{n}=\frac{\vb{t}_1 \cross \vb{t}_2}{\abs{\vb{t}_1 \cross \vb{t}_2}}.
\]

\subsubsection{Distances and angles}
\label{sec:distances_angles}

The metric tensor on the surface is given by
\[
	\tensorq{g}_s = \begin{bmatrix} \vb{t}_1 \vdot \vb{t}_1 & \vb{t}_1 \vdot \vb{t}_2 \\ \vb{t}_1 \vdot \vb{t}_2 & \vb{t}_2 \vdot \vb{t}_2 \end{bmatrix},
\]
or in context of differential geometry, this object is called \textbf{the first fundamental form}.

\subsubsection{Derivatives}
\label{sec:derivatives_on_surfaces}
In $\R^3$, we know how to differentiate tangent vectors (using Christoffel symbols). Can this be helpful to us?
Looking at our surface from $\R^{3},$ we see the (local) orthogonal basis on it is formed by $\vb{t}_1, \vb{t}_2, \vb{n},$ so thhe metric tensor in $\R^3$ is given by 

\[
	\tensorq{g} = 
	\begin{bmatrix}
		\tensorq{g}_s & 0 \\
		0 & 1 
	\end{bmatrix}.
\]
In $\R^{3},$ we have no problem writing the derivatives of the tangent vectors
\[
	\partial_{\alpha} \vb{t}_{\beta} = \Gamma \indices{^\gamma_{\alpha \beta}} \vb{t}_{\gamma}+b_{\alpha \beta} \vb{n},
\]
where we have just denoted $b_{\alpha \beta} = \Gamma \indices{^3 _{\alpha \beta}}.$

What about the derivative of the normal vector? From the length of $\vb{n}$ we know
\[
	\vb{n}\vdot \vb{n} = 1 \Rightarrow \partial_{\alpha} \vb{n}\vdot \vb{n} = 0,
\]
so that means the derivative is perpendicular to the normal direction, so
\[
	\partial_{\alpha}\vb{n}= A \indices{^\gamma_{\alpha}} \vb{t}_{\gamma}.
\]

Next trick is to realize
\[
	0 = \partial_{\alpha}\qty(\vb{n}\vdot \vb{t}_{\beta}) = \partial_{\alpha}\vb{n} \vdot \vb{t}_{\beta}+ \vb{n}\vdot \partial_{\alpha}\vb{t}_{\beta}= A \indices{^{\gamma}_{\alpha}}\vb{t}_{\gamma} \vdot \vb{t}_{\beta}+ \vb{n}\vdot \qty(\Gamma \indices{^{\delta}_{\alpha \beta}} \vb{t}_{\delta}+b_{\alpha \beta} \vb{n}) = A \indices{^\gamma _{\alpha}} g_{\gamma \beta}+ b_{\alpha \beta},
\]
from which it follows

\[
	A \indices{^{\gamma}_{\alpha}} = -g \indices{^{\gamma \beta}}b_{\beta \alpha}.
\]
(those are in fact the components of $\tensorq{g}_s$.)

\subsubsection{Commutation of derivatives}
\label{sec:com_der}

What are the \textit{implications} of
\[
	\partial_{\beta \gamma}\vb{t}_{\alpha} = \partial_{\gamma \beta}\vb{t}_{\alpha}?
\]
Write
\[
	0 =\partial_{\beta \gamma}\vb{t}_{\alpha} - \partial_{\gamma \beta}\vb{t}_{\alpha} = (\, \text{something} \,) \vb{t}_{\delta}+(\, \text{something different} \,) \vb{n},
\]
so we see the whole thing splits into two parts, that are perpendicular to each other; this must however mean that both of them are zero, which can be shown to be equivalent to 
\begin{theorem}[Gauss relation]
	\[
		R \indices{_{\gamma \beta \delta \alpha}} = b_{\alpha \beta}b_{\gamma \delta} - b_{\alpha \delta} b_{\gamma \beta}.
	\]
\end{theorem}

\begin{theorem}[Codazzi-Mainardi relation]
	\[
		\nabla_{\alpha}b_{\beta \gamma}- \nabla_{\gamma}b_{\beta \alpha}=0
	\]
\end{theorem}

\subsubsection{Surfaces evolving in time}
\label{sec:time_evolving_surfaces}

Now the points of the surface are given by
\[
	\vb{x}=\vb*{\Phi}\qty(t,\vb{u}), \, \text{where} \,\vb*{\Phi}: \R \times G \to \R^3.
\]

We can define the \textbf{velocity of the surface}:
\begin{equation*}
	\vb{v}_s = \pdv{\vb*{\Phi}}{t}\qty(t,\vb{u}) = \partial_t \vb*{\Phi}\qty(t, \vb{u}).
\end{equation*}

The basis of everything has always been Gauss theorem; we will be interested in the quantity of the type

\[
	\dv{t} \int_{S(t)}\psi\qty(t,\vb{x})\dd{S},
\]
where $S(t)$ is a time-dependent surface. Let us try the approach from Reynolds:
\[
\dv{t} \int_{S(t)}\psi\qty(t,\vb{x})\dd{S} = \dv{t} \int_{\inverse{\vb*{\Phi}}\qty(t,\vb{x})}\psi(t,\vb*{\Phi}(t,\vb{u})) \sqrt{\det \tensorq{g}_s}\dd{u^1} \dd{u^2} = ,
\]
which is now a time-indepdent integral, meaning we can differentiate through. Let us first calculate the derivatives. Start slow:

\begin{align*}
	\dv{\vb{t}_{\alpha}}{t} &= \pdv{t}\qty(\pdv{\vb{\Phi}}{u^\alpha}\qty(t,\vb{u})) = \pdv{u^{\alpha}}\underbrace{\qty(\pdv{\vb{\Phi}}{t}\qty(t,\vb{u}))}_{= \vb{v}_s(t,\vb{u})} = \pdv{u^{\alpha}}\qty(\vb{v}_{\parallel}+v_{\perp}\vb{n}) = \\
				&= \pdv{\vb{v}_{\parallel}}{u^{\alpha}}+\pdv{\qty(v_{\perp} \vb{n})}{u^{\alpha}} = \pdv{\qty(v^{\beta}_{\parallel}\vb{t}_{\beta})}{u^{\alpha}}+\pdv{v_{\perp}}{u^{\alpha}}\vb{n}+ v_{\perp} \pdv{\vb{n}}{u^{\alpha}} = \\
				&=\pdv{v_{\perp}^{\beta}}{u^{\alpha}}\vb{t}_{\beta}+\pdv{\vb{t}_{\beta}}{u^{\alpha}}v^{\beta}_{\parallel}+\pdv{v_{per}}{u^{\alpha}}\vb{n} - v_{\perp} g^{\gamma \beta}b_{\beta \alpha}\vb{t}_{\gamma} = \\
				& = \pdv{v^{\beta}_{\parallel}}{u^{\alpha}}\vb{t}_{\beta}+v^{\beta}_{\parallel}\Gamma \indices{^{\gamma}_{\alpha \beta}} \vb{t}_{\gamma}+ v_{\parallel}^{\beta}b_{\alpha \beta} \vb{n} + \pdv{v_{\perp}}{u^{\alpha}}\vb{n}- v_{\perp}g^{\gamma \beta}b_{\alpha \beta}\vb{t}_{\gamma} =\\
				& = v^{\beta}_{\parallel}|_{\alpha} \vb{t}_{\beta}- v_{\perp}g^{\gamma \beta} b_{\alpha \beta}\vb{t}_{\gamma}+\qty(v^{\beta}_{\parallel}b_{\alpha \beta}+ \pdv{v_{\perp}}{u^{\alpha}})\vb{n} = \\
				& = \qty(v_{\parallel}^{\beta}|_{\alpha}-v_{\perp}g^{\beta \gamma} b_{\alpha \gamma})\vb{t}_{\beta}+ \qty(v_{\parallel}^{\beta}b_{\alpha \beta} + \pdv{v_{\perp}}{u^{\alpha}})\vb{n}.
\end{align*}
So all in all
\[
	\partial_t\vb{t}_{\alpha}= \qty(\nabla_{\alpha}v_{\parallel}^{\beta}-v_{\perp}g^{\beta \gamma} b_{\alpha \gamma})\vb{t}_{\beta}+ \qty(v_{\parallel}^{\beta}b_{\alpha \beta} + \partial_{\alpha}v_{\perp})\vb{n}.
\]

Next ingredient is the quantity $\dv{t} \tensorq{g}_s$, so in components:
\[
	\partial_t g_{\alpha \beta} = \partial_t\qty(\vb{t}_{\alpha} \vdot \vb{g}_{\beta}) = \dots =\nabla_{\alpha} v_{\parallel}^\delta g_{\delta \beta}+ \nabla_{\beta}v^{\delta}_{\parallel} g_{\delta \alpha}-2v_{\perp}b_{\alpha \beta}.
\]
After some further manipulation, the final formula becomes

\begin{equation}
	\dv{t} \int_{S(t)}\psi(t,\vb{x})\dd{S} = \int_{S(t)}\partial_t \psi\qty(t,\vb{x})+\psi(t,\vb{x})\qty(\divergence{\vb{v}_{\parallel S}} - 2 v_{\perp}\qty(t,\vb{x})K\qty(t,\vb{x}))\dd{S},
\end{equation}
where
\[
	\qty(\divergence{\vb{v}_{\parallel S}})\qty(t,\vb{x}) - 2 v_{\perp}K \coloneq \eval{\nabla_{\beta}v^{\beta}(t,\vb{u})_{\parallel}-2v_{\perp}(t,\vb{u})K(t,\vb{u})}_{\vb{u}=\inverse{\vb*{\Phi}}\qty(t,\vb{x})},
\]

and

\[
	K = \frac{1}{2}g^{\beta \alpha} b_{\alpha \beta}
\]
is the mean curvature.

\end{document}
