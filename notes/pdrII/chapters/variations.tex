% !TEX root = ../main.tex

\section{Calculus of variations}
\label{sec:calculus_of_variations}

Our motivation is the following: search for a point of minimum for a mapping $I:X \mapsto \R$ 
\[
	I(u) = \int_{\Omega}F\qty(x, u, \grad u)\dd{x},
\]
with some basic assumptions:
\begin{itemize}
	\item $\Omega \in \Ckl{0}{1},$
	\item $r \in \qty(1, \infty),$
	\item $X = u_0 + \WkpzeroSet[1][r]{\Omega},$ for some $u_0 \in \WkpSet[1][r]{\Omega},$
	\item $F: \Omega \times \R \times \R^{d} \to \R$ that is Caratheodory in $x$ and $(z,\vb{p})$ with the following coercivity condition:
		\[
			\exists C_1 > 0, C_2 \in \LpSet[1]{\Omega}, \, \text{a.e.} \, x \in \Omega, \forall z \in \R, \forall p \in \R^d : F\qty(x,z,p)\geq C_1|p|^r - C_2(x).
		\]
\end{itemize}

\begin{remark}
	\begin{itemize}
		\item from the assumptions it follows $\int_{\Omega}F\qty(\vdot, u, \grad u)\dd{x}$ is defined $\forall u \in \WkpSet[1][r]{\Omega},$
		\item notice the setting is very similiar to the nonlinear equations previously,
		\item the fact $u \in u_0 + \WkpzeroSet[1][r]{\Omega}, u_0 \in \WkpSet[1][r]{\Omega},$ is nothing new - even it the previous chapter we had $u_0 \in \WkpSet[1][r]{\Omega}, u \in \WkpSet[1][r]{\Omega}\, \text{\textit{s.t.}} \, u-u_0 \in \WkpzeroSet[1][r]{\Omega}.$ Also, since we know $\WkpzeroSet[1][r]{\Omega}$ is a closed subspace of $\WkpSet[1][r]{\Omega},$ the structure $u_0 + \WkpzeroSet[1][r]{\Omega}$ is exactly that of a factorspace.

	\end{itemize}
\end{remark}

\subsection{Euler-Lagrange equations}
\label{sec:euler_lagrange}
The connection between PDE's and calculus of variations is presented in the following lemma.

\begin{lemma}
	Let $\Omega \in C^{0,1}, r \in(1,\infty), X = u_0 + \WkpzeroSet[1][r]{\Omega}, u_0 \in \WkpSet[1][r]{\Omega}, F$ Caratheodory. Moreover, let the following condition hold: $\exists C>0, h \in \LpSet[1]{\Omega}$ such that 
	\[
		 \forall \, \text{a.a} \, x \in \Omega, \forall z \in \R, \forall p \in \R^d: |\grad_p F(x,z,p)|+|\partial_{z} F(x,z,p)| \leq h(x) + C\qty(|z|^r + |p|^r),
	\]
	and let moreover $F(x,\vdot, \vdot) \in C^1\qty(\R^{d+1})$ for \textit{a.a.} $x \in \Omega.$

	Let now $u \in X$ be a local minimizer of $I$ over $X$, i.e., $I(u) < \infty$ and
	\[
		\exists \rho>0: \forall v \in \mathcal{D}\qty(\Omega), \norm{v}_{\WkpSet[1][r]{\Omega}} < \rho \Rightarrow I(u) \leq I(v).
	\]
	Then $u$ is the weak solution to the \textbf{Euler-Lagrange equations}:
		\begin{align*}
			- \divergence{\qty(\grad_p F\qty(x, u, \grad u)+ \partial_{z}F\qty(x, u, \grad u))} &= 0, \, \text{in} \, \Omega \\
			u &= u_0, \, \text{on} \, \partial \Omega,
		\end{align*}
	i.e.,
	\begin{align*}
		\forall \varphi \in \mathcal{D}\qty(\Omega): \int_{\Omega}\grad_p F\qty(x, u, \grad u) \vdot \grad \varphi + \partial_{z}F\qty(x, u, \grad u) \varphi \dd{x} = 0,\\
	\end{align*} 
	and $\trace u = \trace u_0$ on $\partial \Omega.$
\end{lemma}
\begin{proof}(\textit{From: the lectures})
	First, if $u \in u_0 + \WkpzeroSet[1][r]{\Omega},$ then $\tr u = \tr u_0$ and that is true. Now fix some $\varphi \in \mathcal{D}\qty(\Omega)$ and define
	\[
		\iota: \R \to \R^{*}: \iota\qty(t) = \int_{\Omega}F\qty(x, u + t \varphi, \grad\qty(u+ t \varphi))\dd{x}.
	\]
	Since now $u$ minimizes $I,$ we see $\iota$ has a (local) minimum at $0.$ We will show $\iota'(0)$ exists and is equal to the Euler-Lagrange equations. 
	Denote now
	\[
		l(\vdot, t) = F(\vdot, u + t \varphi, \grad \qty(u + t \varphi),
	\]
	for $t \in \R$. To do that, we would like to swap the derivative and the integral, namely to write
	\[
		\partial_t \iota(t)  = \partial_t \int_{\Omega}l\qty(\vdot, t)\dd{x} = \int_{\Omega}\partial_t l\qty(\vdot, t)\dd{x}.
	\]
	To do that, we have to check the assumptions: 
	\begin{itemize}
		\item measurability of $l(\vdot,t),$
		\item differentiability of $l\qty(\vdot,t),$
		\item majorant for the derivative,
		\item convergence of the integral on some neighbourhood of $t$.
	\end{itemize}

	Recall that from our assumptions ($F$ is Caratheodory + smoothness)
	\begin{itemize}
		\item $F\qty(\vdot, z, \vb{p})$ is continuous $\forall (z, \vb{p}) \in \R \times \R^{d} \Rightarrow l(\vdot, t)$ is measurable. so in particular it is measurable on some neighbourhood of $t = 0,$ 
		\item $F(x, \vdot, \vdot) \in \CkSet{1}{\R^{d+1}} \Rightarrow l\qty(\vdot, t)$ is differentable w.r.t $t$.
	\end{itemize}
	It is hence valid to now compute $\partial_{t} l\qty(\vdot, t)$  
	\begin{align*}
		\partial_t l(\vdot, t) &= \partial_{z}F\qty(\vdot, u+t \varphi, \grad \qty(u+t \varphi))\varphi + \grad_p F\qty(\vdot, u+t \varphi, \grad\qty(u+t \varphi))\vdot \grad \varphi.
	\end{align*}
	We have a nice majorant for this expression straight from the assumptions:
	\begin{align*}
		\abs{\partial_t l(\vdot, t)} &=\abs{\partial_{z}F\qty(\vdot, u+t \varphi, \grad \qty(u+t \varphi))\varphi + \grad_p F\qty(\vdot, u+t \varphi, \grad\qty(u+t \varphi))\vdot \grad \varphi} \leq \\
					     &\leq 2\qty(\abs{h(\vdot)} + C\qty(\abs{u + t \varphi}^r + \abs{\grad (u + t \varphi)}^r))\qty(\abs{\varphi} + \abs{\grad \varphi}),
	\end{align*}
	which is in $\LpSet[1]{\Omega}$ for $h \in \LpSet[1]{\Omega}, u \in \WkpSet[1][r]{\Omega}, \varphi \in \DSet{\Omega}.$ Also from the assumption $u$ is a local minimizer we have
	\[
		i(0) = \int_{\Omega}F\qty(\vdot, u, \grad u)\dd{x} < \infty
	\]
	Altogether, we have checked the necessary assumption, and we see that $\iota'\qty(0)$ exists and is equal to
	\[
		\iota'\qty(0) = \int_{\Omega}\partial_{z}F\qty(\vdot, u, \grad u)\varphi + \grad_p F\qty(\vdot, u, \grad u) \vdot \grad \varphi\dd{x} =0.
	\]
\end{proof}

\begin{example}
	Let
	\[
		F\qty(x,z,p) = \frac{1}{r}\qty(1)+|p|^{2})^{\frac{r}{2}} - gz - Gp,
	\]
	then
	\[
		- \grad_p F\qty(x,z,p) = \qty(\frac{r}{2}\frac{1}{r} 2 \qty(1+|p|^{2})^{\frac{r-2}{2}})p -G = \qty(1+|p|^{2})^{\frac{r-2}{2}}p-G, \partial_{z}F\qty(x,z,p) = -g.
	\]
	We have
	\[
		|\qty(1+|p|^{2\frac{r-2}{2}})p| \leq \qty(1+|p|^{2})^{\frac{r-2}{2}}\qty(1+|p|^{2})^{\frac{1}{2}} = \qty(1+|p|^{2})^{\frac{r-1}{2}}\leq C\qty(1+|p|^r).
	\]
	So the estimates are met (somehow with some fantasy).
	The Euler-Lagrange equations are
	\[
		\begin{cases}
			- \divergence{\qty(\qty(1+|\grad u|^{2})^{\frac{r-2}{2}}\grad u)} = - \divergence{G} + g, & \, \text{in} \, \Omega\\
			u = u_0, & \, \text{on} \, \partial \Omega.
		\end{cases},
	\]
	whereas their weak form:
	\[
		\forall \varphi \in \mathcal{D}\qty(\Omega):\int_{\Omega}\qty(1+|\grad u|^{2})^{\frac{r-2}{2}}\grad u \vdot \grad \varphi\dd{x} = \int_{\Omega}\qty(G \vdot \grad\varphi + g \varphi) \dd{x}.
	\]
\end{example}


\subsection{Minimization of (convex) functionals}
\label{sec:minimization}
We have seen that finding a minimizer is equivalent to solving the Euler-Lagrange equations. But how to find the minimizer? The approach is the following:
\begin{remark}[General approach]


	\begin{itemize}
		\item find some minimizing sequence :$ \qty{u_n} \subset X$ such that
			\[
				\lim_{n \to \infty}I\qty(u_n) = \inf_X I.
			\]
		\item we would then like to find the minimizer as some sort of a limit of $\qty{u_n}$ - we need some kind of convergence or compactness. In our case, we will use weak convergence, thus from some compactness result, we will want to show
			\[
				u_n \rightharpoonup u
			\]
			for some $u,$
		\item to check that actually $I(u)$ is the minimum of $I$, we need some kind of continuity of $I$ - in our case, weak sequential lower semicontinuity will be enough:
			\[
				I(u) \leq \liminf_{n \to \infty} I(u_n), \, \text{as} \, u_n \rightharpoonup u.
			\]
	\end{itemize}
\end{remark}

In the following, we try to derive some sufficient conditions for the functional $I = \int_{\Omega}F\qty(x, u, \grad u)\dd{x}$ to be weakly sequentially lower semicontinuous, \textit{i.e.}, some conditions for $F\qty(x, u, \grad u)$. 

\begin{lemma}
	Let $N \in \N, F: \R^N \to \R, F \in C^1\qty(\R^N).$ Then 
	\begin{enumerate}
		\item $F$ is (strictly) convex $\Leftrightarrow$ $\grad F$ is (strictly) monotone
		\item If $F$ is (strictly) convex, then
			\[
				\forall \xi_1, \xi_2 \in \R^N, \xi_1 \neq \xi_2: F\qty(\xi_1)-F\qty(\xi_2)\geq \grad F\qty(\xi_2)\vdot \qty(\xi_1 - \xi_2).
			\]
	\end{enumerate}
\end{lemma}
\begin{proof}(\textit{From: the lectures})
	Fix $\xi_1, \xi_2, \xi_1 \neq \xi_2,$ define $\varphi(t) = F\qty(\xi_2 + t\qty(\xi_1 - \xi_2)).$ Then $\varphi \in C^1\qty(\R)$ and it is also (strictly) convex: $\forall s,t \in \R:$ 
	\begin{align*}
		\varphi\qty(\lambda s + (1- \lambda)t) &= F\qty(\xi_2 + \qty(\lambda t + \qty(1-\lambda)s)\qty(\xi_1 - \xi_2)) = F\qty(\lambda\qty(\xi_2 + t\qty(\xi_1 - \xi_2))- \lambda \xi_2 + \xi_2 + \qty(1-\lambda)s\qty(\xi_1 - \xi_2)) = \\
						       &= F\qty(\lambda\qty(\xi_2 + t\qty(\xi_1 - \xi_2))+ \qty(1- \lambda)\qty(\xi_2 + s\qty(\xi_1 - \xi_2))) \leq \\
						       &\leq \lambda F\qty(\xi_2 + t\qty(\xi_1 - \xi_2)) + \qty(1- \lambda)F\qty(\xi_2 +s \qty(\xi_1 - \xi_2)) = \lambda \varphi(s) + (1-\lambda) \varphi(s).
	\end{align*}
	The derivative of $\varphi$ is 
	\[
		\varphi'(t) = \grad F\qty(\xi_2 +t\qty(\xi_1 - \xi_2))\vdot \qty(\xi_1 - \xi_2),
	\]
	and so
	\[
		"\Rightarrow": \qty(\grad F\qty(\xi_1) - \grad F\qty(\xi_2))\vdot\qty(\xi_1-\xi_2) = \varphi'(1)-\varphi'(0) \geq 0,
	\]
	if $\varphi$ is (strictly) convex.
	
	And $"\Leftarrow":$ Fix $t_1 > t_2$ and compute
	\[
		(t_1 - t_2)	(\varphi'\qty(t_1) - \varphi'\qty(t_2)) = \qty(\grad F\qty(\xi_2 + t_1\qty(\xi_1 - \xi_2))- \grad F\qty(\xi_2 + t_2\qty(\xi_1-\xi_2)))\vdot\qty(\xi_1-\xi_2)(t_1-t_2),
	\]
	define
	\[
		\eta_1 - \eta_2 = \qty(\xi_1 - \xi_2)\qty(t_1-t_2)
	\]
	and we obtain
	\[
		(t_1 - t_2)(\varphi'(t_1) - \varphi'(t_2)) = \qty(\grad F\qty(\eta_1)-\grad F\qty(\eta_2))\vdot \qty(\eta_1 - \eta_2)
	\]
	and so if the $\grad F$ is (strictly) monotonous, the LHS is (strictly) positive, meaning  (strict) convexity of $\varphi$.

	For 2) we already know $F$ (strictly) convex $\Rightarrow \varphi$ (strictly) convex, and moreover realize
	\[
		\varphi(t) = \varphi(t \vdot 1 + \qty(1- t)\vdot0) \leq t \varphi(1) + (1-t) \varphi(0),
	\]
	meaning
	\[
		\Rightarrow \forall t \in \qty(0,\frac{1}{2}): \frac{\varphi(1)-\varphi(0)}{1} \geq \frac{\varphi(t)-\varphi(0)}{t}\to \varphi'(0),
	\]
	as $t\to 0_+.$ And so $\varphi(1) - \varphi(0) \geq \varphi'(0),$ which is exactly the same as
	\[
		F\qty(\xi_1) - F\qty(\xi_2) \geq \grad F\qty(\xi_2)\vdot \qty(\xi_1 - \xi_2).
	\]
\end{proof}

\begin{theorem}[Sequential weak lower semicontinuity]
	Let $N, M \in \N, \Omega$ be open and $F\qty(x, \vb{z}, \vb{p}): \Omega \times \R^{N} \times \R^{M} \to \R$ be Caratheodory and convex in $\vb{p}.$ Let moreover $\exists C(x) \in \LpSet[1]{\Omega}$ such that
	\[
		\forall \, \text{\textit{a.a.}} \, x \in \Omega, \forall \qty(\vb{z}, \vb{p}) \in \R^{N+M}: F\qty(x, \vb{z}, \vb{p}) \geq C(x).
	\]
	Let $\qty{\vb{u_n}} \subset \LpSet[1]{\Omega}$ converge to $\vb{u}$ in $\LpSet[1]{\Omega}, \vb{u}_n \to \vb{u}$ and let $\qty{\vb{U}_n} \subset \LpSet[1]{\Omega}$ converge weakly to $\vb{U}$ in $\LpSet[1]{\Omega}, \vb{U}_n \rightharpoonup \vb{U}.$ Then it holds

	\[
		\int_{\Omega}F\qty(x, \vb{u}, \vb{U})\dd{x} \leq \liminf_{n\to \infty} \int_{\Omega}F\qty(x, \vb{u}_n, \vb{U}_n)\dd{x}
	\]
\end{theorem}

\begin{proof}(\textit{From: \cite{bulicekPartialDifferentialEquations2019}})
	The proof will be given only if moreover $\forall \, \text{\textit{a.e.}} \,x \in \Omega, \forall z \in \R^{M}: F\qty(x,z,\vdot) \in C^1\qty(\R^N).$ The main idea is the following: by the previous lemma:
	\[
		\int_{\Omega}F\qty(\vdot, \vb{u}_n, \vb{U}_n)\dd{x} \geq \int_{\Omega}\qty(F\qty(\vdot, \vb{u}_n, \vb{U}) + \grad_p F\qty(\vdot, \vb{u}_n, \vb{U})\vdot\qty(\vb{U}_n - \vb{U}))\dd{x},
	\]
	Since $\qty{\vb{u}_n}$ converges strongly in $\LpSet[1]{\Omega},$ there is a (not renamed) subsequence converging \textit{a.e.} in $\Omega.$ By Egorov theorem that means
	\[
		\forall \varepsilon>0 \exists \Omega_{\varepsilon} \subset \Omega: \vb{u}_n \to \to \vb{u}_n, \lambda\qty(\Omega / \Omega_{\varepsilon})< \varepsilon.
	\]
	Let us then write (we are using the fact $F\qty(x, \vb{z}, \vb{p}) - C(x) \geq 0.$)
	\begin{align*}
		\int_{\Omega}F\qty(x, \vb{u}_n, \vb{U}_n)\dd{x} &= \int_{\Omega}\qty(F\qty(x, \vb{u}_n, \vb{U}_n) - C(x))\dd{x} + \int_{\Omega}C(x)\dd{x} \geq \\
		&\geq \int_{\Omega_{\varepsilon}}F\qty(x, \vb{u}_n, \vb{U}) - C(x)\dd{x} + \int_{\Omega_{\varepsilon}}F\qty(x, \vb{u}_n, \vb{U}_n) - F\qty(x, \vb{u}_n, \vb{U})\dd{x} + \int_{\Omega}C(x)\dd{x} \geq \\
		&\geq \int_{\Omega_{\varepsilon}}F\qty(x, \vb{u}_n, \vb{U}) - C(x)\dd{x} + \int_{\Omega_{\varepsilon}}\grad_p F\qty(x, \vb{u}_n, \vb{U}) \vdot \qty(\vb{U}_n - \vb{U})\dd{x} + \int_{\Omega}C(x)\dd{x}.
	\end{align*}
	Now we take the limes inferior of both sides and write 
	\begin{align*}
		\liminf_{n\to \infty}\int_{\Omega}F\qty(x, \vb{u}_n, \vb{U}_n)\dd{x} &\geq \liminf_{n \to \infty} \Big(\int_{\Omega_{\varepsilon}}F\qty(x, \vb{u}_n, \vb{U}) - C(x)\dd{x} + \int_{\Omega_{\varepsilon}}\grad_p F\qty(x, \vb{u}_n, \vb{U}) \vdot \qty(\vb{U}_n - \vb{U})\dd{x} + \\
										     &+\int_{\Omega}C(x)\dd{x}Big) \geq \\
										     &\geq \liminf_{n \to \infty} \int_{\Omega_{\varepsilon}}F\qty(x, \vb{u}_n, \vb{U}) - C(x)\dd{x} + \liminf_{n\to \infty}\qty(\int_{\Omega_{\varepsilon}}\grad_p F\qty(x, \vb{u}_n, \vb{U})\vdot\qty(\vb{u_n} - \vb{U})\dd{x}) + \\
										     &+\int_{\Omega}C(x)\dd{x}.
	\end{align*}
	The second integral actually has a limit,
	as $\vb{U}_n \rightharpoonup \vb{U}$ in $\Omega_{\varepsilon}$, so $\vb{U}$ is bounded and $\grad_p F\qty(x, \vb{u}_n, \vb{U}_n) \in \LinfSet{\Omega_{\varepsilon}},$ because $\vb{u}_n \to \to \vb{u}$ on $\Omega_{\varepsilon}$ and $\grad_p F\qty(x, \vb{u}_n, \vb{U})$ is continuous in $\vb{u}_n$ from Caratheodory. As the first integral is nonnegative, we can use Fatou to estimate from below. But since $F(x, \vdot, \vb{p})$ is continuous from Caratheodory property and $\vb{u}_n \to \to \vb{u}$ on $\Omega_{\varepsilon}$, the integrand has a limit and it actually holds:

	\[
		\liminf_{n\to \infty}\int_{\Omega}F\qty(x, \vb{u}_n, \vb{U}_n)\dd{x}\geq \int_{\Omega_{\varepsilon}}F\qty(x, \vb{u}, \vb{U})- C(x)\dd{x} + \int_{\Omega}C(x)\dd{x} = \int_{\Omega_{\varepsilon}}F\qty(x, \vb{u}, \vb{U})\dd{x} + \int_{\Omega / \Omega_{\varepsilon}}C(x)\dd{x},
	\]
	and by taking the limit $\varepsilon \to 0^+$ and using the monotone convergence theorem we actually have
	\[
		\liminf_{n \to \infty}\int_{\Omega}F\qty(x, \vb{u}_n, \vb{U}_n)\dd{x} \geq \int_{\Omega}F\qty(x, \vb{u}, \vb{U})\dd{x}.
	\]

\end{proof}
\begin{remark}
	\begin{itemize}
		\item if $U_n \to U$ strongly $\Rightarrow u_n \to u, U_n \to U$ \textit{a.e.} (up to a subsequence) and the claim follows from the Fatou lemma. \footnote{For Fatou, we need nonnegativity of the integrand, but that can be again met from the assumptions $F-c_2 \geq 0, F-c_2 \in \LpSet[1]{\Omega}$}
		\item norm is weakly lower semicontinuous:
			\[
				\grad u_n \rightharpoonup \grad u \, \text{in} \, \LpSet{\Omega} \Rightarrow \int_{\Omega}|\grad u|^p\dd{x} \leq \liminf_{n \to \infty} \int_{\Omega}|\grad u_n|^p\dd{x}.
			\]
	\end{itemize}
\end{remark}

\begin{theorem}[Existence of a minimizer]
    Let $d \in \N, \Omega \in \Ckl{0}{1}, r \in (1, \infty), u_0 \in \WkpSet[1][r]{\Omega},$ denote $X = u_0 + \WkpzeroSet[1][r]{\Omega}.$ Let $F:\Omega \times \R \times \R^{d} \to \R$ be Caratheodory with the coercivity condition: $\exists C > 0, h \in \LpSet[1]{\Omega}$ \textit{s.t.}:
    \[
	    \forall \, \text{\textit{a.a.}} \, x \in \Omega, \forall \qty(z, \vb{p}) \in \R \times \R^{d}: F\qty(x,z,\vb{p}) \geq C\abs{\vb{p}}^r - h.
    \]
    Let moreover $F$ be convex in the last variable, \textit{i.e.}, let $\vb{p} \mapsto F\qty(x, z, \vb{p})$ be convex $\forall$ \textit{a.a.} $x\in \Omega, \forall z \in \R.$ Then there exists $u \in X$ that minimizes $I$ on $X$, \textit{i.e.}, $\trace u = \trace u_0$ on $\partial \Omega$ and
    \[
	    \forall v \in X: I(u) \leq I(v).
    \]
\end{theorem}

\begin{proof}(\textit{From: \cite{bulicekPartialDifferentialEquations2019}})
    First of all, let us show
    \[
	    \Lambda \coloneq \inf_{u \in X} I(u) = \inf_{u \in X} \int_{\Omega}F\qty(x, u, \grad u)\dd{x},
    \]
    is bounded from below. Coercivity states
    \[
	    F\qty(x, u, \grad u) \geq C \abs{\grad u}^r - h(x),
    \]
    so
    \[
	   \int_{\Omega}F\qty(x, u, \grad u)\dd{x} \geq \int_{\Omega}C \abs{\grad u}^r - h(x)\dd{x} \geq C \norm{\grad u}_{\LpSet[r]{\Omega}}^r - \norm{h}_{\LpSet[1]{\Omega}} \geq - \norm{h}_{\LpSet[1]{\Omega}},
    \]
    and since $\norm{h}_{\LpSet[1]{\Omega}}$ is of course bounded, this means $\Lambda > - \infty.$ Next, we show there exists a minimizing sequence $\qty{u_n} \subset X,$ \textit{i.e.}, a sequence \textit{s.t.}
    \[
	    \lim_{n \to \infty}I\qty(u_n) = \Lambda.
    \]
    Realize that from the definition of infimum, $\forall n \in \N \exists u_n \in X$ such that
    \[
	    I\qty(u_n) \leq \Lambda + \frac{1}{n}.
    \]
    We do not know the limit of $I\qty(u_n)$ actually exists, so we have to be a bit more careful. On the other hand, limes superior is defined always, so we on one hand have
    \[
	    \limsup_{n \to \infty} I\qty(u_n) \leq \Lambda,
    \]
    and from the definition of the infimum
    \[
	    \Lambda \leq I\qty(u_n),
    \]
    we read
    \[
	    \Lambda \leq \liminf_{n \to \infty}I\qty(u_n),
    \]
    and so in total
    \[
	    \Lambda \leq \liminf_{n \to \infty}I\qty(u_n) \leq \limsup_{n \to \infty} \leq \Lambda.
    \]
But this means all the inequalities must in fact be equalities, and the fact $\liminf_{n \to \infty} I(u_n) = \limsup_{n \to \infty}I\qty(u_n)$ is equivalent to the fact $\lim_{n \to \infty}I\qty(u_n)$ exists and is equal to $\Lambda.$ And so we have obtained our minimizing ("infimizing") sequence. 

Now, also from the property of the infimum we may write
    \[
	    I(u_n) = \int_{\Omega}F\qty(x, u_n, \grad u_n)\dd{x} \leq \Lambda + 1,
    \]
    and so upon using coercivity and estimating the integral from below we may write
    \[
	    \int_{\Omega}F\qty(x, u_n, \grad u_n)\dd{x} \geq \int_{\Omega}C \abs{\grad u_n}^r - h\dd{x} \geq C \norm{\grad u_n}^r_{\LpSet[r]{\Omega}} - \norm{h}_{\LpSet[1]{\Omega}},
    \]
    and so
    \[
	    \norm{\grad u_n}_{\LpSet[r]{\Omega}}^r \leq \frac{1}{C}\qty(\Lambda + 1 + \norm{h}_{\LpSet[1]{\Omega}}), \, \text{\textit{i.e.}} \,, \norm{\grad u_n}_{\LpSet[r]{\Omega}} \leq C_1
    \]
    Realize also that
    \[
	    \norm{\grad u_n}_{\LpSet[r]{\Omega}} = \norm{\grad\qty(u_n - u_0) + \grad u_0}_{\LpSet[r]{\Omega}} \geq \norm{\grad\qty(u_n - u_0)}_{\LpSet[r]{\Omega}} - \norm{\grad u_0}_{\LpSet[r]{\Omega}} \geq C_p \norm{u_n - u_0}_{\WkpzeroSet[1][r]{\Omega}} - \norm{\grad u_0}_{\LpSet[r]{\Omega}},
    \]

meaning
\[
	\norm{u_n-u_0}_{\WkpzeroSet[1][r]{\Omega}} \leq \frac{1}{C_p}\qty(\norm{\grad u_n}_{\LpSet[r]{\Omega}}+ \norm{\grad u_0}_{\LpSet[r]{\Omega}}) \leq \frac{1}{C_p}\qty(C_1 + \norm{\grad u_0}_{\LpSet[r]{\Omega}}) = C_2,
\]

where we used Poincare and the fact $\qty{u_n} \subset X,$ so $\trace v_n = \trace u_0.$ Finally, using
\[
	\norm{u_n - u_0}_{\WkpzeroSet[1][r]{\Omega}} \geq \norm{u_n}_{\WkpSet[1][r]{\Omega}} - \norm{u_0}_{\WkpSet[1][r]{\Omega}},
\]
we have obtained
\[
	\norm{u_n}_{\WkpSet[1][r]{\Omega}} \leq C_2 + \norm{u_0}_{\WkpSet[1][r]{\Omega}} = C_3,
\]
which is an uniform bound of the functions from the sequence.  $\qty{u_n}$ is a bounded sequence in a reflexive space\footnote{$X$ is the space $X = u_0 + \WkpzeroSet[1][r]{\Omega},$ and since $\WkpzeroSet[1][r]{\Omega}$ is a closed subspace of $\WkpSet[1][r]{\Omega},$ the space $X$ is actually a factorspace and its reflexivity is implied by the reflexivity of $\WkpSet[1][r]{\Omega}.$}, so
    \[
	    u_n \rightharpoonup u, \, \text{in} \, \WkpSet[1][r]{\Omega},
    \]
    in particular
    \[
	    \grad u_n \rightharpoonup \grad u \, \text{in} \, \LpSet[r]{\Omega},
    \]
    and also
    \[
	    u_n \to u, \, \text{in} \, \LpSet[r]{\Omega}.
    \]
Finally, this means
\[
	\Lambda = \lim_{n \to \infty} \int_{\Omega}F\qty(x, u_n, \grad u_n)\dd{x} \geq \liminf_{n \to \infty} \int_{\Omega}F\qty(x, u_n, \grad u_n)\dd{x} \geq \int_{\Omega}F\qty(x, u, \grad u)\dd{x} \geq \Lambda,
\]

where we used the weak lower sequential semicontinuity of $I$ and the fact $u \in X,$ so $I(u) \geq \Lambda.$ But the above manipulation shows that in fact all the inequalities must be equalities, and thus
\[
	I(u) = \int_{\Omega}F\qty(x, u, \grad u)\dd{x} = \Lambda,
\]
\textit{i.e.}, $u$ is a minimizer.
\end{proof}

