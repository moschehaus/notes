% !TEX root = ../main.tex

\section{Sobolev spaces revisited}
\label{sec:sobolev_revisited}
Let $\Omega \subset \R^d \, \text{open} \,, p \in [1,+\infty], k \in N.$ We define
\[
	\WkpSet{\Omega} = \Big\{f \in \LpSet{\Omega}; D^\alpha f \in \LpSet{\Omega}, \forall |\alpha| \leq k \Big\},
\]
with the norm
\[
	\norm{f}_{\WkpSet{\Omega}}^p = \norm{f}_{\LpSet{\Omega}}^p + \sum_{0< |\alpha| \leq k} \norm{D^\alpha f}_{\LpSet{\Omega}}^p.
\]
Recall that:
\begin{itemize}
	\item	$\WkpSet{\Omega}$ is Banach $\forall p$ and Hilbert for $p=2$. 
	\item $\WkpSet{\Omega}$ is separable if $p < \infty$ and reflexive if $p>1, p<\infty$.
\end{itemize}


\textit{Our goal will be to prove embedding and trace theorems. We will use the density of smooth functions.}

\subsection{Tools from functional analysis}
\label{sec:fa_tools}

\begin{definition}[Regularization kernel]
	The function $\eta$ is called the regularization kernel supposed:
	\begin{itemize}
		\item $\eta \in \mathcal{D}(\R^d)$
		\item $\supp \eta \subset \text{U}(0,1)$
		\item $\eta \geq 0$
		\item $\eta$ is radially symmetric
		\item $\int_{\R^d}\eta(x)\dd{x} = 1$
	\end{itemize}
\end{definition}

\begin{definition}[Regularization of a function]
	Let $\eta$ be a regularization kernel. Set\footnote{Another common choice is $\eta_k = k^d \eta\qty(kx), k \in \N.$}
	\[
		\eta_{\varepsilon}(x) = \frac{1}{\varepsilon^{d}} \eta (x/\varepsilon), \varepsilon >0.
	\]
	We define the smoothing of $f \in \LpSet[1]{\Omega}_{\, \text{loc} \,}$ by
	\[
		f_{\varepsilon}(x) = (f \star \eta_{\varepsilon})(x).
	\]
\end{definition}

\begin{remark}[Propertios of regularization]
	The regularization has the following properties:
	\begin{itemize}
		\item $f \in \LpSet{\Omega} \Rightarrow f_{\varepsilon} \to f \, \text{in} \, \LpSet{\Omega}$ and also a.e
		\item $f \in \LinfSet{\Omega} \Rightarrow f_{\varepsilon} \to f \, \text{a.e and *-weak} \,$
		\item $f_{\varepsilon}(x) = \int_{\R^d}f(y)\eta_{\varepsilon}(x-y)\dd{y} = \int_{\text{U}(x,\varepsilon)}f(y) \eta_{\varepsilon}(x-y)\dd{y}$
		\item $\supp f_{\varepsilon} \subset \overline{U(\Omega,\varepsilon)}, f=0 \, \text{on} \,\text{U}(x,\varepsilon) \Rightarrow f_{\varepsilon}(x)=0$
	\end{itemize}
\end{remark}

\begin{definition}[$\Omega' \subset \subset \Omega$]
	$O \subset \subset \Omega$ means $\overline{O}$ is compact and $\overline{O}\subset \Omega$.
\end{definition}

\begin{definition}[Shift operator]
	For $u \in \LpSet{\Omega}, k \in \{1,\dots,d\}, h >0, $ we introduce the shift operator
	\begin{equation*}
		\tau_h u(x)=u(x+h \vb{e}_k)
	\end{equation*}
\end{definition}

\begin{lemma}[Approximation property of the shift operator]
	For $u \in \LpSet{\Omega}$, it holds $\tau_h u \to u$ in $\LpSet{\Omega}, h\to 0^+$.
\end{lemma}

\begin{lemma}[Partition of unity]
	Let $E \subset \R^d, \mathcal{G}$ be an open covering of $E$ (possibly uncountable.) Then there exists a countable system $\mathcal{F}$ of nonnegative functions $\varphi \in \mathcal{D}(\R^d)$ such that $0 \leq \varphi \leq 1$ and 
	\begin{enumerate}
		\item $\mathcal{F}$ is subordinate to $\mathcal{G}: \forall \varphi \in \mathcal{F} \exists U \in \mathcal{G}: \supp \varphi \subset U$
		\item $\mathcal{F}$ is locally finite\footnote{In other words, $\varphi_K$ is nonzero for at most finitely many $\varphi \in \mathcal{F} \Leftrightarrow$ points in $K$ can be represented by finitely many functions $\varphi \in \mathcal{F}.$}: $\forall K \subset E \, \text{compact} \,, \supp \varphi \cap K \neq \emptyset$ for at most finitely many $\varphi \in \mathcal{F}$.
		\item $\sum_{\varphi \in \mathcal{F}} \varphi(x) = 1, \forall x \in E$ .
	\end{enumerate}
\end{lemma}
\begin{proof}(\textit{From: the lectures})
	(Sketch)
	\textit{Step 1 (If $E$ is compact)}:

	$E$ compact $\Rightarrow \exists m \in \N: \exists U_j \in \mathcal{G} \, \text{\textit{s.t.}} \, E \subset \bigcup_{j=1}^{m}U_j$ . Moreover, $\exists K_j \subset U_j$ compact such that $E \subset \cup_{j=1}^m K_j$. That follows from the exhaustion argument: for $U \subset \R^d$ open, you can approximate it by a compact set:
	\[
		K_m = \Big\{x \in U| \dist\qty(x,\partial \Omega) \geq \frac{1}{m}, \norm{x} \leq m \Big\}.
	\]
	Then clearly $K_1 \subset K_2 \dots $, and they "converge monotonously to $U$.
	Next, find $\phi_j \in C_c(U_j), \phi_j >0 \, \text{on} \, K_j$, e.g. $\phi_j = \theta\qty(\dist(x,\partial U_j)).$ Then use convolution: $\psi_j = (\phi_j)_{\varepsilon}, \varepsilon > 0$ small and take finally
	\[
		\varphi_j = \frac{\psi_j}{\sum_k \psi_k}.
	\]
	

	\textit{Step 2 (If $E$ is open)}:

	Approximate $E$ by $K \subset E$ compact by the exhaustion argument, then the covering will enlarge from finite $\to$ countable (nontrivial reasoning).
\end{proof}


\subsection{Density of smooth functions}
\label{sec:density}

\begin{lemma}[Local approximation by smooth functions (using regularization)]
	Assume $p \in [1, \infty), \Omega \subset \R^d \, \text{open} \,, k \in \N, u \in \WkpSet{\Omega}, \Omega_{\varepsilon} = \qty{x \in \Omega | \dist\qty(x, \partial \Omega)>\varepsilon}.$ Then it holds
	\begin{enumerate}
		\item $D^\alpha(u_{\varepsilon}) = (D^\alpha u)_{\varepsilon}$ \textit{a.e.} in $\Omega_{\varepsilon}, \forall |\alpha| \leq k$
		\item $u_{\varepsilon}\to u$ in $\WkpSet{\Omega}_{\, \text{loc} \,}, \varepsilon \to 0^+$
	\end{enumerate}
\end{lemma}

\begin{proof}(\textit{From: the lectures})
	First of all: (those are classical derivatives at the moment!) \[
		\forall x \in \Omega: D^{\alpha}\qty(u_{\varepsilon}(x)) = D^{\alpha}\qty(\int_{\R^d}u(y)\eta_{\varepsilon}(x-y)\dd{y}) = \int_{\R^d}u(y)D_x^{\alpha}\eta_{\varepsilon}(x-y)\dd{y}, 
	\]
	the integrable majorants are \textit{e.g.} $\norm{\eta_{\varepsilon}}_{\infty}|u|\chi_{\text{U}(0,\varepsilon)}(x) \in \LpSet[1]{\Omega}.$ Now picking $x \in \Omega_{\varepsilon}$ we realize $\forall y \in \R^d / \overline{\Omega}: x-y \geq \dist\qty(x,\partial \Omega) \geq \varepsilon,$ and so $\eta_{\varepsilon}\qty(x-y) = 0.$ Meaning the integrand is zero on the complement of $\Omega,$ and since $\eta_{\varepsilon}$ has a compact support in $\Omega,$ we can integrate over $\Omega \supset \Omega_{\varepsilon}$ instead. Exchanging derivatives and using the definition of the weak derivative then yields
	\begin{align*}
		\int_{\Omega}u(y) D_x^{\alpha}\eta_{\varepsilon}(x-y)\dd{y} &=(-1)^{|\alpha|}\int_{\Omega}u(y)D_y^{\alpha}\eta_{\varepsilon}\qty(x-y)\dd{y} = \int_{\Omega}D^{\alpha}_y u(y) \eta_{\varepsilon}(x-y)\dd{y}= \\
		&=\int_{\R^d}D^{\alpha}_y u(y) \eta_{\varepsilon}(x-y)\dd{y} = \qty(D^{\alpha}u)_{\varepsilon}.
	\end{align*}
	Take $V \subset \subset \Omega$ open, then
	\[
		\norm{u - u_{\varepsilon}}_{\WkpSet{V}} = \sum_{|\alpha|\leq k}\norm{D^{\alpha}u - D^{\alpha}u_{\varepsilon}}_{\LpSet{V}} \to 0,
	\]
	because $D^{\alpha}u_{\varepsilon} = (D^{\alpha}u)_{\varepsilon} \to D^{\alpha}u \, \text{in} \, \LpSet{V},$ from the properties of regularization.
\end{proof}

\begin{theorem}[Global approximation by smooth functions]
	Let $\Omega \subset \R^d$ be open, $k \in \N, p \in [1,\infty).$ Then $C = \Big\{ f \in C^\infty(\Omega), \supp f \, \text{bounded} \,\Big\} \bigcap \WkpSet{\Omega}$ is dense in $\WkpSet{\Omega}, \, \text{\textit{i.e.}} \,$
	\[
		\overline{C \bigcap \WkpSet{\Omega}}^{\norm{\vdot}_{\WkpSet{\Omega}}} = \WkpSet{\Omega}.
	\]
	If moreover $\Omega$ is bounded, it holds:
	\[
		\overline{\CinfSet{\Omega}\bigcap \WkpSet{\Omega}}^{\norm{\vdot}_{\WkpSet{\Omega}}} = \WkpSet{\Omega}.
	\]
\end{theorem}
\begin{proof}(\textit{From: the lectures})
	Let $u \in \WkpSet{\Omega}, \varepsilon >0$. I want to show $\exists v \in C^\infty(\Omega) \bigcap \WkpSet{\Omega} \, \text{\textit{s.t.}} \, \norm{u-v}_{\WkpSet{\Omega}} < \varepsilon$.
For every $j \in \N$ define an open set
	\[
		\Omega_j = \Big\{ x \in \Omega, \dist\qty(x,\partial \Omega) > \frac{1}{j}\Big\}.
	\]
	Clearly, $\Omega_j \subset \Omega_{j+1} \forall j \in \N, \bigcup_{j=1}^\infty \Omega_j =\Omega$. Next, set
	\[
		U_j = \Omega_{j+1} /\,  \overline{\Omega_{j-1}}, j=1,2, \dots,
	\]
	where $\Omega_0 = \Omega_{-1} = \emptyset$. Since $\Omega_j$ are open, $U_j$ are also open and $\Omega \subset \bigcup_{j \in \N}U_j  \Rightarrow \exists \{\varphi_j\}_{j\in \N} $ partition of unity subordinate to $\{U_j\}_{j \in \N}$. We can write $u = \sum_{j \in \N} u \varphi_j$, where $u \varphi_j \in \WkpSet{\Omega}, \supp u \varphi_j \subset U_j \subset \Omega_{j+1} \subset \subset \Omega$. This is ready for convolution with $\varepsilon_j >0$: set $ v_j = (u \varphi_j)_{\varepsilon_j}$ and fix an arbitrary $\delta>0.$ By the properties of regularization, we have
	\[
		\norm{v_j - u \varphi_j }_{\WkpSet{U}} < \frac{\delta}{2^{j-1}},
	\]
	for any $U \subset \subset \Omega,$ for $\varepsilon_j > 0$ sufficiently smalls. In fact, if needed, make the $\varepsilon_j$ smaller so that
	\[
		\supp v_j \subset \Omega_{j+2}/ \overline{\Omega_{j-2}}.
	\]
	That is completely possible: from the properties of regularization, we know
	\[
		\supp v_j \subset \text{B}(0,\varepsilon)+ \overline{U_j},
	\]
	and since
	\[
		\overline{U_j} = \overline{\Omega_{j+1}}/\Omega_{j-1} \subset \overline{\Omega_{j+2}}/\Omega_{j-2} \subset \overline{\overline{\Omega_{j+2}}/\Omega_{j-2}} = \Omega_{j+2}/ \overline{\Omega_{j-2}},
	\]
	the compact set $\overline{U_j}$ is contained in the open set $\Omega_{j+2}/ \overline{\Omega_{j-2}},$ meaning with $\varepsilon_j$ small, the set $\overline{U_j} + \text{B}(0,\varepsilon_j)$ will still be in $\Omega_{j+2}/ \overline{\Omega_{j-2}}.$ Also, let us take possibly $\varepsilon_j$ even smaller to have a nice inequality: for fixed $N \in \N$:
	\[
		\norm{v_j - u \varphi_j}_{\WkpSet{U}} < \frac{1}{2-\qty(\frac{1}{2})^{N+1}} \frac{\delta}{2^{j-1}},
	\]
	meaning of $N \in \N$ will be evident later. 


	Set
	\[
		v = \sum_{j\in \N} v_j,
	\]
	then $v \in C^{\infty}\qty(\Omega),$ (not clearly in $\WkpSet{\Omega}$ however) as $\forall x \in \Omega$ the sum contains at most finitely many terms ($ \mathcal{F}$ is locally finite.)


	Take the $N \in \N$ and estimate the norm $\norm{u-v}_{\WkpSet{\Omega_N}}$. Observe (the sum again contains only finitely many terms)
	\[
		u-v = \sum_{j=1}^\infty(u \varphi_j - v_j),
	\]
	so taking $x \in \Omega_N$ i have
	\[
		(u-v)(x) = \sum_{j=1}^{N+2}(u \varphi_j - v_j),
	\]
		because for $m > N+2, \, \text{\textit{i.e.}}, \, m-2 >N$ it holds the functions $u_m, v_m, $ have their supports in
		\begin{itemize}
			\item $\supp u_m \subset U_m = \Omega_{m+1}/ \overline{\Omega_{m-1}},$
			\item $\sup v_m \subset \Omega_{m+2}/ \overline{\Omega_{m-2}},$
		\end{itemize}
		but $\Omega_{m-1} \subset \Omega_{m-2} \subset \Omega_N,$ for $N < m-2,$ meaning that the set $\Omega_N$ does not lie in the supports of those functions.
	The norm of sum is (recall $\Omega_N \subset \subset \Omega,$ so the above estimate holds)

	\begin{equation*}
		\norm{u-v}_{\WkpSet{\Omega_N}} \leq \sum_{j=1}^{N+2}\norm{u \varphi_j - v_j}_{\WkpSet{\Omega_N}} <\delta \frac{1}{2-\qty(\frac{1}{2})^{N+1}}\sum_{j=1}^{N+2} \frac{1}{2^j} = \delta.
	\end{equation*}
	It only remains to let $N \to \infty$ and realize
	\[
		\norm{u-v}_{\WkpSet{\Omega_N}} \to \norm{u-v}_{\WkpSet{\Omega}}
	\]
	by Lévi's theorem:
	\[
	\sup_{N\in \N}\int_{\Omega_N}|D^\alpha f| \dd{x} = \sup_{N \in \N}\int_{\R^d}|D^{\alpha}f|\chi_{\Omega_N}(x)\dd{x} = \int_{\R^d}\sup_{N \in \N}|D^{\alpha}f| \chi_{\Omega_N}\dd{x} \int_{\R^d}|D^\alpha f|\chi_{\Omega}(x)	\dd{x} = \int_{\Omega}|D^{\alpha}f|\dd{x},
	\]
	since $\Omega_{N-1} \subset \Omega_{N} \forall N \in \N,$ and $|D^{\alpha}f|$ is nonnegative, so the sequence under the integral is nondecreasing. Alltogether,
	\[
		\norm{u-v}_{\WkpSet{\Omega}}\leq \delta, \forall \delta>0
	\]
	from which it follows $v \in \WkpSet{\Omega}$ (this was not totally evident) and thus $v \in \WkpSet{\Omega} \cap C^{\infty}\qty(\Omega)$ so indeed we have showed the desired density.
\end{proof}

\begin{remark}
	It is nice that we only require $\Omega$ to be open (no boundary regularity required), but on the other hand, we don't have any information about the function's behaviour near it.
\end{remark}

\begin{remark}[$\Ckl{k}{\lambda}$ domain]
	Recall we call $\Omega \subset \R^{d}$ to be of class $\Ckl{k}{\lambda}$ if: $\Omega$ is open and bounded, $\exists m \in \N, k \in \N_0, \lambda \in [0,1], \alpha, \beta \in \R^+, \exists \, \text{open sets} \, U_j \subset \R^d, \exists a_j: \text{B}(0,\alpha) \subset \R^{d-1}: \to \R \, \text{\textit{s.t.}} \, a_j \in \CklSet{k}{\lambda}{\text{B}(0,\alpha)}, \exists \tensorq{A}_j \R^{d} \to \R^{d} \, \text{affine orthogonal matrices} \,$ such that 
	\begin{enumerate}
		\item $\partial \Omega \subset \bigcup_{j=1}^m U_j,$
		\item $\forall j \leq m: \emptyset \neq \partial \Omega \cap U_j = \tensorq{A}_j\qty(\qty{(x',a_j\qty(x') \in \R^d | x' \in \text{U}(0,\alpha)\subset \R^{d-1}}),$
		\item $\forall j \leq m: \tensorq{A}_j\qty(\qty{(x', a_j(x') + b) | x' \in \text{U}(0,\alpha), b \in (0,\beta)}) \subset \Omega,$
		\item $\forall j \leq m: \tensorq{A}_j\qty(\qty{(x', a_j(x') - b) | x' \in \text{U}(0,\alpha), b \in (0,\beta)}) \subset \R^d/ \overline{\Omega}.$
\end{enumerate}
	If $\lambda = 0$ we sometimes drop it and write $\Omega \in \Ckl{k}{0} \Leftrightarrow \Omega \in \, \text{C} \,^k,$ if $k = 0, \lambda =1$ we call $\Omega \in \Ckl{0}{1}$ to be a Lipschitz domain.
\textit{Remember that $\lambda\qty(\Omega)<\infty$ is a part of the definition.}

\end{remark}

\begin{theorem}[Global approximation by smooth functions up to the boundary]
	Let $\Omega \in \Ckl{0}{0}$, $k \in \N, p \in [1,\infty)$. Then $\text{C}^{\,\infty}_{\,\overline{\Omega}}\qty(\R^d)$ is dense in $\WkpSet{\Omega}$.


\end{theorem}

\begin{proof}(\textit{From: \cite{bulicekUvodModerniTeorie2018}})
	Let $u \in \WkpSet{\Omega},$ and $\varepsilon >0,$ be given. We wish  to find $v \in \CinfSet{\overline{\Omega}}\, \text{\textit{s.t.}} \,  \norm{u-v}_{\WkpSet{\Omega}} < \varepsilon$.

	The sketch is simple:  
	\begin{enumerate}
		\item covering of $\overline{\Omega},$
		\item partition of unity,
		\item approximation of $u$ on the covering sets,
		\item glue it together.
	\end{enumerate}

	Set $U_0 = \Omega,$ and let $\qty{U_j}_{j=1}^m$ be from the definition of $\Ckl{0}{0}$ boundary. Then\footnote{Our choice $U_0 = \Omega$ is important, as without it the definition of $\Ckl{0}{0}$ boundary only means $\partial \Omega \subset \bigcup_{j=1}^m U_j.$}
	\[
		\overline{\Omega} \subset \bigcup_{j=0}^m U_j,
	\]
	Take $\{\varphi_j\}$ to be the partition of unity on $\overline{\Omega}$, subordinate to $\qty{U_j}_{j=0}^m$. Since
	\[
		u = \sum_{j=0}^m u \varphi_j, \, \text{on} \, \Omega
	\]
	observe that $u_j \coloneq u \varphi_j \in \WkpSet{\Omega}, \supp u_j \subset \supp \varphi_j \subset \subset U_j$. \textbf{Also, we define $u(x) = 0, \forall x \in \R^{d}/\Omega.$} 
	The proofs differs in the cases $j = 0$ and $j \in \qty{1, \dots, m}.$

	\textit{Case $j=0$}.
	We have $ \supp u \varphi_0 \subset \subset U_0 = \Omega.$ That means that after the extension of $u \varphi_0$ by zero outside of $\Omega$, it holds\footnote{This would not hold if the support were not compactly contained in $\Omega$.} $u \varphi_0 \in \WkpSet{\R^d}.$ Since $\WkpSet{\R^d} = \WkpzeroSet{\R^d} = \overline{\DSet{\R^{d}}}^{\norm{\vdot}_{\WkpSet{\R^{d}}}},$ we can find $v_0 \in \DSet{\R^{d}} \, \text{\textit{s.t.}} \,$
	\[
	\norm{v_0 -u_0}_{\WkpSet{\R^{d}}} = \norm{v_0 - u \varphi_0}_{\WkpSet{\Omega}} < \frac{\varepsilon}{m+1}.
	\]
	Notice that naturally
	\[
		\DSet{\R^{d}} \subset \, \text{C} \,^{\infty}_{\overline{\Omega}}\qty(\R^{d}),
	\]
	and so $v_0 \in \, \text{C} \,^{\infty}_{\overline{\Omega}}\qty(\R^{d}).$

	\textit{Case $j \in \{1,\dots,m\} $}.
	We have a problem now: $\qty{U_j}_{j=1}^m$ covers $\partial \Omega,$ which is a \textit{closed} set and we cannot simply use local approximation theorem. One could imagine if we were to mollify in the neighbourhood of $\partial \Omega,$ the kernel would pick up values from outside of $\Omega,$ where $u=0$ and the mollification would not be a good approximation. Instead, we approximate $u_j$ on a larger \textit{open} domain containing $\overline{\Omega}$ and then show this is also a good aproximation of $u_j$ on $\Omega \subset \overline{\Omega}.$

	Set $u_j = u \varphi_j,$ and denote
	\begin{align*}
		S_j &=\tensorq{A}_j\qty(\qty{(x',x_d) | a_j(x') - \frac{\beta}{2}  < x_d < a_j(x') , x' \in \text{U}(0,\alpha)}), \\
	\Omega_j &= \R^{d} / \overline{S_j},
	\end{align*}
	Realize that since $u = 0$ outside of $\Omega,$ also $u_j$ is zero there and in particular it is zero on that "lower strip". Clearly then $u_j \in \WkpSet{\Omega_j}.$ Now pick $\delta \in \qty(0,\frac{\beta}{2}),$ where $\beta$ is from the definition of $\Ckl{0}{0}$ and set
	\begin{align*}
	  		S_{j}^{\delta} &= \tensorq{A}_j\qty(\qty{(x',x_d) | a_j(x') - \frac{\beta}{2} - \delta < x_d < a_j(x') -\delta, x' \in \text{U}(0,\alpha)}), \\
		\Omega_j^{\delta} &= \R^{d} / \overline{S_{j}^{\delta}}, 
	\end{align*}
The trick is to shift the (support of) function $u_j$ "into" $\Omega_{j}^\delta$

	\[
		\tau_{\delta}u_j\qty(\tensorq{A}_j\qty(x', a_j(x'))) = u_j\qty(\tensorq{A}_j\qty(x', a_j(x') + \delta)), x' \in \text{U}(0,\alpha) \subset \R^{d-1}.
	\]
	Realize that in fact
	\[
		\supp\qty(\tau_{\delta}u_j) = \supp\qty(u_j) - \text{B}(0,\delta),
	\]
	from which it follows $\tau_{\delta}u_j \in \WkpSet{\Omega_j^{\delta}};$ we have only shifted the function $u_j$, but since we have also shifted $S_j$, qualitatively there is no difference. Since $\Omega \subset \Omega_j^\delta$ and $\Omega \subset \Omega_j^{\delta} \cap \Omega_j,$ and also $ \Omega \subset \Omega_j,$ $\Omega \subset \Omega_j^{\delta} \cap \Omega_j,$ and the fact $\tau_{\delta}$ is an isometry between Sobolev spaces, we also have $u_j, \tau_{\delta}u_j \in \WkpSet{\Omega_{j} \cap \Omega_j^{\delta}}.$ Moreover, from the properties of the shift operator it follows $\exists \delta >0$ \textit{s.t.}
	\[
		\norm{u_j - \tau_{\delta}u_j}_{\WkpSet{\Omega}}\leq \norm{u_j - \tau_{\delta}u_j}_{\WkpSet{\Omega_j \cap \Omega_j^{\delta}}} < \frac{\varepsilon}{2\qty(m+1)}.
	\]

	We are on a good track. Since we know $\tau_{\delta}u_j$ is already close to $u_j,$ we are done once we approximate $\tau_{\delta}u_j$ by a function from $\CinfSet{\overline{\Omega}}.$ Notice that if we show $\overline{\Omega} \subset \Omega_j^{\delta},$ then clearly $\CinfSet{\Omega_j^{\delta}} \subset \CinfSet{\overline{\Omega}}.$

\textit{Show $\Omega \subset \subset \Omega_j^{\delta}$:}
We already know $\Omega \subset \Omega_j^{\delta},$ so it suffices to show $\partial \Omega \subset \Omega_j^{\delta}.$ Our parametrization of the boundary yields
\[
	\partial \Omega = \bigcup_{k=1}^m \tensorq{A}_k\qty(\qty{(x',x_d)| x_d = a_k(x'), x' \in \text{U}(0,\alpha)}),
\]
and the set $\Omega_j^{\delta}$ is given as $\Omega_{j}^{\delta} = \R^{d} / \overline{S_j},$ where
\[
	S_j = \tensorq{A}_j\qty(\qty{(x',x_d)| a_j(x') - \frac{\beta}{2} - \delta < x_d < a_j(x') - \delta, x' \in \text{U}(0,\alpha)}).
\]
Realize it suffices to show $\partial \Omega \not \subset \overline{S_j}$, as then it wont be excluded from $\R^{d}$ and thus will end up in $\Omega_j^\delta.$ \textit{Thanks to continuity of $a_j$}, we may write
\[
	\overline{S_j} = \tensorq{A}_j\qty(\qty{(x',x_d)| a_j(x') - \frac{\beta}{2} - \delta \leq x_d \leq a_j(x') - \delta, x' \in \text{U}(0,\alpha)}),
\]
\textit{i.e.}, the $"<"$ have changed to $"\leq"$. Since we are doing everything locally, it is enough to show
\[
	\tensorq{A}_j\qty(\qty{(x',x_d)|x_d = a_j(x'), x' \in \text{U}(0,\alpha)}`) \not \subset \tensorq{A}_j\qty(\qty{(x',x_d)| a_j(x') - \frac{\beta}{2} - \delta \leq x_d \leq a_j(x') - \delta, x' \in \text{U}(0,\alpha)}),
\]
which is equivalent to
\[
	\qty((a_j \leq a_j - \delta) \wedge (a_j < a_j - \frac{\beta}{2} - \delta)) \vee \qty((a_j > a_j - \delta) \wedge (a_j \geq a_j - \frac{\beta}{2} - \delta)).
\]
Our choice has been $\delta \in \qty(0, \frac{\beta}{2}),$ and $\beta > 0$ from the definition of $\Omega \in \Ckl{0}{0},$ so the second statement is clearly true $\forall j \in {1, \dots, m}$. Consequently $\partial \Omega \not \subset \overline{S}_j$ which leads to $\partial \Omega \subset \Omega_j^{\delta},$ and since also $\Omega \subset \Omega_j^{\delta},$ we have $\overline{\Omega} \subset \Omega_j^{\delta}.$

\textit{Approximation of $\tau_{\delta}u_j$.}
Since $\Omega_{j}^\delta$ is open, by the local approximation theorem there $\exists v_j \in \CinfSet{\R^{d}}$ such that for any $U \subset \subset \Omega_j^{\delta}$:
\[
	\norm{\tau_{\delta}u_j - v_j}_{\WkpSet[k][p]{U}} \frac{\varepsilon}{2\qty(m+1)},
\]
and so in particular, (as we have shown above $\Omega \subset \overline{\Omega} \subset \Omega_{j}^\delta,$)
\[
	\norm{\tau_{\delta}u_j - v_j}_{\WkpSet{\Omega}}  < \frac{\varepsilon}{2(m+1)}.
\]
Realize that the function $v_j$ is smooth on all $\R^{d},$ so in particular $v_j \in \, \text{C} \,_{\overline{\Omega}}^{\infty}\qty(\R^{d}).$


\textit{Approximation of $u$.}

Finally, let us set
\[
	v = \sum_{j=0}^m v_j.
\]
Then $v \in \, \text{C} \,^{\infty}_{\overline{\Omega}}\qty(\R^{d})$ and it holds
\begin{align*}
	\norm{u-v}_{\WkpSet{\Omega}} &= \norm{\sum_{j=0}^m u_j- \sum_{j=0}^m v_j}_{\WkpSet{\Omega}} = \norm{\sum_{j=0}^m u_j - v_j}_{\WkpSet{\Omega}} \leq \sum_{j=0}^m \norm{u_j - v_j}_{\WkpSet{\Omega}} \leq \\
	& \leq \frac{\varepsilon}{m+1} + \sum_{j=1}^m \norm{v_j-u_j}_{\WkpSet{\Omega}} \leq \frac{\varepsilon}{m+1} + \sum_{j=1}^m \norm{v_j - \tau_{\delta}u_j}_{\WkpSet{\Omega}} + \sum_{j=1}^m \norm{\tau_{\delta}u_j - u_j}_{\WkpSet{\Omega}} < \\
	&< \frac{\varepsilon}{m+1} +  2\sum_{j=1}^m \frac{\varepsilon}{2\qty(m+1)}  = \varepsilon
\end{align*}

\textit{This proof may still have some flaws, but the author has decided to move on.}
\end{proof}	

\begin{remark}[What is $\, \text{C} \,^{\infty}_{\overline{\Omega}}\qty(\R^{d})$]
	Recall
	\[
		C^\infty_{\overline{\Omega}}(\R^d) = \Big\{u|_{\overline{\Omega}}, u \in C^{\infty}(\R^d) \Big\}.
	\]
\end{remark}



\subsection{Extension of Sobolev functions}
\label{sec:extension}

\textit{Problem of extension}: For $u \in \WkpSet{\Omega}$, does there exist
\[
	\overline{u} \in \WkpSet{\R^d}, \, \text{\textit{s.t.}} \,\overline{u}|_{\Omega}=u, \norm{\overline{u}}_{\WkpSet{\R^d}} \leq C(\Omega) \norm{u}_{\WkpSet{\Omega}}
\]

?

The answer is \textbf{yes}, if $\Omega$ is nice enough. Notice however this is not as simple as in the case of Lebesgue spaces, where we could just extend the function by zero trivailly. We are dealing with derivatives, that are somehow regular, and if we extend a nonzero function by zero, it might mess up the regularity of the derivatives.

We will be using $\, \text{C} \,^1$ diffeomorphisms heavily, so we investigate some of their properties first.

\begin{lemma}[Properties of $C^1$ diffeomorphisms]
	Let $U,V \subset \R^d$ be open, $\Phi: U \to V$ be a $C^1$ diffeomorphism and let $\tilde{U} \subset \R^{d}\, \text{\textit{s.t.}} \, \tilde{U} \subset \subset U$. Then
	\begin{enumerate}
		\item $\Phi(\tilde{U}) \subset \subset V,$
		\item if moreover $\tilde{U}$ is compact, then \footnote{For $\tilde{U}$ compact: $\tilde{U} \subset \subset V \Leftrightarrow \tilde{U} \subset V$.} 
			\[
				\exists C>0: \forall u \in C^1\qty(V): \norm{u \circ \Phi}_{\WkpSet[1][p]{\tilde{U}}} \leq C \norm{u}_{\WkpSet[1][p]{\Phi(\tilde{U})}}.
			\]
	\end{enumerate}
\end{lemma}

\begin{proof}(\textit{From: the lectures})

	\textit{Ad 1.}: No proof has been given.


	\textit{Ad 2.}: Just a change of variables formula:
	\[
		\norm{u \circ \Phi}_{\LpSet{\tilde{U}}}^p = \int_{\tilde{U}}|u \circ \Phi|^p\dd{x} = \int_{\Phi\qty(\tilde{U})}|u|^p | \det \grad \Phi|^{-1}\dd{x}.
	\]
	Since $\Phi$ is one-to-one, we know $|\det \grad \Phi| >0$ on $U$, and since $\Phi \in \CkSet{1}{U}$ and $\tilde{U} \subset U \Rightarrow \Phi \in \CkSet{1}{\tilde{U}} \Rightarrow \det \grad \Phi \in \CkSet{0}{\tilde{U}},$ and since $\tilde{U}$ is compact, $|\det \grad \Phi| \geq C_1 > 0 \Leftrightarrow |\det \grad \Phi|^{-1}\leq \frac{1}{C_1}.$ In total	
	\[
		\norm{u \circ \Phi}_{\LpSet{\tilde{U}}}^p \leq \frac{1}{C_1} \int_{\Phi(\tilde{U})}|u|^p\dd{x} = C\norm{u}_{\LpSet{\Phi(\tilde{U})}}^p.
	\]
	As for the derivative, we have $\forall i \in \qty{1,\dots,d}:$
	\begin{align*}
		\int_{\tilde{U}}|\partial_{i}(u \circ \Phi)|^p\dd{x} &\leq \int_{\tilde{U}}|\grad (u \circ \Phi)|^p\dd{x} = \int_{\tilde{U}}| \grad \Phi \Big(\qty(\grad u)\circ \Phi\Big)|^p\dd{x}\leq \\
									 &\leq \norm{\grad \Phi} \int_{\tilde{U}}|(\grad u) \circ \Phi|^p\dd{x} = \norm{\grad \Phi} \int_{\Phi\qty(\tilde{U})}|\grad u|^p |\det \grad \Phi|^{-1}\dd{x} \leq C\norm{\grad \Phi} \int_{\Phi\qty(\tilde{U})}|\grad u|^p\dd{x} \leq \\
										 &\leq C \norm{\grad u}_{\LpSet[p]{\Phi\qty(\tilde{U})}}^p,
	\end{align*}
	where $\norm{\grad \Phi}$ is \textit{e.g.} the operator norm of the matrix $\grad \Phi.$
\end{proof}


\begin{lemma}[Flat extension]
	Let $\alpha, \beta >0, K \subset \subset U(0,\alpha) \times [0,\beta)$ be compact. Then there $\exists C >0 $, a linear operator
	\[
		E:\CkSet{1}{(\text{B}(0,\alpha) \cross [0,\beta])} \to \CkSet{1}{(\text{B}(0,\alpha) \cross [-\beta, \beta])}, 
	\]
	and the set $\tilde{K} \subset \subset \text{U}(0,\alpha) \times [-\beta, \beta)$ such that $\forall u \in \CkSet{1}{\text{B}(0,\alpha) \times [0,\beta]})$ it holds
	\begin{enumerate}
		\item $Eu = u$ on $\text{B}(0,\alpha) \times [0,\beta]$,
		\item $\norm{Eu}_{\WkpSet[1][p]{\text{U}(0,\alpha) \times (-\beta,\beta)}} \leq \norm{u}_{\WkpSet[1][p]{\text{U}(0,\alpha) \times (0, \beta)}} $. Actually,
			\[
				\norm{E}_{\mathcal{L}\qty(\WkpSet[1][p]{\text{U}(0,\alpha)\times (0,\beta)},\WkpSet[1][p]{\text{U}(0,\alpha)\times(-\beta,\beta)})} = 2^{\frac{1}{p}}
			\]
		\item if $\supp u \subset K \, \text{then} \,\supp Eu \subset \tilde{K}$
	\end{enumerate}
\end{lemma}

\begin{proof}(\textit{From: the lectures})
	(The set $\text{U}(0,\alpha) \times [0,\beta)$ is a cylinder of radius $\alpha$ and height $\beta$)

	The proof is constructive: for the assumed $u$ we write ($x = (x',x_d),$ where $x' \in \text{B}(0,\alpha) \subset \R^{d-1}, x_d \in [0,\beta] \subset \R$)

	\begin{equation*}
		Eu(x',x_d) = 
		\begin{cases}
			u(x',x_d),& x_d \geq 0 \\
			-3u(x',-x_d) + 4u(x' ,-\frac{x_d}{2}) ,& x_d < 0.\\
		\end{cases}
	\end{equation*}
	Does $Eu$ lie in $\CkSet{1}{\text{B}(0,\alpha)\times [-\beta, \beta]}$? Since $u \in \CkSet{1}{\text{B}(0,\alpha)\times [0,\beta]}$, it us continuous in the "lower cylinder", check only the transition through the origin plane: take some $a=(x',0) \in \text{B}(0,\alpha)\times \qty{0}$. Then

	\begin{equation*}
		\lim_{x \to a} Eu(x) = 
		\begin{cases}
			u(a), & x_d \geq 0 \\
			-3u(a)+ 4u(a) = u(a), & x_d<0, \\
		\end{cases}
	\end{equation*}
	so $Eu$ is continuous. The derivatives 

	\begin{equation*}
		\lim_{x\to a}\partial_{d} Eu(x',x_d) = 
		\begin{cases}
			\partial_{d}u(x', 0) ,& x_d \geq 0 \\
			\eval{\qty(-3 \partial_{d} u (x',-x_d)(-1) + 4 \partial_{d}u(x',-\frac{x_d}{2}) (-\frac{1}{2}))}_{x_d = 0} = \partial_{d} u(x', 0),& x_d<0,\\
		\end{cases}
	\end{equation*}
	and also for any $i \in \qty{1, \dots, d-1}$ 
	\begin{equation*}
	  \lim_{x\to a}\partial_{i} Eu(x_1, \dots, x_d) = 
	  \begin{cases}
		  = \partial_{i} u (x_1, \dots, 0) , & x_d \geq 0 \\
		  = -3 \partial_{i} u(x_1, \dots, 0) + 4 \partial_{i} u(x_1, \dots, 0) = \partial_{i} u(x_1, \dots, 0)
	  \end{cases}
	\end{equation*}
	
	so the the derivative is also continuous. Thus, we have
	\[
		Eu  \in C^1 \subset \WkpSet[1][p]{U(0,\alpha) \cross (-\beta,\beta)}.
	\]
	The first property is clear from the definition of $Eu$, the estimates of the norm: (all derivatives can in fact be assumed classical)
	\begin{align*}
		\norm{Eu}_{\WkpSet[1][p]{\text{U}(0,\alpha) \times (-\beta,\beta)}}^p &= \norm{Eu}_{\LpSet{\text{U}(0,\alpha)\times (-\beta,\beta)}}^p + \sum_{|\alpha| = 1}\norm{D^{\alpha}Eu}_{\LpSet{\text{U}(0,\alpha)\times (-\beta, \beta)}}^p = \\
		&= \norm{Eu}_{\LpSet{\text{U}(0,\alpha) \times (0,\beta)}}^p +\norm{Eu}_{\LpSet{\text{U}(0,\alpha) \times (-\beta,0)}}^p \\ +
		& + \sum_{|\alpha| = 1}\norm{D^{\alpha}Eu}_{\LpSet{\text{U}(0,\alpha)\times (0, \beta)}}^p + \sum_{|\alpha| = 1}\norm{D^{\alpha}Eu}_{\LpSet{\text{U}(0,\alpha)\times (-\beta, 0)}}^p = \\
		& = \norm{u}_{\LpSet{\text{U}(0,\alpha)\times (0,\beta)}}^p + \norm{4u - 3u}_{\LpSet{\text{U}(0,\alpha)\times (-\beta, 0)}}^p + \\
		& + \sum_{|\alpha| = 1}\norm{D^{\alpha}u}_{\LpSet{\text{U}(0,\alpha)\times (0,\beta)}}^p + \sum_{|\alpha| =1}\norm{D^{\alpha}\qty(4u - 3u)}_{\LpSet{\text{U}(0,\alpha)\times (-\beta,0)}}^p \\
		&= 2 \norm{u}_{\LpSet{\text{U}(0,\alpha)\times (0,\beta)}}^p + 2 \sum_{|\alpha| =1}\norm{D^{\alpha}u}_{\LpSet{\text{U}(0,\alpha)\times (0,\beta)}}^p = 2 \norm{u}_{\WkpSet[1][p]{\text{U}(0,\alpha)\times (0,\beta)}}^p
	\end{align*}
	and so
	\[
		\norm{Eu}_{\WkpSet[1][p]{\text{U}(0,\alpha)\times (-\beta,\beta)}} = 2^{\frac{1}{p}} \norm{u}_{\WkpSet[1][p]{\text{U}(0,\alpha)\times (0,\beta)}}.
	\]
	Where we have used the obvious fact
	\[
		\int_{\text{U}(0,\alpha)\times (0,\beta)}|f|^p\dd{x} = \int_{\text{U}(0,\alpha)\times (-\beta,0)}|f|^p\dd{x},
	\]

	We will assume (although this is \textit{with} a loss of generality), that
	\[
		Eu\qty(x', x_d) = 0 \Leftrightarrow (u(x', x_d) = 0 \vee 3u(x',-x_d) = 0 \vee 4u(x', - \frac{x_d}{2})= 0),
	\]
	\textit{i.e.}, that support lies in
	\begin{align*}
		\supp Eu &\subset\qty{(x',x_d) \in \text{U}(0,\alpha) \times [0, \beta] | u(x', x_d) = 0} \cup \qty{(x',x_d) \in \text{U}(0,\alpha) \times [-\beta,0]| u(x', - x_d) \neq 0}\cup \\
			 &\cup \qty{(x',x_d) \in \text{U}(0,\alpha) \times [-\beta,0]| u(x', - \frac{x_d}{2}) \neq 0}.
	\end{align*}
	Denote $\Phi_1, \Phi_2: \text{U}(0,\alpha) \times (0,\beta) \to \text{U}(0,\alpha) \times (-\beta,\beta)$ to be the mappings
	\begin{align*}
		\Phi_1 (x',x_d) &= (x', -x_d),\\
		\Phi_2 (x',x_d) &= (x', - \frac{x_d}{2}),
	\end{align*}
	then clearly $\Phi_1, \Phi_2$ are $\, \text{C} \,^1$ diffeomorphisms and
	\[
		u(x',-x_d) = \qty(u \circ \Phi_1)(x',x_d), u\qty(x', - \frac{x_d}{2}) = \qty(u \circ \Phi_2)\qty(x',x_d),
	\]
	\textit{i.e.},
	\[
		u\qty(x',x_d) = \qty(u \circ \Phi_1^{-1})\qty(x', -x_d) = \qty(u \circ \Phi_2^{-1})\qty(x',-\frac{x_d}{2}),
	\]
	and so
	\[
		\supp Eu \subset \supp u \cup \Phi_1^{-1}\qty(\supp u) \cup \Phi_2^{-1}\qty(\supp u) \subset K \cup \Phi_1^{-1}\qty(K) \cup \Phi_2^{-1}\qty(K),
	\]
	as $\supp u \subset \subset K.$ Let us define
	\[
		\tilde{K} \coloneq K \cup \Phi_1^{-1}\qty(K) \cup \Phi_2^{-1}\qty(K),
	\]
	Then we see
	\[
		\supp Eu \subset K \cup \Phi_1^{-1}\qty(K) \cup \Phi_2^{-1}\qty(K) = \tilde{K},
	\]
	And, finally, we have $K \subset \subset \text{U}(0,\alpha) \times [0,\beta) \Rightarrow K \subset \subset \text{U}(0,\alpha) \times (-\beta,\beta) \Rightarrow \Phi_1^{-1}\qty(K), \Phi_2^{-1}\qty(K)\subset \subset \text{U}(0,\alpha)\times (0,\beta),$ which really implies
	\[
		\tilde{K} \subset \subset \text{U}(0,\alpha) \times [-\beta, \beta).
	\]
\end{proof}

Let us prove the main result.

\begin{theorem}[Extension of Sobolev functions]
	Let $\Omega \in C^{k-1,1}, k \in \N, p \in [1,\infty], V \subset \R^d$ open such that $\Omega \subset \subset V.$ Then there is $E: \WkpSet{\Omega} \to \WkpSet{\R^d}$ bounded linear operator such that
	\begin{enumerate}
		\item $ \forall u \in \WkpSet{\Omega}: Eu=u \, \text{\textit{a.e.}} \,\, \text{in} \, \Omega$,
		\item $ \forall u \in \WkpSet{\Omega}: \supp Eu \subset V \supset \supset \Omega$,
		\item $\norm{E}_{\mathcal{L}\qty(\WkpSet{\Omega},\WkpSet{\R^{d}})} \leq C, C=C(p,\Omega,V).$
	\end{enumerate}
\end{theorem}

\begin{proof}(\textit{From: the lectures, with some incompletions})
	\textbf{Will only be presented for $k=1, \Omega \in C^1, p < \infty$.} The strategy is:
	\begin{enumerate}
		\item covering of $\overline{\Omega}$ \& partition of unity
		\item obtain a diffeomorphism from the fact $\Omega \in \Ckl{1}{0},$
		\item suitable composition \& cut off,
		\item flat extension,
		\item show existence of $E: \, \text{C} \,^{\infty}_{\overline{\Omega}}\qty(\R^{d})\to \WkpSet{\R^{d}}$ with the desired properties,
		\item extend via density
	\end{enumerate}

	\textit{Covering of $\Omega$:}
	In the following,  $U_j,a_j, \tensorq{A}_j, \alpha, \beta$ are as in the definition of a $\Ckl{1}{0}$ domain. Set $U_0 = \Omega$ and realize
	\[
		\overline{\Omega} \subset \bigcup_{j=0}^m U_j,
	\]
	\textit{i.e.}, $\qty{U_j}_{j=0}^m$ is an open covering of $\overline{\Omega}$. Denote $\{\varphi_j \}_{j=0}^m$ as the partition of unity subordinate to $\{U_j\}_{j=0}^m.$


	\textit{Diffeomorphism \& flat extension}
	For $j \in \{1, \dots, m\}$ we define $\Phi_j : \text{U}(0,\alpha) \cross (-\beta,\beta) \to U_j$ by
	\[
		\Phi_j\qty(y',y_d) = \tensorq{A}_j (y', a_j(y') + y_d), y' \in \R^{d-1}, y_d \in \R.
	\]

	(A bit confusingly, we will however be "interpreting" $\Phi_j$ as $\Phi_j : \text{U}(0,\alpha) \times \qty(-\beta, \beta) \to \R^{d},$ with it being extended by zero on $\R^{d}/ \overline{U_j},$ as we need $\Phi_j^{-1}$ to be defined on the whole $\R^{d}.$)

	As $\Omega \in \Ckl{1}{0},$ we know $a_j \in \CkSet{1}{\text{B}(0,\alpha)}\subset \CkSet{1}{\text{U}(0,\alpha)},$ and so $\phi_j$ is a $\text{C}^1$ diffemorphism. Let us denote by $\tilde{E}: \CkSet{1}{\text{B}(0,\alpha)\times [-\beta, \beta]} \to \CkSet{1}{\text{B}(0,\alpha)\times [-2 \beta, \beta]}$ the extension operator from the Flat extension lemma. Then we for any $u \in C^{\infty}_{\overline{\Omega}}\qty(\R^d): u = \sum_{j=1}^m \varphi_j u$ define
	\[
		Eu=\varphi_0 u + \sum_{j=1}^m \qty(\eta \tilde{E}\qty((\varphi_j u) \circ \Phi_j)) \circ \inverse{\Phi_j},
	\]
	where $\eta \in \CinfSet{\text{U}(0,\alpha)\times \R}$ is a cut-off function
	\begin{equation*}
		\eta\qty(y',y_d)  
		\begin{cases}
			=1 \, \text{on} \, &y_d \geq 0,\\
			= 0 \, \text{on} \,& y_d \leq -2h, \\
			\in (0,1) \, &\text{else} \,.
		\end{cases},
	\end{equation*}
for some parameter $h>0$ which will be defined later.
With this definition, $\forall x \in \Omega$ it holds
\begin{align*}
	\qty(Eu)(x) &= \qty(\varphi_0 u)(x) + \sum_{j=1}^{m}\qty(\eta \tilde{E}\qty(u \varphi_j \circ \Phi_j))\underbrace{\qty(\Phi_j^{-1}(x))}_{\in \text{U}(0,\alpha)\times \qty(-\beta, \beta)} =\\
	&= \qty(\varphi_0 u)(x) + \sum_{j=1}^m \qty(u \varphi_j \circ \Phi_j)\qty(\Phi_j^{-1}(x)) = \qty(\varphi_0 u)(x) + \sum_{j=1}^m \qty(u \varphi_j)(x) = \sum_{j=0}^m \qty(u \varphi_j)(x) = u(x) + \sum_{j=0}^m \varphi_j(x) = u(x),
\end{align*}
since $\eta(y) = 1, \tilde{E}\qty(\varphi_j u \circ \Phi_j) (y) = \qty(\varphi_j u \circ \Phi_j)(y)$ for $y \in \text{U}(0,\alpha) \times (-\beta, \beta), $ according to our definition of $\eta$ and the properties of the extension $\tilde{E}.$

The motivation behind the cutoff is the following: we know $\supp \varphi_j \subset U_j,$ so since $\Phi_j: \text{U}(0,\alpha)\times(-\beta,\beta) \to U_j, $ we have $\supp u \varphi_j \circ \Phi_j \subset \text{U}(0,\alpha)\times (-\beta, \beta)$ and from the properties of the flat extension operator we also have $\supp \tilde{E}\qty(u \varphi_j \circ \Phi_j) \subset \text{U}(0,\alpha)\times (-2 \beta, \beta)$. Since moreover $\supp \eta = \text{U}(0,\alpha) \times (-2h, \infty),$ in total
\[
 \supp \eta \tilde{E}\qty(u \varphi_j \circ \Phi_j) = \supp \eta \cap \supp \tilde{E}\qty(u \varphi_j \circ \Phi_j) = \text{U}(0,\alpha) \times (-2h, \infty) \cap (-\beta, \beta),
\]
and also \footnote{Generally, $\supp v \circ \Psi^{-1} = \overline{\qty{x \in \R^{d}| u\qty(\Psi^{-1}(x)) \neq 0}} = \overline{\qty{x \in \R^{d}| \Psi^{-1}(x) \in \supp u}} = \overline{\qty{x \in \R^{d}| x \in \Psi\qty(\supp u)}} = \Psi\qty(\supp u).$}
\[
	\supp Eu \subset \Phi_j\qty(\supp \eta \tilde{E}\qty(u \varphi_j \circ \Phi_j)).
\]
We need to prove $\supp Eu \subset V,$ where $V$ is some neighbourhood of $\Omega$. If it holds
\[
	\supp \eta \tilde{E}\qty(u \varphi_j \circ \Phi_j) \subset \text{U}(0,\alpha) \times (-\beta,\beta),
\]
the desired property holds, as then
\[
	\supp Eu \subset \supp u_0 \cup \bigcup_{j=1}^m\Phi_j \qty( \supp \eta \tilde{E}\qty(u \varphi_j \circ \Phi_j)) \subset U_0 \cup \bigcup_{j=1}^m \Phi_j\qty(\text{U}(0,\alpha)\times (-\beta, \beta)) \subset \bigcup_{j=0}^m U_j,
\]
and from the assumptions $\overline{\Omega} \subset \bigcup_{j=0}^m U_j,$ and the union is open. Meaning, the support is contained within an open set, in which $\Omega$ is compactly contained.\footnote{This might not exactly be the formulation of the theorem, but is pretty close. But come on, we are already proving something different then the original formulation...}
In total, if
\[
	\supp \eta \tilde{E}\qty(u \varphi_j \circ \Phi_j) = \text{U}(0,\alpha) \times (-2h, \infty) \cap (-\beta, \beta) \subset \text{U}(0,\alpha) \times (-\beta,\beta).
\]
So if $h$ is such that $-2h > -\beta \Leftrightarrow h < \frac{\beta}{2}$, we can guarantee $\supp Eu \subset \bigcup_{j=0}^m U_j$ and that is what we want.

Finally, $E$ is clearly linear, its norm: (we are using the lemma about flat extension and the properties of $\, \text{C} \,^1$ diffeomorphisms together with the facts $ \eta \leq 1 \, \text{on} \, \text{U}(0,\alpha) \times (-\beta, \beta),$ $\Phi_j\qty(\text{U}(0,\alpha)\times (0,\beta)) \subset U_{j},$$ \Phi_j^{-1}\qty(\R^{d}) \subset \footnote{We can somehow extend $\Phi_j^{-1}$ by zero from $U_j \subset \Omega \subset \R^{d}$ to be defined on the whole $\R^{d}$ (i guess)}, \text{U}(0,\alpha)\times \qty(-\beta,\beta)$, \footnote{...even though the assumptions to use those are not totally valid... but doc. Kaplicky is okay with that})

\begin{align*}
\norm{Eu}_{\WkpSet[1][p]{\R^{d}}} &= \norm{\qty(\eta \tilde{E}\qty(\varphi u_j \circ \Phi_j))\circ \Phi_j^{-1}}_{\WkpSet[1][p]{\R^{d}}} \leq C\norm{\eta \tilde{E} \qty(\varphi u_j \circ \Phi_j)}_{\WkpSet[1][p]{\text{U}(0,\alpha)\times (-\beta,\beta)}} = \\
				  &= C \norm{\tilde{E}\qty(u \varphi_j \circ \Phi_j)}_{\WkpSet[1][p]{\text{U}(0,\alpha)\times (-\beta, \beta)}} \leq C \norm{u \varphi_j \circ \Phi_j}_{\WkpSet[1][p]{\text{U}(0,\alpha)\times (0,\beta)}} \leq \\
				  &\leq C \norm{u \varphi_j}_{\WkpSet[1][p]{U_j \cap \Omega}} \leq C \norm{u}_{\WkpSet[1][p]{\Omega}},
\end{align*}
from which it clearly follows $\norm{E}_{\mathcal{L}\qty(\WkpSet[1][p]{\Omega}, \WkpSet[1][p]{\R^d})} \leq C.$

	So all the properties hold for $u \in C^{\infty}_{\overline{\Omega}}\qty(\R^d)$, let us show them also for $u \in \WkpSet[1][p]{\Omega}$.
	Pick an arbitrary $u \in \WkpSet[1][p]{\Omega}, \, \text{find} \, \{u_k\} \subset C^{\infty}_{\overline{\Omega}}\qty(\R^d): u_k \to u \, \text{in} \, \WkpSet[1][p]{\Omega}.$ Since $E$ is continous, we know $\lim_{k\to \infty}E u_k$ exists. Let us set
	\[
		E u \coloneq \lim_{k\to \infty}E u_k,
	\]
	where $Eu_k$ is defined above for $u_k \in \, \text{C} \,^{\infty}_{\overline{\Omega}}\qty(\R^{d}).$

	Ad 3): Clearly $E \qty(\alpha u) = \lim_{k \to \infty}E \qty(\alpha u_k) = \alpha E u, E \qty(u + v) = \lim_{k\to \infty}E\qty(u_k + v_k) = \lim_{k \to \infty}Eu_k + Ev_k = Eu + Ev,$ since $E$ on the continous functions is linear. Also


	\[
		\norm{Eu}_{\WkpSet[1][p]{\R^{d}}} = \norm{\lim_{k\to \infty}E u_k}_{\WkpSet[1][p]{\R^{d}}} = \lim_{k\to \infty} \norm{E u_k}_{\WkpSet[1][p]{\R^{d}}} \leq \norm{E} \lim_{k\to \infty}\norm{u_k}_{\WkpSet[1][p]{\R^{d}}} = \norm{E} \norm{u}_{\WkpSet[1][p]{\R^{d}}},
	\]
	(we are using $\qty{E u_k}$ has a limit); we see our above definition truly yields a continuous linear operator.\\
	Ad 1): $\forall \, \text{\textit{a.a.}} \, x \in \Omega: Eu(x) = \lim_{k\to \infty}Eu_k(x) = \lim_{k \to \infty}u_k(x) = u(x),$\\
	Ad 2): $\supp Eu_k \subset \text{U}(\Omega,\varepsilon) \Rightarrow \supp Eu \subset \overline{\text{U}(\Omega,\varepsilon)} \subset V.$
	\[
		\supp Eu = \qty{x \in \R^{d}| \lim_{k\to \infty}Eu_k \neq 0}\subset \bigcap_{k=1}^{\infty}\underbrace{\supp Eu_k}_{\subset V} \subset V.
	\]
	We are done.

\end{proof}

\begin{remark}[$\Omega \in C^{0,1}$ suffices]
	The theorem is still valid if we assume only $\Omega \in \Ckl{0}{1}$ and $p \in (1,\infty), k \in \N,$ but the construction of the extension must be different. "It seems the result is not known for $\Omega \in \Ckl{0}{1}$ and $p = 1,$ or $p = \infty.$"
\end{remark}

\subsection{Embedding theorems}
\label{sec:embedding}

From the definition of $\WkpSet{\Omega}$ it immediately follows $\WkpSet{\Omega} \subset \LpSet{\Omega}.$ Can we obtain $\WkpSet{\Omega} \subset \LpSet[q]{\Omega}$ for some $q > p?$ The answer \textbf{yes}, if $\Omega$ is again nice enough (and there will also be some dependence on the dimension of $\R^{d}.$)

\subsubsection{Theorems for $p \leq d$}
\label{sec:embedding_p_leq_q}

\begin{example}
	Let $u \in \mathcal{D}\qty(\R^2)$. Then 

	\begin{equation*}
		u\qty(x_1,x_2) = \int_{-\infty}^{x_1} \partial_{1}u\qty(s,x_2)\dd{s} = \int_{-\infty}^{x_2} \partial_{2}u(x_1,s)\dd{s},
	\end{equation*}
	so 
	\begin{align*}
		\norm{u}_{\LpSet[2]{\R^{2}}}^{2} = \int_{\R^2}|u(x_1,x_2)|^{2}\dd{x_1} \dd{x_2} &= \int_{\R}\int_{\R}|u(x_1,x_2)|^{2}\dd{x_1}\dd{x_2}\leq \\
		&\leq \qty(\int_{\R}\int_{-\infty}^{x_1}|\partial_{1}u(s,x_2)|\dd{s}\dd{x_2}) \qty(\int_{\R}\int_{-\infty}^{x_2}|\partial_{2} u(x_1,s)|\dd{s}\dd{x_2}) \leq \\
		&\leq \qty(\int_{\R}\int_{\R}|\partial_{1}u(s,x_2)|\dd{s}\dd{x_2})\qty(\int_{\R}\int_{\R}|\partial_{2}u(x_1,s)|\dd{s}\dd{x_2}) \leq \\
		& \leq \qty(\int_{\R}\int_{\R}|\grad u(s,x_2)|\dd{s}\dd{x_2})\qty(\int_{\R}\int_{\R}|\grad u(x_1,s)|\dd{s}\dd{x_2}) = \\
		& = \qty(\int_{\R^2}|\grad u|\dd{x}) \qty(\int_{\R^{2}}|\grad u|\dd{x}) = \qty(\int_{\R^{2}}|\grad u|\dd{x})^{2} = \\
		& = \norm{\grad u}_{\LpSet[1]{\R^{2}}}^{2},
	\end{align*}
	so
	\[
		\norm{u}_{\LpSet[2]{\R^{2}}} \leq \norm{\grad u}_{\LpSet[1]{\R^{2}}}.
	\]
\end{example}

This can be generalized in two ways: 
\begin{itemize}
	\item $d > 2,$
	\item less smoothness.
\end{itemize}

\begin{lemma}[Gagliardo]
	Let $d\geq 2$. Let $\hat{u}_i:\R^{d-1} \to \R$ be nonnegative and measurable for $j \in \{1,\dots,d\}.$ We define
	\[
		\hat{x}_j = (x_1, \dots, x_{j-1}, x_{j+1}, \dots, x_d), \hat{\dd{x}_j} = \dd{x}_1 \dots \dd{x}_{j-1} \dd{x}_{j+1} \dots \dd{x}_d.
	\]
	Consider the functions $u_j:\R^d \to \R, u_j\qty(x)=\hat{u}_j(\hat{x}_j).$ Then
	\begin{equation}
		\int_{\R^d}\prod_{j=1}^d u_j\qty(x)\dd{x} \leq \prod_{j=1}^d \qty(\int_{\R^{d-1}}\qty(\hat{u}_j\qty(\hat{x}_j))^{d-1}\hat{\dd{x}_j})^{\frac{1}{d-1}}.
	\end{equation}
	(Both integrals can be infinity.)
\end{lemma}

\begin{proof}(\textit{From: the lectures})
	Induction by dimension:
	\textit{The case $d = 2$.}:
	\begin{align*}
		\int_{\R^{2}}u_1\qty(x_1,x_2)u_2\qty(x_1,x_2)\dd{x}_1 \dd{x}_2 &= \int_{\R^{2}}\hat{u}_1\qty(x_2) \hat{u}_2\qty(x_1)\dd{x}_1 \dd{x}_2 = \qty(\int_{\R}\hat{u}_1\qty(x_2)\dd{x}_2) \qty(\int_{\R}\hat{u}_2\qty(x_1)\dd{x}_1) = \\
		&=\qty(\int_{\R}\hat{u}_1\qty(\hat{x}_1)\hat{\dd{x}_1})\qty(\int_{\R}\hat{u}_2\qty(\hat{x}_2)\hat{\dd{x}_2}).
	\end{align*}
	(an equality in fact.) We have used Fubini once, which is permitted, as we have measurability + nonnegativity.

	\textit{The case $d \to d+1$}
	Before we procceed, recall the "generalized Holder", all functions are nonnegative

	\[
		\int_{\Omega}\prod_{j=1}^d f_j\dd{x} \leq \prod_{j=1}^d \qty(\int_{\Omega}f_j^{p_j}\dd{x})^{\frac{1}{p_j}},
	\]
	where $\sum_{j=1}^d \frac{1}{p_j}  = 1.$ See that if we take $p_j = d,$ then $\sum_{j=1}^d \frac{1}{d} = \frac{1}{d} \sum_{j=1}^d 1 =1,$ so
	\[
		\int_{\Omega}\prod_{j=1}^d f_j \dd{x} \leq \prod_{j=1}^d\qty(\int_{\Omega}f_j^d\dd{x})^{\frac{1}{d}}.
	\]
	Let us compute:
	\begin{align*}
		\int_{\R^{d+1}}\prod_{j=1}^{d+1} u_j\qty(x)\dd{x}&=\int_{\R^d}\int_{\R}\prod_{j=1}^d u_j \overbrace{u_{d+1}}^{=\hat{u}\qty(\hat{x}_{d+1})}\overbrace{\dd{x}_1 \dots \dd{x}_d}^{=\hat{\dd{x}}_{d+1}}\dd{x}_{d+1} = \int_{\R^{d}}\underbrace{\qty(\int_{\R}\prod_{j=1}^d u_j\dd{x}_{d+1})}_{\text{Holder}}u_{d+1}\hat{\dd{x}}_{d+1} \leq \\
								 &\leq \int_{\R^d}\underbrace{\qty(\prod_{j=1}^d \int_{\R}u_j^d\dd{x_{d+1}})^{\frac{1}{d}}u_{d+1}}_{\text{Holder}}\hat{\dd{x_{d+1}}} \leq \qty(\int_{\R^{d}}u_{d+1}^d\dd{\hat{x}_{d+1}})^{\frac{1}{d}}\qty(\int_{\R^d}\qty(\prod_{j=1}^d \int_{\R}u_j^d(x)\dd{x_{d+1}})^{\frac{1}{d}\frac{d}{d-1}}\dd{\hat{x}_{d+1}})^{\frac{d-1}{d}} \leq \\
								 &\leq \qty(\int_{\R^d}u_{d+1}^d\dd{\hat{x}_{d+1}})^{\frac{1}{d}}\qty(\prod_{j=1}^d\int_{\R^{d-1}}\qty(\int_{\R}u_j^d\dd{x_{d+1}})^{\frac{1}{d-1}\frac{d-1}{1}}\dd{\hat{x}_j}\dd{\hat{x}_{d+1}})^{\frac{d-1}{d} \frac{1}{d-1}} = \\
								 &= \qty(\int_{\R^{d}}u_{d+1}^d \dd{\hat{x}_{d+1}})^{\frac{1}{d}}\prod_{j=1}^d\qty(\int_{\R^{d-1}}\int_{\R}u_j^d\dd{x}_{d+1}\dd{\hat{x}_{j}}\dd{\hat{x}_{d+1}})^{\frac{1}{d}} = \qty(\int_{\R^{d}}u_{d+1}^d\dd{\hat{x}_{d+1}})^\frac{1}{d} \prod_{j=1}^d \qty(\int_{\R^{d}}u_j^d\dd{\hat{x}_j})^{\frac{1}{d}} = \\
								 &= \prod_{j=1}^{d+1} \qty(\int_{\R^{d}}u_j^d\dd{\hat{x}_j})^{\frac{1}{d}},
	\end{align*}
	where the induction step is taken for the function
	\[
		v_j=\qty(\int_{\R}u_j^d\dd{x_{d+1}})^{\frac{1}{d-1}},
	\]
	that is clearly nonnegative and measurable.
\end{proof}

\begin{remark}
	Sometimes, the lemma is stated as: $\hat{u}_i \in \LinfSet{\R^{d-1}}, \supp \hat{u}_i$ is compact $\forall i \in \qty{1,\dots,d}.$ Then
	\[
		\int_{\R^{d}}\prod_{j=1}^d|u_j(x)|\dd{x} \leq \prod_{j=1}\qty(\int_{\R^{d-1}}|\hat{u}_j(\hat{x}_j)|^{d-1}\hat{\dd{x_i}})^{\frac{1}{d-1}} = \prod_{j=1}^d \norm{\hat{u}_j}_{\LpSet[d-1]{\R^{d-1}}}.
	\]
	The difference is that in our version, we have nonnegativity in the assumptions and do not requiry compact supports and essential boundedness, as we work with integrals that are possibly infinite.
\end{remark}

\begin{theorem}[Gagliardo-Nirenberg]
	Let $p \in [1,d).$ Then $\forall u \in \WkpSet[1][p]{\R^d}$:
	\[
		\norm{u}_{\LpSet[p^{*}]{\R^d}}\leq \frac{p(d-1)}{d-p} \norm{\grad u}_{\LpSet{\R^d}}, 
	\]
	where $p^{*}=\frac{dp}{d-p}.$
\end{theorem}

\begin{proof}(\textit{From: the lectures})
	Estimate for $u \in \mathcal{D}\qty(\R^d),$ then use density, as $\WkpSet[1][p]{\R^{d}} = \WkpzeroSet[1][p]{\R^{d}}.$
	\[
		\forall j \in \{1,\dots,d\}, x \in \R^d: u\qty(x)=\int_{-\infty}^{x_j}\partial_{j}u\qty(x_1,\dots,x_{j-1},s,x_{j+1},\dots,x_d)\dd{s}.
	\]
	This estimate is independent of $j \in \qty{1, \dots,d},$ so it holds 
	\[
		|u\qty(x)| \leq \int_{\R}|\grad u|\qty(\dots,s,\dots)\dd{s}.
	\]
	Next, consider $p=1, p^{*}=\frac{d}{d-1}$ and estimate:
	\[
		|u|^{\frac{d}{d-1}}\leq \qty(\int_{\R^{d}}|\grad u| \qty(\dots, s, \dots)\dd{s})^{\frac{d}{d-1}} = \prod_{j=1}^d\qty(\int_{\R}|\grad u| (\dots, s, \dots) \dd{s})^{\frac{1}{d-1}}.
	\]
	Denote
	\[
		u_j(x_1, \dots, x_{j-1}, x_{j+1}, \dots, x_d) = \qty(\int_{\R^{d}}|\grad u|\overbrace{\qty(\dots, x_j, \dots)}^{=x \, \text{in fact} \,}\dd{x_j})^{\frac{1}{d-1}},
	\]
	which is a function independent of $x_j, u_j \equiv u_j\qty(\hat{x}_j).$
	So the integral (the $\LpSet[\frac{d}{d-1}]{\R^{d}}$ norm)
	\begin{align*}
		\int_{\R^d}|u|^{\frac{d}{d-1}}\dd{x} &\leq \int_{\R^d}\prod_{j=1}^d u_j \dd{x} \underbrace{\leq}_{\text{Gagliardo lemma}}\qty(\prod_{j=1}^d \int_{\R^{d-1}}u_j^{d-1}\qty(\hat{x}_j)\dd{\hat{x}_j})^{\frac{1}{d-1}}=\\
						     &=\qty(\prod_{j=1}^d \int_{\R^{d-1}}\int_{\R}|\grad u|\qty(x)\dd{x_j}\dd{\hat{x}_j})^{\frac{1}{d-1}} = \qty(\int_{\R^d}|\grad u|\dd{x})^{\frac{d}{d-1}},
	\end{align*}
	and so
	\[
		\norm{u}_{\LpSet[\frac{d}{d-1}]{\R^{d}}}^{\frac{d}{d-1}}\leq \norm{\grad u}_{\LpSet[1]{\R^{d}}}^{\frac{d}{d-1}},
	\]
	meaning ($1^{*} = \frac{d}{d-1})$
	\[
		\norm{u}_{\LpSet[1^{*}]{\R^{d}}}\leq 1 \norm{\grad u}_{\LpSet[1]{\R^{d}}}.
	\]
	If now $p \in (1,d),$ we investigate for what $q$ can we obtain estimate of $\norm{|u|^q}_{\LpSet[\frac{d}{d-1}]{\R^{d}}}$:
	\begin{align*}
		\norm{u}_{\LpSet[\frac{qd}{d-1}]{\R^d}}^q = \norm{|u|^q}_{\LpSet[\frac{d}{d-1}]{\R^d}} \leq \norm{\grad (|u|^q)}_{\LpSet[1]{\R^d}} = \int_{\R^d}q|u|^{q-1} |\grad u| \dd{x} \underbrace{\leq}_{\, \text{Holder} \,}&q \norm{\grad u}_{\LpSet[p]{\R^{d}}}\norm{|u|^{q-1}}_{\LpSet[p']{\R^{d}}}= \\
		=&q \norm{\grad u}_{\LpSet[p]{\R^d}} \norm{u}_{\LpSet[(q-1)p']{\R^d}}^{q-1}.
	\end{align*} We want $(q-1)p' = \frac{qd}{d-1}$, so we can divide both sides:
	\[
		q \qty(\frac{p}{p-1} - \frac{d}{d-1}) = \frac{p}{p-1}, \Leftrightarrow q\frac{pd-p-pd+d}{(p-1)(d-1)}=\frac{d-p}{\qty(p-1)\qty(d-1)}=\frac{p}{p-1} \Leftrightarrow q=\frac{d-1}{d-p}p.
	\]
	Also
	\[
	\frac{q}{d-1} = \frac{p}{d-p} \Leftrightarrow \frac{qd}{d-1} = \frac{dp}{d-p} = p^{*},	q = \frac{p\qty(d-1)}{d-p}
	\]
	and thus
	\[
		\norm{u}_{\LpSet[p^{*}]{\R^{d}}}^q \leq q \norm{\grad u }_{\LpSet[p]{\R^{d}}} \norm{u}_{\LpSet[p^{*}]{\R^{d}}}^{q-1} \Leftrightarrow \norm{u}_{\LpSet[p^{*}]{\R^{d}}}\leq \frac{p\qty(d-1)}{d-p}\norm{\grad u}_{\LpSet[p]{\R^{d}}}.
	\]
	$\Rightarrow$ statement holds for $u \in \mathcal{D}\qty(\R^d)$. To finish, use density of $\mathcal{D}\qty(\R^d) \, \text{in} \, \WkpSet[1][p]{\R^d}:$ let $u \in \WkpSet[1][p]{\R^{d}},$ be arbitrary. Then $\exists \qty{u_k} \subset \DSet{\R^{d}}: u_k \to u \, \text{in} \,\WkpSet[1][p]{\R^{d}}.$ Moreover, we have showed that $\forall k \in \N:$
	\[
		\norm{u_k}_{\LpSet[p^{*}]{\R^{d}}} \leq \frac{p\qty(d-1)}{d-p}\norm{\grad u}_{\LpSet[p]{\R^{d}}},
	\]
	so passing to the (strong) limit and using the continuity of the norm indeed yields

	\[
		\norm{u}_{\LpSet[p^{*}]{\R^{d}}} \leq \frac{p\qty(d-1)}{d-p}\norm{\grad u}_{\LpSet[p]{\R^{d}}}.
	\]
	We are done.


\end{proof}


\begin{remark}

	\begin{itemize}
		\item It is evident that nonzero constants are not in $\WkpSet[1][p]{\R^d}$ and that also the inequality does not hold for them.
		\item the set $\R^d$ is of course unbounded, so we have no ordering of $\LpSet{\Omega}$ spaces.
		\item of course, we require no smoothness of the domain
	\end{itemize}


\end{remark}

\begin{theorem}
	Let $\Omega \subset \R^d$ be open. Then $\forall u \in \WkpzeroSet[1][p]{\Omega}, \forall p \in [1,d)$ the statement of the previous theorem holds.
\end{theorem}
\begin{proof}(\textit{From: the lectures})
	An immediate corollary of the previous theorem: we have showed the inequality for $u \in \DSet{\R^{d}},$ but WLOG it holds also for $u \in \DSet{\Omega}$ (i can keep the integrals over $\R^{d}$, but in the end only values from $\Omega$ count) and since $\WkpzeroSet[1][p]{\Omega} = \overline{\DSet{\Omega}}^{\norm{\vdot}_{\WkpSet[1][p]{\Omega}}},$ i can again extend it on the whole $\WkpzeroSet[1][p]{\Omega}.$
\end{proof}

\begin{remark}
	In the proof of theorem we showed that $\forall u \in \WkpSet[1][p]{\R^d}$ it holds
	\[
		\norm{u}_{\LpSet[\frac{qd}{d-1}]{\Omega}}^q \leq q\norm{\grad u}_{\LpSet{\Omega}} \norm{u}_{\LpSet[\frac{p\qty(q-1)}{p-1}]{\Omega}}^{q-1},
	\]
	for $q$ such that $\frac{qd}{d-1}\leq p^{*}$.
\end{remark}


\begin{definition}[Continuous \& compact embeddings]
    Let $X,Y$ be linear normed spaces. We say 

    \begin{itemize}
	    \item $X$ is continuously embedded into $Y$, $X \hookrightarrow Y$, provided $X \subset Y$ (is a subspace) and
		    \[
			    \forall x \in X: \norm{x}_Y \leq C \norm{x}_X.
		    \]
	    \item $X$ is compactly embedded into $Y$, $X \hookrightarrow \hookrightarrow Y$, provided $X \subset Y$ (is a subspace) and
		    \[
			    \forall A \subset X \, \text{bounded} \,: \overline{A}^{Y} \, \text{is compact in} \, Y.
		    \]
    \end{itemize}
    This is the same as saying\footnote{Really, we have $\norm{x}_{Y} = \norm{\, \text{id} \,x}_Y \leq \norm{\text{id}}_{\mathcal{L}\qty(X,Y)}\norm{x}_X,$ and if $A \subset X$ is bounded, than from the definition of $\, \text{id} \, \in \mathcal{K}\qty(X,Y): \text{id}(A) = A \subset Y$ is relatively compact in $Y$.} $X \subset Y$ (is a subspace) and the identity $\, \text{id} \,: X \to Y$ is
    \begin{itemize}
	    \item a bounded linear operator, $\, \text{id} \, \in \mathcal{L}\qty(X,Y)$
	    \item is a compact linear operator, $\, \text{id} \, \in \mathcal{K}\qty(X,Y)$
    \end{itemize}
\end{definition}

\begin{theorem}[Embedding theorem for $p\leq d$]
	Let $\Omega \in \Ckl{0}{1}, p^{*} = \frac{dp}{d-p}.$ Then

	\begin{itemize}
		\item if $p \in [1,d),$ then
			\[
				\WkpSet[1][p]{\Omega} \hookrightarrow \LpSet[q]{\Omega} \forall q \in [1,p^{*}],
			\]
		\item if $p \in [1,d),$ then
			\[
				\WkpSet[1][p]{\Omega} \hookrightarrow \hookrightarrow \LpSet[q]{\Omega}, \forall q \in [1,p^{*}),
			\]
		\item if $p = d,$ then
			\[
				\WkpSet[1][d]{\Omega} \hookrightarrow \LpSet[q]{\Omega} \forall q \in [1, \infty),
			\]
		\item if $p = d, $ then
			\[
				\WkpSet[1][d]{\Omega} \hookrightarrow \hookrightarrow \LpSet[q]{\Omega} \forall q \in [1,\infty),
			\]
			(the same as above, \textit{i.e.}, every continuous embedding is also compact.)
	\end{itemize}
\end{theorem}

\begin{proof}(\textit{From: the lectures \& \cite{bulicekUvodModerniTeorie2018}})
	We would like to use the previous lemmas + extension. 


	\textit{Ad continuity for $p<d$}:

	Recall that the composition of continuous operators yields a continuous operator. In our case:
	\begin{itemize}
		\item the extension operator $E:\WkpSet[1][p]{\Omega} \to \WkpSet[1][p]{\R^d}$ is continuous
		\item identity $I_1:\WkpSet[1][p]{\R^d}\to \LpSet[p^{*}]{\R^d}$ is continous (Gagliardo-Nirenberg: $\norm{u}_{\LpSet[p^{*}]{\R^{d}}}\leq \frac{p(d-1)}{d-p}\norm{\grad u}_{\LpSet[p]{\R^{d}}} \leq C \norm{u}_{\WkpSet[1][p]{\R^{d}}}.$)
		\item restriction $I_2: \LpSet[p^{*}]{\R^d}\to \LpSet[p^{*}]{\Omega}$ is continuous (monotonicity of the L. integral: $\Omega \subset \R^{d} \Rightarrow \norm{u}_{\LpSet[p^{*}]{\Omega}} \leq \norm{u}_{\LpSet[p^{*}]{\R^{d}}}.$)
		\item identity $I_3: \LpSet[p^{*}]{\Omega} \to \LpSet[q]{\Omega}$ is continous (embedding of Lebesgue spaces: $\Omega$ is bounded $\Rightarrow \LpSet[p^{*}]{\Omega} \hookrightarrow \LpSet[q]{\Omega} \forall q \in [1,p^{*}]$)
	\end{itemize}
	Together, the mapping
	\[
		\text{id}:\WkpSet[1][p]{\Omega} \to \LpSet[q]{\Omega}, \text{id}=I_3 \circ I_2 \circ I_1 \circ E
	\]
	is continuous, and so $\WkpSet[1][p]{\Omega} \hookrightarrow \LpSet[q]{\Omega}, \forall q \in [1, p^{*}].$\\


	\textit{Ad continuity for $p =d$}:

	If $p=d$, we have (this holds generally)  $\WkpSet[1][d]{\Omega} \hookrightarrow \WkpSet[1][r]{\Omega} \forall r \in [1,d),$ (embedding of Lebesgue spaces, $\LpSet[d]{\Omega}\hookrightarrow \LpSet[r]{\Omega}, \forall r \in [1,d)$). Notice $r^{*} = \frac{r d}{r-d} \to \infty$ as $r\to d-,$ which means we can for all $q \in [1,\infty)$ find $r \in [1,d) \, \text{s.t.} \, r^{*} >q.$ Consequently,
	\[
		\forall q \in [1,\infty) \exists r \in [1,d) \, \text{\textit{s.t.}} \, \LpSet[r^{*}]{\Omega} \hookrightarrow \LpSet[q]{\Omega}.
	\]
	Notice also that $\forall r \in [1,d)$ we always have
	\[
		\WkpSet[1][r]{\Omega} \hookrightarrow \LpSet[r^{*}]{\Omega}
	\]
	(that's just renaming $p$ with $r$ in embedding for $p<d$). Then it holds
	\[
		\WkpSet[1][d]{\Omega} \hookrightarrow \WkpSet[1][r]{\Omega} \hookrightarrow \LpSet[r^{*}]{\Omega} \hookrightarrow \LpSet[q]{\Omega},
	\]
	and so $\WkpSet[1][d]{\Omega} \hookrightarrow \LpSet[q]{\Omega}, \forall q \in [1,\infty).$\\

	\textit{Ad compactness for $p<d$}:

	It suffices to show $U = \text{U}_{\WkpSet[1][p]{\Omega}}(0,1)$ is relatively compact in $\LpSet[q]{\Omega},$ which is, since $\LpSet[q]{\Omega}$ is complete, equivalent to $U$ being totally bounded in $\LpSet[q]{\Omega}.$\footnote{A metric space $P$ is totally bounded if there exists a finite $\varepsilon-$net: a finite open covering of $P$ by balls centered in $P$ of radii smaller than $\varepsilon$.} Extend the functions to $\WkpSet[1][p]{\R^{d}}$ using the extension operator $E$, so $EU \subset \WkpSet[1][p]{\R^{d}}.$ Take some yet undetermined $\delta>0$ and denote by $(EU)_{\delta}$ the set of regularized functions from $EU$ with some kernel $\eta$. Our next strategy is the following:

	\begin{enumerate}
		\item show $\qty(EU)_{\delta}$ is totally bounded in $\LpSet[1]{\text{U}(0,R)},$
		\item show $\WkpSet[1][p]{\Omega} \hookrightarrow \hookrightarrow \LpSet[1]{\Omega},$
		\item show $\WkpSet[1][p]{\Omega} \hookrightarrow \hookrightarrow \LpSet[q]{\Omega}.$
	\end{enumerate}

	Since the supports of the functions from $EU$ are uniformly bounded\footnote{From the properties of extension, we know $ \forall u \in \WkpSet[1][p]{\Omega}:\supp Eu \subset V$ with $V$ open \textit{s.t.} $\Omega \subset \subset V.$}, we know \footnote{Properties of mollification include $\supp (Eu)_{\delta} \subset \text{B}(0,\delta) + \supp Eu.$} $\exists R>0$ \textit{s.t.}
	\[
		\forall v \in \qty(EU)_{\delta}: \supp v \subsetneq \text{B}(0,R) \subset \R^d.
	\]
	(Also remember that $\Omega \subset \text{B}(0,R).$)
	Moreover, from the properties of mollification\footnote{We have $(EU)_{\delta} \subset \CinfSet{\R^{d}}$.} it follows: 
	\[
		\qty(EU)_{\delta} \subset \CkSet{1}{\qty(\text{B}(0,R))}.
	\]

	Next up, calculate for $v \in (EU)_{\delta}$ the norm $\norm{v}_{\CkSet{1}{\text{B}(0,1)}}.$ Realize that in fact $v = (Eu)_{\delta}$ for some $u \in U,$ and so
	\[
		|\int_{\R^{d}}Eu(y) \eta_{\delta}(x-y)\dd{y}| \leq \norm{Eu}_{\LpSet{\R^{d}}}\norm{\eta_{\delta}}_{\LpSet[p']{\R^{d}}} \leq C \norm{Eu}_{\WkpSet[1][p]{\R^{d}}} \leq C \norm{E}_{\mathcal{L}\qty(\WkpSet[1][p]{\Omega}, \WkpSet[1][p]{\R^{d}})} \norm{u}_{\WkpSet[1][p]{\Omega}} \leq C,
	\]
	as $U = \text{U}_{\WkpSet[1][p]{\Omega}}(0,1).$ Also
	\begin{align*}
		|\grad \int_{\R^{d}}Eu(y) \eta_{\delta}(x-y)\dd{y}| &=|\int_{\R^{d}}Eu(y) \grad_x \eta_{\delta}(x-y)\dd{y}| \leq \norm{Eu}_{\LpSet{\R^{d}}} \norm{\grad \eta_{\delta}}_{\LpSet[p']{\R^{d}}} \leq \\
		&\leq C \norm{Eu}_{\WkpSet[1][p]{\R^{d}}} \leq C \norm{E}_{\mathcal{L}\qty(\WkpSet[1][p]{\Omega}, \WkpSet[1][p]{\R^{d}})} \norm{u}_{\WkpSet[1][p]{\Omega}} \leq C,
	\end{align*}
	using the same arguments. In total
	\[
		\forall v \in \qty(EU)_{\delta}: \norm{v}_{\CkSet{1}{\text{B}(0,R)}}\leq C,
	\]
	or in other words, all functions from $\qty(EU)_{\delta}$ are uniformly bounded in $\CkSet{1}{\text{B}(0,R)} \Rightarrow$ they are uniformly bounded and uniformly equicontinuous (that is implied by uniform boundedness of the derivatives). Thus we can use Arzela-Ascoli theorem and state
	\[
		(EU)_{\delta} \subset \subset \CkSet{0}{\text{B}(0,R)}.
	\]
	Since $\CkSet{0}{\text{B}(0,R)}$ is complete, this also means $(EU)_{\delta}$ is totally bounded in $\CkSet{0}{\text{B}(0,R)}.$ Using the fact ($\text{B}(0,R)$ is compact) \footnote{Take some sequence $\qty{f_k} \subset \CkSet{0}{\text{B}(0,R)}$ that is bounded, then
		\[
			\lim_{k \to \infty}\int_{\text{U}(0,R)}|f_k|\dd{x} = \int_{\text{U}(0,R)}\lim_{k \to \infty} |f_k|\dd{x} = \int_{\text{U}(0,R)}|f|\dd{x} = \int_{\text{B}(0,R)}|f|\dd{x} \leq \infty,
		\]
	the majorant being \textit{e.g.} $\max_{k \in \N}\norm{f_k}_{\infty}.$ Hence every bounded sequence in $\CkSet{0}{\text{B}(0,R)}$ has a converging (sub)sequence in $\LpSet[1]{\text{U}(0,R)}$.}
	
	\[
		\CkSet{0}{\text{B}(0,R)} \hookrightarrow \hookrightarrow \LpSet[1]{\text{U}(0,R)},
	\]
	we also see that $(EU)_{\delta}$ is totally bounded in $\LpSet[1]{\text{U}(0,R)}.$

	Next, take an arbitrary $u \in U$ and compute (we are using the fact $Eu = u \, \text{\textit{a.e.}} \, \, \text{in} \, \Omega, \Omega \subset \text{U}(0,R).$)
	\begin{align*}
		\norm{u-\qty(Eu)_{\delta}}_{\LpSet[1]{\Omega}} &\leq \norm{\overbrace{Eu}^{\coloneq v}-(Eu)_{\delta}}_{\LpSet[1]{\text{U}(0,R)}} = \int_{\text{U}(0,R)}|v-v_{\delta}|\dd{x} = \int_{\text{U}(0,R)}|v(x) - \int_{\R^d}v(y) \eta_{\delta}(x-y)\dd{y}|\dd{x}= \\
							       &=\int_{\text{U}(0,R)}|v(x) - \int_{\R^{d}}v(x-y)\eta_{\delta}(y)\dd{y}|\dd{x} = \Big[ x \mapsto x +y \Big] = \int_{\text{U}(0,R)}|\int_{\R^{d}}v(x+y) - v(x) \underbrace{\eta_{\delta}(y)}_{\leq 1}\dd{y}|\dd{x} \leq \\
							       &\leq \int_{\text{U}(0,R)}|\int_{\R^{d}}(v(x+y) - v(y))\eta_{\delta}(y)\dd{y}|\dd{x}\leq \int_{\text{U}(0,R)}\int_{\R^d}\frac{|v(x+y)-v(x)|}{|y|}|\eta_{\delta}(y)| |y| \dd{y}\dd{x}\leq \\
									& \underbrace{\leq}_{\, \text{Fubini} \,} \int_{\R^d}\int_{\text{U}(0,R)}\frac{|v(x+y)-v(x)|}{|y|}\dd{x}|y| \eta_{\delta}(y)\dd{y}.
	\end{align*}
	Estimate the inner integral: assume $v$ is smooth, $v \in \DSet{\text{U}(0,R)}$ and write

	\begin{align*}
		\int_{\text{U}(0,R)}\frac{|v(x+y) - v(x)|}{|y|}\dd{x} &= \int_{\text{U}(0,R)}\frac{1}{|y|}|\int_{0}^1 \underbrace{\dv{s} \qty(v(x+sy))}_{\grad v\qty(x+sy) \vdot y}\dd{s}|\dd{x} \underbrace{\leq}_{\, \text{C-S} \,} \int_{\text{U}(0,R)}\frac{1}{|y|}\int_{0}^1 |\grad v (x+sy)||y|\dd{s}\dd{x} =\\
								      &= \int_{\text{U}(0,R)}\int_0^1 |\grad v(x+sy)|\dd{s}\dd{x} = \int_{0}^1 \int_{\text{U}(0,R)}|v(x+sy)|\dd{x}\dd{s}. 
	\end{align*}
	The last integral can be further manipulated by using the change of variables $z \coloneq x + sy \in \qty{x + sy | x \in \text{U}(0,R)} = \text{U}(0,R) + sy = \text{U}(sy,R).$ Since $\grad v \in \DSet{\text{U}(0,R)},$ the integral is nonzero only for $z \in \text{U}(sy,R) \cap \text{U}(0,R) \subset \text{U}(0,R)$ so we can write
	\begin{align*}
		\int_0^1 \int_{\text{U}(0,R)}|\grad v(x+sy)|\dd{x}\dd{s} &= \int_0^1 \int_{\text{U}(sy,R) \cap \text{U}(0,R)}|\grad v(z)|\dd{z}\dd{s} \leq \int_0^1 \int_{\text{U}(0,R)}|\grad v(z)|\dd{z}\dd{s}\leq \\
									 &\leq \int_0^1 \norm{\grad v}_{\LpSet{\text{U}(0,R)}}\qty(\lambda\qty(\text{U}(0,R)))^{\frac{1}{p'}}\dd{s} \leq C(R) \norm{\grad v}_{\LpSet{\text{U}(0,R)}},
	\end{align*}
	and so we have shown
	\[
		\forall v \in \DSet{\text{U}(0,1)}: \int_{\text{U}(0,R)}\frac{|v(x+y)-v(x)|}{|y|}\dd{x} \leq C(R) \norm{\grad v}_{\LpSet{\text{U}(0,R)}}.
	\]
	
	Now, take\footnote{Recall we have $v \in \qty(EU)_{\delta}$ and so $\supp v \subsetneq \text{B}(0,R),$ meaning it is "zero on $\text{S}(0,R)"$ - in the sense of traces.} $v \in \WkpzeroSet[1][p]{\text{U}(0,R)}, $ then $\exists \{v_k\} \subset \mathcal{D}\qty(\text{U}(0,R)): v_k \to v \, \text{in} \, \WkpzeroSet[1][p]{\text{U}(0,R)}.$ So
	\[
		\forall y \in \R^d: \int_{\R^d}\frac{|v_k(x+y)-v_k(x)|}{|y|}\dd{x}\leq C(R) \norm{\grad v_k}_{\LpSet{\text{U}(0,R)}}\to C(R) \norm{\grad v}_{\LpSet{\text{U}(0,R)}}.
	\]
	Putting it all together:
	\begin{align*}
		\norm{u-(Eu)_{\delta}}_{\LpSet[1]{\Omega}} &\leq \int_{\R^d}\int_{\text{U}(0,R)}\frac{|v(x+y)-v(x)|}{|y|}\dd{x} |y| \eta_{\delta}(y)\dd{y} \leq C(R) \norm{\grad v}_{\LpSet{\text{U}(0,R)}} \int_{\R^d}\underbrace{|y|}_{\leq \delta} \eta_{\delta}(y)\dd{y} \leq \\
		&\leq C(R) \delta \norm{\grad v}_{\LpSet{\text{U}(0,R)}}\int_{\R^{d}}\eta_{\delta}(y)\dd{y} = C(R) \delta \norm{v}_{\WkpSet[1][p]{\text{U}(0,R)}} = C(R) \delta \norm{Eu}_{\WkpSet[1][p]{\text{U}(0,R)}} \leq \\
		&\leq C_1 \delta \norm{u}_{\WkpSet[1][p]{\Omega}} \leq C_1 \delta.
	\end{align*}
	where we have used the properties of the reg. kernel $\eta_{\delta},$ the extension operator $E$ and the fact $u \in U.$

	Now fix $\varepsilon >0, $ find $\qty{\qty(Eu_k)_{\delta}}_{k=1}^m$ a finite $\frac{\varepsilon}{2}$-net in $(EB)_{\delta}$ in $\LpSet[1]{\text{U}(0,R)}$ (which is possible, since we have total boundedness in $\LpSet[1]{\text{U}(0,R)}$.) We will show $\qty{u_k}_{k=1}^m$is a (finite) $\varepsilon-$net in $\LpSet[1]{\Omega}.$

	Up to now, $\delta >0$ has been undetermined; now comes the time -  set
	\[
		\delta >0 \, \text{s.t.} \, C_1 \delta < \frac{\varepsilon}{4}.
	\]
	Fix an arbitrary $u \in U,$ and find $j \in \qty{1,\dots,m}$ \textit{s.t.} $\norm{(Eu)_{\delta}-(Eu_j)_{\delta}}_{\LpSet[1]{\text{U}(0,R)}} < \frac{\varepsilon}{2}.$ Compute
	\[
		\norm{u-u_j}_{\LpSet[1]{\Omega}} \leq \norm{u-\qty(Eu)_{\delta}}_{\LpSet[1]{\Omega}} + \norm{\qty(Eu)_{\delta}-\qty(Eu_j)_{\delta}}_{\LpSet[1]{\Omega}}+ \norm{\qty(Eu_j)_{\delta}-u_j}_{\LpSet[1]{\Omega}}\leq C_1 \delta + \frac{\varepsilon}{2} C_1 \delta < \frac{\varepsilon}{4} + \frac{\varepsilon}{4} + \frac{\varepsilon}{2} < \varepsilon,
	\]
	where we have used the above estimate and the fact $\Omega \subset \text{B}(0,R).$
	Thus, we have shown $U$ is totally bounded in $\LpSet[1]{\Omega}$ and so
	\[
		\WkpSet[1][p]{\Omega} \subset \subset \LpSet[1]{\Omega}.
	\]
	It remains to show the validity for a general $q \in [1,p^{*})$. Using the interpolation theorem on Lebesgue spaces \footnote{In the case $q \in [r,s)$ it holds $\norm{u}_{\LpSet[q]{\Omega}} \leq \norm{u}_{\LpSet[r]{\Omega}}^{\theta}\norm{u}_{\LpSet[s]{\Omega}}^{1-\theta}, \frac{1}{q} = \frac{\theta}{r} + \frac{1-\theta}{s}$.} we obtain
	\[
		\norm{u}_{\LpSet[q]{\Omega}} \leq \norm{u}_{\LpSet[1]{\Omega}}^{\theta}\norm{u}_{\LpSet[p^{*}]{\Omega}}^{1-\theta},
	\]
	where $\frac{1}{q} = \frac{\theta}{1} + \frac{1-\theta}{p^{*}}.$  Let us now show $U$ is totally bounded in $\LpSet[q]{\Omega}$, \textit{i.e.}, $\forall \epsilon >0$ there exists a finite $\epsilon-$ net in $U$ in $\LpSet[q]{\Omega}.$ Pick $\qty{u_j}_{j=1}^m \subset U$ that is an $\beta >0$ net in $\LpSet[1]{\Omega},$ where $\beta$ will be determined later. Then it holds
	\[
		\norm{u-u_j}_{\LpSet[q]{\Omega}}\leq \norm{u-u_j}_{\LpSet[1]{\Omega}}^\theta \norm{u-u_j}_{\LpSet[p^{*}]{\Omega}}^{1-\theta} \leq  \beta^\theta \norm{u- u_j}_{\LpSet[p^{*}]{\Omega}}^{1-\theta}.
	\]
	Since we have already shown $\WkpSet[1][p]{\Omega} \hookrightarrow \LpSet[p^{*}]{\Omega},$ we know (again $u, u_j$ are in $U$)
	\[
		\norm{u-u_j}_{\LpSet[p^{*}]{\Omega}}^{1-\theta} \leq C_2 \norm{u - u_j}_{\WkpSet[1][p]{\Omega}}^{1-\theta} \leq C_2 2^{1-\theta},
	\]
	and so
	\[
		\norm{u-u_j}_{\LpSet[q]{\Omega}} \leq \beta^{\theta}C_2 2^{1-\theta}.
	\]
	We see that if we choose $\beta$ \,\text{\textit{s.t.}} \,
	\[
		\beta < \qty(\frac{\epsilon}{C_2 2^{1-\theta}})^{\frac{1}{\theta}},
	\]
	we obtain
	\[
		\norm{u - u_j}_{\LpSet[q]{\Omega}} \leq \epsilon,
	\]
	\textit{i.e.}, $\qty{u_j}_{j=1}^m$ is a $\epsilon-$set in $\LpSet[q]{\Omega}$. Thus $\WkpSet[1][p]{\Omega} \hookrightarrow \hookrightarrow \LpSet[q]{\Omega}, \forall q \in [1, p^{*}).$ \\

	\textit{Ad compactness for $p=d$}

	Finally, let us show the last result. It holds
	\[
		\WkpSet[1][d]{\Omega} \hookrightarrow \WkpSet[1][r]{\Omega}, \forall r \in [1,d).
	\]
	Moreover, we have just shown
	\[
		\WkpSet[1][r]{\Omega} \hookrightarrow \hookrightarrow \LpSet[s]{\Omega}, \forall s \in [1,r^{*}).
	\]

	Notice that $r^{*} = \frac{rd}{r-d} \to \infty$ as $r \to d^-,$ so $\forall q \in [1,\infty)$ fixed $\exists r \in [1,d): r^{*} > q,$ \textit{i.e.}, $q \in [1, r^{*}).$ But then
	\[
		\WkpSet[1][d]{\Omega} \hookrightarrow \WkpSet[1][r]{\Omega} \hookrightarrow \hookrightarrow \LpSet[q]{\Omega}, \forall q \in [1,\infty).
	\]
	And realize that implies $\WkpSet[1][d]{\Omega} \hookrightarrow \hookrightarrow \LpSet[q]{\Omega} \forall q \in [1,\infty):$ if $\qty{u_n}$ is bounded in $\WkpSet[1][d]{\Omega},$ then it is bounded in $\WkpSet[1][r]{\Omega},$ as the identity between those spaces is continuous, and the above compact embedding tells us $\qty{u_{n_k}}$ is convergent in $\LpSet[q]{\Omega}$ for some $\qty{n_k}.$ In total, $\qty{u_n} \subset \WkpSet[1][d]{\Omega}$ has a subsequence $\qty{u_{n_k}}$ convergent in $\LpSet[q]{\Omega}.$ We are done.
\end{proof}

\subsubsection{Theorems for $p>d$}
\label{sec:emedding_p_ge_q}
We know we will encounter Holder spaces. Let us recall some of their properties in a remark.


\begin{remark}[Properties of Holder spaces]
	Let $\Omega \subset \R^{d}$ be open and bounded, $k \in \N_0, \lambda \in [0,1]$. The norm on the space $\CklSet{0}{1}{\overline{\Omega}}$ is defined as
	\[
		\norm{u}_{\CklSet{k}{\lambda}{\overline{\Omega}}} = \norm{u}_{\CkSet{k}{\overline{\Omega}}} + \sum_{|\alpha| = k}\sup_{x \neq y, x,y, \in \overline{\Omega}}\frac{|D^{\alpha}u(x) - D^{\alpha}(y)|}{|x-y|^{\lambda}},
	\]
	and the space
	\[
		\CklSet{k}{\lambda}{\overline{\Omega}} \coloneq \qty{u \in \CkSet{k}{\overline{\Omega}}| \norm{u}_{\CklSet{k}{\lambda}{\overline{\Omega}}} \leq \infty},
	\]
	where we identify
	\[
		\CklSet{k}{0}{\overline{\Omega}} = \CkSet{k}{\overline{\Omega}}.
	\]

	Moreover, we have the following embeddings: $\forall \alpha \in [0,1]$ it holds

	\[
		\CklSet{0}{\alpha}{\overline{\Omega}} \hookrightarrow \CklSet{0}{\beta}{\Omega}, \forall \beta \in [1, \alpha],
	\]
	and 
	\[
		\CklSet{0}{\alpha}{\overline{\Omega}} \hookrightarrow \hookrightarrow \CklSet{0}{\beta}{\overline{\Omega}}, \forall \beta \in [1,\alpha).
	\]


\end{remark}

A fresh start of a new chapter calls for a fresh new lemma.

\begin{lemma}[Morrey]
	Let $u \in \DSet{\R^{d}}$. Then $\forall x_1, x_2 \in \R^{d}, \forall \mu \in (0,1]$ it holds

	\[
		|u(x_1) - u(x_2)| \leq \frac{2 \sqrt{d}}{\mu}|x_1 - x_2|^{\mu} \qty[\grad u]_{\LpSet[1,\mu]{\R^{d}}},
	\]
	with
	\[
		\qty[\grad u]_{\LpSet[1,\mu]{\R^{d}}} = \sup_{x \in \R^{d}} \sup_{\rho >0} \int_{[0,\rho]^d}\frac{|\grad u(x+y)|}{\rho^{d-1+\mu}}\dd{y}.
	\]
\end{lemma}

\begin{proof}(\textit{From: \cite{bulicekUvodModerniTeorie2018}})
	Pick arbitrary but fixed $x_1, x_2 \in \R^{d}.$ Denote by $C_{\rho}$ the closed cube with a side of length $\rho$ \textit{s.t.} $x_1$ and $x_2$ lie on opposite faces. Then $\rho \leq |x_1 - x_2| \leq \rho \sqrt{d}$ (it is not closer then the height and not further then the diagonal.)

Let us begin by first computing the deviation of $u(x_i)$ from the mean value of $u$ on $C_{\rho}$:
	\[
		|\frac{1}{\lambda\qty(C_{\rho})} \int_{C_{\rho}}u(x)\dd{x} - u(x_i)| = |\int_{C_{\rho}}\frac{u(x)-u(x_i)}{\rho^d}\dd{x}| \leq \int_{C_{\rho}}\frac{|u(x)-u(x_i)|}{\rho^d}\dd{x}.
	\]

	What other can we use when estimating differences than Newton - Leibniz, right? Since $u \in \DSet{\R^{d}}\subset \CkSet{1}{\R^{d}}$ it holds for $i \in \qty{1,2}$ and $\forall x \in C_{\rho}:$
	\begin{align*}
		|u(x) - u(x_i)| &\leq |\int_0^1 \dv{s} u\qty(x_i + s\qty(x-x_i))\dd{s}| = |\int_0^1 \grad u\qty(x_i + s(x-x_i)) \vdot (x-x_i)\dd{s}|\leq \\
				&\leq \int_0^1 |\grad u(x_i + s(x-x_i))| |x-x_i| \dd{s} \leq \rho \sqrt{d} \int_0^1 |\grad u(x_i + s(x-x_i))| \dd{s}.
	\end{align*}
	Notice that it is important $x \in C_{\rho}$ and $x_1, x_2$ are on the opposite sides.	With this estimate for $|u(x) - u(x_i)|$ and Fubini we can write for the deviattion
	\[
		|\frac{1}{\rho^d}\int_{C_{\rho}}u(x)\dd{x} - u(x_i)| \leq \sqrt{d} \int_{C_{\rho}}\frac{1}{\rho^{d-1}}\int_0^1 |\grad u(x_i + s(x - x_i)|\dd{s}\dd{x} = \sqrt{d}\int_0^1 \int_{C_{\rho}}\frac{|\grad u(x_i + s(x-x_i))|}{\rho^{d-1}}\dd{x}\dd{s}.
	\]
	This calls for a sensible change of variables. Denote $z = x_i + s\qty(x-x_i)$, then under this transformation the cube $C_{\rho}$ becomes
	\[
		z \in x_i + s(C_{\rho}-x_i) = x_i\qty(1-s) + C_{s \rho} = x_i\qty(1-s) + [0, s \rho]^{d} \coloneq C^i_{s \rho}
	\]
	which, since $x_i$ is taken from the faces and $s \leq 1$, is a cube with its "origin" somewhere in $C_{\rho}$ and a side of length $s \rho \leq \rho,$ which implies $C_{s \rho}^i \subset [0,R]^d$ for some $R$. The integral then becomes (clearly $|\det \grad_x z| = s^d),$

	\[
		\sqrt{d}\int_0^1\int_{C_{\rho}}\frac{|\grad u(x_i + s(x - x_i)|}{\rho^{d-1}}\dd{x} \dd{s} =\sqrt{d} \int_0^1 \int_{C^i_{s \rho}}\frac{|\grad u(z)|}{s^d \rho^{d-1}}\dd{z}\dd{s} =\sqrt{d} \rho^{\mu} \int_0^1 s^{\mu-1} \int_{C_{s \rho}^i}\frac{|\grad u(z)|}{\qty(s \rho)^{d-1 + \mu}}\dd{z}\dd{s},
	\]
	where we are being a bit suggestively imaginative. The "deviation estimate" then becomes
	\begin{align*}
		|\frac{1}{\rho^d}\int_{C_{\rho}}u(x)\dd{x} - u(x_i)| &\leq \sqrt{d}\rho^{\mu}\int_0^1 s^{\mu-1} \underbrace{\int_{C^i_{s \rho}}\frac{|\grad u(z)|}{\qty(s \rho)^{d-1+\mu}}\dd{z}}_{\leq \qty[\grad u]_{\LpSet[1,\mu]{\R^{d}}}}\dd{s} \leq \sqrt{d} \rho^{\mu} \qty[\grad u]_{\LpSet[1,\mu]{\R^{d}}}\int_0^1 s^{\mu-1}\dd{s}= \\
		&= \frac{\sqrt{d}}{\mu}\rho^{\mu}\qty[\grad u]_{\LpSet[1,\mu]{\R^{d}}},
	\end{align*}
where we used that $C_{s \rho}^i \subset [0,R]^{d}$ and that $0 \in \R^{d},$ and so we could estimate the integral over $C_{s \rho}^i$ by $\qty[\grad u]_{\LpSet[1,\mu]{\R^{d}}}.$ Triangle inequality and the fact $\rho \leq |x_1 - x_2|$ concludes our proof:
\begin{align*}
	|u(x_1) - u(x_2)| &\leq |\frac{1}{\rho^d}\int_{C_{\rho}}u(x)\dd{x} - u(x_1)| + |\frac{1}{\rho^d}\int_{C_{\rho}}u(x)\dd{x} - u(x_2)| \leq \frac{2 \sqrt{d}}{\mu}\rho^{\mu} \qty[\grad u]_{\LpSet[1, \mu]{\R^{d}}}\leq \\
			  &\leq \frac{2 \sqrt{d}}{\mu}|x_1 - x_2|^{\mu} \qty[\grad u]_{\LpSet[1,\mu]{\R^{d}}}.
\end{align*}
\end{proof}
\begin{remark}
    It is sufficient when $u \in \text{C}^1_0\qty(\R^{d}).$
\end{remark}

Gagliardo had in fact two lemmas, so let us even the game for Morrey.


\begin{lemma}
	Let $p \in (d, \infty),$ and let $\mu = 1 - \frac{d}{p}.$ Then $\forall u \in \DSet{\R^{d}}$ it holds
	\[
		\norm{u}_{\CklSet{0}{\mu}{\R^{d}}} \leq \qty(1+ \frac{4 \sqrt{d}}{\mu})\norm{u}_{\WkpSet[1][p]{\R^{d}}},
	\]
	where
	\[
		\norm{u}_{\CklSet{0}{\mu}{\R^{d}}} = \sup_{x \in \R^{d}}|u(x)| + \sup_{x,y \in \R^{d}, x \neq y}\frac{|u(x)-u(y)|}{|x-y|^{\mu}}.
	\]
\end{lemma}
\begin{proof}(\textit{From: \cite{bulicekUvodModerniTeorie2018}})
	We prove the assertion by estimating both terms in the above norm. To obtain those specific constants, we will pay some more attention to our procceeding. Let us also state the trivial: $\mu = 1 - \frac{d}{p} \in (0,1)$ for $p \in (d, \infty).$

	Begin with the differences: choose an arbitrary $\rho>0$ and compute
	\[
		\int_{[0,\rho]^d}\frac{|\grad u(x+y)|}{\rho^{d-1 + \mu}}\dd{y} \leq \qty(\int_{[0,\rho]^d}\qty(\frac{|\grad u(x+y)| }{\rho^{d-1+\mu}})^{p}\dd{y})^{\frac{1}{p}}\qty(\lambda\qty([0,\rho]^d))^{\frac{p-1}{p}} = \norm{\grad u}_{\LpSet[p]{\R^{d}}} \frac{\rho^{\frac{d(p-1)}{p}}}{\rho^{d-1+\mu}} = \norm{\grad u}_{\LpSet{\R^{d}}},
	\]
	because $\frac{d(p-1)}{p}-d+1-\mu = \frac{dp - d}{p} -d +1 -1 + \frac{d}{p} = \frac{dp}{p}-d = 0.$ Taking the suprema yields
	\[
		\qty[\grad u]_{\LpSet[1,\mu]{\R^{d}}} \leq \norm{\grad u}_{\LpSet{\R^{d}}}.
	\]
	Going with this into the first Morrey lemma, we see
	\[
		|u(x_1) - u(x_2)| \leq \frac{2 \sqrt{d}}{\mu} |x_1 - x_2|^{\mu}\norm{\grad u}_{\LpSet{\R^{d}}},
	\]
	meaning
	\[
		\sup_{x_1, x_2 \in \R^{d}, x_1 \neq x_2}\frac{|u(x_1) - u(x_2)|}{|x_1-x_2|^{\mu}} \leq \frac{2 \sqrt{d}}{\mu}\norm{\grad u}_{\LpSet{\R^{d}}}.
	\]

	To estimate the infinity norm, we can actually exploit the above result aswell: pick $x \neq y \in \R^{d}$ and write 

	\[
		|u(x)| - |u(y)| \leq |u(x) - u(y)| \leq \frac{2 \sqrt{d}}{\mu} \norm{\grad u}_{\LpSet{\R^{d}}},
	\]
	and so
	\[
		|u(x)| \leq |u(y)| + \frac{2 \sqrt{d}}{\mu} \norm{\grad u}_{\LpSet{\R^{d}}}.
	\]
	Now fix $\rho \geq 2 |x-y| > 0$, integrate both sides w.r.t $y$ over $[0,\rho]^d$ and obtain

	\[
		|u(x)| \rho^d \leq \int_{[0,\rho]^d}|u(y)|\dd{y} + \frac{2 \sqrt{d}}{\mu}\rho^d \norm{\grad u}_{\LpSet{\R^{d}}},
	\]
	which upon using Holder in the integral becomes

	\[
		|u(x)| \rho^d \leq \norm{u}_{\LpSet{\R^{d}}} \rho^{\frac{d\qty(p-1)}{p}} + \frac{2 \sqrt{d}}{\mu}\rho^{d} \norm{\grad u}_{\LpSet[p]{\R^{d}}}.
	\]
	Since we have lost $y$, we can in fact choose $\rho = 1,$ and upon taking the supremum write

	\[
		\sup_{x\in \R^{d}} |u(x)| \leq \norm{u}_{\LpSet{\R^{d}}} + \frac{2 \sqrt{d}}{\mu} \norm{\grad u}_{\LpSet[p]{\R^{d}}},
	\]
	and so in total
	\[
		\norm{u}_{\CklSet{0}{\mu}{\R^{d}}} \leq \norm{u}_{\LpSet{\R^{d}}} + \frac{4 \sqrt{d}}{\mu}\norm{\grad u}_{\LpSet{\R^{d}}} = \qty(1+\frac{4 \sqrt{d}}{\mu})\norm{u}_{\WkpSet[1][p]{\R^{d}}}.
	\]
\end{proof}

\begin{remark}
    Since $\DSet{\R^{d}}$ is dense in $\WkpSet[1][p]{\R^{d}},$ the above lemma holds also for $u \in \WkpSet[1][p]{\R^{d}}.$ We have to be careful to pick a good representant though.
\end{remark}



\begin{theorem}[Embedding theorems for $p>d$]
	Let $\Omega \in \Ckl{0}{1}, d \in \N, p > d, \, \text{\textit{i.e.}} \,, p \in (d, \infty]$. Denote $\mu^{*} = 1 - \frac{d}{p} \in (0,1).$ Then
	\[
		\WkpSet[1][p]{\Omega} \hookrightarrow \CklSet{0}{\alpha}{\overline{\Omega}}, \forall \alpha \in [0,\mu^{*}],
	\]
	and
	\[
		\WkpSet[1][p]{\Omega} \hookrightarrow \hookrightarrow \CklSet{0}{\alpha}{\overline{\Omega}}, \forall \alpha \in [0, \mu^{*}).
	\]
\end{theorem}

\begin{proof}(\textit{From: \cite{bulicekUvodModerniTeorie2018}})
    \textit{Ad continuous:}

    Since $\Omega \in \Ckl{0}{1}$, we are able to use the extension theorem; recall $\forall u \in \WkpSet[1][p]{\Omega}: \supp Eu \subset V,$ where $\overline{\Omega} \subset V.$ Let us deal with the case $p \in (d, \infty)$ first. We have shown
    \[
	    \norm{u}_{\CklSet{0}{\mu^{*}}{\overline{\Omega}}} \leq \norm{u}_{\CklSet{0}{\mu^{*}}{\R^{d}}} \leq \qty(1+ \frac{4 \sqrt{d}}{\mu^{*}})\norm{u}_{\WkpSet[1][p]{\R^{d}}}, \forall u \in \WkpSet[1][p]{\R^{d}}.
    \]
    Realize that in fact
    \[
	    \norm{u}_{\CklSet{0}{\mu^{*}}{\overline{\Omega}}} = \norm{Eu}_{\CklSet{0}{\mu^{*}}{\overline{\Omega}}},
    \]
    as $\partial \Omega$ is a set of zero Lebesgue measure and so in fact $Eu = u \, \text{on} \, \overline{\Omega},$ which together with the obvious fact $Eu \in \WkpSet[1][p]{\R^{d}}$ gives
    \[
	    \norm{u}_{\CklSet{0}{\mu^{*}}{\overline{\Omega}}} = \norm{Eu}_{\CklSet{0}{\mu^{*}}{\overline{\Omega}}} \leq \qty(1 + \frac{4 \sqrt{d}}{\mu^{*}})\norm{Eu}_{\WkpSet[1][p]{\R^{d}}} \leq C \norm{u}_{\WkpSet[1][p]{\Omega}}.
    \]
    This exactly means
    \[
	    \WkpSet[1][p]{\Omega} \hookrightarrow \CklSet{0}{\mu^{*}}{\overline{\Omega}}.
    \]
    Realize also that

    \[
	    \CklSet{0}{\mu^{*}}{\overline{\Omega}} \hookrightarrow \CklSet{0}{\alpha}{\overline{\Omega}}, \forall \alpha \in [0,\mu^{*}],
    \]

    and so
    \[
	    \WkpSet[1][p]{\Omega} \hookrightarrow \CklSet{0}{\mu^{*}}{\overline{\Omega}} \hookrightarrow \CklSet{0}{\alpha}{\overline{\Omega}}, \forall \alpha \in [0,\mu^{*}].
    \]

    If now $p = \infty,$ realize that (by embedding of Lebesgue spaces)
    \[
	    \WkpSet[1][\infty]{\Omega} \hookrightarrow \WkpSet[1][q]{\Omega}, \forall q \in [1, \infty).
    \]
    From the previous result it follows
    \[
	    \WkpSet[1][q]{\Omega} \hookrightarrow \CklSet{0}{\alpha}{\overline{\Omega}}, \forall \alpha \in \qty[0, 1 - \frac{d}{q}],
    \]
    and notice that $1- \frac{d}{q} \to 1$ as $q \to \infty.$ This means that $\forall q \in [1,\infty)$ it holds
    \[
	    \WkpSet[1][\infty]{\Omega} \hookrightarrow \WkpSet[1][q]{\Omega} \hookrightarrow \CklSet{0}{\alpha}{\overline{\Omega}}, \forall \alpha \in [0, 1- \frac{d}{q}].
    \]
    Since $q \in [1, \infty)$ was arbitrary, we conclude it must be


    \[
	    \WkpSet[1][\infty]{\Omega} \hookrightarrow \CklSet{0}{\alpha}{\overline{\Omega}}, \forall \alpha \in [0,1].
    \]


    \footnote{The norm

	    \[
		    \norm{u}_{\CklSet{0}{1- \frac{d}{q}}{\overline{\Omega}}} = \sup_{x \in \overline{\Omega}}|u(x)| + \sup_{x_1 \neq x_2 \in \overline{\Omega}} \frac{|u(x_1) - u(x_2)|}{|x_1 - x_2|^{1- \frac{d}{q}}},
	    \]
    actually allows one to pass to the limit $q \to \infty,$ since the supremum is independent of the exponent in the denonimanotor, the function on the RHS is continuous. So if we know $\norm{u}_{\CklSet{0}{1-\frac{d}{q}}{\overline{\Omega}}}\leq C \norm{u}_{\WkpSet[1][\infty]{\Omega}}, \forall q \in [1, \infty),$ we can pass to the limit on the LHS and obtain $\norm{u}_{\CklSet{0}{1}{\overline{\Omega}}}\leq C \norm{u}_{\WkpSet[1][\infty]{\Omega}},$ which is the only missing possibility.}\\


    \textit{Ad compactness}
    This will be a bit of a cheating: we know
    \[
	    \WkpSet[1][p]{\Omega} \hookrightarrow \CklSet{0}{\beta}{\overline{\Omega}}, \forall \beta \in [0,\mu^{*}],
    \]
    and for Holder spaces it also holds
    \[
	    \CklSet{0}{\beta}{\overline{\Omega}} \hookrightarrow \hookrightarrow \CklSet{0}{\alpha}{\overline{\Omega}}, \forall \alpha \in [0,\beta),
    \]
    which means if we choose $\beta = \mu^{*},$ we in fact obtain
    \[
	    \WkpSet[1][p]{\Omega} \hookrightarrow \CklSet{0}{\mu^{*}}{\overline{\Omega}} \hookrightarrow \hookrightarrow \CklSet{0}{\alpha}{\overline{\Omega}}, \forall \alpha \in [0, \mu^{*}).
    \]
    Using the same arguments as in the case of the compact embedding for $p <d, $ we can conclude
    \[
	    \WkpSet[1][p]{\Omega} \hookrightarrow \hookrightarrow \CklSet{0}{\alpha}{\overline{\Omega}}, \forall \alpha \in [0, \mu^{*}).
    \]
    We are done.
\end{proof}

\begin{remark}
    Note that in the case of $p > d,$ from
    \[
	    \WkpSet[1][p]{\Omega} \hookrightarrow \hookrightarrow \CklSet{0}{\alpha}{\overline{\Omega}}, \forall \alpha \in [0,\mu^{*}),
    \]
    and\footnote{Clearly,
	    \[
		    \norm{u}_{\LinfSet{\Omega}} \leq \norm{u}_{\CklSet{0}{\alpha}{\overline{\Omega}}} = \norm{u}_{\LinfSet{\Omega}} + \sup_{x_1 \neq x_2}\frac{\abs{u(x_1) - u(x_2)}}{\abs{x_1 - x_2}^{\alpha}}.
	    \]
    }
    \[
	    \CklSet{0}{\alpha}{\overline{\Omega}} \hookrightarrow \LinfSet{\Omega}, \forall \alpha \in [0,1]
    \]
    it follows\footnote{Again, composition of a compact and continuous (linear) operators yields a compact operator independently of the order.}
    \[
	    \WkpSet[1][p]{\Omega} \hookrightarrow \hookrightarrow \LinfSet{\Omega}, \forall p > d.
    \]
    But that of course means $(\Omega$ is bounded) that
    \[
	    \WkpSet[1][p]{\Omega} \hookrightarrow \hookrightarrow \LpSet[q]{\Omega}, \forall q \in [1,\infty],
    \]
    whenever $p>d.$
\end{remark}

\begin{remark}[Summary]
    Let us summarize the obtained results. If $d \in \N, \Omega \in \Ckl{0}{1},$ then upon denoting
    \[
	    p^{*} = \frac{dp}{d-p}, \mu^{*} = 1- \frac{d}{p},
    \]
    it holds

    \begin{itemize}
	    \item if $p < d$, then
		    \[
			    \WkpSet[1][p]{\Omega} \hookrightarrow \LpSet[q]{\Omega}, \forall q \in [1,p^{*}],
		    \]
		    and
		    \[
			    \WkpSet[1][p]{\Omega} \hookrightarrow \hookrightarrow \LpSet[q]{\Omega}, \forall q \in [1,p^{*}),
		    \]
	    \item if $p = d,$ then
		    \[
			    \WkpSet[1][d]{\Omega} \hookrightarrow \LpSet[q]{\Omega}, \forall q \in [1, \infty)
		    \]
		    and
		    \[
			    \WkpSet[1][d]{\Omega} \hookrightarrow \hookrightarrow \LpSet[q]{\Omega}, \forall q \in [1, \infty),
		    \]
	    \item if $p > d,$ then
		    \[
			    \WkpSet[1][p]{\Omega} \hookrightarrow \CklSet{0}{\alpha}{\overline{\Omega}}, \forall \alpha \in [0, \mu^{*}],
		    \]
		    and
		    \[
			    \WkpSet[1][p]{\Omega} \hookrightarrow \hookrightarrow \CklSet{0}{\alpha}{\overline{\Omega}}, \forall \alpha \in [0, \mu^{*}),
		    \]
		    which also imply
		    \[
			    \WkpSet[1][p]{\Omega} \hookrightarrow \LinfSet{\Omega},
		    \]
		    and
		    \[
			    \WkpSet[1][p]{\Omega} \hookrightarrow \hookrightarrow \LinfSet{\Omega}.
		    \]
		    
    \end{itemize}
\end{remark}

\begin{remark}[Summary - embedding into Lebesgue spaces]
    It can be guiding to look only at the embeddings into some Lebesgue spaces. Let $d \in \N, \Omega \in \Ckl{0}{1},$ denote
    \[
	    p^{*} = \frac{dp}{d-p}, \mu^{*} = 1 - \frac{d}{p}.
    \]
    Then it holds
    \[
	    \WkpSet[1][p]{\Omega} \hookrightarrow \LpSet[q]{\Omega},
    \]
    where $q$ is given: 
    \begin{itemize}
	    \item if $p<d,$ then $q \in [1, p^{*}],$
	    \item if $p=d,$ then $q \in [1, \infty),$
	    \item if $p>d,$ then $q \in [1,\infty].$
    \end{itemize}

    Also, it holds
    \[
	    \WkpSet[1][p]{\Omega} \hookrightarrow \hookrightarrow \LpSet[q]{\Omega},
    \]
    where $q$ is given as 
    \begin{itemize}
	    \item if $p<d,$ then $q \in [1, p^{*}),$
	    \item if $p=d,$ then $q \in [1,\infty),$
	    \item if $p>d,$ then $q \in [1, \infty].$
    \end{itemize}
\end{remark}

\subsection{Trace theorems}
\label{sec:traces}

There are many many troubles with the boundary. Another one we have yet not encountered arises with \textit{e.g.} homogenous Dirichlet boundary coniditons: it should hold
\[
	u = 0 \, \text{on} \, \partial \Omega,
\]
but typically, $u$ is an element of some Sobolev space on $\Omega,$ and $\lambda_{d}\qty(\partial \Omega) = 0.$ We cannot sensibly talk about pointwise values on a set of measure zero, they can be arbitrary. This can be dealt with provided - as expected - when $\Omega, \, \text{\textit{i.e.}} \,, \partial \Omega$ is benevolent enough. 

Realize also that for $\Omega$ at least $\Ckl{0}{0}$ we also have the density of $\, \text{C} \,^{\infty}_{\overline{\Omega}}\qty(\R^{d})$ in $\WkpSet[1][p]{\Omega},$ meaning the values on $\partial \Omega$ are well defined at least for a dense subset of $\WkpSet[1][p]{\Omega}.$ It should not be too difficult to extend to the whole $\WkpSet[1][p]{\Omega},$ right...?

Moreover, in the case $p>d, \Omega \in \Ckl{0}{1}$ we already know $\forall u \in \WkpSet[1][p]{\Omega}$ there exists $u^{*} \in \CkSet{0}{\overline{\Omega}}$, such that $u = u^{*} \, \text{\textit{a.e.}} \,$ in $\Omega,$ and so in these cases, the values $u^{*}\restriction_{\partial \Omega}$ are well defined. What if $p \leq d$?

\begin{remark}[The space $\LpSet{\mathcal{H}_{d-1}, \partial \Omega}$]
	In the winter semester, we have defined
	\[
		\LpSet{\partial \Omega} \equiv \LpSet{\mathcal{H}_{d-1}, \partial \Omega},
	\]
	\textit{i.e.}, the lebesgue spaces are taken w.r.t the $d-1$ dimensional (normalized complete) Hausdorff measure $\mathcal{H}_{d-1}.$
\end{remark}
\begin{theorem}[Continuous trace theorem]
	Let $\Omega \in \Ckl{0}{1}, p \in [1,\infty],$ denote $p^{\#} = \frac{dp-p}{d-p}.$ Then there is a continuous linear operator $\trace: \WkpSet[1][p]{\Omega} \to \LpSet[q]{\partial \Omega},$ with $q$ being

	\begin{itemize}
		\item if $p<d$, then $q \in [1, p^{\#}],$
		\item if $p=d$, then $q \in [1, \infty)$,
		\item if $p>d,$ then $q \in [1, \infty]$.
	\end{itemize}
	Moreover, $\forall u \in \CinfSet{\overline{\Omega}}$ it holds
	\[
		\trace u = u\restriction_{\partial \Omega},
	\]
	meaning $\trace: \WkpSet[1][p]{\Omega} \to \LpSet{\partial \Omega}$ is an extension of $\tilde{\trace}: \CinfSet{\overline{\Omega}} \to \CkSet{0}{\partial \Omega}.$
\end{theorem}

\begin{proof}(\textit{From: \cite{bulicekUvodModerniTeorie2018}})
	The strategy is the following

	\begin{enumerate}
		\item define $\tilde{\trace}$ for smooth functions,
		\item obtain estimates for $\tilde{\trace}$ using embedding theorems,
		\item extend $\tilde{\trace}$ to the whole space, which defines $\trace$.
	\end{enumerate}

	\textit{Case $p<d$:}

	As we have mentioned, the case for functions smooth up to the boundary is evident. Let us so define $\tilde{\trace}:\, \text{C} \,^{\infty}_{\overline{\Omega}}\qty(\R^{d}) \to \CkSet{0}{\partial \Omega},$ by
	\[
		\tilde{\trace}\, u = u \restriction_{\partial \Omega}.
	\]
Then clearly $\tilde{\trace}$ is a well defined linear  \footnote{Be careful about the continuity of $\tilde{\trace}$ here. $\, \text{C} \,^{\infty}_{\overline{\Omega}}\qty(\R^{d})$ is not a normed linear space. We could state the continuity in a different setting, but we do not actually need it for our purposes.} operator. 

	(We are using the notation from the definition of a $\Ckl{0}{1}$ domain). Let us for clarity define (and also recall)

	\begin{align*}
		G_j &= \tensorq{A}_j\qty(\qty{(x',a_j(x')| x' \in \text{U}(0,\alpha)}),\\
		G_j^+ &= \tensorq{A}_j\qty(\qty{x', a_j(x') + b| x' \in \text{U}(0,\alpha), b \in (0, \beta)}), \\
		G_j^{-} &= \tensorq{A}_j\qty(\qty{x', a_j(x') - b| x' \in \text{U}(0,\alpha), b \in (0,\beta)}).
	\end{align*}
	Within this notation, $G_j \subset \partial \Omega, G_j^{+} \subset \Omega, G_j^{-} \subset \R^{d}/ \overline{\Omega}$ and $U_j = G_j \cup G_j^{+} \cup G_j^{-}.$ Moreover, $\qty{U_j}_{j=1}^m$ are open sets \textit{s.t.} $\partial \Omega \subset \bigcup_{j=1}^m U_j.$ Denote $\qty{\varphi_j}_{j=1}^m \subset \DSet{\R^{d}}$ to be the partition of unity subordinate to this (open) covering. Realize moreover that since $\mathcal{H}_{d-1}\qty(\partial \Omega) < \infty,$ it holds
	\[
		\LpSet[p^{\#}]{\partial \Omega} \hookrightarrow \LpSet[q]{\partial \Omega}, \forall q \in [1, p^{\#}].
	\]
	So if we are able to show
	\[
		\norm{u}_{\LpSet[p^{\#}]{\partial \Omega}}\leq  C \norm{u}_{\WkpSet[1][p]{\Omega}},
	\]
	we have the rest of the estimates for free.

	Take $u \in \, \text{C} \,^{\infty}_{\overline{\Omega}}\qty(\R^{d}),$ and denote $u_j = u \varphi_j, j \in \qty{1, \dots, m}.$ Let for the moment $p>1,$ so $p^{\#} = \frac{dp-p}{d-p} > \frac{d-1}{d-1} = 1.$ As $u_j \in \CinfSet{\R^{d}},$ it holds $u_j \in \CkSet{1}{\overline{G_j^+}}.$ Moreover, since $\supp u_j \subset U_j$ and as $U_j \cap \tensorq{A}_j\qty(\qty{(x', a_j(x') + b| x' \in \text{U}(0,\alpha), b \in [0, \infty)}) = \emptyset,$ it holds
	\[
		u_j\qty(\tensorq{A}_j\qty(x', a_j(x')+\beta)) = 0.
	\]
	With those qualities, if we denote (for an arbitrary $j \in \qty{1, \dots, m}$; this will become $\norm{u_j}_{\LpSet[p^{\#}]{\partial \Omega}}$)
	\[
		v\qty(x') = \abs{u_j\qty(\tensorq{A}_j\qty(x', a_j(x')))}^{p^{\#}} = \abs{u_j\qty(\tensorq{A}_j\qty(x',a_j(x')))}^{\frac{dp-p}{d-p}},
	\]
	we can write (recall $\tensorq{A}_j$ is orthogonal and $x' \in \text{U}(0,\alpha)$)
	\begin{align*}
		\abs{v(x')} &= \abs{\abs{u_j\qty(\tensorq{A}_j\qty(x',a_j\qty(x')))}^{\frac{dp-p}{d-p}} - \abs{u_j\qty(\tensorq{A}_j\qty(x',a_j\qty(x')+\beta))}^{\frac{dp-p}{d-p}}} \leq \abs{\int^{a_j\qty(x')}_{a_j\qty(x')+\beta} \pdv{s} \abs{u_j\qty(\tensorq{A}_j\qty(x',s))} \dd{s}} = \\
			    &= \abs{\int_{a_j(x')+\beta}^{a_j(x')}p^{\#}\abs{u_j\qty(\tensorq{A}_j\qty(x',s))}^{\frac{dp-d}{d-p}}\signum\qty(u_j\qty(\tensorq{A}_j\qty(x',s)))\grad u_j\qty(\tensorq{A}_j\qty(x',s))\vdot \tensorq{A}_j\qty(x',1)\dd{s}} \leq \\
			    &\leq p^{\#}\int^{a_j(x')+\beta}_{a_j(x')}\abs{u_j\qty(\tensorq{A}_j\qty(x',s))}^{\frac{dp-d}{d-p}}\abs{\grad u_j\qty(\tensorq{A}_j\qty(x',s))}\underbrace{\abs{\tensorq{A}_j\qty(x',1)}}_{=\abs{\qty(x',1)}}\dd{s} \leq \\
			    &\leq p^{\#}\sqrt{1+\alpha^{2}}\int_{a_j(x')}^{a_j(x')+\beta}\abs{u_j\qty(\tensorq{A}_j\qty(x',s))}^{\frac{dp-d}{d-p}}\abs{\grad u_j\qty(x',s)}\dd{s}.
	\end{align*}
Integrate this inequality over $\text{U}(0,\alpha)$ and write (recall the definition of $G_j^{+}$, we are using Fubini, substitution theorem and the fact $\abs{\det \tensorq{A}_j = 1}$)
	\begin{align*}
		\int_{\text{U}(0,\alpha)}\abs{v\qty(x')}\dd{x'} &\leq p^{\#}\sqrt{1+\alpha^{2}} \int_{\text{U}(0,\alpha)}\int_{a_j(x')}^{a_j(x')+\beta}\abs{u_j\qty(\tensorq{A}_j\qty(x',s))}^{\frac{dp-d}{d-p}}\abs{\grad u_j\qty(\tensorq{A}_j\qty(x',s))}\dd{s}\dd{x'} \leq \\
								&\leq p^{\#}\sqrt{1+\alpha^{2}}\int_{\tensorq{A}_j\qty(\qty{(x', a_j(x')+b)|x' \in \text{U}(0,\alpha), b \in [0,\beta]})}\abs{u_j}^{\frac{dp-d}{d-p}}\abs{\grad u_j(x)}\dd{x} = \\
								& = p^{\#}\sqrt{1+\alpha^{2}}\int_{G_j^{+}}\abs{u_j(x)}^{\frac{dp-d}{d-p}}\abs{\grad u(x)}\dd{x} \leq p^{\#}\sqrt{1+\alpha^{2}} \norm{\grad u}_{\LpSet[p]{G_j^+}} \norm{\abs{u_j}^{\frac{dp-d}{d-p}}}_{\LpSet[p']{G_j^{+}}},
	\end{align*}
	and since $\frac{dp-d}{d-p} p' = \frac{dp-d}{d-p} \frac{p}{p-1} = \frac{dp}{d-p} = p^{*},$ we have	
	\[
		\int_{\text{U}(0,\alpha)}\abs{v(x')}\dd{x'} = p^{\#}\sqrt{1+\alpha^{2}}\norm{\grad u}_{\LpSet[p]{G_j^{+}}}\norm{u_j}_{\LpSet[p^{*}]{G_j^{+}}}^{\frac{dp-d}{d-p}}.
	\]
	Since $G_j^{+} \in \Ckl{0}{1},$ the last term can be estimated using the continuous embedding theorems:
	\[
		\norm{u_j}_{\LpSet[p^{*}]{G_j^+}}^{\frac{dp-d}{d-p}} \leq C \norm{u_j}_{\WkpSet[1][p]{G_j^+}}^{\frac{dp-d}{d-p}},
	\]
	whereas the integral on the LHS actually is\footnote{Careful, this is a bit inaccurate (although not mentioned in the reference also); we should also include the volume form from the definition of the surface integral into our computation. However, that can be estimated by a constant(and included in $C$), since our $\Omega \in \Ckl{0}{1}.$}
	\[
		\int_{\text{U}(0,\alpha)}\abs{v(x')}\dd{x'} = \int_{\text{U}(0,\alpha)}\abs{u_j\qty(\tensorq{A}_j\qty(x',a_j(x')))}^{p^{\#}}\dd{x} = \int_{\tensorq{A}_j\qty(\qty{(x',a_j(x')|x' \in \text{U}(0,\alpha)})}\abs{u_j(x)}^{p^{\#}}\dd{x} = \norm{u_j}_{\LpSet[p^{\#}]{G_j}}^{p^{\#}}, 
	\]
	and so we write
	\[
		\norm{u_j}_{\LpSet[p^{\#}]{G_j}}^{p^{\#}}\leq C \norm{\grad u_j}_{\LpSet[p]{G_j^+}} \norm{u_j}_{\WkpSet[1][p]{G_j^+}}^{\frac{dp-d}{d-p}} \geq C \norm{u_j}_{\WkpSet[1][p]{G_j^+}}^\frac{dp-d+d-p}{d-p} = C \norm{u_j}_{\WkpSet[1][p]{G^+_j}}^{p^{\#}},
	\]
	and so
	\[
		\norm{u_j}_{\LpSet[p^{\#}]{G_j}}\leq C \norm{u_j}_{\WkpSet[1][p]{G_j^+}}.
	\]
	This has been done for $p>1,$ but in fact taking the limit $p \to 1^+$ is allowed here (without a proof). Hence the above estimate holds $\forall p \in [1,d).$

	The estimates have so far been local - let us glue them together. Recall $\partial \Omega = \bigcup_{j=1}^m G_j,G_j^+ \subset \Omega,$ and so
	\begin{align*}
		\norm{u}_{\LpSet[p^{\#}]{\partial \Omega}} &= \norm{\sum_{j=1}^m u_j}_{\LpSet[p^{\#}]{\partial \Omega}}\leq \sum_{j=1}^m \norm{u_j}_{\LpSet[p^{\#}]{\partial \Omega}} = \sum_{j=1}^m \norm{u_j}_{\LpSet[p]{\supp u_j \cap \partial \Omega}} = \sum_{j=1}^m \norm{u_j}_{\LpSet[p]{G_j}} \leq \\
		&\leq C \sum_{j=1}^m \norm{u_j}_{\WkpSet[1][p]{G_j^{+}}} \leq C \norm{u}_{\WkpSet[1][p]{\Omega}},
	\end{align*}
	where we have used the fact $0 \leq \varphi_j \leq 1$ in the last inequality. And so we have shown
	\[
		\norm{u}_{\LpSet[p^{\#}]{\partial \Omega}} \leq \norm{u}_{\WkpSet[1][p]{\Omega}}, \forall u \in \, \text{C} \,^{\infty}_{\overline{\Omega}}\qty(\R^{d}).
	\]
	Now let $u \in \WkpSet[1][p]{\Omega}$ be arbitrary. Since $\Omega \in \Ckl{0}{1},$ there $\exists \qty{u_k} \subset \, \text{C}^{\infty}_{\overline{\Omega}}\qty(\R^{d}) \, \text{\textit{s.t.}} \, u_k \to u \, \text{in} \, \WkpSet[1][p]{\Omega}.$ Set
	\[
		\trace u = \lim_{k\to \infty}\tilde{\trace}\, u_k.
	\]
	It is not totally evident that the definition is sensible, \textit{i.e.}, that $\tilde{\trace} u_k$ converges. But using our derived estimate 

	\[
		\norm{u}_{\LpSet[q]{\partial \Omega}}\leq C \norm{u}_{\WkpSet[1][p]{\Omega}},
	\]
	it is easy to show that the sequence $\qty{\tilde{\trace}u_k}$ is Cauchy in $\LpSet[q]{\partial \Omega},$ and so it is convergent.
	Also, from the arithmetic of the limits and linearity of $\tilde{\trace}$ we see $\trace$ is linear. Next, check
	\[
		\norm{\trace u}_{\LpSet[q]{\partial \Omega}} = \norm{\lim_{k\to \infty}\tilde{\trace}u_k}_{\LpSet[q]{\partial \Omega}} = \lim_{k \to \infty}\norm{u_k \restriction_{\partial \Omega}}_{\LpSet[q]{\partial \Omega}} \leq C \lim_{k\to \infty}\norm{u_k}_{\WkpSet[1][p]{\Omega}} = C \norm{u}_{\WkpSet[1][p]{\Omega}}, \forall q \in [1, p^{\#}].
	\]
	and so inded $\trace: \WkpSet[1][p]{\Omega} \to \LpSet[q]{\partial \Omega}, \forall q \in [1,p^{\#}]$ and it surely is bounded. \\


	\textit{Case $p=d$} 

	In this case, we have $\idop \in \mathcal{L}(\WkpSet[1][d]{\Omega},\WkpSet[1][r]{\Omega}), \forall r \in [1,d)$ and the previous result tells us $\trace \in \mathcal{L}\qty(\WkpSet[1][r]{\Omega},\LpSet[q]{\partial \Omega}), \forall q \in [1,r^{\#}].$ Observe that $r^{\#} = \frac{dr-r}{d-r} \to \infty$ as $r \to d^-,$ meaning $\forall q \in [1, \infty)$ there exists $r \in [1,d) \, \text{\textit{s.t.}} \, r^{\#} > q, \, \text{\textit{i.e.}} \, q \in [1,r^{\#}),$ which in fact means
	\[
		\trace \in \mathcal{L}\qty(\WkpSet[1][r]{\Omega}, \LpSet[q]{\partial \Omega}),\forall r \in [1,d), \forall q \in [1,\infty).
	\]
	But then $\trace \circ \idop: \WkpSet[1][d]{\Omega} \to \LpSet[q]{\partial \Omega}$ is a continous linear operator $\forall q \in [1,\infty),$ as it as a composition of continuous linear operators.
	\\


	\textit{Case $p>d$}

	This is the easiest case: we know that $\WkpSet[1][p]{\Omega} \hookrightarrow \CklSet{0}{\alpha}{\overline{\Omega}}, \forall \alpha \in [0,\mu^{*}],$ so in particular
	\[
		\WkpSet[1][p]{\Omega} \hookrightarrow \CkSet{0}{\overline{\Omega}} \subset \CkSet{0}{\partial \Omega} \subset \LinfSet{\partial \Omega},
	\]
	and so in total
	\[
		\WkpSet[1][p]{\Omega} \hookrightarrow \LinfSet{\partial \Omega},
	\]
	which together with the fact ($\mathcal{H}_{d-1}\qty(\partial \Omega) < \infty)$
	\[
		\LinfSet{\partial \Omega} \hookrightarrow \LpSet[q]{\partial \Omega}, \forall q \in [1, \infty),
	\]
	concludes the proof.
\end{proof}

\begin{remark}
	These results are very similiar to the results obtained in the case of embedding theorems - and it is not surprising, since it might seem we have in fact shown the embeddings of some Sobolev spaces into some Lebesgue spaces. One needs to be careful however, as it \textit{is not an embedding} - we are taking $\Omega$ open, so $\partial \Omega \not \subset \Omega.$ It only makes sense in the case of $p > d.$
\end{remark}
The chapter will be concluded by stating the compact analogue to the embedding theorems.


\begin{theorem}[Compact trace theorem]
    Let $d \in \N, \Omega \in \Ckl{0}{1},$ denote $p^{\#} = \frac{dp-p}{d-p},$ and let $\trace$ be the trace operator from the previous theorem. Then
    \[
	    \trace \in \mathcal{K}\qty(\WkpSet[1][p]{\Omega}, \LpSet[q]{\partial \Omega}),
    \]
    where $q$ is
\begin{itemize}
	\item if $p < d$, then $q \in [1, p^{\#}),$
	\item if $p = d$, then $q \in [1, \infty),$
	\item if $p > d,$ then $q \in [1, \infty],$
\end{itemize}
\end{theorem}

\begin{proof}(\textit{From: \cite{bulicekUvodModerniTeorie2018}})
	\textit{Case $p<d$:}
	Let us adopt the custom that when talking about the properties of $u \in \WkpSet[1][p]{\Omega},$ in the space $\LpSet[q]{\partial \Omega},$ we are always talking about the properties of $\trace u$ in $\LpSet[q]{\partial \Omega}$...\\

It will be pretty similiar to the continuous case, so let us skip only to the key estimate. We are not apriori sure in which space we will be able to comfortably show the compactness, so let first $q \in [1, \infty)$ and compute. Then we might use some interpolation estimates...
\begin{align*}
	\int_{\text{U}(0,\alpha)}\abs{u_j\qty(\tensorq{A}_j\qty(x',a_j\qty(x')))}^q\dd{x'} &\leq \abs{\int_{\text{U}(0,\alpha)}\int_{a_j(x')}^{a_j(x')+\beta}\pdv{s} \abs{u_j\qty(\tensorq{A}_j\qty(x',s))}^q \dd{s}\dd{x'}}\leq \\
											   &\leq q \sqrt{1+\alpha^{2}}\int_{\text{U}(0,\alpha)}\int_{a_j(x')}^{a_j(x')+\beta}\abs{u_j\qty(\tensorq{A}_j\qty(x', s))}\abs{\grad u_j\qty(\tensorq{A}_j\qty(x',s))}\dd{s}\dd{x'} \leq \\
											   &\leq q \sqrt{1+\alpha^{2}}\int_{G_j^{+}}\abs{u_j(x)}^{q-1}\abs{\grad u_j(x)}\dd{x} \leq q \sqrt{1+\alpha^{2}}\norm{\grad u_j}_{\LpSet[q]{G_j^+}}\norm{u_j}_{\LpSet[q'(q-1)]{G_j^+}}^{q-1},
\end{align*}
where $q'(q-1) = \frac{q}{q-1}(q-1) = q,$ so all in all
\[
	\norm{u_j}_{\LpSet[q]{G_j}}\leq C(q, \Omega) \norm{\grad u_j}_{\LpSet[q]{G_j^+}}^\frac{1}{q}\norm{u_j}_{\LpSet[q]{G_j^{+}}}^{1-\frac{1}{q}}\leq C(q,\Omega) \norm{u_j}_{\WkpSet[1][q]{G_j^+}}^{\frac{1}{q}}\norm{u_j}_{\LpSet[q]{G_j^+}}^{1-\frac{1}{q}}
\]
which leads to 
\begin{align*}
	\norm{u}_{\LpSet[q]{\partial \Omega}} &= \norm{\sum_{j=1}^m u_j}_{\LpSet[q]{\partial \Omega}} \leq \sum_{j=1}^m \norm{u_j}_{\LpSet[q]{\supp u_j \cap \partial \Omega}} \leq \sum_{j=1}^m \norm{u_j}_{\LpSet[q]{G_j}} \leq \\
					      &\leq C \sum_{j=1}\norm{u_j}_{\WkpSet[1][q]{G_j^{+}}}^{\frac{1}{q}}\norm{u_j}_{\LpSet[q]{G_j^+}}^{1-\frac{1}{q}} \leq C_1 \norm{u}_{\WkpSet[1][p]{\Omega}}^{\frac{1}{q}}\norm{u}_{\LpSet[q]{\Omega}}^{1-\frac{1}{q}}.
\end{align*}

Denote $U = \text{U}_{\WkpSet[1][p]{\Omega}}(0,1)$; we will show $\trace U$ is totally bounded in $\LpSet[q]{\partial \Omega}$ for \textit{some} $q \in [1,\infty).$ Let $\varepsilon > 0$ be given. Realize that $\forall p \in [1,d)$ it is $p < p^{*},$ and so it always holds $\WkpSet[1][p]{\Omega} \hookrightarrow \hookrightarrow \LpSet{\Omega}.$ For the moment, we pick $q = p.$
Denote $\qty{u_k}_{k=1}^m$ to be the $\delta-$net from $U$ in $\LpSet{\Omega},$ where $\delta$ will be chosen suitably later. Let now $u \in U$ be arbitrary and find $u_i \in U$ \textit{s.t.} $\norm{u-u_i}_{\LpSet{\Omega}} \leq \delta.$ Using the estimate from above we have

\[
	\norm{u-u_i}_{\LpSet[p]{\partial \Omega}} \leq C_1 \norm{u-u_i}_{\WkpSet[1][p]{\Omega}}^{\frac{1}{q}}\norm{u-u_i}_{\LpSet[p]{\Omega}}^{1- \frac{1}{q}}\leq C_1 2^{\frac{1}{q}} \delta^{1-\frac{1}{q}},
\]
where we have used the fact $u, u_i \in U.$ We see that upon choosing
\[
	\delta < \qty(\frac{\varepsilon}{C_1 2^{\frac{1}{q}}})^{\frac{1}{1-q}},
\]
we in fact have
\[
	\norm{u-u_i}_{\LpSet{\partial \Omega}} < \varepsilon,
\]
and so $\qty{\trace u_i}_{i=1}^m \subset U$ is a $\varepsilon-$net in $\LpSet{\partial \Omega},$ meaning
\[
	\trace \in \mathcal{K}\qty(\WkpSet[1][p]{\Omega}, \LpSet{\partial \Omega}).
\]

Since now $\LpSet[p]{\partial \Omega} \hookrightarrow \LpSet[q]{\partial \Omega}, \forall q \in [1,p],$ we have also shown
\[
	\trace \in \mathcal{K}\qty(\WkpSet[1][p]{\Omega}, \LpSet[q]{\partial \Omega}), \forall q \in [1,p].
\]

The remaining case is when $q \in (p, p^{\#}).$ As in the case of compact embedding of Sobolev spaces, a suitable interpolation theorem will do the job for us. It holds

\[
	\norm{u}_{\LpSet[q]{\partial \Omega}}\leq \norm{u}_{\LpSet[p]{\partial \Omega}}^{\theta} \norm{u}_{\LpSet[p^{\#}]{\partial \Omega}}^{1-\theta},
\]
where $\frac{1}{q} = \frac{\theta}{p} + \frac{1-\theta}{p^{\#}}.$ Let now $\qty{u_i}_{i=1}^m \subset U$ be such that $\qty{\trace u_i}_{i=1}^m$ is the $\beta-$net in $\LpSet{\partial \Omega}$ whose existence we have just proven $\forall \beta >0.$ Recall also that since $U$ is bounded and $\trace \in \mathcal{L}\qty(\WkpSet[1][p]{\Omega}, \LpSet[q]{\partial \Omega}) \forall q \in [1,p^{\#}],$ there exists $0< C_2 < \infty$ \textit{s.t.}
\[
	\norm{u}_{\LpSet[q]{\partial \Omega}} \leq C_2, \forall u \in U.
\]
and so $\norm{u}_{\LpSet[p^{\#}]{\partial \Omega}} \leq C_2, \forall u \in U$ in particular.

Finally, $\forall \in U$ it holds
\[
	\norm{u-u_i}_{\LpSet[q]{\partial \Omega}} \leq \norm{u-u_i}_{\LpSet{\partial \Omega}}^{\theta}\norm{u-u_i}_{\LpSet[p^{\#}]{\partial \Omega}}^{1-\theta} \leq \beta^{\theta}\qty(2C_2)^{1-\theta},
\]
so if we choose
\[
	\beta <\qty(\frac{\varepsilon}{\qty(2C_2)^{1-\theta}})^{\frac{1}{\theta}},
\]
then
\[
	\norm{u-u_i}_{\LpSet[q]{\partial \Omega}} \leq \varepsilon, \forall u \in U.
\]
which concludes the proof $\qty{\trace u_i}_{i=1}^m$ is an $\varepsilon-$net in $U$ in $\LpSet[q]{\partial \Omega}$ also for $q \in (p,p^{\#})$. In total, we have showed the compactness of the trace operator for all $q \in [1, p^{\#}).$\\

\textit{Case $p = d$}

In this case $\trace \in \mathcal{L}(\WkpSet[1][d]{\Omega}, \LpSet[q]{\partial \Omega}), \forall q \in [1,\infty).$ We also know $\idop \in \mathcal{L}\qty(\WkpSet[1][d]{\Omega}, \WkpSet[1][r]{\Omega}), \forall r \in [1,d)$ and that
\[
	\trace \in \mathcal{K}\qty(\WkpSet[1][r]{\Omega}, \LpSet[q]{\partial \Omega}), \forall q \in [1, r^{\#}).
\]
Repating the same arguments, we see $r^{\#} \to \infty$ as $r \to d^-,$ meaning $\forall q \in [1,\infty)$ there exists $r \in [1,d)$ \textit{s.t.} $r^{\#} > q,$ and consequently this implies
\[
	\forall q \in [1, \infty) \exists r \in [1,d) \, \text{\textit{s.t.}} \, \trace \in \mathcal{K}\qty(\WkpSet[1][r]{\Omega}, \LpSet[q]{\partial \Omega}).
\]
But then the operator $\trace \circ \idop: \WkpSet[1][d]{\Omega} \to \LpSet[q]{\partial \Omega}$ is compact $\forall q \in [1, \infty),$ as it is a composition of a (linear) continuous and compact operator. \\

\textit{Case $p > d$}

This case is again trivial: from the embedding theorems, it holds
\[
	\WkpSet[1][p]{\Omega} \hookrightarrow \hookrightarrow \CklSet{0}{\alpha}{\overline{\Omega}}, \forall \alpha \in [0,\mu),
\]
so in particular
\[
	\WkpSet[1][p]{\Omega} \hookrightarrow \hookrightarrow \CkSet{0}{\overline{\Omega}} \subset \CkSet{0}{\partial \Omega} \subset \LinfSet{\partial \Omega},
\]
meaning
\[
	\WkpSet[1][p]{\Omega} \hookrightarrow \hookrightarrow \LinfSet{\partial \Omega},
\]
which together with the embedding
\[
	\LpSet[q]{\partial \Omega} \hookrightarrow \LinfSet{\partial \Omega}, \forall q \in [1, \infty).
\]
completes the proof for any $q \in [1, \infty].$ We are done once again.

\end{proof}

\subsection{Fine properties of Sobolev spaces}
\label{sec:fine_properties}
In this last chapter, we present some other useful properties of Sobolev spaces.

\subsubsection{Composition of Sobolev functions}
\label{sec:composition}

In the chapter about extension of functions from Sobolev spaces we have seen: $U,V \subset \R^{d}$ open, $u \in \WkpSet[1][p]{U}, \Phi: U \to V$ a $\, \text{C} \,^1$-diffemorphism $\Rightarrow u \circ \Phi \in \WkpSet[1][p]{V}.$ Is there any other class of mappings that guarantee the composition remains in some Sobolev spaces?

\begin{theorem}[Derivative of superposition]
	Let $\Omega \subset \R^{d}$ open, $p \in [1, \infty]$, $u: \Omega \to \R^{d}$ be in $\WkpSet[1][p]{\Omega}.$ Denote for an arbitrary $a \in \R$ the set
	\[
		\Omega_a = \qty{x \in \Omega | u(x) = a}.
	\]
	Then 
	\begin{enumerate}
		\item $\grad u = 0$ on $\Omega_a,$ 
		\item if $f \in \CklSet{0}{1}{\R^{d}} \, \text{\textit{s.t.}} \, \norm{f}_{\LinfSet{\Omega}} < \infty$ (it is a Lipschitz continuous function), then $f \circ u - f(0) \in \WkpSet[1][p]{\Omega}$ and it holds
			\[
				\grad \qty(f \circ u) = \begin{cases}
					\qty(f' \circ u)\grad u, & \, \text{\textit{a.e.}} \,\, \text{in} \, \qty{u \notin S},\\
					0, & \, \text{\textit{a.e.}} \, \, \text{in} \, \qty{u \in S}
				\end{cases},
			\]
			where 
			\[
				S = \qty{s \in \R| f'(s) \, \text{does not exist} \,}
			\]
			(in the strong sense).
	\end{enumerate}
\end{theorem}

\begin{proof}(\textit{From: the lectures \& \cite{bulicekUvodModerniTeorie2018}})
	The proof has been presented for the case $f \in \CkSet{1}{\R}, \norm{f}_{\LinfSet{\R}} < \infty$. In the general setting, one can play with \textit{a.e.} convergence and Rademacher theorem.

	Denote $L = \norm{f}_{\LinfSet{\R}}.$ Since $f$ is continuous, $f \circ u$ is measurable. Moreover, from the Lipschitzity on the whole $\R$ it follows
	\[
		\abs{(f \circ u)(x) - f(0)} \leq L \abs{u(x) - 0} = L \abs{u(x)},
	\]
	and since $u \in \LpSet{\Omega},$ also $f\circ u - f(0) \in \LpSet{\Omega}.$ To obtain the wanted result, we need to show $\grad\qty(f \circ u)$ exists, so let us move on to the second claim. The strategy is to work with some regularization and then pass to the limit with the support size. In concrete terms, pick $\varphi \in \DSet{\Omega},$ and regularization $u_{\varepsilon} \, \text{\textit{s.t.}} \, \dist\qty(\supp \varphi, \partial \Omega) > 2 \varepsilon.$ Then we can write
	\[
		- \int_{\Omega}\qty(f \circ u_{\varepsilon})(x) \partial_{i}\varphi(x)\dd{x} = \int_{\Omega}\qty(f' \circ u_{\varepsilon})(x)\partial_{i}u_{\varepsilon}(x)\varphi \dd{x},
	\]
using per partes and chain rules (those are strong derivatives). Now we would like to pass to the limit $\varepsilon \to 0^+.$ Take the LHS:
\[
	\lim_{\varepsilon\to 0^{+}}|\int_{\Omega}\qty(f \circ u_{\varepsilon} - f \circ u)\partial_{i}\varphi \dd{x}| \leq \lim_{\varepsilon \to 0^{+}}\int_{\Omega}\abs{f \circ u_{\varepsilon} - f \circ u}\abs{\partial_{i}\varphi}\dd{x} \leq L \norm{\varphi}_{\WkpSet[1][\infty]{\Omega}}\lim_{\varepsilon \to 0^+}\int_{\Omega \cap \supp \varphi}|u_{\varepsilon} - u|\dd{x} = 0,
\]
as $u_{\varepsilon} \to u$ in $\LpSet[1]{\Omega \cap \supp \varphi}$ (notice from the theorem on mollification we actually obtain this convergence in $\LpSet[1]{\Omega_{\varepsilon}},$ but we have made a good choice of $\varphi$) The RHS can be manipulated

\begin{align*}
	&\lim_{\varepsilon \to 0^+}\abs{\int_{\Omega}\qty(f' \circ u_{\varepsilon})\partial_{i}u_{\varepsilon}\varphi - \qty(f' \circ u)\partial_{i}u \varphi\dd{x}} \leq \lim_{\varepsilon \to 0^{+}}\int_{\Omega}\abs{\qty(f' \circ u_{\varepsilon} - f' \circ u)}|\partial_{i}u| |\varphi|\dd{x} + \\
	&+\lim_{\varepsilon \to 0^{+}}\int_{\Omega}\abs{f' \circ u_{\varepsilon}}\abs{\partial_{i}u_{\varepsilon} - \partial_{i}u}\abs{\varphi}\dd{x}\leq  \\
																				    &\leq \norm{\varphi}_{\LinfSet{\Omega}}\lim_{\varepsilon \to 0^{+}}\int_{\Omega \cap \supp \varphi}\abs{f' \circ u_{\varepsilon} -f' \circ u}\abs{\partial_{i}u}\dd{x} + L \norm{\varphi}_{\LinfSet{\Omega}} \lim_{\varepsilon \to 0^+} \int_{\Omega \cap \supp \varphi}\abs{\partial_{i}u_{\varepsilon} - \partial_{i}u}\dd{x} = \\
																				    & = \norm{\varphi}_{\LinfSet{\Omega}}\lim_{\varepsilon \to 0^+} \int_{\Omega}\abs{f' \circ u_{\varepsilon} - f' \circ u}\abs{\partial_{i}u}\dd{x},
\end{align*}
where we have used $\grad u_{\varepsilon} \to \grad u$ in $\LpSet[1]{\Omega \cap \supp \varphi}.$ As for the second integral, recall that $u_{\varepsilon} \to u $ \textit{a.e.} in $\Omega \cap \supp \varphi $ and that $f'$ is globally continous. Thus with the majorant
\[
	\abs{f' \circ u_{\varepsilon}- f' \circ u}|\partial_{i} u| \leq L \norm{u}_{\WkpSet[1][p]{\Omega}} \abs{u_{\varepsilon} - u},
\]
that is integrable on the bounded set $\supp \varphi \cap \Omega$ we see the second integral is zero aswell. All in all, we have shown
\[
	- \int_{\Omega}\qty(f \circ u)\partial_{i}\varphi\dd{x} = \int_{\Omega}\qty(f' \circ u)\partial_{i}u \varphi\dd{x}, \forall \varphi \in \DSet{\Omega},
\]
which is exactly what we need to state that the $\partial_{i}$-weak derivative of $f \circ u$ exists and is equal to $(f' \circ u) \partial_{i}u$. Finally, the following estimate holds:
\[
	\norm{\underbrace{\grad\qty(f \circ u)}_{= \grad\qty(f \circ u - f(0))}}_{\LpSet{\Omega}} = \norm{\qty(f' \circ u)\grad u}_{\LpSet{\Omega}} \leq L \norm{\grad u}_{\LpSet{\Omega}} \leq L \norm{u}_{\WkpSet[1][p]{\Omega}},
\]
and so
\[
	f\circ u - f(0) \in \WkpSet[1][p]{\Omega}.
\]

Let us now deal with the first assertion. We will first show $\grad u = 0$ \textit{a.e.} on $\Omega_0$. For that, choose a special function

\[
	f(x) =
	\begin{cases}
		x, & x>0,\\
		0, & x\leq 0,
	\end{cases}
\]
and so $f \circ u = u^{+}.$ If we now set
\[
	f_{\varepsilon}(x) =
	\begin{cases}
		\qty(x^{2}+\varepsilon^{2})^{1/2} - \varepsilon, & x > 0, \\
		0, & x\leq 0,
	\end{cases}
\]
with the derivative being
\[
	f'_{\varepsilon}(x) =
	\begin{cases}
		\frac{x}{\qty(x^{2}+\varepsilon^{2})^{1/2}}, & x >0, \\
		0, x \leq 0
	\end{cases}
\]
and so $\lim_{\varepsilon \to 0^+}f_{\varepsilon}(x) = f(x),$ and $\lim_{\varepsilon \to 0^+}f'_{\varepsilon}(x) = \chi_{\R^+}(x) $ \textit{a.e.} in $\R$, meaning
\[
	\lim_{\varepsilon \to 0^+}f_{\varepsilon}\circ u = u^+, \lim_{\varepsilon \to 0^+}f'_{\varepsilon}\circ u = \chi_{x \in \Omega| u(x) > 0},
\]
\textit{a.e.} in $\Omega$. Using the preivously obtained general expression and passing to the limit yields

\[
	- \int_{\Omega}\qty(f \circ u)\grad \varphi\dd{x} = \int_{\Omega}\grad u \varphi \chi_{\qty{x\in \Omega| u(x) > 0}}\dd{x},
\]
and realizing\footnote{Recall $u^+ = \max\qty(0, u), u^- = -\min\qty(0, u)$ are both nonnegative functions, $u = u^+ - u^-, \abs{u} = u^+ + u^-.$ It holds trivially $u^- = - \min\qty(0, u) = \max\qty(0, -u) = \qty(-u)^{+}.$} $u^- = (-u)^+$, we also have
\[
	- \int_{\Omega}\qty(f \circ u)\grad \varphi\dd{x} = -\int_{\Omega}\grad u \varphi \chi_{\qty{x\in \Omega| u(x) < 0}}\dd{x},
\]
meaning (in the weak sense)
\[
	\grad u^+ = \grad u \chi_{\qty{x \in \Omega | u(x) >0}}, \grad u^- = -\grad u\chi_{\qty{x \in \Omega | u(x) <0}}.
\]
But since $u = u^+ - u^-$, it holds $\grad u = \grad u^+ - \grad u^-$ \textit{a.e.}. We see $\grad u^- = \grad u^+ = 0,$ on $\Omega_0$, and so $\grad u = 0$ on \textit{a.e.} on $\Omega_0.$ Taking $\tilde{u} = u + c$ for $c \in \R$ arbitrary, we see $\grad \tilde{u} = \grad u = 0$ \textit{a.e.} on $\Omega_0$, meaning $\grad \tilde{u} = 0$ \textit{a.e.} on $\qty{x \in \Omega| \tilde{u}-c = 0} = \Omega_c.$ Since $\tilde{u}, c$ were arbitrary, we are done with this proof.

If $u \notin S$, the above presentation works just fine. If $u(x) \in S,$ we know $\grad u = 0$ \textit{a.e.} on $\Omega_u$, and since $\lambda\qty(S) = 0$ by Rademacher theorem, we have also the above conclusion. This argumentation is sloppy, but has not been presented in full detail.

\end{proof}

\subsubsection{Difference quotients}
\label{sec:difference_quotients}

There is some interplay between the weak derivative and difference quotients, which can be useful in some proofs, \textit{e.g.}, regularity results for elliptic problems.
We have seen that if a function posseses classical derivatives, it also has weak derivatives and they conincide \textit{a.e.}. Is something there something to be said about the opossite implication?

\begin{definition}[Difference quotient]
    Let $u: \R^{d} \to \R, h \in \R.$ For each $i \in \qty{1, \dots, d}$ we define the difference quotient as
    \[
	    \Delta_{h}^i u(x) = \frac{u\qty(x+h e_i) - u(x)}{h},
    \]
    where $e_i$ is the canoncal base vector of $\R^{d}.$
\end{definition}

\begin{remark}
	If $\partial_{i}u(x)$ exists, then clearly
	\[
		\partial_{i}u(x) = \lim_{h\to 0}\Delta_h^i u(x).
	\]
\end{remark}

Let us jump straight to the main result.

\begin{theorem}(\textit{From: \cite{bulicekUvodModerniTeorie2018}})
	Let $\Omega \subset \R^{d}$ open, $p \in [1,\infty]$ and $u \in \LpSet{\Omega}.$ Denote $\forall \delta>0$ the set
	\[
		\Omega_{\delta} = \qty{x \in \Omega | \dist\qty(x, \partial \Omega)>\delta}.
	\]
	Then it holds

	\begin{enumerate}
		\item If moreover $u \in \WkpSet[1][p]{\Omega},$ then $\forall i \in \qty{1, \dots, d}, \delta \in (0,1), h \in \qty(0,\frac{\delta}{2})$
			\[
				\norm{\Delta_h^i u}_{\LpSet{\Omega_{\delta}}}\leq \norm{\partial_{i}u}_{\LpSet{\Omega}}.
			\]
		\item Let $p \in (1, \infty]$ and let there exist $\qty{C_i}_{i=1}^d$ constants \textit{s.t.} $\forall i \in \qty{1, \dots, d}, \delta \in (0,1), h \in \qty(0, \frac{\delta}{2})$ it holds
			\[
				\norm{\Delta_h^i u}_{\LpSet{\Omega_{\delta}}}\leq C_i.
			\]
			Then $u \in \WkpSet[1][p]{\Omega}$ and moreover $\forall i \in \qty{1, \dots, d}$ we have
			\[
				\norm{\partial_{i}u}_{\LpSet{\Omega}} \leq C_i.
			\]
	\end{enumerate}
\end{theorem}
\begin{proof}
	\textit{Ad 1.}
	Let first $p \in [1, \infty).$ \textit{What else do we use to estimate differences then..., right?} If needed, extend $u$ by zero outside of $\Omega$ and set $u_{\varepsilon} = u \star \omega_{\varepsilon}.$ Recall that $u_{\varepsilon} \to u$ in $\WkpSet[1][p]{\Omega_{\frac{\delta}{2}}}.$ Realize that since $u_{\varepsilon} \in \CkSet{1}{\R^{d}},$ we have
	\[
		\Delta_h^i u_{\varepsilon}(x) = \frac{u_{\varepsilon}\qty(x+h e_i)- u_{\varepsilon}\qty(x)}{h} = \frac{1}{h}\int_0^h \partial_{t} u_{\varepsilon}\qty(x+te_i)\dd{t}.
	\]
	Let us compute $\norm{\Delta_h^i}_{\LpSet[p]{\Omega_{\delta}}}.$ We have
	\begin{align*}
		\abs{\Delta_h^i u_{\varepsilon}}^{p} &= \abs{\frac{1}{h}\int_0^h \partial_{t} u_{\varepsilon}\qty(x+te_i)\dd{t}}^p \leq \frac{1}{h^p}\qty(\int_0^h \abs{\partial_t u_{\varepsilon}\qty(x+te_i)}\dd{t})^p \leq \frac{1}{h^p}\qty(\qty(\int_0^h \abs{\partial_t u_{\varepsilon}\qty(x+te_i)}^p \dd{t})^{\frac{1}{p}}\qty(\int_0^h 1 \dd{t})^{\frac{p-1}{p}})^p = \\ 
				       &=\frac{1}{h^p}\int_0^h \abs{\partial_t u_{\varepsilon}(x+te_i)}^p \dd{t}h^{p-1} = \frac{1}{h}\int_0^h \abs{\partial_t u_{\varepsilon}\qty(x+te_i)}^p\dd{t},
	\end{align*}
	and so integrating over $\Omega_{\delta}$ yields
	\[
		\norm{\Delta_h^i u_{\varepsilon}}_{\LpSet{\Omega_{\delta}}}^p \leq \frac{1}{h} \int_{\Omega_{\delta}}\int_0^h \abs{\partial_t u_{\varepsilon}\qty(x+te_i)}^p\dd{t}\dd{x} = \frac{1}{h}\int_0^h \int_{\Omega_{\delta}}\abs{\partial_t u_{\varepsilon}\qty(x+te_i)}^p\dd{x}\dd{t},
	\]
	denote $z = x + te_i$, then $z \in \Omega_{\delta} + te_i,$ meaning the points have shifted towards $\Omega$ in one direction by at most $t \leq h < \frac{\delta}{2}.$ If we integrate over $\Omega_{\frac{\delta}{2}}$ instead, we make the domain larger and so 
	\[
		\norm{\Delta_h^i u_{\varepsilon}}_{\LpSet{\Omega_{\delta}}}^p\leq \frac{1}{h}\int_0^h \int_{\Omega_{\frac{\delta}{2}}}\abs{\partial_t u_{\varepsilon}(z)}^p\dd{z}\dd{t}\leq \frac{1}{h} h \norm{\partial_t u_{\varepsilon}}_{\LpSet[p]{\Omega_{\frac{\delta}{2}}}}^p,
	\]
	so if we pass to the limit $\varepsilon \to 0^+$ (we are in $\Omega_{\delta}$), we see

	\[
		\norm{\Delta_h^i u}_{\LpSet{\Omega_{\delta}}} \leq \norm{\partial_t u}_{\LpSet{\Omega_{\frac{\delta}{2}}}} \leq \norm{\partial_t u}_{\LpSet{\Omega}}.
	\]
	If now $p = \infty,$ the things are not that simple - $\Omega$ is not bounded apriori, so we have no ordering of Lebesgue spaces. Let us fix that: take some $R>0$ and denote $\Omega^R = \Omega \cap \text{U}(0,R).$ If $u \in \WkpSet[1][\infty]{\Omega},$ it must be $u \in \WkpSet[1][p]{\Omega^R}$ for all $p \in [1, \infty).$ On $\Omega^R_{\delta}$ we have from the above
	\[
		\norm{\Delta_h^i u}_{\LpSet{\Omega_{\delta}^R}}\leq \norm{\partial_t u}_{\LpSet{\Omega^R}},
	\]
	and since $\Omega_{\delta}^R$ is bounded, we can pass to the limit $p \to \infty$ and write
	\[
		\norm{\Delta_h^i u}_{\LinfSet{\Omega_{\delta}^R}}\leq \norm{\partial_t u}_{\LinfSet{\Omega^R}},
	\]
	and passing to the limit $R \to \infty$ we obtain
	\[
		\norm{\Delta_{h}^i u}_{\LinfSet{\Omega_{\delta}}} \leq \norm{\partial_t u}_{\LinfSet{\Omega}}.
	\]

	\textit{Ad 2.}
	Once again extend $u$ by zero outside of $\Omega$ and let for the moment $p \in \qty(1,\infty)$. Fix $\qty{h_n} \subset \qty(0, \frac{\delta}{2}) \, \text{\textit{s.t.}} \, h_n \to 0$ and consider the functions
	\[
		v_n^i = \Delta_{h_n}^i u \chi_{\Omega_{2 h_n}}.
	\]
	Clearly (meaning from the assumptions)
	\[
		\norm{v_n}^i_{\LpSet{\Omega}} = \norm{\Delta^i_{{h_n}} u}_{\LpSet{\Omega_{2 h_n}}}  \leq C_i,
	\]
	and from the reflexivity of $\LpSet{\Omega}, p \in \qty(1, \infty),$ we know $\exists v_i \in \LpSet{\Omega}$ such that
	\[
		v_n^i \rightharpoonup v^i,
	\]
	as $n \to \infty.$ Since the norm is weak lower semicontinuous, we also have the estimate
	\[
		\norm{v_i}_{\LpSet{\Omega}} \leq C_i.
	\]
	It remains to show now that $v_i$ are weak derivatives of $u$. Before we procced, let us mention the following identity: let $\varphi \in \DSet{\Omega},$ then
	\[
		\int_{\R^{d}}\Delta_h^i u \varphi\dd{x} = \int_{\R^{d}}u \Delta_{-h}^i \varphi\dd{x}.
	\]
	Really, it holds \textbf{THIS NEEDS TO BE FINISHED}
	\[
		\int_{\R^{d}}\Delta_h^i u \varphi\dd{x} = \int_{\R^{d}}\frac{u\qty(x+he_i) - u\qty(x)}{h}\varphi\dd{x} = \int_{\R^{d}}\frac{\varphi(x) u\qty(x+he_i) - \varphi(x) u(x)}{h}\dd{x},
	\]
	change the variables $y = x+ he_i,$ then the above integral is equal to
	\[
		\int_{\R^{d}}\frac{\varphi\qty(y - he_i)u(y) - \varphi(y - he_i) u\qty(y-he_i)}{h}\dd{y} =
	\]
	Using this, we can obtain 
	\begin{align*}
		\int_{\Omega}v_i \varphi\dd{x} &= \lim_{n \to \infty}\int_{\Omega}v_n^i \varphi \dd{x} = \lim_{n \to \infty}\int_{\Omega_{2h_n}}\Delta_{h_n}^i u \varphi\dd{x} = \lim_{n \to \infty} \int_{\R^{d}}\Delta_{h_n}^i u \varphi\dd{x} = \\
					       &= \lim_{n\to \infty}\int_{\R^{d}}u \Delta_{-h_n}^i \varphi \dd{x} = \lim_{n \to \infty}\int_{\R^{d}}u \frac{\varphi(x- h_n e_i) - \varphi(x)}{h_n}u(x)\dd{x} =  -\int_{\R^{d}}u \partial_{i}\varphi\dd{x} = \\
					       &=- \int_{\Omega}u \partial_{i} \varphi \dd{x}
	\end{align*}
which really means $v_i = \partial_{i} u$ in the weak sense. The fact we can integrate over $\R^{d}$ instead of $\Omega_{2h_n}$ comes from the fact $\supp \varphi \subset \Omega_{2h_n}$ for some $n$ large enough, and since the whole integrand is zero on any larger set then $\Omega_{2h_n},$ meaning we without a doubt expand the integration domain to the whole space; the interchange of differentiation and integration is simple, because $\varphi$ has a compact support. All in all, we have shown $v_i = \partial_{i} u$ weakly and $\norm{v_i}_{\LpSet{\Omega}} \leq C_i,$ so we are done with the case $p \in (1, \infty).$

If $p = \infty,$ we can repeat the same arguments as in the above case: create a bounded domain $\Omega_R = \Omega \cap \text{U}(0,R),$ realize that
\[
	\norm{v_i}_{\LpSet{\Omega^R}} \leq C_i, \forall p \in (1, \infty),
\]
and then pass to the limits $R \to \infty, p \to \infty.$

\end{proof}
\begin{remark}
    We have thus shown that if $p \in (1, \infty),$ then
    \[
	    \Delta_h^i u \rightharpoonup \partial_{i} u, h \to 0^+.
    \]
\end{remark}

\subsubsection{Representation of duals}
\label{sec:dual_representation}

The last stop within our adventure through Sobolev spaces will be the dual. We present a representation theorem for dual spaces of certain Sobolev spaces. The proof will use techniques developed later, in the chapter \ref{sec:monotone_operator}.

\begin{theorem}
	Let $\Omega \in C^{0,1}.$ Let $X = \WkpzeroSet[1][r]{\Omega}, r \in \qty(1,\infty)$ with equivalent norm $|\norm{u}| = \norm{\grad u}_{\WkpzeroSet[1][r]{\Omega}}.$ Then $\forall \Phi \in X^{*} \exists \vb{F} \in \LpSet[r']{\Omega}$ such that
	\[
		\forall \varphi \in \WkpzeroSet[1][r]{\Omega}: \Phi(\varphi) = \int_{\Omega}\vb{F} \vdot \grad \varphi\dd{x},
	\]
	and moreover it holds $ \norm{\Phi}_{X^{*}} = \norm{\vb{F}}_{\LpSet[r']{\Omega}}.$
\end{theorem}
\begin{proof}(\textit{From: the lectures})
	We solve the problem
	\begin{equation}
		\begin{cases}
			- \divergence{\qty(|\grad u|^{r-2}\grad u)} &= \Phi,  \, \text{in} \, \Omega, \\
			u &= 0,  \, \text{on} \, \partial \Omega
		\end{cases}.
	\end{equation}
	Let us check the solution exists: denote
	\[
		a_0 =0, \vb{a}\qty(x,z,\vb{p}) = \abs{\vb{p}}^{r-2}\vb{p}.
	\]
	Then clearly
	\[
		\abs{a_i\qty(x, z, \vb{p})} \leq \abs{\vb{p}}^{r-1},
	\]
	so we have the right grow conditions, next:
	\[
		\vb{a}\qty(x,z,p) \vdot \vb{p} = \abs{\vb{p}}^{r-2} \vb{p} \vdot \vb{p} = \abs{\vb{p}}^r,
	\]
	so we have coercivity, and finally monotonicity; realize that actually\footnote{$\grad \abs{\vb{p}}^r = r \abs{\vb{p}}^{r-1} \abs{\vb{p}}^{-1} \vb{p} = r \abs{\vb{p}}^{r-2}$}
	\[
		\vb{a}(x,z,p) = \frac{1}{r} \grad \abs{\vb{p}}^r,
	\]
and for $r>1$ is the function $\vb{p} \mapsto \abs{\vb{p}}^r$ strictly convex. So from the classification of convex functions we know
\[
	\frac{1}{r}\abs{\vb{p}}^r \, \text{is strictly convex} \, \Leftrightarrow \grad \frac{1}{r} \abs{\vb{p}}^r = \vb{a} \, \text{is strictly monotone} \,.
\]

By the existence \& uniqueness theorem on problems with monotone operator, we conclude such $u \in \WkpzeroSet[1][r]{\Omega}$ exists and is unique by. The weak formulation of the above problem is:
	\[
		\forall \varphi \in \WkpzeroSet[1][r]{\Omega}: \int_{\Omega}|\grad u|^{r-2} \grad u \vdot \grad \varphi \dd{x} = \Phi\qty(\varphi).
	\]
	So it seems proimising to set $\vb{F} = |\grad u|^{r-2} \grad u.$ If we test now the weak formulation with $u$ itself (recall what norm are we using on $X$)
	\[
		\norm{\grad u}_{\LpSet[r]{\Omega}}^r = \Phi(u) \leq \norm{\Phi}_{X^{*}} \abs{\norm{u}} = \norm{\Phi}_{X^{*}} \norm{\grad u}_{\LpSet[r]{\Omega}}.
	\]
	If now $\norm{\grad u}_{\LpSet[r]{\Omega}} = 0$, then $\Phi = 0$ and we are finished; if it is nonzero, then we can divide and write
	\[
		\norm{\grad u}_{\LpSet[r]{\Omega}}^{r-1} \leq \norm{\Phi}_{X^{*}}.
	\]
	Realize now
	\[
		\norm{\grad u}_{\LpSet[r]{\Omega}}^{r-1} = \norm{| \grad u|^{r-1}}_{\LpSet[\frac{r}{r-1}]{\Omega}} = \norm{\vb{F}}_{\LpSet[r']{\Omega}} \Rightarrow \norm{\vb{F}}_{\LpSet[r']{\Omega}} \leq \norm{\Phi}_{X^{*}}.
	\]
	On the other hand:
	\[
		\norm{\Phi}_{X^{*}} = \sup_{\text{B}_{X}(0,1)} \abs{\Phi(\varphi)} = \sup_{\text{B}_{X}(0,1)} \abs{\int_{\Omega}\vb{F} \vdot \grad \varphi \dd{x}} \leq \sup_{\text{B}_{X}(0,1)} \norm{\vb{F}}_{\LpSet[r']{\Omega}} \norm{\grad \varphi}_{\LpSet[r]{\Omega}} = \norm{\vb{F}}_{\LpSet[r']{\Omega}},
	\]
	which concludes $\norm{\Phi}_{X^{*}} = \norm{\vb{F}}_{\LpSet[r']{\Omega}}.$
\end{proof}




