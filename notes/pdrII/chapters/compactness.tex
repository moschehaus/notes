% !TEX root = ../main.tex

\section{Nonlinear elliptic equations - compactness methods}
\label{sec:nonlinear_elliptic_compact}

So far we have dealt with only linear equations, \textit{i.e.}, linear operators. When dealing with nonlinear operators, things get much more involved. This chapter will be dedicated to the study of nonlinear problems using compactness methods, which will especially allow us to solve problems with nonlinearities in the non-leading terms. \textbf{Be especially careful about the assumptions on the operators - they do not need to be linear.}

Let us start with some technical, mostly known lemmas.

\subsection{Nemytskii operators}
\label{sec:nemytskii_operators}

\begin{lemma}[Fatou]
	Let $\Omega \subset \R^{d}$ be measurable, let $\qty{f_n}$ be a sequence of measurable nonnegative functions. Then it holds
	\[
		\int_{\Omega}\liminf_{n\to \infty}f_n\dd{x} \leq \liminf_{n \to \infty}\int_{\Omega}f_n\dd{x}.
	\]
	(Both integrals can be infinity.)
    
\end{lemma}

\begin{theorem}[Iegorov]
	Let $\Omega \subset \R^{d}$ be measurable and of finite Lebesgue measure, let $\qty{f_n}$ and $f$ be measurable and finite \textit{a.e.} in $\Omega$. Then the following statements are equivalent
\begin{enumerate}
	\item $f_n(x) \to f(x), \forall \, \text{\textit{a.a.}} \,x \in \Omega,$
	\item $\forall \varepsilon >0 \exists G \subset \Omega \, \text{open} \, \, \text{\textit{s.t.}} \, \lambda(G) < \varepsilon$ and $f_n \rightrightarrows f$ on $\Omega / G.$
\end{enumerate}
\end{theorem}

\begin{theorem}[Vitali]
    Let $\Omega \subset \R^{d}$ be measurable and of finite Lebesgue measure, let $\qty{f_n}$ and $f$ be measurable \textit{s.t.} $f_n(x) \to f(x)$for \textit{a.a.} $x \in \Omega$. Let moreover be true
    \[
	    \forall \varepsilon > 0 \exists \delta >0: \forall n \in \N, \forall H \subset \Omega \, \text{\textit{s.t.}} \, \lambda(H) < \delta \Rightarrow \int_{H}|f_n|\dd{x} \leq \varepsilon.
    \]
    Then
    \[
	    \lim_{n\to \infty} \int_{\Omega}f_n\dd{x} = \int_{\Omega}f\dd{x}.
    \]
\end{theorem}

\begin{remark}[What is $f\qty(\vdot, x)$?]
    In the following definitions and throughout the chapter in general, we use the "$\vdot$" to denote variables in the following sense. Just to make it clear, this means the following.
    If $U,V,W$ are some sets and $f: U \times V \to W,$ we write
    \begin{itemize}
	    \item $f\qty(u, \vdot): V \to W,$ as the mapping $V \ni v \mapsto f(u,v) \in W$ for some $u \in U$ fixed,
	    \item $f\qty(\vdot, v): U \to W,$ as the mapping $U \ni u \mapsto f(u,v) \in W$ for some $v \in V$ fixed.
    \end{itemize}
\end{remark}

\begin{definition}[Caratheodory function]
	Let $\Omega \subset \R^{d}$ be measurable, $N \in \N.$ We say the function $f: \Omega \times \R^{N} \to \R$ is Caratheodory if
	\begin{enumerate}
		\item $f\qty(x, \vdot): \R^{N} \to \R$ is continuous for almost all $x \in \Omega,$
		\item $f\qty(\vdot, y): \Omega \to \R$ is measurable for all $y \in \R^{N}.$
	\end{enumerate}
\end{definition}

\begin{definition}[Nemytskii operator]
	Let $\Omega \subset \R^{d}$ be measurable, $N \in \N$ and $f: \Omega \times \R^{N} \to \R$ be Caratheodory. For $\vb{u}: \Omega \to \R^N, x \in \Omega$ we define the Nemytskii operator $N_f$: as
    \[
	    N_f: \vb{u}(x) \mapsto f(x,\vb{u}(x)),
    \]
    meaning $N_f$ is an operator between function spaces, whose image is the function $N_f\vb{u}:\Omega \to \R $ \textit{s.t.}
    \[
	    N_f\vb{u}(x) = f\qty(x,\vb{u}(x)), x \in \Omega.
    \]
    
\end{definition}

\begin{remark}
    We write $\vb{u}$ in bold to stress $u: \Omega \to \R^{N}$ is a vector valued mapping, even though we might drop the notation later on.
\end{remark}

\begin{remark}
    This definition is similiar to the one used in the Caratheodory theory of ordinary differential equations, but not the same.
\end{remark}

\begin{theorem}[On Nemytskii operators]
  Let $\Omega \subset \R^{d}$ be measurable, $N \in \N, f: \Omega \times \R^{N} \to \R$ be Caratheodory. Then

  \begin{enumerate}
	  \item if $\vb{u}: \Omega \to \R^{N}$ is measurable, then $N_f \vb{u}: \Omega \to \R$ is also measurable,
	  \item if there are numbers $p_i \in [1, \infty), i \in \qty{1, \dots, N},$ an exponent $p \in [1, \infty),$ a function $g \, \text{\textit{s.t.}} \, g \in \LpSet{\Omega},$ and a constant $C\geq 0$ such that for almost all $x \in \Omega$ and all $y \in \R^{N}$ it holds
		  \[
			  \abs{f(x,y)}\leq g(x) + C \sum_{i=1}^N\abs{y_i}^{\frac{p_i}{p}},
		  \]
		  then the Nemytskii operator $N_f: \vb{u} \mapsto N_f\vb{u}$ is continuous from $\LpSet[p_1]{\Omega} \times \LpSet[p_2]{\Omega} \times \dots \LpSet[p_N]{\Omega}$ to $\LpSet{\Omega}$ and moreover it maps sets bounded in $\LpSet[p_1]{\Omega} \times \LpSet[p_2]{\Omega} \times \dots \LpSet[p_N]{\Omega}$ to sets bounded in $\LpSet{\Omega}.$ We write
		  \[
			  N_f \in \mathcal{C}\qty(\LpSet[p_1]{\Omega}, \dots, \LpSet[p_N]{\Omega}; \LpSet{\Omega}).
		  \]
  \end{enumerate}
\end{theorem}

\begin{proof}(\textit{From: the lectures})
	No proof
\end{proof}

\begin{remark}
	The information about boundedness of images of bounded sets is not trivial - \textit{the Nemytskii operator is not linear in general.}
\end{remark}

\subsection{Fixed point theorems}
\label{sec:fixed_points}
The majority of the problems we will be dealing with will be solved by using some fixed points theorem. Here we state two of them (without proofs).

\begin{definition}[Compact operator]
	Let $X,Y$ be normed linear spaces, $M \subset X.$ The mapping $F:M \to Y$ is called a compact operator on $M$ into $Y$ provided
	\begin{enumerate}
		\item $F$ is continuous,
		\item $F\qty(M \cap K) \subset Y$ is relatively compact in $Y$ for any bounded $K \subset X.$
	\end{enumerate}
\end{definition}

\begin{remark}

	\begin{itemize}
		\item	We have no linearity of $F$! So continuity cannot follow from compactness and we have to assume it explicitely. In general, compactness $\Rightarrow $ boundedness $\neq$ continuity for nonlinear operators.
		\item one might wonder why just not state $F(K) \subset Y$ is relatively compact in $Y$ for $K \subset M $ bounded. First of all, this is pretty much the same, but in the applications, this formulation will be more suitable for us. Secondly, in general $M$ need not be a subspace of $X$, let alone a normed linear space;in such cases the notion of boundedness of $K \subset M$ is not defined at all. On the other hand, boundedness of $K \subset X$ can be measured in the metric of $X.$
		\item	The definition is from \textit{Drábek, Milota: Methods of Nonlinear Analysis, Def 5.2.2}
	\end{itemize}
\end{remark}

\begin{theorem}[Brouwer fixed point theorem]
	Let $N \in \N$ and $K \subset \R^N,$ be a nonempty convex closed bounded subset of $\R^{N}$. Assume further that $F: K \to K$ is continuous. Then $F$ has a fixed point in $K$, \textit{i.e.},
	\[
		\exists x_0 \in K: F\qty(x_{0}) = x_0.
	\]
\end{theorem}

\begin{proof}(\textit{From: the lectures})
	No proof.
\end{proof}

\begin{theorem}[Schauder fixed point theorem]
	Let $X$ be a linear normed space and $K \subset X$ be a nonempty convex closed bonded subset of $X$. Assume further that $F: K \to K$ is compact on $K$ into $K$ and $F\qty(K) \subset K.$ Then there is fixed point of $F$ in $K,$ \textit{i.e.},
	\[
		\exists x_0 \in K: F\qty(x_0) = x_0.
	\]
\end{theorem}

\begin{proof}(\textit{From: the lectures})
	No proof.
\end{proof}

\begin{remark}
\begin{itemize}
	\item for Brouwer, $K \subset \R^N$ so since it is closed and bouded, it is automatically compact, and since $F: K \to K$ is continuous, $F$ is compact\footnote{Image of a compact set under continuous mapping is a compact set}. For Schauder, we have to assume this extra.
	\item realize again that in Schauder, the set $K$ does not have to be a subspace of $X$.
	\item proof of Brouwer with N=1 is easy, based on Darboux property.
\end{itemize}
\end{remark}

\subsection{Problem protypes}
\label{sec:prototypes}

In this chapter some nonlinear elliptic equations are discussed. The strategy of solving them (\textit{i.e.}, proving a solution exists) will be always the same - define a suitable operator and show it has a fixed point.

\begin{example}
	Suppose the following problem: 
	\[
		\begin{cases}
			-\laplace u + g(u) = f & \, \text{in} \, \Omega\\
			u = 0 & \, \text{on} \, \partial \Omega,
		\end{cases}
	\]
	where $f \in \qty(\WkpzeroSet[1][2]{\Omega})^{*}, g: \R \to \R$ is continuous and has a controlled grow \textit{s.t.} $\exists C>0, \exists \alpha \in (0,1]$:
	\[
		\forall s \in \R: |g(s)|\leq C\qty(1+|s|^{\alpha}).
	\]
	Even though $g$ might be nonlinear, we assume only (sub)linear growth of it - so this is kind of cheating.

	\begin{theorem}[Existence]
		Let $ \Omega \in C^{1,1}, f \in \qty(\WkpzeroSet[1][2]{\Omega})^{*}, g$ is as above. Then there is a weak solution to the above problem, i.e., it holds:
		\[
			\forall \varphi \in \WkpzeroSet[1][2]{\Omega}: \int_{\Omega}\grad u \vdot \grad \varphi + g(u) \varphi\dd{x} = <f,\varphi>_{(\WkpzeroSet[1][2]{\Omega})^{*}}.
		\]
		If $f \in \LpSet[2]{\Omega}$, then the solution $u \in \WkpSet[2][2]{\Omega}.$
	\end{theorem}
	\begin{proof}(\textit{From: the lectures})
		We define $S: \LpSet[2]{\Omega} \to \LpSet[2]{\Omega}$ such that\footnote{Meaning the image $Sw$ of $w$ is the function $u$ such that the integral equality holds. We could also assume $S: \LpSet[2]{\Omega} \to \WkpSet[1][2]{\Omega},$ but for compactness we need the origin and target spaces to be the same.}
		\[
			Sw = u \Leftrightarrow \forall \varphi \in \WkpzeroSet[1][2]{\Omega}: \int_{\Omega}\grad u \vdot \grad \varphi\dd{x} = <f,\varphi> - \int_{\Omega}g(w) \varphi\dd{x}.
		\]
		Clearly, when $Sw = w,$ \textit{i.e.}, $S$ has a fixed point and then
		\[
			Sw = w \Leftrightarrow \forall \varphi \in \WkpzeroSet[1][2]{\Omega}: \int_{\Omega}\grad w \vdot \grad \varphi \dd{x} = <f, \varphi> - \int_{\Omega}g(w) \varphi\dd{x},
		\]
		and so the solution $w$ to our problem exists. $S$ is defined in an implicit way, but it might be guiding to think of it, written in strong formulation, as
		\[
			Sw = -\qty(\laplace)^{-1}(f - g(w),
		\]
		meaning $Sw$ is the solution $u$ to (modulo boundary conditions)
		\[
			-\laplace u = f- g(w),
		\]
		also meaning $Sw$ is the solution $u$ to the Poisson equation with the RHS "build from $w$."

		First, let us show $S$ is well defined:
		\[
			\abs{\int_{\Omega}g(w) \varphi\dd{x} + <f, \varphi>} \leq \norm{f}_{(\WkpzeroSet[1][2]{\Omega})^{*}} \norm{\varphi}_{\WkpSet[1][2]{\Omega}} + \norm{\varphi}_{\LpSet[2]{\Omega}} \norm{g(w)}_{\LpSet[2]{\Omega}},
		\] and
		\[
			\int_{\Omega}|g(w)|^{2}\dd{x} \leq \int_{\Omega}C^{2}\qty(1+|w|^{\alpha})^{2}\dd{x} \leq \int_{\Omega}2C^{2}\qty(1+|w|^{2 \alpha})\dd{x} \leq \int_{\Omega}2C^{2}\qty(1+|w|^2)\dd{x} < \infty,
		\]
		where we used the Young inequality\footnote{In the form $\qty(a+b)^{2} \leq 2\qty(a^{2}+b^{2}).$}, $\alpha \leq 1$ and the fact $S$ is defined on $\LpSet[2]{\Omega},$ meaning $w \in \LpSet[2]{\Omega}.$ With that information, we see the last integral is finite and so truly $S: \LpSet[2]{\Omega} \to \LpSet[2]{\Omega}$ is well defined.

		Also, the estimate $\abs{g(w)} \leq C(1+\abs{w}^{\alpha}), \alpha \in (0,1]$ together with the fact $C(1+w^{\alpha}) \in \LpSet[2]{\Omega}$ for $w \in \LpSet[2]{\Omega}$ means the Nemytskii operator
		\[
			w \mapsto g(w) \in \mathcal{C}\qty(\LpSet[2]{\Omega}; \LpSet[2]{\Omega}),
		\]
		is continuous from $\LpSet[2]{\Omega}$ to $\LpSet[2]{\Omega}.$

		Let us now show that $S$ is continuous as a whole. Define $I_g \in \qty(\WkpzeroSet[1][2]{\Omega})^{*}$ as
		\[
			I_g\qty(\varphi) = \int_{\Omega}g(w) \varphi\dd{x} - <f, \varphi>.
		\]
		We are interested in the continuity of the (nonlinear!) mapping $g \mapsto I_g$ as a mapping from $\LpSet[2]{\Omega}$ to $\qty(\WkpzeroSet[1][2]{\Omega})^{*}.$ Let $\qty{g_k} \subset \LpSet[2]{\Omega}$ be a sequence \textit{s.t.} $g_k \to g$ in  $\LpSet[2]{\Omega}.$ Then
		\begin{align*}
			\norm{I_{g_k}- I_g}_{\qty(\WkpzeroSet[1][2]{\Omega})^{*}} &= \sup_{\varphi \in \text{U}_{\WkpzeroSet[1][2]{\Omega}}(0,1)}\abs{\qty(I_{g_k}-I_g)\qty(\varphi)} = \sup_{\text{U}_{\WkpzeroSet[1][2]{\Omega}}(0,1)}\abs{\qty(\int_{\Omega}g_k \varphi\dd{x} - <f,\varphi> - \int_{\Omega}g \varphi\dd{x}+<f,\varphi>)} \leq \\
										  &\sup_{\text{U}_{\WkpzeroSet[1][2]{\Omega}}(0,1)}\int_{\Omega}\abs{g - g_k}|\varphi|\dd{x} \leq \norm{g-g_k}_{\LpSet[2]{\Omega}} \to 0,
		\end{align*}
		so we have shown $g \to g_k$ in $\LpSet[2]{\Omega}$ $\Rightarrow I_g \to I_{g_k}$ in $\qty(\WkpzeroSet[1][2]{\Omega})^{*},$ and so $\qty(g \mapsto I_g) \in \mathcal{C}\qty(\LpSet[2]{\Omega}; \qty(\WkpzeroSet[1][2]{\Omega})^{*}).$ Finally, define the operator $L: \qty(\WkpzeroSet[1][2]{\Omega})^{*} \to \WkpzeroSet[1][2]{\Omega},$ as
		\[
			Lh = u \Leftrightarrow \forall \varphi \in \WkpzeroSet[1][2]{\Omega}: \int_{\Omega}\grad u \vdot \grad \varphi \dd{x} = <h, \varphi>,
		\]
		or in other words, $L$ maps $h \mapsto u$, where $u$ solves
		\[
			\begin{cases}
				-\laplace u = h, & \, \text{in} \, \Omega, \\
				u = 0, & \, \text{on} \, \partial \Omega
			\end{cases}.
		\]
		or in other other, informal, words
		\[
			Lh = - \qty(\laplace)^{-1}h.
		\]
		Realize that this approach been studiud in the chapter about Fredholm theory extensively. In this simple case however, recall the apriori estimates that the solution satisfies (that is guaranteed from the properties of the Poisson equation)
		\[
			\norm{u}_{\WkpzeroSet[1][2]{\Omega}}\leq C \norm{h}_{\qty(\WkpzeroSet[1][2]{\Omega})^{*}},
		\]
		which together with the linearity of $L$ checks $h \mapsto u$ is continuous. Putting it all together:
		\begin{itemize}
			\item $\qty(w \mapsto g(w)) \in \mathcal{C}\qty(\LpSet[2]{\Omega};\LpSet[2]{\Omega})$,
			\item $\qty(g(w) \mapsto f - g(w))  \in \mathcal{C}\qty(\LpSet[2]{\Omega}; \qty(\WkpzeroSet[1][2]{\Omega})^{*})$, \footnote{In the sense we interpret the RHS as a point in the dual space. Precisely, we should write something like $\qty(g(w) \mapsto \qty(\varphi \in \WkpzeroSet[1][2]{\Omega} \mapsto <f,\varphi>-\int_{\Omega}g(w) \varphi\dd{x})) \in \mathcal{C}\qty(\LpSet[2]{\Omega}; \qty(\WkpzeroSet[1][2]{\Omega})^{*})$.} 
			\item $\qty(f - g(w) \mapsto u) \in \mathcal{C}\qty(\qty(\WkpzeroSet[1][2]{\Omega})^{*}; \WkpzeroSet[1][2]{\Omega})$, 
		\end{itemize}
		In total, the composition is continuous and yields $S$.

		Next, we would like to show $S$ is a compact (nonlinear) operator. We start with showing $S$ maps bounded sets in $\LpSet[2]{\Omega}$ to bounded sets in $\WkpzeroSet[1][2]{\Omega} \hookrightarrow \hookrightarrow \LpSet[2]{\Omega}$; for that we need apriori estimates: test the weak formulation with $u$: (we are doing this a bit more careful that might seem needed, but come on, it is the first example...)
\begin{align*}
	\underbrace{\norm{\grad u}_{\LpSet[2]{\Omega}}^{2}}_{C_p^{2} \norm{u}_{\WkpzeroSet[1][2]{\Omega}}^{2} \leq} &\leq\norm{f}_{\qty(\WkpzeroSet[1][2]{\Omega})^{*}}\norm{u}_{\WkpzeroSet[1][2]{\Omega}} + \norm{g(w)}_{\LpSet[2]{\Omega}} \norm{u}_{\LpSet[2]{\Omega}}\leq \\
														    &\leq \frac{\norm{f}_{\qty(\WkpzeroSet[1][2]{\Omega})^{*}}}{\sqrt{2 \varepsilon_1}}\sqrt{2 \varepsilon_1}\norm{u}_{\WkpzeroSet[1][2]{\Omega}} + \frac{\norm{g(w)}_{\LpSet[2]{\Omega}}}{\sqrt{2 \varepsilon_2}} \sqrt{2 \varepsilon_2}\norm{u}_{\WkpzeroSet[1][2]{\Omega}}\leq   \\
														    &\leq \qty(\varepsilon_1 + \varepsilon_2)\norm{u}_{\WkpzeroSet[1][2]{\Omega}}^{2} + \frac{1}{4 \varepsilon_1} \norm{f}_{\qty(\WkpzeroSet[1][2]{\Omega})^{*}}^{2} + \frac{1}{4 \varepsilon_2} \norm{g(w)}_{\LpSet[2]{\Omega}}^{2} \leq \\
														    &\leq \qty(\varepsilon_1 + \varepsilon_2)\norm{u}_{\WkpzeroSet[1][2]{\Omega}}^{2}+ \frac{1}{4\min\qty(\varepsilon_1, \varepsilon_2)}\qty(\norm{f}_{\qty(\WkpzeroSet[1][2]{\Omega})^{*}}^{2}+ \norm{g(w)}_{\LpSet[2]{\Omega}}^{2}),
\end{align*}
and since $\norm{g(w)}_{\LpSet[2]{\Omega}}^{2} \leq \norm{C\qty(1+\abs{w}^{\alpha})}_{\LpSet[2]{\Omega}}^{2} \leq 4C^{2}\qty(\qty(\lambda\qty(\Omega))^{2}+\norm{w}_{\LpSet[2]{\Omega}}^{2}) $ and Poincare inequality, we can write
\[
	\norm{u}_{\WkpzeroSet[1][2]{\Omega}}^{2} \leq \frac{1}{4\qty(C_p^{2} - \qty(\varepsilon_1 + \varepsilon_2))\min\qty(\varepsilon_1, \varepsilon_2)}\qty(\norm{f}_{\qty(\WkpzeroSet[1][2]{\Omega})^{*}}^{2}+ 4\qty(\qty(\lambda\qty(\Omega))^{2}+\norm{w}_{\LpSet[2]{\Omega}}^{2})),
\]
which, for $\varepsilon_1, \varepsilon_2$ sufficiently small, just an estimate of the type
\[
	\norm{u}_{\WkpzeroSet[1][2]{\Omega}}^{2} \leq C\qty(1+\norm{f}_{\qty(\WkpzeroSet[1][2]{\Omega})^{*}}^{2} + \norm{w}_{\LpSet[2]{\Omega}}^{2}).
\]
As we can see, if $w \in \LpSet[2]{\Omega}$ is bounded, also $u \in \WkpzeroSet[1][2]{\Omega}$ is bounded, so $S$ maps bounded sets from $\LpSet[2]{\Omega}$ to bounded sets in $\WkpzeroSet[1][2]{\Omega}.$ Since moreover $\WkpzeroSet[1][2]{\Omega} \hookrightarrow \hookrightarrow \LpSet[2]{\Omega}, $ this means $S$ is compact from $\LpSet[2]{\Omega}$ to $\LpSet[2]{\Omega}$.

To conclude the proof, we need that there exists $K \subset \LpSet[2]{\Omega}$ closed convex bounded nonempty \textit{s.t.} $S(K) \subset K,$ so we can use Schauder. That clearly will be the case when we show $S\qty(\text{B}_{\LpSet[2]{\Omega}}(0,R)) \subset \text{B}_{\LpSet[2]{\Omega}}(0,R)$ for some $R>0.$ But this is simple - the estimate

\[
	\norm{u}_{\LpSet[2]{\Omega}}^{2} \leq \norm{u}_{\WkpzeroSet[1][2]{\Omega}}^{2} \leq C\qty(\underbrace{1+\norm{f}_{\qty(\WkpzeroSet[1][2]{\Omega})^{*}}^{2}}_{\coloneq C_1} + \norm{w}_{\LpSet[2]{\Omega}}^{2}),
\]
tells us if the $R$ were to exist, it must hold
\[
	R^{2} \leq C C_1 + C R^{2} \Leftrightarrow R^{2}\qty(1-C)\leq C C_1.
\]
If $1-C >0,$ then just take $R \leq \sqrt{\frac{C C_1}{1-C}},$ if $1-C <0,$ any $R>0$ will do.

And so we see such an $R$ exists in all cases  $\Rightarrow$ the image of a ball is in a ball for some $R \Rightarrow$ $S$ is compact on $\text{B}_{\LpSet[2]{\Omega}}(0,R)$ into $\text{B}_{\LpSet[2]{\Omega}}(0,R) \Rightarrow $ it has a fixed point by Schauder $\Leftrightarrow$ the solution exists.

For the regularity part of the assertion, realize that $u_0$ solves 
\[
	\begin{cases}
		-\laplace u_0 = f-g(u_0), & \, \text{in} \, \Omega \\
		u_0 = 0, & \, \text{on} \, \partial \Omega. 
	\end{cases},
\]
and if $f \in \LpSet[2]{\Omega},$ then $f-g\qty(u_0) \in \LpSet[2]{\Omega}$ and so from the regularity theory for elliptic equations we get
\[
	u \in \WkpSet[2][2]{\Omega}.
\]
\end{proof}

\begin{theorem}[Uniqueness]
	Let $u_1, u_2 \in \WkpzeroSet[1][2]{\Omega}$ be weak solutions to the above problem. Let $f \in \qty(\WkpzeroSet[1][2]{\Omega})^{*}$ and let $g$ be continuous. Let either extra 
	\begin{enumerate}
		\item $g$ is nondecreasing, or
		\item $g \in C^1\qty(\R), \norm{g'}_{\LinfSet{\R}}$ small.
	\end{enumerate}
	Then $u_1 = u_2$.
\end{theorem}
\begin{proof}(\textit{From: the lectures})
	In the linear case, $u_1 - u_2$ would be a solution to the problem with zero data. In the nonlinear case, we have now certainity $u_1 - u_2$ solves anything, so we will have to be more careful.
	
Subtract the equations for $u_1, u_2$ and test the weak formulation with $u_1 - u_2.$:
	\[
		\int_{\Omega}|\grad(u_1 - u_2)|^{2} + (g(u_1)-g(u_2))\qty(u_1-u_2)\dd{x} = 0.
	\]
	In the first case, the second term is nonnegative, so for the whole thing to be zero it has to hold 
	\[
		0=\norm{\grad(u_1 - u_2)}_{\LpSet[2]{\Omega}}^{2} \geq C_p \norm{u_1 - u_2}_{\WkpSet[1][2]{\Omega}}^{2} \Rightarrow u_1 - u_2 = 0 \in \WkpzeroSet[1][2]{\Omega}.
	\]

	In the second case, the second does not have a sign, but we can estimate it
	\[
		\abs{\int_{\Omega}(g(u_1)-g(u_2)(u_1-u_2))\dd{x}} \leq \int_{\Omega}\norm{g'}_{\LinfSet{\R}}\abs{u_1 - u_2|^{2}}\dd{x} \leq \norm{g'}_{\LinfSet{\R}} C_p \norm{\grad(u_1 - u_2)}_{\LpSet[2]{\Omega}}^{2}, 
	\]
	meaning
	\[
		\norm{\grad\qty(u_1 - u_2)}_{\LpSet[2]{\Omega}}^{2} \pm \norm{g}_{\LinfSet{\R}}C_p \norm{\grad\qty(u_1 - u_2)}_{\LpSet[2]{\Omega}}^{2}  = (1 \pm \norm{g}_{\LinfSet{\R}}C_p)\norm{\grad\qty(u_1 - u_2)}_{\LpSet[2]{\Omega}}^{2} \leq 0.
	\]
	If the term is positive we again obtain $u_1 = u_2$ in $\WkpzeroSet[1][2]{\Omega}$ from the Poincare inequality, if it is negative, we require
	\[
		\norm{g'}_{\LinfSet{\R}}C_p < 1,
	\]
	and then we have the same result.
\end{proof}
\end{example}

\begin{example}
	Suppose the following problem
	\[
		\begin{cases}
			- \laplace u + b\qty(\grad u) = f, & \, \text{in} \, \Omega \\
			u = 0, & \, \text{on} \, \partial \Omega.
		\end{cases}
	\]
	where $f \in (\WkpzeroSet[1][2]{\Omega})^{*},$ and the function $b: \R^{d} \to \R$ is continuous and (essentialy) bounded. The weak formulation is: 

	\[
	\, \text{find} \,	u \in \WkpzeroSet[1][2]{\Omega} \, \text{\textit{s.t.}} \,: \forall \varphi \in \WkpzeroSet[1][2]{\Omega}: \int_{\Omega}\grad u \vdot \grad \varphi + b\qty(\grad u) \varphi\dd{x} = <f,\varphi>_{\qty(\WkpzeroSet[1][2]{\Omega})^{*}},
	\]

	\begin{theorem}
		Let $f \in \qty(\WkpzeroSet[1][2]{\Omega})^{*}, \Omega \in C^{0,1}, b: \R^d \to \R$ continuous and essentialy bounded. Then there is a weak solution to the above problem.
	\end{theorem}
	\begin{proof}(\textit{From: the lectures})
		As in the previous: define the operator $S: \WkpzeroSet[1][2]{\Omega} \to \WkpzeroSet[1][2]{\Omega},$  such as
		\[
			Sw = u \Leftrightarrow \forall \varphi \in \WkpzeroSet[1][2]{\Omega}: \int_{\Omega}\grad u \vdot \grad \varphi \dd{x} = <f, \varphi> - \int_{\Omega}b\qty(\grad w)\varphi \dd{x},
		\]
		or equivalently
		\[
			Sw = u \Leftrightarrow u \, \text{is the solution to } \,
			\begin{cases}
				- \laplace u = f - b\qty(\grad w) & \, \text{in} \, \Omega \\
				u = 0 & \, \text{on} \, \partial \Omega.
			\end{cases}, 
		\]
		or informally
		\[
			Sw = -\qty(\laplace)^{-1}\qty(f - b(\grad w)),
		\]
		modulo boundary conditions. Once again, if we are able to show $S$ has a fixed point, this information is equivalent to the considered problem having a solution.

		Let us first show $S$ is well defined. Holder yields
		\[
			\norm{\grad u}_{\LpSet[2]{\Omega}}\norm{\grad \varphi}_{\LpSet[2]{\Omega}} \leq \norm{f}_{(\WkpzeroSet[1][2]{\Omega})^{*}} \norm{\varphi}_{\WkpzeroSet[1][2]{\Omega}}+ \norm{b}_{\LinfSet{\R^{d}}} \norm{\varphi}_{\LpSet[1]{\Omega}},
		\]
		use Poincare inequality and $\WkpzeroSet[1][2]{\Omega} \hookrightarrow \LpSet[1]{\Omega},$ to obtain
		\[
			\norm{u}_{\WkpzeroSet[1][2]{\Omega}} \leq \underbrace{C\qty(\norm{f}_{\qty(\WkpzeroSet[1][2]{\Omega})^{*}}+ \norm{b}_{\LinfSet{\R^{d}}})}_{\coloneq R},
		\]
		and so we see that the target space really is $\WkpzeroSet[1][2]{\Omega}$. Also, the above estimate is extra nice, as it does not depend on $w$ whatsoever. If we denote the RHS by $R>0$, we immediately see $\norm{u}_{\WkpzeroSet[1][2]{\Omega}} = \norm{Sw}_{\WkpzeroSet[1][2]{\Omega}}\leq R$, meaning that for sure $S\qty(\text{B}_{\WkpzeroSet[1][2]{\Omega}}(0,R)) \subset \text{B}_{\WkpzeroSet[1][2]{\Omega}}(0,R).$
		Let us show other properties needed for compactness.


	First, $\grad: \WkpzeroSet[1][2]{\Omega} \to \LpSet[2]{\Omega;\R^{d}}, w \mapsto \grad w$ is linear and bounded,
	\[
		\norm{\grad w}_{\LpSet[2]{\Omega}} \leq \norm{w}_{\WkpzeroSet[1][2]{\Omega}},
	\]
	and so continuous, the map $\vb{y} \mapsto b\qty(\vb{y})$ satisfies the growth
		\[
			\abs{b\qty(\vb{y})} \leq \underbrace{\norm{b}_{\LinfSet{\R^{d}}}}_{" = g(x)"} + 0 \vdot \sum_{i=1}^d |y_i|^{2/2},
		\]
		and since $\lambda\qty(\Omega) < \infty,$ we have (in particular) $\norm{b}_{\LinfSet{\R^{d}}} \in \LpSet[2]{\Omega}.$ Putting this together means $N_b: \grad w \mapsto b\qty(\grad w)$ is continuous from\footnote{We should be more precise and write it is continuous from $\LpSet[2]{\Omega; \R^{d}}$ to $\LpSet[2]{\Omega;\R},$ as $b: \R^{d} \to \R.$} $\LpSet[2]{\Omega; \R^{d}}$ to $\LpSet[2]{\Omega;\R}$,
		\[
			\qty(\grad w \mapsto b\qty(\grad w)) \in \mathcal{C}\qty(\LpSet[2]{\Omega; \R^{d}}; \LpSet[2]{\Omega; \R})
		\]
		by Nemytskii. Next, the mapping
		\[
			b\qty(\grad w) \mapsto b\qty(\grad w) + f
		\]
		from $\LpSet[2]{\Omega}$ to $\qty(\WkpzeroSet[1][2]{\Omega})^{*}$ in the sense
		\[
			b\qty(\grad w) \mapsto \qty(\WkpzeroSet[1][2]{\Omega} \ni \varphi \mapsto <f, \varphi>_{\qty(\WkpzeroSet[1][2]{\Omega})^{*}}+ \int_{\Omega}b\qty(\grad w)\varphi\dd{x}),
		\]
		is also continous: let $\qty{b_k}\subset \LpSet[2]{\Omega} \, \text{\textit{s.t.}} \, b_k \to b,$ in $\LpSet[2]{\Omega},$ then
		\begin{align*}
			\sup_{\text{U}_{\WkpzeroSet[1][2]{\Omega}}(0,1)}\abs{<f, \varphi> -\int_{\Omega}b_k\qty(\grad )\varphi \dd{x} - <f, \varphi> + \int_{\Omega}b\qty(\grad w)\varphi\dd{x}} &\leq \sup_{\text{U}_{\WkpzeroSet[1][2]{\Omega}}(0,1)}\int_{\Omega}\abs{b\qty(\grad w)- b_k\qty(\grad w)}\abs{\varphi}\dd{x} \leq \\
																								 &\leq \norm{b_k - b}_{\LpSet[2]{\Omega}} \to 0,
		\end{align*}
		and so the mapping $b\qty(\grad w) \mapsto b\qty(\grad w) + f$ is continuous from $\LpSet[2]{\Omega}$ to $\LpSet[2]{\Omega}.$ Finally, the mapping $L: \qty(\WkpzeroSet[1][2]{\Omega})^{*} \to \WkpzeroSet[1][2]{\Omega},$
		\[
			Lh = u \Leftrightarrow \forall \varphi \in \WkpzeroSet[1][2]{\Omega}: \int_{\Omega}\grad u \vdot \grad \varphi\dd{x} = <h, \varphi>,
		\]
		\textit{i.e.},
		\[
			Lh = u \Leftrightarrow u \, \text{solves} \, \begin{cases}
				- \laplace u = h, & \, \text{in} \, \Omega, \\
				u = 0, & \, \text{on} \, \partial \Omega,
			\end{cases}
		\]
		is also continuous between the spaces - it is just again the solution operator to the Poisson equation with a RHS $h \in \qty(\WkpzeroSet[1][2]{\Omega})^{*}.$ Altogether
		\begin{itemize}
			\item $\qty(w \mapsto \grad w) \in \mathcal{C}\qty(\WkpzeroSet[1][2]{\Omega}; \LpSet[2]{\Omega;\R^{d}}),$
			\item $\qty(\grad w \mapsto b(\grad w)) \in \mathcal{C}\qty(\LpSet[2]{\Omega;\R^{d}}; \LpSet[2]{\Omega;\R})$,
			\item $\qty(b\qty(\grad w) \mapsto f - b\qty(\grad w)) \in \mathcal{C}\qty(\LpSet[2]{\Omega}; \qty(\WkpzeroSet[1][2]{\Omega})^{*})$,
			\item $\qty(f - b\qty(\grad w) \mapsto u) \in \mathcal{C}\qty(\qty(\WkpzeroSet[1][2]{\Omega})^{*}; \WkpzeroSet[1][2]{\Omega})$,
		\end{itemize}
		and so $S$ as a composition of those mappings is continuous from $\WkpzeroSet[1][2]{\Omega}$ to $\WkpzeroSet[1][2]{\Omega}.$ It remains to show $S$ is compact: we already have continuity, consider now $\{w_k\}_{k \in \N} \subset \WkpzeroSet[1][2]{\Omega}$ bounded. Then, by the compact embedding $\WkpzeroSet[1][2]{\Omega} \hookrightarrow \hookrightarrow \LpSet[1]{\Omega},$ there $\exists \{u_k\} \subset \qty{w_k} \subset \WkpzeroSet[1][2]{\Omega}$ (also bounded) \textit{s.t.} $u_k \to u$ in $\LpSet[1]{\Omega}$. To conclude, use the following trick (as when showing uniqueness to the previous problem): subtract equation for $u_k$ from equation for $u_l$ and test with $u_l - u_k$
		\begin{align*}
			\int_{\Omega}\grad u_l \vdot \grad\qty(u_l - u_k)\dd{x} &+ \int_{\Omega}b\qty(\grad u_l)\qty(u_l - u_k)\dd{x} - <f, u_l -u_k> \\
										&-\qty(\int_{\Omega}\grad u_k \vdot \grad\qty(u_l - u_k)\dd{x}+ \int_{\Omega}b\qty(\grad u_k)\qty(u_l - u_k)\dd{x} - <f, u_l - u_k),
		\end{align*}
		meaning
		\[
			C \norm{u_l - u_k}_{\WkpzeroSet[1][2]{\Omega}}^{2} \leq	\norm{\grad\qty(u_l - u_k)}_{\LpSet[2]{\Omega}}^{2} \leq \int_{\Omega}|b(\grad u_l) - b\qty(\grad u_k)| |u_l - u_k|\dd{x} \leq 2 \norm{b}_{\LinfSet{\Omega}}\norm{u_l-u_k}_{\LpSet[1]{\Omega}},
		\]
		so 
		\[
			\norm{u_l-u_k}_{\WkpzeroSet[1][2]{\Omega}}^{2} \leq C\norm{u_l-u_k}_{\LpSet[q]{\Omega}}.
		\]
	And we are finished - the sequence $\qty{u_n}$ converges in $\LpSet[1]{\Omega},$ so is Cauchy in $\LpSet[1]{\Omega},$ and the above inequality proves that also $\qty{u_n}$ is Cauchy in a complete space $\WkpzeroSet[1][2]{\Omega},$ meaning $\qty{u_n}$ converges in $\WkpzeroSet[1][2]{\Omega}.$ All in all, $\qty{u_n} \subset \WkpzeroSet[1][2]{\Omega}$ is a convergent subsequence of a bounded sequence $\qty{w_k} \subset \WkpzeroSet[1][2]{\Omega},$ meaning $S$ is compact on $\text{B}_{\WkpzeroSet[1][2]{\Omega}}(0,R)$ into $\text{B}_{\WkpzeroSet[1][2]{\Omega}}(0,R)$ and so must possess a fixed point by the Schauder fixed point theorem.
	\end{proof}
\end{example}


