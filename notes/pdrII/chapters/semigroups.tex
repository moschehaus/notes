% !TEX root = ../main.tex

\section{Semigroup theory}
\label{sec:semigroups}

The lands of nonlinear equations are hostile, let us retreat back to our welcoming safeplace - \textit{linear equations}. We will study an elegant framework; consider the following problem

\begin{align*}
	u' &= Au, A \, \text{is a linear operator} \,\\
	u(0) &= u_0,
\end{align*}
where $u:[0, \infty) \to \R.$ We know that for example if $Au = au, a \in \R$ then
\[
	u(t) = u_0 e^{at}.
\]
If $\vb{u}: [0,\infty) \to \R^{d}, A\vb{u} = \tensorq{A}\vb{u}, \tensorq{A} \in \R^{d \times d}$, then
\[
	\vb{u}(t) = \exp\qty(t\tensorq{A})\vb{u}_0, \exp\qty(t \tensorq{A}) = \sum_{k=0}^{\infty}\frac{1}{k!}\tensorq{A}^k t^k.
\]
It is not surprising that this can be extended to functions $u:[0, \infty) \to X,$ taking values in an arbitrary Banach space X. In particular, for $A \in \mathcal{L}\qty(X), $ it is sensible to set
\[
	u(t) = \exp\qty(tA)u_0,
\]
with
\[
	\exp\qty(tA) = \sum_{k=0}^{\infty}\frac{\qty(tA)^k}{k!},
\]

This works well for \textit{bounded} operators. But in the context of PDE's, most operators are differential and these are generally unbounded, take for example 
\[
	X = \LpSet[2]{\Omega}, Au = \laplace u.
\]

Our divine inspiration tells us we can \textit{guess} the solution should be
\[
	u(t) = \exp\qty(\laplace t)u_0,
\]
but what is
\[
	\exp(\laplace t)?
\]

To explore this question, we will have to develop some theory first.

\subsection{(Unbounded) linear operators and ($c_0-$)semigroups}
\label{sec:linops_semigroups}

\begin{definition}[Linear operator and its domain]
Let $X$ be a Banach space over $\mathbb{K}$. A linear operator on $X$ is a couple $(A, \mathcal{D}\qty(A))$, where $\mathcal{D}\qty(A)$ is a subspace of $X$ and $A: \mathcal{D}\qty(A) \to X$ is linear. We call $\mathcal{D}\qty(A)$ the domain of $A.$
\end{definition}

\begin{remark}
	Evidently, it is important that $\mathcal{D}(A)$ is a subspace of $X$ (and not \textit{e.g.} a subset); when evaluating $A\qty(\alpha x + \beta y), \alpha, \beta \in \mathbb{K}, x, y \in \mathcal{D}(A),$ we need to be sure $\alpha x + \beta y \in \mathcal{D}\qty(A)$ also.
\end{remark}

\begin{definition}[Semigroup, $c_0$ semigroup]
	A family of linear bounded operators
	\[
		\qty{S(t)}_{t \geq 0} \subset \mathcal{L}\qty(X)
	\]
	is called a semigroup, provided 
	\begin{enumerate}
		\item $S\qty(0) = \, \text{id} \,$
		\item $\forall s,t \geq 0: S(t) S(s) = S(t+s).$
	\end{enumerate}
	If moreover
	\[
		\forall x \in X: \lim_{t \to 0^+}S(t)x = x,
	\]
	we call $\qty{S(t)}$ a strongly continuous semigroup, written as $c_0-$ semigroup.
\end{definition}
\begin{remark}
	\begin{itemize}
		\item even though we have talked about unbounded operators, we stress that always
			\[
				\qty{S(t)}_{t \geq 0} \subset \mathcal{L}(X),
			\]
		\item 	$\qty{S(t)}_{t\in \R}$ with the two conditions, an inverse element 
			\[
				\qty(S(t))^{-1} = S\qty(-t),
			\]
			is an Abelian group $\qty(\{S(t)\}_{t\in \R}, \circ)$ with respect to composition $\circ.$
	\end{itemize}

\end{remark}

\begin{remark}[Notation-wise]
	\begin{itemize}
		\item \textbf{in the following, $X$ is always a Banach space,}
		\item the exponential function defined on numbers is written as $e^t,$ whereas the operator exponential will be written as $\exp\qty(tA).$
	\end{itemize}
\end{remark}

\begin{lemma}[Properties of strongly continuous semigroups]
	Let $\qty{S(t)}_{t\geq0}$ be a $c_0$-semigroup in $X$. Then 
	\begin{enumerate}
		\item $\exists M \geq 1, \omega \in \R \, \text{\textit{s.t.}} \, \forall S \in \qty{S(t)}_{t \geq 0}, \forall t \geq 0: \norm{S(t)}_{\mathcal{L}\qty(X)} \leq M e^{\omega t},$
		\item $\forall x \in X$ the mapping
			\[
				t \mapsto S(t)x 
			\]
			is continuous from $[0, \infty)$ to $X$.
	\end{enumerate}
\end{lemma}
\begin{proof}(\textit{From: the lectures})
	Ad 1.: No proof has been given.

	Ad 2.: Fix $t > 0, x \in X$ and compute
	\begin{align*}
		\lim_{h\to 0_+} \norm{S\qty(t+h)x - S(t)x}_X &= \lim_{h \to 0_+} \norm{S(t)\qty(S(h)x - x)}_X \leq \lim_{h\to 0_+}\norm{S(t)}_{\mathcal{L}\qty(X)}\norm{S(h)x - x}_{X} = 0, \\
		\lim_{h\to 0_+}\norm{S(t-h)x - S(t)x}_{X} &= \lim_{h \to 0_+}\norm{S(t-h)(x-S(h)x)}_X \leq \lim_{h \to 0^+} \underbrace{\norm{S(t-h)}_{\mathcal{L}(X)}}_{\leq Me^{\omega(t-h)}}\norm{x-S(h)x}_X = 0,
	\end{align*}
	so
	\[
		\lim_{h \to 0}\norm{S(t+h)x - S(t)x}_X = 0.
	\]
\end{proof}


\begin{definition}[Infinitesimal generator]
	A linear operator $\qty(A, \mathcal{D}\qty(A))$ is called an infinitesimal generator of the semigroup $\qty{S(t)}_{t\geq 0},$ provided
	\[
		\forall x \in \mathcal{D}\qty(A): Ax = \lim_{h \to 0^+} \frac{S(h)x - x}{h},
	\]
	where 
	\[
		\mathcal{D}\qty(A) = \qty{ x \in X | \lim_{h\to0^+} \frac{S(h)x-x}{h} \, \text{exists in} \, X},
	\]
\end{definition}

\begin{remark}
	Realize that the limit from the definition actually is
	\[
		Ax = \lim_{h \to 0^+} \frac{S(h)x - x}{h} = \lim_{h \to 0^+} \frac{S\qty(h+ 0)x - S(0)x}{h} = \lim_{h \to 0^+}\frac{S\qty(h+0) - S(0)}{h} x= \dv{t}S(0) x,
	\]
	\textit{i.e.},
	\[
		A = \dv{t}S(0).
	\]
\end{remark}

\begin{theorem}
	Let $\qty(A, \mathcal{D}\qty(A))$ be the infinitesimal generator of a $c_0$-semigroup $\qty{S(t)}_{t \geq 0}$ in $X$. Then 
	\begin{enumerate}
		\item  If $x \in \mathcal{D}\qty(A)$ then $\forall t \geq 0: S(t)x \in \mathcal{D}\qty(A)$ and moreover it holds
			\[
				A S(t) x = S(t) A x = \dv{t}\qty(S(t)x).
			\]
		\item For $x \in X, t\geq 0$ set
			\[
				x_t = \int_0^t S(s)x \dd{s}.
			\]
			Then $x_t \in \mathcal{D}(A)$ and moreover it holds
			\[
				A\qty(x_{t}) = S(t)x - x.
			\]
	\end{enumerate}
\end{theorem}

\begin{proof}(\textit{From: the lectures})
    Fix $x \in \mathcal{D}\qty(A), t \geq 0.$ Calculate 
    \[
	    \lim_{h \to 0_+}\frac{S(h)S(t)x - S(t)x}{h} = \footnote{$S(h)S(t) = S(h+t) = S(t+h) = S(t) S(h)$}\lim_{h \to 0_+} S(t) \frac{S(h)x -x}{h} = S(t)Ax.
    \]
    This means $S(t)x \in \mathcal{D}\qty(A)$ and $AS(t)x = S(t) Ax,$ moreover, if $t>0$:
    \[
	    \lim_{h\to 0_+}\frac{S(t-h) x - S(t)x}{-h} - S(t) Ax = \lim_{h\to 0_+}S(t-h) \qty(\frac{x-S(h)x}{-h}-S(h)Ax),
    \]
    estimate,
    \begin{align*}
	    \norm{\lim_{h\to 0_+}\frac{S(t-h) x - S(t)x}{-h} - S(t) Ax} &= \norm{\lim_{h\to 0^+}S(t-h) \qty(\frac{x-S(h)x}{-h}-S(h)Ax)} \leq \\
									&=\lim_{h\to 0^+}\norm{S(t-h)}_{\mathcal{L}\qty(X)} \norm{\qty(\frac{x-S(h)x}{-h}-S(h)Ax)}_X\leq \\
									&\leq\lim_{h \to 0^+} M e^{\omega\qty(t-h)}\norm{\frac{x-S(h)x}{-h}-S(h)Ax} = \lim_{h \to 0^+} Me^{\omega\qty(t-h)}\qty(Ax - Ax) = 0,
    \end{align*}
    as $S(t)$ is continous and $S(0) = \, \text{id} \,.$ We have thus shown the derivative exists, is equal to
    \[
	    \dv{t} \qty(S(t)x) = AS(t)x,
    \]
    and since the RHS is continuos \footnote{We can easily check
	    \[
		    \lim_{h \to 0^+}\norm{AS\qty(t+h)x - AS(t)x}_X \leq \lim_{h \to 0^+} \norm{S(t)}_{\mathcal{L}(X)}\norm{AS(h)x - Ax}_X = 0,
	    \]
	    and very similiarly for the other limit.
    }we have also showed $t \mapsto S(t)x$ is $C^1\qty([0, \infty)).$ This justifies our computation above, that concludes
    \[
	    \dv{t}\qty(S(t)x) = S(t)S'(0)x = S(t)Ax.
    \]

    To show the second part, compute
    \[
	    \lim_{h \to 0_+}\frac{1}{h}\qty(S(h)x_t - x_t) = \lim_{h\to 0_+} \frac{1}{h}\qty(S(h)\int_0^{t}S(s)x \dd{s} - \int_0^t S(s)x \dd{s}),
    \]
    and realize that
    \[
	    S(h) \int_0^t S(s)x \dd{s} =\int_0^t S(h) S(s) x\dd{s} = \int_0^t S(h+s) x \dd{s} = \int_h^{t+h}S(s) x \dd{s},
    \]
    which follows from the properties of the Bochner integral with respect to composition with linear bounded operators. Hence the previous computation continues as follows
    \[
	    = \lim_{h \to 0_+}\frac{1}{h}\qty(\int_t^{t+h}S(s)x\dd{s}-\int_0^h S(s)x \dd{s}) = S(t)x - S(0)x = S(t)x -x, 
    \]
which means $ x_t \in \mathcal{D}\qty(A)$ and the formula is proven. We used some kind of Lebesgue differentiation theorem, that also works on Banach spaces (with our strognly continuous semigroups).
\end{proof}

\begin{definition}[Closed operator]
	We say that a linear operator $\qty(A, \mathcal{D}(A))$ is closed if $\forall \qty{u_n} \subset \mathcal{D}\qty(A): u_n \to u \wedge A u_n \to v,$ for some $u,v \in X,$ then it most hold
	\[
		u \in \mathcal{D}\qty(A) \wedge Au = v.
	\]
	This also means that
	\[
		\, \text{graph} \, A = \qty{(x,Ax) | x \in \mathcal{D}\qty(A)} \subset X \times X
	\]
	is closed in $\qty(X \times X,\norm{\vdot}_1).$
\end{definition}

\begin{example}
	Let $ \Omega \in C^{1,1}, X = \LpSet[2]{\Omega}, \mathcal{D}\qty(A) = \WkpSet[2][2]{\Omega} \cap \WkpzeroSet[1][2]{\Omega}, Au = \laplace u.$ Then $(A, \mathcal{D}(A)$ is closed. Really, take $\qty{u_n} \subset \LpSet[2]{\Omega}: u_n \to u$ in $\LpSet[2]{\Omega}$ for some $u \in \LpSet[2]{\Omega}.$ Suppose $Au_n \to v$ in $\LpSet[2]{\Omega}, v \in \LpSet[2]{\Omega}.$ 
	Suppose the following equation: find $u_n$ such that
	\begin{align*}
		-\laplace u_n &= Au_n, \, \text{in} \,\Omega \\
		u_n &= 0 \, \text{on} \, \partial \Omega.
	\end{align*}
	From the regularity theory for elliptic problems, we know that $\norm{u_n}_{\WkpSet[2][2]{\Omega}} \leq C \norm{Au_n}_{\LpSet[2]{\Omega}} \leq C,$ so we can extract $u_{n_k} \rightharpoonup u$ in $\WkpSet[2][2]{\Omega}.$ Realize moreover
	\[
		\int_{\Omega}\laplace u_n \varphi \dd{x} = \int_{\Omega}u_n \laplace \varphi\dd{x}, \forall \varphi \in \mathcal{D}\qty(\Omega),
	\]
	and the limit of this is
	\[
		\int_{\Omega}v \varphi\dd{x} = \int_{\Omega}u \laplace \varphi\dd{x} = \int_{\Omega} \varphi \laplace u \dd{x},
	\]
	which means $\laplace u = v \, \text{\textit{a.e.}} \, \, \text{in} \, \Omega$ and that $u \in \mathcal{D}\qty(A), Au = v.$
\end{example}

\begin{theorem}[Closedness of infinitesimal generators]
	Let $\qty(A, \mathcal{D}\qty(A))$ be the infinitesimal generator of a $c_0$-semigroup $\qty{S(t)}_{t\geq 0} \subset \mathcal{L}\qty(X).$ Then

	\begin{enumerate}
		\item $\mathcal{D}\qty(A)$ is dense in $X$,
		\item $\qty(A, \mathcal{D}(A))$ is closed.
	\end{enumerate}
\end{theorem}

\begin{proof}(\textit{From: the lectures})
	Ad 1.:
	From the previous theorem it follows $\forall x \in X, \forall t\geq0 :x_t \in \mathcal{D}\qty(A),$ so
	\[
	\frac{1}{t}x_t = \frac{1}{t}\int_0^t S(s)x \dd{s} \in \mathcal{D}\qty(A), 
	\]
	and moreover
	\[
		\lim_{t \to 0^+} \frac{x_t}{t} \lim_{t \to 0^+} \int_0^t S(s) \dd{s} = S(0) x \in X
	\]
	, and so we have showed (again with the help of Lebesgue differentiation theorem)
	\[
		\forall x \in X \exists \qty{\frac{x_t}{t}} \subset \mathcal{D}\qty(A): \frac{x_t}{t} \to x,
	\]
	\textit{i.e.}, $\mathcal{D}\qty(A)$ is dense in $X.$

	Ad 2.:
	Take $\qty{x_n} \subset \mathcal{D}(A): x_n \to x$ in $X, Ax \to v \, \text{in} \,X$ and compute first \footnote{This "Newton-Leibniz formula" does not hold trivially, but doc. Kaplicky says it does; you have to realize that $X$ is a Banach space and work with some functionals and Bochner integrals or whatever}
	\begin{align*}
		\frac{\qty(S(h) - \idop)x}{h} &= \lim_{n \to \infty} \frac{\qty(S(h) - \idop)x_n}{h}= \frac{1}{h}\int_0^h \dv{s}\qty(S(s)x_n)\dd{s} = \frac{1}{h}\int_0^h AS(s)x_n \dd{s} =\frac{1}{h}\int_0^h S(s)Ax_n, 
	\end{align*}
so taking the limit yields 
\[
	A x = \lim_{h \to 0^+}\frac{\qty(S(h)-\text{id})x}{h} =\lim_{h \to 0^+} \frac{1}{h}\int_0^h S(s)v \dd{s} = v,
\]
	Altogether, $x \in \mathcal{D}(A), Ax = v.$	
\end{proof}

\begin{lemma}
	Let $\qty(A, \mathcal{D}(A))$ be the infinitesimal generator of $c_0$-semigroups $\qty{S(t)}_{t\geq0}, \qty{\tilde{S}(t)}_{t\geq 0}.$ Then
	\[
		\qty{S(t)}_{t\geq 0} = \qty{\tilde{S}(t)}_{t \geq 0}.
	\]
\end{lemma}
\begin{proof}(\textit{From: the lectures})
    We want to show
    \[
	    \forall x \in X, \forall t\geq 0: S(t)x = \tilde{S}(t)x.
    \]
    Fix $x \in \mathcal{D}(A), t >0.$ Then $g(s) \coloneq S(s) \tilde{S}\qty(t-x)x$ satisfies $g \in C^1\qty([0,t];X), g'(s) = S'(s)\tilde{S}(t-s)x - S(s) \tilde{S}'(t-s)x = AS(s) \tilde{S}(t-s)x - S(s) A \tilde{S}\qty(t-s)x = 0,$ as $A,S$ commute. This means $g(0) = g(1)$ and from this it follows $S(t)x = \tilde{S}(t)x, \forall x \in \mathcal{D}(A).$ Since $\overline{\mathcal{D}(A)} = X, S$ continous $\Rightarrow S(t)x = \tilde{S}(t)x \forall x \in X, $ and since $t \geq 0$ was arbitrary, we are done.
\end{proof}

\subsection{Resolvent set \& operator}
\label{sec:resolvent}

\begin{definition}[Resolvent of a linear operator]
	Let $\qty(A, \mathcal{D}(A))$ be a linear (possibly unbounded) operator on $X$. We define 
	\begin{enumerate}
		\item resolvent set
			\[
				\rho(A) = \qty{\lambda \in \mathbb{K}| \lambda \, \text{id} \, - A \, \text{is invertible and} \, \inverse{\qty(\lambda \text{id} - A)} \in \mathcal{L}(X)},
			\]
		\item resolvent operator $R(\lambda, A):X \to \mathcal{D}(A): R\qty(\lambda, A) =\inverse{\qty(\lambda \text{id} - A)},$ for $\lambda \in \rho(A).$
	\end{enumerate}
	\end{definition}
\begin{remark}
	If $(A, \mathcal{D}(A))$ is a closed linear operator: $\lambda \in \rho(A) \Leftrightarrow \lambda \text{id} - A$ is a bijection of $\mathcal{D}(A) \, \text{onto} \, X.$
\end{remark}

\begin{lemma}
    Let $\qty(A, \mathcal{D}(A))$ be a linear operator on $X$. Then it holds

    \begin{enumerate}
	\item $\forall x \in X, \forall \lambda \in \rho(A): AR(\lambda, A)x = \lambda R(\lambda, A)x - x,$
	\item $\forall x \in \mathcal{D}(A), \forall \lambda \in \rho(A): R(\lambda, A)Ax = \lambda R(\lambda, A)x - x,$
	\item $\forall \lambda, \eta \in \rho(A): R(\lambda, A) - R(\eta, A) = (\eta - \lambda) R(\lambda, A) R(\eta, A),$ and $R(\lambda, A) R(\eta, A) = R(\eta, A) R(\lambda, A)$,
	\item If moreover $(A, \mathcal{D}(A))$ is the infinitesimal generator of a $c_0$-semigroup $\qty{S(t)}_{t\geq 0} \, \text{\textit{s.t.}} \, \forall t\geq 0: \norm{S(t)}_{\mathcal{L}(X)} \leq Me^{\omega t}$, then $\forall \lambda > \omega$ it holds
		\begin{itemize}
			\item $\lambda \in \rho(A),$ 
			\item
				\[
					R(\lambda, A) = \int_0^\infty e^{-\lambda t} S(t) \dd{t},
				\]
			\item $\norm{R(\lambda, A)}_{\mathcal{L}(X)} \leq \frac{M}{\lambda- \omega}.$
		\end{itemize}
    \end{enumerate}
\end{lemma}

\begin{remark}
    The point 4 says that under some conditions, the resolvent operator is the Laplace transformation of the semigroup operator.
\end{remark}

\begin{proof}(\textit{From: the lectures})
    Ad 1.: 
    \[
	    A R(\lambda, A)x = \qty(A- \lambda \text{id})\underbrace{R(\lambda, A)}_{ = \inverse{\qty(\lambda \text{id}-A)}}x + \lambda R(\lambda, A)x = \lambda R(\lambda, A)x - x.
    \]
    Ad 2.:
    The same as 1.

    Ad 3.:
    \begin{align*}
	    R(\lambda, A) - R\qty(\eta, A) &= R(\lambda, A)\qty(\text{id} - \qty(\lambda \text{id} - A))R\qty(\eta,A) = R\qty(\lambda,A)\qty(\eta \text{id} - A - \lambda \text{id} +A)R\qty(\eta,A) =\\
	    &=\qty(\eta - \lambda)R\qty(\lambda, A)R\qty(\eta,A)
    \end{align*}
    For $\lambda \neq \eta$ we also have
    \[
	    R\qty(\lambda,A) R\qty(\eta A) = \frac{R\qty(\lambda,A)-R\qty(\eta,A)}{\eta-\lambda} = \frac{R\qty(\eta,A)-R\qty(\lambda, A)}{\lambda-\eta} = R\qty(\eta,A) R\qty(\lambda,A).
    \]


    Ad 4.:
    WLOG asume $\omega = 0,$ meaning $\norm{S(t)}_{\mathcal{L}(X)}\leq M \forall t \geq 0.$ Denote $\tilde{S}(t) = e^{- \omega t}S(t).$



    Define
    \[
	    \tilde{R}x = \int_0^\infty e^{-\lambda t}S(t)x \dd{t}.
    \]
    First of all, this is well defined as
    \[
	    \norm{\tilde{R}x}_X \leq \int_0^{\infty}e^{-\lambda t}M \norm{x}_X \dd{t} = \frac{M}{\lambda}\norm{x}_X,
    \]
    and so $\norm{\tilde{R}}_{\mathcal{L}(X)} \leq \frac{M}{\lambda}, \tilde{R} \in \mathcal{L}(X).$ Next, we want to show
    \[
	    \forall x \in X: \tilde{R}x \in \mathcal{D}(A) \wedge A \tilde{R}x = \lambda \tilde{R}x -x \Leftrightarrow \text{id} = \qty(\lambda \text{id}-A)\tilde{R}.
    \]


    For $x \in X, h >0$ fixed compute
    \begin{align*}
	    \frac{1}{h}\qty(S(h)\tilde{R}x - \tilde{R}x) &= \frac{1}{h}\qty(\int_0^{\infty}e^{-\lambda t}S(t+h)x - e^{-\lambda t}S(t)x \dd{t})=\\
							 &= \frac{1}{h}\qty(\int_h^\infty e^{-\lambda\qty(t-h)}S(t) x \dd{t} - \int_0^{\infty} e^{-\lambda t}S(t) x \dd{t}) =\\
							 & = \int_h^{\infty} \frac{e^{-\lambda(t-h)}-e^{-\lambda t}}{h}S(t)x \dd{t} - \frac{1}{h}\int_0^h e^{-\lambda t}S(t) x\dd{t}, \\
							 & = \int_h^{\infty}e^{-\lambda t} \frac{e^{\lambda h}-1}{h}S(t)x \dd{t} - \frac{1}{h}\int_0^h e^{-\lambda t}S(t) x\dd{t} = \\
							 & = \frac{e^{\lambda h}-1}{h} \int_h^{\infty}e^{-\lambda t}S(t)x \dd{t} - \frac{1}{h}\int_0^h e^{-\lambda t}S(t) x\dd{t} = 
    \end{align*}
    and so 
    \[
	    \lim_{h \to 0^+} \frac{1}{h}\qty(S(h)\tilde{R}x - \tilde{R}x) = \lim_{h \to 0^+} \frac{e^{\lambda h}-1}{h}\int_h^{\infty} e^{-\lambda t}S(t)x \dd{t} - \frac{1}{h}\int_0^h e^{-\lambda t}S(t) x\dd{t} = \lambda \int_0^{\infty}e^{- \lambda t} S(t) x \dd{t} -x = \lambda \tilde{R}x -x,
    \]
    meaning truly
    \[
	    \tilde{R}x \in \mathcal{D}(A), A \tilde{R}x = \lambda \tilde{R} x -x \Rightarrow \idop = \qty(\lambda \idop - A)\tilde{R}.
    \]
    \textit{i.e.} 

        To show that $\tilde{R}$ is also the second inverse, we need the following theorem:
    \[
	    x \in \mathcal{D}(A), A \, \text{closed} \,: A \tilde{R}x = A\qty(\int_0^{\infty}e^{-\lambda t}S(t)x \dd{t}) = \int_0^{\infty}e^{- \lambda t}\underbrace{A S(t)}_{= S(t)A} x \dd{t} = \tilde{R}Ax,
    \]
    which has been stated but not proved \footnote{It could be shown by first constructing a approximating sequence of the Bochner integral, like a Riemann sum, do the calculation on this level and then pass to the limit.}. Using this however, one has
    \[
	    \idop = \lambda \tilde{R} - A \tilde{R} = \lambda \tilde{R} - \tilde{R}A = \tilde{R}\qty(\lambda \idop - A),
    \]
    and so in total 
    \[
	    \idop = \tilde{R}\qty(\lambda \idop - A) = \qty(\lambda \idop - A) \tilde{R}.
    \]
    And so $\lambda \idop - A$ has an inverse, which together with the fact $\tilde{R}$ is bounded concludes, $\lambda \in \rho\qty(A)$ and $\tilde{R} = R(\lambda,A).$ The fact that $R\qty(\lambda,A)$ follows from the remark above.
\end{proof}

\begin{definition}[Contraction semigroup]
	We say that $\qty{S(t)}_{t\geq 0}$ is a contraction semigroup if
	\[
		\forall t\geq: \norm{S(t)}_{\mathcal{L}(X)} \leq 1.
	\]
\end{definition}

\begin{theorem}[Hille-Yosida]
	Let $M \geq 1, \omega \in \R.$ A linear $\qty(A, \mathcal{D}(A))$ on a Banach space $X$ generates a $c_0$-semigroup (meaning it is its infinitesimal generator) satysfing $\forall t \geq 0: \norm{S(t)}_{\mathcal{L}(X)} \leq Me^{\omega t}$ \textbf{if and only if}
	\begin{enumerate}
		\item $\qty(A, \mathcal{D}(A))$ is closed,
		\item $\mathcal{D}(A)$ is dense in $X$,
		\item $\forall \lambda > \omega, n \in \N: \lambda \in \rho(A)$ and it holds $\norm{R^n\qty(\lambda,A)}_{\mathcal{L}(X)} \leq \frac{M}{\qty(\lambda-\omega)^n}.$
	\end{enumerate}
\end{theorem}
\begin{proof}(\textit{From: the lectures})
	\textit{Proof from: lectures}
	If $M = 1, \omega = 0, $ then $\norm{R\qty(\lambda,A)}_{\mathcal{L}(X)}\leq \frac{1}{\lambda} \Rightarrow \norm{R^n\qty(\lambda, A)}_{\mathcal{L}(X)} \leq \frac{1}{\lambda^n}.$
	$"\Rightarrow"$ has been proven, now show the other direction. The plan is to 
\begin{enumerate}
	\item approximate $A$ by $\qty{A_n}\subset \mathcal{L}(X)$,
	\item construct $S_n$ for $A_n$ as previously,
	\item estimate and limit passage.
\end{enumerate}

\textit{Approximation:}
See the analogy: $a \in \R: \frac{n}{n-a} \to 1,$ we would like $n R(n,A)  = n \qty(n \idop - A)^{-1} \to \, \text{id} \,,$ as then our obvious candidate for the aproximation of $A$ would be
\[
	A_n x = A\qty(n R(n,A))x,
\]
and thanks to the property $AR\qty(\lambda,A) = \lambda R\qty(\lambda,A) - \idop,$ we actually see $A n R\qty(n,A) = n \qty(n R(n,A) - \idop),$ which clearly is a linear bounded operator.

Let us so calculate the norm of $nAR(n,A) = n\qty(nR(n,A)- \idop) \in \mathcal{L}(X) \forall n \in \N,$ (This approx. is called the Yosida approximation.) 

Fix some $x \in \mathcal{D}(A)$ and write
\[
	\norm{n R(n,A)x - x}_X = \norm{R(n,A)Ax}_X \leq \norm{R(n,A)}_{\mathcal{L}(X)} \norm{Ax}_X \leq \frac{1}{n}\norm{Ax}_X \to 0 \, \text{as} \, n\to \infty.
\]
If
\[
	y \in X: \norm{nR\qty(n,A)y -y}_X \leq \norm{nR(n,A)(y-x)}_X + \norm{nR(n,A)x-x}_X + \norm{x-y}_X \leq 2 \norm{y-x} + \underbrace{\norm{nR(n,a)x -x}_X}_{\to 0},
\]
but $\norm{y-x}_X$ can be made arbitrarily small from density of $\mathcal{D}(A) \, \text{in} \,X, $ so in fact
\[
	n R(n,A)y \to y \, \text{in} \, X, \forall y \in X.
\]
And so $n R(n,A)$ really approximates $\, \text{id} \,.$ 

Using this gives us
\[
	\forall x \in \mathcal{D}(A): A_n x = n AR(n,A)x = n \overbrace{R(n,A)A}^{=AR(n,A)}x \to Ax \, \text{in} \,X
\]
pointwisely. Define now (this is only possible since $\qty{A_n} \subset \mathcal{L}(X)$)
\[
	S_n(t) = \sum_{k=0}^{\infty}\frac{\qty(A_n t)^k}{k!} \in \mathcal{L}(X) \forall t>0,
\]
which has a norm
\[
	\norm{S_n(t)}_{\mathcal{L}(X)} = \norm{\sum_{k=0}^{\infty}\frac{1}{k!}\qty(tA_n)^k}_{\mathcal{L}(X)} = \norm{\sum_{k=0}^{\infty} \frac{1}{k!}\qty(-n t \text{id} + n^{2} t R(n,A))^k}_{\mathcal{L}(X)}
\]
and we claim this is equal to
\[
	= \norm{\sum_{k=0}^{\infty}\frac{1}{k!}(-n t \text{id})^k \sum_{k=0}^\infty \frac{\qty(n^{2} t R(n,A))^k}{k!}}_{\mathcal{L}(X)},
\]
which follows from the Cauchy theorem on products of series. Estimating this gives
\[
	\norm{S_n(t)}_{\mathcal{L}(X)}\leq e^{-nt}\, \text{id} \, \sum_{k=0}^{\infty}\frac{\qty(nt)^k}{k!}\norm{nR(n,A)}^k_X \leq e^{-nt} e^nt = 1,
\]
as $\norm{nR(n,A)}^k \leq 1.$ This means
\[
	\qty{S_n(t)}_{\mathcal{L}(X)} \leq 1 = Me^{\omega t},
\]
for our case $M=1, \omega = 0.$
Now show that this converges: fix $x \in \mathcal{D}(A)$, compute
\begin{align*}
	\norm{S_n(t)x - S_m(t)x}_X &= \norm{\int_0^t \dv{s} \qty(S_n(s) S_m(t-s)x)\dd{s}}_X = \norm{\int_0^t S_n(s)(A_n - A_m)S_m(t-s)x \dd{s}}_X \leq \\
				   &\underbrace{\leq}_{\norm{S_l}_{\mathcal{L}(X)}\leq 1} t\norm{(A_n - A_m)x}_X, 
\end{align*}
and since $\qty{A_n x}_{n \in \N}$ converges, it is Cauchy and thus ${S_n(t)x} \subset X$ is Cauchy, but since $X$ is Banach, it also converges. Finally, for $y \in X,$ we have
\[
	\norm{S_n(t)y - S_m(t)y}_X \leq \norm{S_n(t)(y-x)}_X + \norm{S_n(t)x - S_m(t)x}_X + \norm{S_m(x-y)}_X \leq 2 \norm{x-y}_X + t\norm{(A_n - A_m)x}_X,
\]
and since again $\mathcal{D}(A)$ is dense in $X$, the norm $\norm{x-y}$ can be made aribtrarily small $\Rightarrow \qty{S_n(y)}$ converges $\forall y \in X.$ By Banach-Steinhaus we can now argue that since $S_n$ are linear continuous operators on a Banach space, that are uniformly bounded by $1$ and coverge pointwisely $\forall y \in X,$ the operators must also converge in $\mathcal{L}(X),$ meaning
\[
	\exists S(t) \in \mathcal{L}(X) \, \text{\textit{s.t.}} \, S_n(t)y \to S(t)y, \forall t \geq 0, \forall y \in X,
\]
and so ${S(t)}_{t \geq 0}$ really is a $c_0-$semigroup.

It remains to answer this question. Is $\qty(A, \mathcal{D}(A))$ the infinitesimal generator of $\qty{S(t)}_{t\geq 0}$? Let $(\tilde{A}, \mathcal{D}\qty(\tilde{A}))$ be the infinitesimal generator of $\qty{S(t)}_{t\geq 0}$. Compute
\[
	S_n(t)x - x = \int_0^t \dv{s} S_n(s) x \dd{s} = \int_0^t S_n(x)A_n x \dd{s},
\]
realize that
\[
	\norm{S_n(t) A_n x - S(s) A x}_X \leq \norm{S_n(s)(A_n - A)x}_X + \norm{\qty(S_n(s)-S(s))Ax}_X \to 0,
\]
from the previously shown convergences, and so (we have taken the limit of the LHS also)

\[
	S(t)x -x = \int_0^t S(s)Ax \dd{s}.
\]
This allows us to compute
\[
	\forall x \in \mathcal{D}(A): \lim_{t\to 0_+}\frac{S(t)x- x}{t} = Ax \Rightarrow \mathcal{D}(A) \subset \mathcal{D}(\tilde{A} \wedge A = \tilde{A} \, \text{on} \, \mathcal{D}(A).
\]
The opposite inclusion is simple: fix $\lambda >0: \lambda \in \rho(A) \cap \rho\qty(\tilde{A}),$ and so $\lambda \text{id} - A: \mathcal{D}(A) \to X$ is onto, but also $\lambda \text{id} - A = \lambda \text{id} - \tilde{A}$ on $\mathcal{D}(A)$, and so $\lambda \text{id}-\tilde{A}: \mathcal{D}(A) \to X$ is onto. From the previous theorem, we know $\lambda \text{id} - \tilde{A}$ is one-to-one, so $\mathcal{D}\qty(\tilde{A}) = \mathcal{D}(A).$ Altogether, $A = \tilde{A}, \mathcal{D}(A) = \mathcal{D}(\tilde{A}).$


\end{proof}

