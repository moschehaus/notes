\documentclass[reqno, a4paper]{article}
\usepackage{amsmath}
\usepackage{amssymb}
\usepackage{amsthm}

% PAGE DIMENSION

% BIBLIOGRAPHY
\usepackage{natbib}
\usepackage{bibentry} % inline refereces

% ENCODING, LANGUAGE
\usepackage[english]{babel}
\usepackage[utf8]{inputenc}

% GRAPHICS
\usepackage{subfig}
\usepackage{graphicx}
\usepackage{tikz}

% HYPERTEXT, SOURCE CODE SPECIALS
\usepackage[unicode]{hyperref}
\usepackage[active]{srcltx} % use TeX-souce-specials-mode

% SYMBOLS, FONTS
\usepackage{mathbbol}
\usepackage{bm} % sophisticated \boldsymbol
\usepackage{physics}
%\usepackage{stmaryrd}
\usepackage{MnSymbol} % \lsem, \rsem, tensor product :
\usepackage{gensymb}
\usepackage{eurosym}

% UNITS, TYPESETTING TENSORS
\usepackage{units}
\usepackage{tensor}
\usepackage{accents}

% COMPACT LIST ENVIRONMENT
\usepackage{enumitem}

% LINE NUMBERS
\usepackage{lineno}

\usepackage{multicol}

% SELECTIVELY INCLUDE/EXCLUDE PARTS OF TEXT
\usepackage{comment}

% FLOAT BARRIER
\usepackage{placeins}

%\makeatletter
% \@ifpackageloaded{tensor}% tensor is a package for a better typesetting of tensors
% {
% \renewcommand{\tnsr@Aux}[3][]{%
% \mathpalette{\tnsr@Plt{#1}{#3}}{\mathrm #2}%
% \tnsr@Wrn
% }%\tnsr@Aux
% }{%
% \relax%
% }
% \makeatother


\theoremstyle{definition}
\newtheorem{definition}{Definition}
\newtheorem*{example}{Example}

\theoremstyle{plain}
\newtheorem{lemma}{Lemma}
\newtheorem{theorem}{Theorem}

\theoremstyle{remark}
\newtheorem*{remark}{Remark}

% operators
\DeclareMathOperator{\Sym}{Sym}
\DeclareMathOperator{\signum}{sign}
\DeclareMathOperator{\supp}{supp}
\DeclareMathOperator{\diam}{diam}
\DeclareMathOperator{\cof}{cof} % cofactor
\DeclareMathOperator{\residue}{res}
\DeclareMathOperator{\ad}{ad} % adjoint ad_X (Y) = [X,Y]  
\DeclareMathOperator{\dist}{dist} % distance in a metric space

% Load xparse (if not already loaded)
\usepackage{xparse}

% Continuous functions


\newcommand{\CkSet}[2]{%
	\ensuremath{\text{C}^{#1}\!\,\left(#2 \right)}}%

\newcommand{\CinfSet}[1]{%
	\ensuremath{\text{C}^{\infty}\!\,\left(#1 \right)}}%

\newcommand{\DSet}[1]{%
	\ensuremath{\mathcal{D}\!\,\left(#1 \right)}}%

\newcommand{\CklSet}[3]{%
	\ensuremath{\text{C}^{\!\, \,#1,#2}\!\,\left(#3 \right)}}%

\newcommand{\Ckl}[2]{%
\ensuremath{\text{C}^{\!\,\, #1,#2}}}%



%%%%%%%%%%%%%%%%%%%%%%%%%%%%%%%%%%%%%%%%%%%%%%%
% Lebesgue Spaces and Their Norms
%%%%%%%%%%%%%%%%%%%%%%%%%%%%%%%%%%%%%%%%%%%%%%%

% Generic Lebesgue space on a set.
\DeclareDocumentCommand{\LpSet}{ o m }{%
	\ensuremath{\text{L}_{\IfNoValueTF{#1}{\text{p}}{#1}}\!\left( #2 \right)}%
}


\newcommand{\LinfSet}[1]{%
	\ensuremath{\text{L}_{\infty}\!\,\left(#1 \right)}}%


% Norm in a Lebesgue space on a set.
\DeclareDocumentCommand{\NormLpSet}{ O{p} m m }{%
	\ensuremath{\norm{#2}_{\text{L}_{\IfNoValueTF{#1}{\text{p}}{#1}}\!\left( #3 \right)}}%
}

%%%%%%%%%%%%%%%%%%%%%%%%%%%%%%%%%%%%%%%%%%%%%%%
% Lebesgue-Bochner Spaces and Their Norms
%%%%%%%%%%%%%%%%%%%%%%%%%%%%%%%%%%%%%%%%%%%%%%%

%Generic Lebesgue - Bochner space on a set.
\newcommand{\LpIntX}[4]{%
	\ensuremath{\text{L}_{\text{#1}}\!\,\Bigl( (#2,#3);#4 \Bigr)}%
}

\newcommand{\LinfIntX}[3]{%
	\ensuremath{\text{L}_{\infty}\!\,\Bigl( (#1,#2);#3 \Bigr)}%
}

% Norm in a Lebesgue space on a set.
\newcommand{\NormLpIntX}[5]{%
	\ensuremath{\norm{#1}_{\text{L}_{\text{#2}}\!\,\left( (#3,#4);#5 \right)}}%
}

\newcommand{\NormLinfIntX}[4]{%
	\ensuremath{\norm{#1}_{\text{L}_{\infty}\!\,\left( (#2,#3);#4 \right)}}%
}


%%%%%%%%%%%%%%%%%%%%%%%%%%%%%%%%%%%%%%%%%%%%%%%
% Sobolev Spaces and Their Norms
%%%%%%%%%%%%%%%%%%%%%%%%%%%%%%%%%%%%%%%%%%%%%%%

% Generic Sobolev space on a set.
\DeclareDocumentCommand{\WkpSet}{ o o m }{%
	\ensuremath{\text{W}^{\IfNoValueTF{#1}{\text{k}}{#1},\IfNoValueTF{#2}{\text{p}}{#2}}\!\left( #3 \right)}%
}


% Sobolev space with zero boundary conditions on a set.
\DeclareDocumentCommand{\WkpzeroSet}{ o o m }{%
	\ensuremath{\text{W}^{\IfNoValueTF{#1}{\text{k}}{#1},\IfNoValueTF{#2}{\text{p}}{#2}}_0\!\left( #3 \right)}%
}

% Norm in a Sobolev space on a set.
\DeclareDocumentCommand{\NormWkpSet}{ O{k} O{p} m m }{%
	\ensuremath{\norm{#3}_{W^{\IfNoValueTF{#1}{\text{k}}{#1},\IfNoValueTF{#2}{\text{p}}{#2}}\!\left( #4 \right)}}%
}


\newcommand{\WminfSet}[2]{%
	\ensuremath{\text{W}^{#1, \infty}\!\,\left(#2 \right)}}%
% Norm in a Sobolev space with zero boundary conditions on a set.
\DeclareDocumentCommand{\NormWkpzeroSet}{ O{k} O{p} m m }{%
	\ensuremath{\norm{#3}_{W^{\IfNoValueTF{#1}{\text{k}}{#1},\IfNoValueTF{#2}{\text{p}}{#2}}_0\!\left( #4 \right)}}%
}

% Differential operators
\DeclareMathOperator{\laplace}{\bigtriangleup}
% Kernel, range, rank
\DeclareMathOperator{\kernelop}{{\mathcal N}}
\DeclareMathOperator{\rangeop}{{\mathcal R}}
\DeclareMathOperator{\rankop}{rank}
% jump
\newcommand{\jumpdis}[1]{\ensuremath{\left\lsem #1 \right\rsem}} % difference between function values at the point of jump discontinuity

% hyperbolic functions
\DeclareMathOperator{\arcsinh}{arcsinh}
\DeclareMathOperator{\arccosh}{arccosh}
\DeclareMathOperator{\arctanh}{arctanh}
\DeclareMathOperator{\arccoth}{arccoth}

% sinc function
\DeclareMathOperator{\sinc}{sinc}

% invariants of second order tensor
\DeclareMathOperator{\invariantI}{I_1}
\DeclareMathOperator{\invariantII}{I_2}
\DeclareMathOperator{\invariantIII}{I_3}

% big o
\newcommand{\bigo}[1]{\ensuremath{O\left(#1 \right)}}
\newcommand{\smallo}[1]{\ensuremath{o\left(#1 \right)}}


% imaginary unit
\newcommand{\iunit}{\ensuremath{\mathrm{i}}}


% real and imaginary part
\newcommand{\realp}{\mathrm{real}}
\newcommand{\imagp}{\mathrm{imag}}

%\newcommand{\Real}{\Re}
%\newcommand{\Imag}{\Im}
\providecommand{\Real}{\Re}
\providecommand{\Imag}{\Im}

% predicates
\newcommand{\charac}{\ensuremath{\mathrm{char}}} % characteristic quantity such as length scale, etc.
\newcommand{\reference}{\mathrm{ref}}
\newcommand{\boundary}{\mathrm{bdr}}
\newcommand{\initial}{\mathrm{init}}
\newcommand{\crit}{\mathrm{crit}}
\newcommand{\bydefinition}{\mathrm{def}}
\newcommand{\traceless}[1]{{#1}_{\delta}}

% dimensionless variables and functions
\newcommand{\dimless}[1]{#1^\star}

% derivatives
\newcommand{\diff}{\mathrm{d}}
\newcommand{\Diff}[1][]{\mathrm{D}_{#1}} % For Frechet and Gateaux derivative
\newcommand{\hDiff}[2][]{\mathrm{D}^{#1}_{#2}} % Higher order Frechet and Gateaux derivative

% inexact differential
\newcommand{\dbar}{{\mathchar'26\mkern-12mu \diff}}
\newcommand{\idiff}{\dbar}

% body
\newcommand{\body}{{\mathcal B}}

% vectors and tensors
\renewcommand{\vec}[1]{\ensuremath{\mathbf{#1}}}
\newcommand{\greekvec}[1]{\ensuremath{\boldsymbol{#1}}}
\makeatletter
\@ifpackageloaded{bm}% 
{\renewcommand{\vec}[1]{\ensuremath{\bm{#1}}}%
\renewcommand{\greekvec}[1]{\ensuremath{\bm{#1}}}%
}{%
\relax% do nothing
}
\makeatother

\newcommand{\tensorq}[1]{\ensuremath{\mathbb{#1}}}      % tensorial quantity
\newcommand{\tensorc}[1]{\ensuremath{\mathrm{#1}}}      % tensorial quantity components  

\newcommand{\conjugate}[1]{#1^\star}
\newcommand{\transpose}[1]{#1^\top}
\newcommand{\transposei}[1]{#1^{-\top}}
\newcommand{\inverse}[1]{#1^{-1}}

% Identity matrix and zero matrix
\newcommand{\identity}{\ensuremath{\tensorq{I}}} % identity
\newcommand{\tensorzero}{{\mathbb{O}}} % zero tensor

% Cauchy stress
\newcommand{\cstress}{\tensorq{T}}
\newcommand{\cstressc}{\tensorc{T}}

% Cauchy stress, thermodynamically determined part
%\DeclareMathSymbol{\robustrho}{\mathord}{letters}{"1A} % If I want to write \fid{\thcstressrho} it sometimes happens that the greek letters in subscripts get crippled, this happens especially in MDPI class. This trick protects \rho. It would work also for other greek letters; the codes are given in fontdef.dtx
% Sometimes it also helps to swith of the bm package.
\newcommand{\thcstress}{\ensuremath{\cstress_{\mathrm{th}}}} 
%\newcommand{\thcstressrho}{\ensuremath{\cstress_{\mathrm{th},\, \robustrho}}} % thermodynamically determined part divided by rho
\newcommand{\thcstressrho}{\ensuremath{\cstress_{\mathrm{th},\, \mathnormal{\rho}}}} % thermodynamically determined part divided by rho
\newcommand{\tracelessthcstress}{\traceless{\left(\thcstress\right)}} % traceless part
\newcommand{\tracelessthcstressrho}{\traceless{\left(\cstress_{\mathrm{th},\, \rho}\right)}} % traceless part divided by rho

% Extra stress tensor
\newcommand{\ecstress}{\tensorq{S}}
\newcommand{\ecstressc}{\tensorc{S}}

% First Piola stress tensor
\newcommand{\fpstress}{\tensorq{T}_{\mathrm{R}}}
\newcommand{\fpstressc}{\tensorc{T}_{\mathrm R}}

% Second Piola--Kirchhoff stress tensor
\newcommand{\spstress}{\tensorq{S}_{\mathrm{R}}}
\newcommand{\spstressc}{\tensorc{S}_{\mathrm{R}}}

% Couple stress tensor
\newcommand{\couplestress}{\tensorq{M}}
\newcommand{\couplestressc}{\tensorc{M}}

% deformation, deformation gradient
\newcommand{\deformation}{\greekvec{\chi}}
\newcommand{\deformationc}{\tensorc{\chi}}

\newcommand{\fg}{\tensorq{F}}
\newcommand{\detf}{\det\, \fg}
\newcommand{\fgradc}{\tensorc{F}}
\newcommand{\fgradrel}[3][]{\fgrad^{#1}_{#2}\left(#3\right)}

% determinant of deformation gradient, Jacobian
\newcommand{\detfgrad}{J}

% displacement
\newcommand{\displacement}{\vec{U}}
\newcommand{\displacementc}{\tensorc{U}}

% right Cauchy-Green tensor
\newcommand{\rcg}{\tensorq{C}}
\newcommand{\rcgc}{\tensorc{C}}        
\newcommand{\rcgrel}[3][]{\rcg^{#1}_{#2}\left(#3\right)}

\newcommand{\rcgb}{\overline{\rcg}} % rescaled right Cauchy--Green tensor, theory of slightly compressible materials
\newcommand{\rcgbc}{\overline{\rcgc}} % rescaled right Cauchy--Green tensor, theory of slightly compressible materials, components

% left Cauchy-Green tensor
\newcommand{\lcg}{\tensorq{B}}
\newcommand{\lcgc}{\tensorc{B}}        
\newcommand{\lcgrel}[3][]{\lcg^{#1}_{#2}\left(#3\right)}

\newcommand{\lcgb}{\overline{\lcg}} % rescaled left Cauchy--Green tensor, theory of slightly compressible materials
\newcommand{\lcgbc}{\overline{\lcgc}} % rescaled left Cauchy--Green tensor, theory of slightly compressible materials, components


%\newcommand{\piolastrain}{\tensorq{b}} % Piola deformation tensor (inverse of right Cauchy--Green)
%\newcommand{\fingerstrain}{\tensorq{c}} % Finger deformation tensor (inverse of left Cauchy--Green)

% rotation
\newcommand{\rotation}{\tensorq{R}}
\newcommand{\rotationrel}[3][]{\rotation^{#1}_{#2}\left(#3\right)}

% stretch
\newcommand{\stretchu}{\tensorq{U}}
\newcommand{\stretchurel}[3][]{\stretchu^{#1}_{#2}\left(#3\right)}
\newcommand{\stretchv}{\tensorq{V}}
\newcommand{\stretchvrel}[3][]{\stretchv^{#1}_{#2}\left(#3\right)}

% linearized strain (symmetric part of displacement gradient), skew-symmetric part of displacement gradient
% THIS MUST BE FIXED
\makeatletter
\@ifpackageloaded{bm}% 
{%
\newcommand{\linstrain}{\bbespilon} %requires \usepackage[bbgreekl]{mathbbol}
% YES, the spelling is wrong, but this is how it is coded in the package
}{%
\newcommand{\linstrain}{\bbespilon} %requires \usepackage[bbgreekl]{mathbbol}
%\newcommand{\linstrain}{\tensorq{\varepsilon}}
}

\@ifpackageloaded{bm}%
{%
\newcommand{\skewdgradient}{\bbomega} 
}{%
\newcommand{\skewdgradient}{\tensorq{\omega}}
}

\@ifpackageloaded{bm}%
{%
\newcommand{\linstress}{\bbtau} % stress in linearised elasticity
}{%
\newcommand{\linstress}{\bbtau} % stress in linearised elasticity
%\newcommand{\linstress}{\tensorq{\tau}}
}
\makeatother

\newcommand{\linstrainc}{\mathrm{\varepsilon}}
\newcommand{\linstressc}{\mathrm{\tau}}
\newcommand{\skewdgradientc}{\mathrm{\omega}}

% Lagrangean and Eulerian strain
\newcommand{\lstrain}{\tensorq{E}} % Green--Saint-Venant strain
\newcommand{\lstrainc}{\tensorc{E}} % Green--Saint-Venant strain, components
\newcommand{\estrain}{\tensorq{e}} % Euler--Almansi strain, components
\newcommand{\estrainc}{\tensorc{e}} % Euler--Almansi strain, components

% Hencky strain
\newcommand{\henckystrain}{\tensorq{H}} % Hencky strain
\newcommand{\henckystrainc}{\tensorc{H}} % Hencky strain, components

\newcommand{\henckystrainb}{\overline{\tensorq{H}}} % Hencky strain for rescaled left Cauchy--Green tensor
\newcommand{\henckystrainbc}{\overline{\tensorc{H}}} % Hencky strain for rescaled left Cauchy--Green tensor, components

\newcommand{\devhencky}{\overline{\tensorq{H}}} % Hencky strain, deviatoric part via deviatoric deformation
\newcommand{\devhenckyc}{\overline{\tensorc{H}}} % Hencky strain, deviatoric part via deviatoric deformation, components

% Hencky strain, Lagrangian
\newcommand{\henckystrainr}{\tensorq{G}} % Hencky strain 
\newcommand{\henckystrainrc}{\tensorc{G}} % Hencky strain, components

\newcommand{\henckystrainrb}{\overline{\tensorq{G}}} % Hencky strain for rescaled right Cauchy--Green tensor
\newcommand{\henckystrainrbc}{\overline{\tensorc{G}}} % Hencky strain for rescaled right Cauchy--Green tensor, components

\newcommand{\devhenckyr}{\overline{\tensorq{G}}} % Hencky strain, deviatoric part via deviatoric deformation
\newcommand{\devhenckyrc}{\overline{\tensorc{G}}} % Hencky strain, deviatoric part via deviatoric deformation, components

% Rivlin-Ericksen tensor
\newcommand{\rivlin}{{\tensorq{A}}}

% generic tensor quantity
\newcommand{\generictensor}{{\tensorq{A}}}
\newcommand{\generictensorc}{\tensorc{A}} % component of the tensor

% deviatoric part of Cauchy stress
\newcommand{\dcstress}{\cstress - \left( \frac{1}{3}\Tr \cstress \right) \identity}
\newcommand{\dcstresssymb}{\traceless{\cstress}}

% mean normal stress
\newcommand{\cstressnorm}{\frac{1}{3}\Tr \cstress}

% velocity and velocity gradient, (skew)symmetric part of velocity gradient
\newcommand{\vecv}{\ensuremath{\vec{v}}}
\newcommand{\gradv}{\ensuremath{\nabla \vecv}}
\newcommand{\gradasym}{\ensuremath{\tensorq{W}}}
\newcommand{\gradsym}{\ensuremath{\tensorq{D}}}
\newcommand{\dgradsymsymb}{\ensuremath{\gradsym_{\delta}}}
\newcommand{\gradvl}{\ensuremath{\tensorq{L}}}

% logarithmic spin
\newcommand{\logspin}{\ensuremath{\tensorq{\Omega}}^{\mathrm{log}}}

% surface velocity
\newcommand{\unders}[1]{\ensuremath{\underaccent{\mathrm{s}}{#1}}}

\newcommand{\gradsymop}{\nabla_{\mathrm{sym}}}
\newcommand{\gradasymop}{\nabla_{\mathrm{asym}}}

\newcommand{\vecvc}{\tensorc{v}}

% velocity and velocity gradient, (skew)symmetric part of velocity gradient, COMPONENTS
\newcommand{\gradsymc}{\tensorc{D}}
\newcommand{\gradasymc}{\tensorc{W}}

% functionals
\newcommand{\functional}[1]{{\mathfrak #1}}
\newcommand{\fhistory}[3]{{\functional{#1}_{#2}^{#3}}}

% base vectors
\newcommand{\bvec}[1]{\vec{e}_{#1}} % current configuration
\newcommand{\Bvec}[1]{\vec{E}_{#1}} % reference configuration

% dual base vectors
\newcommand{\bvecd}[1]{\vec{e}^{#1}} % current configuration
\newcommand{\Bvecd}[1]{\vec{E}^{#1}} % reference configuration

% Cartesian basis, current configuration
\newcommand{\bvecx}{\bvec{\hat{x}}}
\newcommand{\bvecy}{\bvec{\hat{y}}}
\newcommand{\bvecz}{\bvec{\hat{z}}}

% Cartesian basis, reference configuration
\newcommand{\BvecX}{\Bvec{\hat{X}}}
\newcommand{\BvecY}{\Bvec{\hat{Y}}}
\newcommand{\BvecZ}{\Bvec{\hat{Z}}}

% Cartesian dual basis, reference configuration
\newcommand{\BvecdX}{\Bvecd{\hat{X}}}
\newcommand{\BvecdY}{\Bvecd{\hat{Y}}}
\newcommand{\BvecdZ}{\Bvecd{\hat{Z}}}

% Cartesian dual basis, current configuration
\newcommand{\bvecdx}{\bvecd{\hat{x}}}
\newcommand{\bvecdy}{\bvecd{\hat{y}}}
\newcommand{\bvecdz}{\bvecd{\hat{z}}}

% same as above but now in cylindrical coordinates
\newcommand{\bvecr}{\bvec{\hat{r}}}
\newcommand{\bvect}{\bvec{\hat{\theta}}}
\newcommand{\bvecp}{\bvec{\hat{\varphi}}}
%\newcommand{\bvecz}{\bvec{\hat{z}}}

\newcommand{\bvecdr}{\bvecd{\hat{r}}}
\newcommand{\bvecdt}{\bvecd{\hat{\theta}}}
\newcommand{\bvecdp}{\bvecd{\hat{\varphi}}}

\newcommand{\BvecR}{\Bvec{\hat{R}}}
\newcommand{\BvecP}{\Bvec{\hat{\Phi}}}
%\newcommand{\BvecZ}{\Bvec{\hat{Z}}}

\newcommand{\BvecdR}{\Bvecd{\hat{R}}}
\newcommand{\BvecdP}{\Bvecd{\hat{\Phi}}}
%\newcommand{\BvecdZ}{\Bvecd{\hat{Z}}}

% components
\newcommand{\vhatx}[1][\vecvc]{{#1}^{\hat{x}}}
\newcommand{\vhaty}[1][\vecvc]{{#1}^{\hat{y}}}
%\newcommand{\bvhatz}{\vhat{e}_{\hat{z}}}

\newcommand{\vhatr}[1][\vecvc]{{#1}^{\hat{r}}}
\newcommand{\vhatt}[1][\vecvc]{{#1}^{\hat{\theta}}}
\newcommand{\vhatp}[1][\vecvc]{{#1}^{\hat{\varphi}}}
\newcommand{\vhatz}[1][\vecvc]{{#1}^{\hat{z}}}

% indices
\newcommand{\hatx}{\hat{x}}
\newcommand{\haty}{\hat{y}}
\newcommand{\hatz}{\hat{z}}
\newcommand{\hatr}{\hat{r}}
\newcommand{\hatp}{\hat{\varphi}}
\newcommand{\hatt}{\hat{\theta}}
\newcommand{\hatX}{\hat{X}}
\newcommand{\hatY}{\hat{Y}}
\newcommand{\hatZ}{\hat{Z}}

% inner and outer radius (for some calculations)
\newcommand{\Rin}{R_{\mathrm{in}}}
\newcommand{\Rout}{R_{\mathrm{out}}}
\newcommand{\rin}{r_{\mathrm{in}}}
\newcommand{\rout}{r_{\mathrm{out}}}
 
% base vectors, abstract covariant and contravariant basis, current configuration
\newcommand{\cobvec}[1]{\vec{g}_{#1}} % covariant base vector
\newcommand{\conbvec}[1]{\vec{g}^{#1}} % contravariant base vector
\newcommand{\cobvecn}[1]{\vec{g}_{\hat{#1}}} % covariant base vector, normalised
\newcommand{\conbvecn}[1]{\vec{g}^{\hat{#1}}} % contravariant base vector, normalised

% base vectors, abstract covariant and contravariant basis, reference configuration
\newcommand{\coBvec}[1]{\vec{G}_{#1}} % covariant base vector
\newcommand{\conBvec}[1]{\vec{G}^{#1}} % contravariant base vector
\newcommand{\coBvecn}[1]{\vec{G}_{\hat{#1}}} % covariant base vector, normalised
\newcommand{\conBvecn}[1]{\vec{G}^{\hat{#1}}} % contravariant base vector, normalised

% current configuration
\newcommand{\mtensor}{\tensorq{g}}  % metric tensor
\newcommand{\mtensorc}{{\mathrm g}} % metric tensor, components

% reference configuration
\newcommand{\mTensor}{\tensorq{G}}  % metric tensor
\newcommand{\mTensorc}{{\mathrm G}} % metric tensor, components

% Christoffel symbols
\newcommand{\christoffel}[2]{\tensor{\Gamma}{^{#1}_{#2}}}

% mean curvature
\newcommand{\meancurvature}{\mathrm{K}} % mean curvature

\newcommand{\mtensorref}{\tensorq{G}}  %metric tensor in reference configuration
\newcommand{\mtensorrefc}{{\mathrm G}} %metric tensor in reference configuration, components

% Kronecker delta, Levi--Civitta symbol
\newcommand{\kdelta}[1]{\tensor{\delta}{#1}}
\newcommand{\lcepsilon}[1]{\tensor{\epsilon}{#1}}

% distributions
\newcommand{\diracdelta}{\delta}
\newcommand{\Heaviside}{H}
\newcommand{\UnitTriangle}{U_{\mathrm{Triangle}}}

% hypergeometric function
\newcommand{\hypergeom}[4]{\ensuremath{ \mathrm{F}\left( \left[#1, #2 \right]; \left[ #3 \right]; #4\right)}}

% sets
\newcommand{\R}{\ensuremath{{\mathbb R}}}
%\@ifpackageloaded{hyperref}% \C is defined in hyperref package
%{\renewcommand{\C}{\ensuremath{{\mathbb C}}}%
%}{%
%\newcommand{\C}{\ensuremath{{\mathbb C}}}%
%}
%\renewcommand{\C}{\ensuremath{{\mathbb C}}}% The lines above are no longer needed?
\newcommand{\Q}{\ensuremath{{\mathbb Q}}}
\newcommand{\N}{\ensuremath{{\mathbb N}}}
\newcommand{\Z}{\ensuremath{{\mathbb Z}}}


% Reynolds, Womersley number, etc.
\newcommand{\Reynolds}{\mathrm{Re}}
\newcommand{\Womersley}{\mathrm{Wo}}
\newcommand{\Rayleigh}{\mathrm{Ra}}
\newcommand{\RayleighSqrt}{\mathrm{R}}
\newcommand{\Prandtl}{\mathrm{Pr}}
\newcommand{\Grashof}{\mathrm{Gr}}
\newcommand{\Mach}{\mathrm{Ma}}
\newcommand{\Froude}{\mathrm{Fr}}
\newcommand{\Peclet}{\mathrm{Pe}}
\newcommand{\Eckert}{\mathrm{Ec}}
\newcommand{\Brinkman}{\mathrm{Br}}
\newcommand{\Nusselt}{\mathrm{Nu}}

% Young modulus, Poisson ratio
\newcommand{\Young}{\mathrm{E}}
\newcommand{\Poisson}{\mathrm{\nu}}

% bulk modulus, shear modulus
\newcommand{\bulkm}{\mathrm{K}}
\newcommand{\shearm}{\mathrm{G}}

% Symetric and antisymetric tensors
\newcommand{\asym}[1]{\ensuremath{\Asym \left( #1 \right)}}
\newcommand{\sym}[1]{\ensuremath{\Sym \left( #1 \right)}}

% Energy, free energy, entropy, temperature
\newcommand{\tenergy}{\ensuremath{e}_{\mathrm{tot}}} % specific total energy (energy per unit mass), sum of specific internal energy and the specific kinetic energy
\newcommand{\ienergy}{\ensuremath{e}} % specific internal energy (energy per unit mass)
\newcommand{\menergy}{\ensuremath{e}_{\mathrm{mech}}} % specific mechanical energy (energy per unit mass), kinetic energy plus internal energy minus thermal contribution
\newcommand{\kenergy}{\ensuremath{e_{\mathrm{kin}}}} % specific kinetic energy (kinetic energy per unit mass)
\newcommand{\fenergy}{\ensuremath{\psi}} % specific free energy
\newcommand{\entropy}{\ensuremath{\eta}} % specific entropy
\newcommand{\entalphy}{\ensuremath{h}} % specific enthalpy
\newcommand{\gibbs}{\ensuremath{g}} % specific Gibbs free energy

% Decomposition of Helmholtz free energy to thermal and mechancial part
\newcommand{\fenergyth}{\fenergy^{\mathrm{thermal}}} % purely thermal part of Helmholtz free energy
\newcommand{\fenergymech}{\fenergy^{\mathrm{mech}}} % deformation-dependent part of Helmholtz free energy

\newcommand{\temp}{\ensuremath{\theta}} % temperature, Eulerian description
\newcommand{\Temp}{\ensuremath{\Theta}} % temperature, Lagrangian description
\newcommand{\thpressure}{\ensuremath{p_{\mathrm{th}}}} % thermodynamic pressure

\newcommand{\pressure}{\ensuremath{p}} % pressure -- incompressible fluids

\newcommand{\mns}{\ensuremath{m}} % mean normal stress
\newcommand{\temptoref}{\ensuremath{\vartheta}} % (temperature - reference temperature)/(reference temperature)

% Net energy, free energy, entropy, ...
\newcommand{\nettenergy}{\ensuremath{E}_{\mathrm{tot}}} % net total energy
\newcommand{\netmenergy}{\ensuremath{E}_{\mathrm{mech}}} % net mechanical energy
\newcommand{\netthenergy}{\ensuremath{E}_{\mathrm{therm}}} % net thermal energy
\newcommand{\netienergy}{\ensuremath{E}} % net internal energy
\newcommand{\netkenergy}{\ensuremath{E_{\mathrm{kin}}}} % net kinetic energy
\newcommand{\netentropy}{\ensuremath{S}} % net entropy
\newcommand{\netheat}{\ensuremath{Q}} % net heat

% Specific molar gas constant
\newcommand{\Rspecific}{\ensuremath{R_{\mathrm{s}}}}
\newcommand{\Rmol}{\ensuremath{R_{\mathrm{m}}}}
 
% Specific heat at constant volume 
\newcommand{\cheatvol}{\ensuremath{c_{\mathrm{V}}}}
\newcommand{\cheatvolref}{\ensuremath{c_{\mathrm{V},\, \reference}}} % reference value

% Specific heat at constant pressure 
\newcommand{\cheatpressure}{\ensuremath{c_{\mathrm{P}}}}
\newcommand{\cheatpressureref}{\ensuremath{c_{\mathrm{P},\, \reference}}} % reference value

% Density in reference configuration
\newcommand{\rhor}{\rho_{\mathrm{R}}}

% Energy flux, heat flux, entropy flux
\newcommand{\efluxc}{\vec{j}_{e}} % energy flux, current configuration
\newcommand{\eflux}{\vec{J}_{e}} % energy flux, reference configuration

\newcommand{\hfluxc}{\vec{j}_{q}}     % heat flux, current configuration
\newcommand{\hfluxcc}{\tensorc{j}_{q}}     % heat flux, current configuration, components
\newcommand{\hflux}{\vec{J}_{q}}     % heat flux, reference configuration

\newcommand{\entfluxc}{\vec{j}_{\entropy}} % entropy flux, current configurtion 
\newcommand{\entflux}{\vec{J}_{\entropy}} % entropy flux, reference configuration

% Energy source, entropy source
\newcommand{\esourcec}{\ensuremath{q_{e}}} % energy source, current configuration
\newcommand{\hsourcec}{\ensuremath{q}} % heat source, current configuration
\newcommand{\entsourcec}{\ensuremath{q_{\entropy}}} % entropy source, current configuration

% Thermodynamical fluxes and affinities
\newcommand{\thfluxc}[1]{\vec{j}_{#1}} % thermodynamic flux, current configuration
\newcommand{\thaffinityc}[1]{\vec{a}_{#1}} % thermodynamic affinity, current configuration

% Entropy production
\newcommand{\entprodc}{\xi} % entropy production, current configuration
%  The entropy evolution equation is written as \rho \dd{\entropy}{t} + \divx \entfluxc = \entprodc
\newcommand{\entprodctemp}{\zeta} % entropy production times temperature, current configuration

% Upper convected (Oldroyd) derivative
\newcommand{\fid}[1]{\ensuremath{\accentset{\triangledown}{#1}}}
\newcommand{\fidd}[1]{\ensuremath{\accentset{\triangledown \! \triangledown}{#1}}}

% Lower convected derivative
\newcommand{\lfid}[1]{\ensuremath{\accentset{\vartriangle}{#1}}}
\newcommand{\lfidd}[1]{\ensuremath{\accentset{\vartriangle \! \vartriangle}{#1}}}

% Jaumann derivative
\newcommand{\jfid}[1]{\ensuremath{\accentset{\medcircle}{#1}}}
\newcommand{\jfidd}[1]{\ensuremath{\accentset{\medcircle \! \medcircle}{#1}}}

% Logarithmic corrotational derivative
\newcommand{\logfid}[1]{\ensuremath{\accentset{\medcircle_{\mathrm{log}}}{#1}}}
\newcommand{\logfidd}[1]{\ensuremath{\accentset{\medcircle_{\mathrm{log}} \! \medcircle _{\mathrm{log}}}{#1}}}

% Green--Naghdi derivative
\newcommand{\gfid}[1]{\ensuremath{\accentset{\medsquare}{#1}}}
\newcommand{\gfidd}[1]{\ensuremath{\accentset{\medsquare \! \medsquare}{#1}}}

% Truesdell derivative
\newcommand{\tfid}[1]{\ensuremath{\accentset{\meddiamond}{#1}}}
\newcommand{\tfidd}[1]{\ensuremath{\accentset{\meddiamond \! \meddiamond}{#1}}}

% Generic objective derivative
\newcommand{\genericfid}[1]{\ensuremath{\accentset{\star}{#1}}}

% Material derivative (\dot with \overline)
\newcommand{\mdif}[1]{\ensuremath{\dot{\overline{#1}}}}

\makeatletter
\@ifpackageloaded{tensor}% tensor is a package for a better typesetting of tensors
{
\newcommand{\codev}[2]{\ensuremath{\left. {#1} \right|\indices{_{#2}}}}
}{%
\newcommand{\codev}[2]{\ensuremath{\left. {#1} \right|_{#2}}}
}
\makeatother

\makeatletter
\@ifpackageloaded{tensor}% tensor is a package for a better typesetting of tensors
{
\newcommand{\contradev}[2]{\ensuremath{\left. {#1} \right|\indices{^{#2}}}}
}{%
\newcommand{\contradev}[2]{\ensuremath{\left. {#1} \right|^{#2}}}
}
\makeatother


% Bessel and Kelvin functions

\newcommand{\BesselI}[2]{\ensuremath{{\mathrm I}_{#1}\left(#2\right)}} 
\newcommand{\BesselK}[2]{\ensuremath{{\mathrm K}_{#1}\left(#2\right)}}
\newcommand{\BesselJ}[2]{\ensuremath{{\mathrm J}_{#1}\left(#2\right)}}
\newcommand{\BesselY}[2]{\ensuremath{{\mathrm Y}_{#1}\left(#2\right)}}

\newcommand\BesselRoot[2]{\ensuremath{{\rm j}_{#1,#2}}}

\newcommand{\KelvinBer}[2]{\ensuremath{{\mathrm{ber}}_{#1}\left(#2\right)}} 
\newcommand{\KelvinBei}[2]{\ensuremath{{\mathrm{bei}}_{#1}\left(#2\right)}}
\newcommand{\KelvinKer}[2]{\ensuremath{{\mathrm{ker}}_{#1}\left(#2\right)}}
\newcommand{\KelvinKei}[2]{\ensuremath{{\mathrm{kei}}_{#1}\left(#2\right)}}

% Chebyshev polynominals
\newcommand{\Chebyshevp}[3]{\ensuremath{{\mathrm T}_{#1}^{#2}\left(#3\right)}} 
\newcommand{\Chebyshev}[2]{\Chebyshevp{#1}{}{#2}} 


% distance
\newcommand{\distance}[3][]{\distanceop_{#1}\left(#2, #3\right)} % distance in a metric space

% volume
\makeatletter
\@ifundefined{volume}{%
\newcommand{\volume}[1][\Omega]{\ensuremath{#1}}}%
{%
\renewcommand{\volume}[1][\Omega]{\ensuremath{#1}}}
\makeatother

% surface and volume elements (reference configuration)
\newcommand{\svolume}[1][\Omega]{\ensuremath{\partial #1}}
\newcommand{\volumee}{\diff \mathrm{V}}
\newcommand{\surfacee}{\diff \vec{S}}
\newcommand{\surfacees}{\diff \mathrm{S}}
\newcommand{\linee}{\diff \vec{X}}

% surface and volume elements (current configuration)
\newcommand{\cvolumee}{\diff \mathrm{v}}
\newcommand{\csurfacee}{{\diff \vec{s}}}
\newcommand{\csurfacees}{\diff \mathrm{s}}
\newcommand{\clinee}{{\diff \vec{x}}}

% volume and surface integral
\newcommand{\intvolume}[2][\volume]{\int_{#1} #2\; \volumee} % volume integral, reference configuration
\newcommand{\intcvolume}[2][\volume]{\int_{#1} #2\; \cvolumee} % volume integral, current configuration
\newcommand{\intsvolume}[2][\svolume]{\int_{#1} #2\; \surfacee} % surface integral, reference configuration
\newcommand{\intcsvolume}[2][\svolume]{\int_{#1} #2\; \csurfacee} % surface integral, current configuration
\newcommand{\intcsvolumes}[2][\svolume]{\int_{#1} #2\; \csurfacees} % surface integral, current configuration, scalar
\newcommand{\intsvolumes}[2][\svolume]{\int_{#1} #2\; \surfacees} % surface integral, reference configuration, scalar

% surface Jacobian
\newcommand{\surfacej}{\mathrm{j}}

% products
\newcommand{\tensortensor}[2]{\ensuremath{#1 \otimes #2}}

\makeatletter

\@ifpackageloaded{MnSymbol} % : as binary operator needs MnSymbol package
{
\newcommand{\tensordot}[2]{\ensuremath{#1 \vdotdot #2}} 
}{%
\newcommand{\tensordot}[2]{\ensuremath{#1 : #2}} 
}

\@ifpackageloaded{MnSymbol} % : as binary operator needs MnSymbol package
{
  \newcommand{\tensorddot}[2]{\ensuremath{#1 \vdots #2}} 
}{%
  \newcommand{\tensorddot}[2]{\ensuremath{#1 \vdots #2}} 
}
\makeatother

\newcommand{\tensortensorbox}[2]{\ensuremath{#1 \boxtimes #2}}
\newcommand{\vectordot}[2]{\ensuremath{#1 \bullet #2}}
\newcommand{\vectorcross}[2]{\ensuremath{#1 \times #2}}
\newcommand{\tensorschur}[2]{\ensuremath{#1 \circ #2}} % Schur/Hadamard product

\newcommand{\vectordotalt}[3]{\ensuremath{#1 \bullet_{#3} #2}} % alternative scalar product

\newcommand{\liebracket}[2]{\ensuremath{\left[#1, #2\right]}}

% function spaces
\newcommand{\scont}[2][\Omega]{\ensuremath{{\mathcal C}^{#2} \left(#1 \right)}} % space of continuous functions
\newcommand{\sdist}[1][\Omega]{\ensuremath{{\mathcal D} \left(#1 \right)}} % space of smooth functions with compact support
\newcommand{\sdistd}[1][\Omega]{\ensuremath{{\mathcal D}^\prime \left(#1 \right)}} % dual to the space of smooth functions with compact support

\newcommand{\schwartzd}[1][\Omega]{\ensuremath{{\mathcal S}^\prime \left(#1 \right)}}   % Schwartz space
\newcommand{\schwartz}[1][\Omega]{\ensuremath{{\mathcal S} \left(#1 \right)}}           % dual to Schwartz space           

\newcommand{\scdiv}[1][\Omega]{\ensuremath{{\mathcal V} \left(#1 \right)}}

\newcommand{\loc}{\mathrm{loc}}

\newcommand{\slebl}[2]{\ensuremath{L}^{#1}_{\loc} \left(#2 \right)}     % Lebesgue space, locally
\newcommand{\sleb}[2]{\ensuremath{L}^{#1} \left(#2 \right)}             % Lebesgue space


\newcommand{\ssob}[3]{\ensuremath{W}^{#1, #2} \left(#3 \right)}         % Sobolev space
\newcommand{\ssobzero}[3]{\ensuremath{W}_{0}^{#1, #2} \left(#3 \right)} % Sobolev space, functions with zero trace


% dualities, scalar products
\newcommand{\fadual}[4]{\left\langle #1, #2\right\rangle_{#3, #4}}
\newcommand{\fascal}[4]{\left\langle #1, #2\right\rangle_{#3, #4}}
\newcommand{\fascalalt}[2]{\left\langle #1, #2 \right\rangle} % alternative scalar product
\newcommand{\ddual}[2]{\left\langle #1, #2\right\rangle} % duality in distribution theory


% dual space
\newcommand{\dspace}[1]{#1^{\star}}

% tensorial function
\newcommand{\tensorf}[1]{{\mathfrak{#1}}}

% normal stress differences
\newcommand{\firstnsd}{N_1}
\newcommand{\secondnsd}{N_2}

% Laplace and Fourier transform
\newcommand{\laplacetransforms}{{\mathcal L}}
\newcommand{\laplacetransform}[2]{\laplacetransforms \left[#1\right]\left(#2\right)}
\newcommand{\inverselaplacetransform}[2]{\inverse{\laplacetransforms} \left[#1\right]\left(#2\right)}

\newcommand{\fouriertransforms}{{\mathcal F}}
\newcommand{\fouriertransform}[2]{\fouriertransforms \left[#1\right]\left(#2\right)}
\newcommand{\inversefouriertransform}[2]{\inverse{\fouriertransforms} \left[#1\right]\left(#2\right)}

% Radon transformation
\newcommand{\radontransforms}{{\mathcal R}}
\newcommand{\radontransform}[2]{\radontransforms \left[#1\right]\left(#2\right)}
\newcommand{\inverseradontransform}[2]{\inverse{\radontransforms} \left[#1\right]\left(#2\right)}

% Hilbert transformation
\newcommand{\hilberttransforms}{{\mathcal H}}
\newcommand{\hilberttransform}[2]{\hilberttransforms \left[#1\right]\left(#2\right)}

% Convolution
\newcommand{\convolution}[2]{#1 \ast #2}

% Lagrangian
\newcommand{\lagrangian}{{\mathcal L}}
\newcommand{\lpotential}{V}
\newcommand{\lkinetic}{T}


\begin{document}

\title{Thermodynamics and mechanics of solids}

\date{\today}

\author{Kamil Belan}



\begin{comment}
\begin{abstract}
  % \input{article-abstract}
  Here comes the abstract.
\end{abstract}
\end{comment}

\maketitle

\tableofcontents

\section{TODO}
\label{sec:todo}

\begin{itemize}
	\item include missing lecture about potential forces 
	\item include missing lecture about rank one convexity
	\item include weak convergence symbol
	\item fix bold greek letters
\end{itemize}

\section{Geometry}
\label{sec:geometry}

\section{Deformation}
\label{sec:deformation}

Suppose we are given an abstract body $\Omega \subset \R^d, d=2,3$. Choosing a particular state, we denote it the \textbf{reference configuration}. After the acting of forces, the body deforms into the \textbf{current, deformed configuration}. 

\begin{tikzpicture}
 \draw (0,0) .. controls (1,2) and (3,2) .. (3,0) 
                        .. controls (3,-2) and (1,-2) .. (0,-0.5)
                        .. controls (-1,-2) and (-2,-1) .. (-2,0)
                        .. controls (-2,1) and (-1,1) .. (0,0);
\node at (2,2) {$\Omega$};

 % Second blob (on the right)
\draw (7,0) .. controls (8,2) and (10,2) .. (11,0) 
                        .. controls (12,-2) and (6,-2) .. (7,-0.5)
                        .. controls (4,-5) and (1,0) .. (5,0)
                        .. controls (3,1) and (6,1) .. (7,0);

    % Label for the second blob
\node at (7,2) { $\vb{y}(\Omega)$};

    % Curved arrow connecting the two blobs
\draw[->, thick] (3,0) .. controls (4,1.5) and (4,1.5) .. (5,0);
\node at (4,1.5) {$\vb{y}:\overline{\Omega} \to \R^d$};
\end{tikzpicture}

The mapping that produces the deformation is called the \textbf{deformation}, denoted $\vb{y}$, i.e.
\[
	\vb{y}: \overline{\Omega} \to \R^d.
\]
Of large interest will be the \textbf{deformation gradient}
\[
	\fgrad(\vb{x}) = \grad{y}(\vb{x}), (\grad{y})_{ij}=\pdv{y^i}{x^j},
\]
on which we put some physically sound restrictions, such as$\det \fgrad> 0$. This means in particular that the determinant is nonzero, but also that preserves orientations of bases:
\[
(\vb{e}_1 \cross \vb{e}_2) \vdot \vb{e}_3 > 0 \Rightarrow(\fgrad\vb{e}_1 \cross \fgrad \vb{e}_2) \vdot \fgrad \vb{e}_3 >0.
\]

\begin{example}
	Suppose the deformation is given as
	\[
		\vb{y}(\vb{x}) = \begin{bmatrix}-1 & 0 \\ 0 & 1\end{bmatrix} \vb{x},
	\]
i.e., $\fgrad=\begin{bmatrix}-1 & 0 \\ 0 & 1\ \end{bmatrix}, \detf = -1$. This is an example of a deformation that is \textit{forbidden}.

\begin{tikzpicture}
	\draw (-3,0) -- (3,0);
	\draw (0,0) -- (0,2);
	\draw (0,0) rectangle (1,1) node[above right]{$\Omega$};
	\draw[dashed] (0,0) rectangle (-1,1) node[above left] {$\vb{y}(\Omega)$};
\end{tikzpicture}
Imagine it is a sheet of paper in a plane - you cannot reflect it without lifting it from the plane.
\end{example}

\subsection{Displacement}
\label{sec:displacement}
Another useful way of describing the deformation is by using the \textbf{displacement vector} $\vb{u}$:
\[
	\vb{u}(\vb{x}) = \vb{y}(\vb{x}) - \vb{x},
\]
so taking the gradient gives
\[
	\grad{\vb{u}(\vb{x})} = \fgrad(\vb{x}) - \identity.
\]
\begin{remark}
It is interesting that we do not place any restrictions on the determinant of the displacement gradient.
\end{remark}

\subsection{Changes of measures}
\label{sec:changes}

We need to examine how the volume, area and lengths change under deformation. In what follows, for a set $\omega \subset \R^d$ in the reference configuration we denote $\omega^y \subset \R^d$ to be the deformed body in the current configuration, i.e.
\[
	\omega^y = \vb{y}(\omega).
\]

\subsubsection{Change of volume}
\label{sec:Change of volume}
Using the change of variable theorem we obtain

\begin{equation*}
	\lambda(\omega^y) = \int_{\omega^y} 1 \dd{\vb{x}^y} = \int_\omega \detf(\vb{x}) \dd{\vb{x}},
\end{equation*}
so we write $\dd{\vb{x}^y} = \detf \dd{\vb{x}}$. This motivates "our" definition of the determinant of the deformation gradient:

\begin{equation}
\label{eq:detfdefinition}
	\detf (\vb{x}) = \lim_{r\to 0} \frac{\lambda(\vb{y}\qty(B(\vb{x},r))}{\lambda(B(\vb{x},r))},
\end{equation}
where $B(\vb{x},r)$ is a (closed) ball centered at $\vb{x}$ of radius $r$.

\subsubsection{Change of lengths}
\label{sec:chlengths}
Suppose the line segment $\vb{x}+\Delta \vb{x}$ undergoes deformation. How does its length change?
Taylor expansion yields:
\begin{equation*}
	\vb{y}(\vb{x}+\vb{\Delta x})= \vb{y}(\vb{x}) + \fgrad(\vb{x})\Delta \vb{x} + \, \text{h.o.t.},
\end{equation*}
where h.o.t. stands for higher order terms. Using the expansion, we can write

\[
	\norm{\vb{y}(\vb{x}+\Delta \vb{x}) - \vb{y}(\vb{x})}^2 = \transpose{(\Delta \vb{x})} \transpose{\fgrad} \fgrad \Delta \vb{x} = 
	\transpose{(\Delta \vb{x})} \tensorq{C}(\vb{x}) \Delta \vb{x},
\]
where we have denoted
\[
	\tensorq{C}(\vb{x}) = \transpose{\fgrad}(\vb{x})\fgrad(\vb{x}),
\]
as the \textbf{Right Cauchy Green tensor}. 

\begin{example}
	Let the deformation $\vb{y}$ be given as $\vb{y}(\vb{x}) = \tensorq{R} \vb{x}+ \vb{v}, \vb{v} \in \R^d, \tensorq{R} \in \, \text{SO(d)} = \Big\{\tensorq{A} \in \R^{d \cross d}, \transpose{\tensorq{A}} \tensorq{A} = \tensorq{A} \transpose{\tensorq{A}} = \identity, \det \tensorq{A} =1\footnote{From the fact $\tensorq{A}$ is orthogonal automatically follows $\det \tensorq{A} = \pm 1$.} \det \tensorq{A}>0\Big\}$.
	Then \fgrad = \tensorq{R}, \rcg = \identity.
\end{example}


\subsubsection{Change of surfaces}
\label{sec:chsurfaces}
For $\tensorq{A} \in \R^{d \cross d}$ regular we define the \textbf{cofactor matrix} $\cof \tensorq{A}$ as
\[
	\cof \tensorq{A} = (\det \tensorq{A}) \transposei{\tensorq{A}},
\]
which is an interesting quantity whatsoever; we will use the following theorem

\begin{theorem}[Piola's identity]
	Let $\vb{y} \in \text{C}^2(\Omega;\R^d)$, then $\forall \vb{x}\in \Omega:$
	\[
		\divergence({\cof \grad{\vb{y}(\vb{x})}}) = \vb{0}.
	\]
\end{theorem}
For a regular matrix $\tensorq{A}$, we also have the identity
\begin{equation}
	\label{eqn:inv}
	\tensorq{A}^{-1} = \frac{1}{\det \tensorq{A}} (\transpose{\cof \tensorq{A})},
\end{equation}
What about the determinant of the cofactor? Clearly
\[
	\det \cof \tensorq{A} = (\det \tensorq{A})^d \det \transposei{\tensorq{A}} = (\det \tensorq{A})^{d-1},
\]
so we can also express equation \ref{eqn:inv} in a different way

\begin{equation}
    \label{eqn:invbetter}
    \tensorq{A}^{-1} = \frac{\transpose{(\cof \tensorq{A})}}{(\det \cof \tensorq{A})^{1/d-1}}.
\end{equation}
From geometry, recall the change of variables for surface integration:
\[
	\int_{\partial \omega^y}\vb{n}^y \dd{S}^y = \int_{\partial \omega}\cof \fgrad\vb{n} \dd{S},
\]
where $\vb{n}^y$ is the outward unit normal to the deformed boundary $\omega^y$. Informally, we write $\vb{n}^y \dd{S}^y = \cof \fgrad \vb{n} \dd{S}$. We can also explicitely express the normal to the deformed boundary as

\begin{equation}
    \label{eqn:deformed_normal}
    \vb{n}^y(\vb{x}^y)=\frac{\cof \fgrad (\vb{x}) \vb{n}(\vb{x})}{\norm{\cof \fgrad(\vb{x}) \vb{n}(\vb{x})}}, \vb{x} \in \partial \omega, \vb{y}(\vb{x}) \in \partial \omega^y.
\end{equation}

\subsection{Affine transformations}
\label{sec:affine_transf}
An example of deformation is the so called \textbf{affine transformation}.
\begin{example}
	Suppose the folowing deformation:
	\[
		\vb{y}(\vb{x}) = \tensorq{A} \vb{x} + \vb{v}, \tensorq{A}\in \R^{d \cross d}, \vb{v} \in \R^d, \detf > 0.
	\]
	Clearly then $\fgrad(\vb{x}) = \tensorq{A}.$
\end{example}
It is crucial to realize how $\fgrad, \transpose{\fgrad}, \transposei{\fgrad}$ work.

\begin{itemize}
	\item $\fgrad$ takes a vector $\vb{x}-\vb{0}$ from the \textit{reference configuration} and maps it to the vector $\fgrad \vb{x}- \fgrad \vb{0}$ in the \textit{current configuration}
	\item $\fgrad^{-1}$ takes the vector $\fgrad \vb{x} - \fgrad \vb{0}$ from the \textit{current configuration} and maps it to the vector $\vb{x}-\vb{0}$ from the \textit{reference configuration}
	\item $\transpose{\fgrad}$ is defined through: $\fgrad \vb{u} \vdot \vb{w} = \vb{u} \vdot \transpose{\fgrad}\vb{w}$, and since $\fgrad$ is defined on the reference configuration, $\transpose{\fgrad}$ must take something from the \textit{current configuration} and return something from the \textit{reference configuration}.
	\item $\transposei{\fgrad}$ consequently takes something from the \textit{referrence configuration} and maps it to something from the \textit{current configuration}.
\end{itemize}	
\begin{example}
	What when $\rcg = \identity$? Can we say something about $\fgrad$? Write $\rcg = \transpose{\fgrad}\fgrad = \identity$, so $\transpose{\fgrad} = \fgrad^{-1}, \det \fgrad > 0$. From this we have $\fgrad(\vb{x}) = \tensorq{R}(\vb{x}), \vb{x} \in \Omega$, where $\tensorq{R}$ is a rotation tensor. Investigate the cofactor of the deformation gradient:
	\[
		\cof \fgrad = \det \fgrad \transposei{\fgrad} = \cof \tensorq{R} = 1 \tensorq{R}(\vb{x}) = \fgrad (\vb{x}).
	\]
This implies $\cof \fgrad = \fgrad$. Recall Piola's identity:
\[
	\vb{0} = \divergence{\cof \fgrad} = \divergence{\fgrad (\vb{x})} = \laplacian \vb{y}(\vb{x}).
\]
We have the identity:
\begin{comment}
\begin{equation}
	\frac{1}{2}(\norm{\grad{\vb{y}})^{2} = \norm{\grad \grad \vb{y}}^{2} + \grad{\vb{y}} \vdot \grad \laplacian \vb{y},
\end{equation}
\end{comment}
and since the LHS is zero, we also have $\norm{\grad \grad \vb{y}} = 0 \Rightarrow \vb{y}(\vb{x}) = \tensorq{R}\vb{x}+\vb{v}$.
Let $\tensorq{R}$ be piecewise affine. Then $\tensorq{R}_1 (\identity - \vb{n} \otimes \vb{n}) = \tensorq{R}_2(\identity - \vb{n}\otimes \vb{n})$, so $\tensorq{R}_1 - \tensorq{R}_2 = (\tensorq{R}_1 = \tensorq{R}_2) = (\tensorq{R}_1 - \tensorq{R}_2) \vb{n} \otimes \vb{n} = \vb{a} \otimes \vb{b}$, but that is not possible for two rotations; the rank of the RHS is one, whereas the LHS is not.
\end{example}

\section{Forces}
\label{sec:forces}

\subsection{Forces in the deformed configuration}
\label{sec:forcesdeformed}

Recall $\vb{y}: \overline{\Omega} \to \overline{\Omega}^y$.We can define the \textbf{volume density of applied forces} $\vb{f}^y: \overline{\Omega}^y \to \R^3$ (in newtons per cubic meters, e.g. gravity).
The same on the boundary $\vb{g}^y : \Gamma^y_N \to \R^3$ (\textbf{surface density of applied forces} (in newtons per square meters = Pascals, e.g. hydrostatic pressure.)

\subsubsection{Cauchy stress tensor}
\label{sec:cstress}

\begin{lemma}[Stress principle of Euler and Cauchy]
	There exists a (Cauchy) stress vector function $\vb{t}^y : \overline{\Omega}^y \times \mathcal{S}^{d-1} \to \R^d $ with the following properties.
\begin{enumerate}
	\item If $\vb{x}^y \in \Gamma^y_N$, then $\vb{t}^y(\vb{x}^y, \vb{n}^y) = \vb{g}^y(\vb{x}^y),$ where $\vb{n}^y$ is the unit outer normal vector to $\partial \Omega^y$ at $\vb{x}^y$.
	\item $\forall \omega^y \subset \Omega^y$ it holds that $\int_{\omega^y} \vb{f}(\vb{x^y})\dd{\vb{x}^y} + \int_{\partial \omega^y} \vb{t}^y(\vb{x}^y, \vb{n}^y)\dd{S}^y = 0.$ (\textit{Balance of forces in static equilibrium.}
	\item $\forall \omega^y \subset \Omega^y$ it holds that $\int_{\omega^y} \vb{x}^y \cross \vb{f}^y(\vb{x}^y) \dd{\vb{x}^y} + \int_{\partial \omega^y} \vb{x}^y \cross \vb{t}^y(\vb{x}^y, \vb{n}^y) \dd{S}^y = \vb{0}$. (\textit{Balance of moment of forces in static equilibrium.})
\end{enumerate}
Euler says that the direct consequence of this is the existence of $\tensorq{T}^y(\vb{x}^y)$ such that

\begin{equation}
	\label{eqn:cauchy}
\vb{t}^y(\vb{x}^y,\vb{n}^y) = \tensorq{T}^y(\vb{x}^y) \vb{n}^y, 
\end{equation}
where the tensorial quantity $ \tensorq{T}$ is called the \textbf{Cauchy stress tensor}.
\end{lemma}

\subsubsection{Balance equations in the deformed configuration}
\label{sec:balance_equations_cauchy}
Classical physics gives us 2 fundamental relations: Newtons second law for momenta and for angular momenta. We examine these in the continuum mechanics setting.

From second property it follows:

\begin{equation}
	\int_{\omega^y} \vb{f}^y(\vb{x}^y) \dd{\vb{x}^y} + \int_{\partial \omega^y} \tensorq{T}^y(\vb{x}^y) \vb{n}^y \dd{S}^y  =\int_{\omega^{y}}\vb{f}^{y}\qty(\vb{x})^{y}\dd{\vb{x}^{y}}+ \int_{\omega^y} \divergence{\tensorq{T}^y}(\vb{x}^{y}) \dd{\vb{x}^{y}} = 0,
\end{equation}

so using the localization theorem we get

\begin{equation*}
	\label{eqn:balanceofforces}
	\vb{f}^y(\vb{x}^y) + \divergence{\vb{T}^y}(\vb{x}^y) = \vb{0}, \forall \vb{x}^y \in \Omega^y.
\end{equation*}
From the third property it follows

\begin{align*}
&	\int_{\omega^y} \vb{x}^y \cross \vb{f}^y(\vb{x}^y) \dd{\vb{x}^y} + \int_{\partial \omega^y} \vb{x}^y \cross \tensorq{T}^y(\vb{x}^y) \vb{n}^{y} \dd{S}^y = \\ &= \int_{\omega^y} \varepsilon_{ijk} x^y_j f^y_k \dd{\vb{x}^y} + \int_{\omega^y}\varepsilon_{ijk}x^y_j (T^y_{km} n^y_m) \dd{S}^y = \int_{\omega^y} \varepsilon_{ijk} x^y_j f^y_k \dd{\vb{x}^y}=\int_{\omega^y} \varepsilon_{ijk} \pdv{(x^y_j T^y_{km})}{x^y_m}\dd{\vb{x}^y} = \\& = \int_{\omega^y}\varepsilon_{ijk}x^y_j f^y_k \dd{\vb{x}^y} + \int_{\omega^y}\varepsilon_{ijk}x^y_j \pdv{T_km}{x^y_m}\dd{\vb{x}^y} + \int_{\omega^y}\varepsilon_{ijk}\delta_{jm}T^y_{km} \dd{\vb{x}^y} = \vb{0}.
\end{align*}
The last term implies

\begin{equation*}
	\int_{\omega^y} \varepsilon_{ijk}T^y_{kj} = 0,
\end{equation*}
and using the localization theorem, we obtain

\begin{equation}
	T^y_{ij} (\vb{x}^y) = T^y_{ji}(\vb{x}^y), \quad{i.e.} \tensorq{T}^y(\vb{x}^y) = \transpose{(\tensorq{T}^y(\vb{x}^y))}.
\end{equation}
The \textbf{Cauchy stress tensor is symmetric.}

\subsection{Forces in the undeformed configuration}
\label{sec:forces_undeformed}

We have obtained the equations in the deformed configuration. That is however unconvenient - we solve the equations to find the deformed configuration. This brings us to find a new way to write the equations - in the reference configuration. The construction is a bit synthetic, our intuition will be guided by the requirement to obtain similair equations as in the current configuration.

\subsubsection{Piola-Kirchhoff stresses}
\label{sec:pkstresses}

\begin{definition}[First Piola-Kirchhoff stress tensor]
Given the Cauchy stress tensor $\tensorq{T}^y(\vb{x^y})$, we define the \textbf{First Piola Kirchhoff stress tensor}
\[
	\tensorq{T}: \overline{\Omega} \to \R^{3 \times 3}, \tensorq{T}(\vb{x}) = \tensorq{T}^y(\vb{x}^y) \cof \fgrad (\vb{x}) = \det \fgrad (\vb{x}) \tensorq{T}^y(\vb{x}^y) \transposei{\fgrad}(\vb{x}).
\]
\end{definition}

\begin{definition}[Second Piola-Kirchhoff stress tensor]
The quantity
	\[
		\tensorq{S}(\vb{x}) = \fgrad^{-1} \tensorq{T}(\vb{x}) = \transpose{\tensorq{S}(\vb{x})},
	\]
	is called the \textbf{second Piola-Kirchhoff stress tensor.}
\end{definition}

\begin{remark}
	The first PK tensor $\tensorq{T}$ is \textit{not symmetric in general.}, but the second $\tensorq{S}(\vb{x}) = \fgrad^{-1} = \det \fgrad (\vb{x}) \fgrad^{-1} \tensorq{T}^y(\vb{x}^y) \transposei{\fgrad}(\vb{x})$ \textit{is}. Also, we see that not every matrix can serve as $\pkstress$; it must hold $\pkstress\qty(\vb{x}) \qty(\inverse{\cof \fgrad})$ is symmetric.
\end{remark}

\begin{remark}
We have the following identity (using Piola's identity):
\begin{equation}
	\divergence{\pkstress} \qty(\vb{x}) = \detf \qty(\vb{x}) \divergence{\cstress}\qty(\vb{x}^{y})^{y}.
\end{equation}
\end{remark}

\subsubsection{Balance equations in the deformed configuration}
\label{sec:balance_equations_piola}
Recall the balance of momentum

\begin{equation*}
    -\divergence{\qty(\cstress\qty(\vb{x}^{y}))}=\vb{f}^{y}\qty(\vb{x}^{y}) \, \text{in} \, \Omega^{y},
\end{equation*}
multiplying by $\detf >0$ yields

\begin{equation}
    \detf \divergence{\qty(\cstress\qty(\vb{x}^{y}))} = \detf \qty(\vb{x}) \vb{f}^{y}\qty(\vb{x}^{y}),
\end{equation}
which \textit{begs} for the definition

\begin{equation*}
    \vb{f}\qty(\vb{x})= \detf \qty(\vb{x}) \vb{f}^{y}\qty(\vb{y}\qty(\vb{x})),
\end{equation*}
as the force in the \textit{referential configuration}.

In total, the total acting body force on the body can be written as
\[
	\int_{\vb{y}\qty(\omega)}\vb{f}^{y}\qty(\vb{x}^{y})\dd{\vb{x}^{y}} = \int_{\omega}\vb{f}^{y}\qty(\vb{y}\qty(\vb{x}))\detf \qty(\vb{x})\dd{x} = \int_{\omega}\vb{f}\qty(\vb{x})\dd{\vb{x}}.
\]

We can do the same for the balance of surface forces; the total contact force acting on the body is
\begin{align*}
	\int_{\Gamma_N^{y}}\vb{g}^{y}\qty(\vb{x}^{y})\dd{S}^{y} &=\int_{\partial \omega^{y}}\vb{t}^{y}\qty(\vb{x}^{y},\vb{n}^{y})\dd{S}^{y} = \int_{\partial \vb{y}\qty(\omega)}\cstress\qty(\vb{x}^{y})\vb{n}^{y}\dd{S}^{y}=\\
								&=\int_{\partial \omega}\cstress(\vb{y}(\vb{x}))\cof \fgrad(t,\vb{x}) \vb{n}\dd{S} = \int_{\partial \omega}\pkstress\qty(\vb{x})\vb{n}\dd{S},
\end{align*}
so if we define
\[
	\vb{g}(\vb{x}) = \pkstress\qty(\vb{x})\vb{n}\qty(\vb{x}),
\]
as the contact force in the \textit{referential configuration}, we formally have a similiar expression.

\section{Elasticity}
\label{sec:elasticity}

\begin{definition}[Elasticity]
We say that a material is \textbf{elastic (or Cauchy elastic)} if there is a response function $\tilde{\tensorq{T}}^D:\Omega \cross \R^{3 \cross 3}_+ \to \R^{3 \cross 3}_{\, \text{sym} \,}$ such that
\[
	\cstress\qty(\vb{x}^{y})= \tilde{\tensorq{T}}^D\qty(\vb{x},\fgrad).
\]
The response function is also called the \textbf{constitutive law}.
\end{definition}

\begin{remark}
	If we know the material is elastic, we also have the information about the First Piola-Kirchhoff stress, as $\pkstress\qty(\vb{x}) = \cstress\qty(\vb{x}^{y})\cof \fgrad,$ so
	\begin{equation}
		\label{eq:response_undeformed}
		\pkstress\qty(\vb{x}) = \tilde{\tensorq{T}}^D\qty(\vb{x}, \fgrad) \cof \fgrad = \tilde{\tensorq{T}}\qty(\vb{x},\fgrad).
	\end{equation}
\end{remark}

\subsection{Frame invariance principle}
\label{sec:frame_invariance}
The frame invariance principle states:
\[
	\tilde{\tensorq{T}}^D\qty(\vb{x}, \tensorq{R} \vb{x}) = \tensorq{R}\tilde{\tensorq{T}}^D\qty(\vb{x},\fgrad)\transpose{\tensorq{R}}, \forall \tensorq{R}\in \, \text{SO(3)} \,, \forall \vb{x}\in \overline{\Omega},
\]
from which it follows ($\tilde{\tensorq{T}}$ is defined in \ref{eq:response_undeformed})
\[
	\tilde{\tensorq{T}}\qty(\vb{x}, \tensorq{R} \fgrad) = \det\qty(\tensorq{R} \fgrad) \tilde{\tensorq{T}}^D\qty(\vb{x}, \tensorq{R}\fgrad)\transposei{\qty(\tensorq{R} \fgrad)} = \det\qty(\tensorq{R} \tensorq{F}) \tensorq{R} \tilde{\tensorq{T}}^D(\vb{x},\fgrad) \transpose{\tensorq{R}} \tensorq{R} \transposei{\fgrad} = \detf \tensorq{R}\tilde{\tensorq{T}}^D\qty(\vb{x},\fgrad)\transposei{\fgrad}=\tensorq{R}\tilde{\tensorq{T}}\qty(\vb{x},\fgrad),
\]
thus
\[
	\tilde{\tensorq{T}}\qty(\vb{x}, \tensorq{R} \fgrad) = \tensorq{R}\tilde{\tensorq{T}}\qty(\vb{x},\fgrad), \, \text{i.e.} \, \transpose{\tensorq{R}}\tilde{\tensorq{T}}\qty(\vb{x},\tensorq{R}\tensorq{F}) = \tilde{\tensorq{T}}\qty(\vb{x},\fgrad), \forall \tensorq{R} \in \, \text{SO(3)} \,, \forall \fgrad \in \R^{3 \cross 3}_+.
\]

\subsection{Isotropic material}
\label{sec:isotropic_materials}
Recall $\cstress(\vb{x}^{y})=\tilde{\tensorq{T}}^D\qty(\vb{x},\fgrad), \vb{y}:\overline{\Omega} \to \Omega^y = \vb{y}(\Omega).$ Take $\vb{x}_0 \in \overline{\Omega}$ general but fixed, take $\vb{v}\qty(\vb{z})=\vb{x}_0 + \transpose{R}\qty(\vb{z}-\vb{x}_0)$ for some $\tensorq{R} \in \, \text{SO(3)} \,$ and define a \textit{new deformation}
\[
	\tilde{\vb{y}}=\vb{y}\circ \inverse{\vb{v}}: \vb{v}\qty(\overline{\Omega}) \to \vb{y}\qty(\overline{\Omega}), \tilde{\vb{y}}\qty(\tilde{\vb{x}})=\vb{y}\qty(\vb{x}_0+\tensorq{R}\qty(\tilde{\vb{x}}-\vb{x}_0)).
\]
This implies
\[
	\vb{x}_0^{y} = \vb{x}_0^{\tilde{y}}, \cstress\qty(\vb{x}_0^{y})=\tilde{\tensorq{T}}^D\qty(\vb{x}_0, \fgrad)= \tensorq{T}^{\tilde{y}}(\vb{x}^{\tilde{y}}_0) = \tilde{\tensorq{T}}^D\qty(\vb{x}_0, \tilde{\fgrad}\qty(\vb{x}_0)) = \tilde{\tensorq{T}}^D\qty(\vb{x}_0, \fgrad\qty(\vb{x}_0) \tensorq{R}).
\]

\begin{definition}[Isotropic material]
	We cal the material \textbf{isotropic} if it holds
	\[
		\tilde{\tensorq{T}}^D\qty(\vb{x},\fgrad) = \tilde{\tensorq{T}}^D\qty(\vb{x},\fgrad \tensorq{R}), \forall \tensorq{R} \in \, \text{SO(3)} \,, \forall \fgrad \in \R^{3 \cross 3}_+.
	\]
\end{definition}
\begin{remark}
	For the first Piola-Kirchhoff we obtain: $\tensorq{T}^D\qty(\vb{x},\fgrad \tensorq{R}) = \tensorq{T}^D\qty(\vb{x},\fgrad) \tensorq{R}, $ which means
	\[
		\tensorq{T}^D\qty(\vb{x}, \tensorq{Q}\fgrad \tensorq{R}) = \tensorq{Q}\tilde{\tensorq{T}}^D \tensorq{R}, \forall \tensorq{R}, \tensorq{Q} \in \, \text{SO(3)} \,, \forall \fgrad \in \R^{3 \cross 3}_+.
	\]
\end{remark}

\subsection{Hyperelastic materials}
\label{sec:hyperelasticity}
\begin{definition}
	We say that a material is hyperelastic if there is a function $W:\overline{\Omega} \cross \R^{3 \cross 3}_+ \to \R$ such that
	\[
		\pkstress\qty(\vb{x}) = \tilde{\tensorq{T}}\qty(\vb{x}, \fgrad) = \pdv{W\qty(\vb{x},\fgrad)}{\fgrad}, \fgrad = \grad \vb{y}(\vb{x}).
	\]

The function $W, [W] = \frac{J}{m^3} = \frac{N m}{m^3} = \text{Pa}$ is called \textbf{stored energy density.}
\end{definition}
	
\begin{remark}
    Evidently, $W$ has a potential.
\end{remark}

\subsection{Properties of W}
\label{sec:propetiesW}
It is physical to assume
\begin{enumerate}
	\item $W\geq 0$ (energy is nonnegative)
	\item $W\qty(\vb{x},\fgrad) = W\qty(\vb{x},\tensorq{R} \fgrad), \forall \tensorq{R} \in \, \text{SO(3)} \,, \forall \vb{x} \in \overline{\Omega}, \forall \fgrad \in \R^{3 \cross 3}_+.$ (energy does not change under rotations \footnote{If this was not true, you could create infinite energy by just spinning a rubber.}
	\item $W\qty(\vb{x}, \tilde{\tensorq{R}} \tensorq{U}) = W\qty(\vb{x}, \tensorq{U}), \tensorq{U}=\sqrt{\rcg}.$ (matrices are from the polar decomposition)
	\item $W\qty(\vb{x},\fgrad) \to \infty \, \text{if} \, \detf \to 0_+$ (it takes infinite energy to deform the body to a point)
	\item $W\qty(\vb{x},\fgrad) \geq \alpha\qty(\norm{\fgrad}^p + \norm{\cof \fgrad}^q + \qty(\detf)^r)-d, \forall \alpha > 0, \forall p,q,r \geq 1, \forall d \in \R, \forall \vb{x} \in \overline{\Omega}, \forall \fgrad \in \R^{3 \cross 3}_+.$
\end{enumerate}

\begin{definition}[Natural state of the body]
    The natural state of the body is the state in which
    \begin{equation}
	    \label{eq:natural}
	    W(\vb{x},\fgrad) = 0 \wedge \pdv{W}{\fgrad}\qty(\vb{x},\fgrad) = \tensorq{0}.
    \end{equation}
\end{definition}

\begin{remark}[Unnatural vegetable]
    Not all materials (bodies) have its natural states. Zum Beispiel, carrot does not have a natural state.
\end{remark}

From the previous work, we can write $\transpose{\tensorq{R}} \pdv{W\qty(\vb{x}, \tensorq{R} \fgrad)}{\fgrad} = \pdv{W\qty(\vb{x}, \fgrad)}{\fgrad},$ and for brevity denote $W\qty(\vb{x}, \tensorq{R}\fgrad) = W_R(\vb{x}, \fgrad).$ Next, we \textit{suppose we can Taylor expand}:
\begin{align*}
	W_R\qty(\vb{x}, \fgrad + \tilde{\fgrad}) &= W\qty(\vb{x}, \tensorq{R}\fgrad + \tensorq{R}\tilde{\fgrad}) = W\qty(\vb{x}, \tensorq{R} \fgrad) + \pdv{W\qty(\vb{x},\tensorq{R} \fgrad)}{\fgrad}:\qty(\tensorq{R} \tilde{\fgrad}) + \, \text{h.o.t.} \, \\&= W_R\qty(\vb{x}, \fgrad)+ \transpose{\tensorq{R}} \pdv{W\qty(\vb{x}, \tensorq{R}\fgrad)}{\fgrad}:\tilde{\fgrad} + \, \text{h.o.t.} \,.
\end{align*}
Moreover
\[
	W_R\qty(\vb{x},\fgrad+\tilde{\fgrad}) = W_R\qty(\vb{x},\fgrad)+ \pdv{W_R\qty(\vb{x},\fgrad)}{\fgrad}: \tilde{\fgrad}.
\]

Altogether
\[
	\pdv{\fgrad}\qty(W_R\qty(\vb{x},\fgrad)-W_R\qty(\vb{x},\fgrad)) = \tensorq{0},
\]
from which it follows \footnote{The set of matrices with positive determinant is connected.}
\[
	W\qty(\vb{x},\tensorq{R} \fgrad) = W\qty(\vb{x},\fgrad) + k\qty(\tensorq{R}).
\]
Take $\fgrad = \identity,$ then
\[
	W\qty(\vb{x}, \tensorq{R}^{2}) = W\qty(\vb{x},\tensorq{R}\tensorq{R}) = W\qty(\vb{x},\tensorq{R})+k\qty(\tensorq{R}) = W\qty(\vb{x},\identity)+2 k\qty(\tensorq{R}),
\]
 so
\[
	W\qty(\vb{x},\tensorq{R}^n)=W\qty(\vb{x},\identity)+nk\qty(\tensorq{R}).
\]
Since the set of rotations is closed and bounded, it is compact, so there exists a convergent subsequence of $\{\tensorq{R}^n\}$. Moreover, we assume $W$ to be continuous (we took the derivative...), so $\lim_{n \to \infty}W\qty(\vb{x},\tensorq{R}^n)$ exists and from the properties of $W$ we get it is finite. But then $k\qty(\tensorq{R}) = 0$, as otherwise $nk\qty(\tensorq{R})\to \infty.$ All in all, we have shown
\begin{equation}
    W\qty(\vb{x}, \tensorq{R}\fgrad) = W\qty(\vb{x},\fgrad).
\end{equation}

\begin{definition}[Energy functional]
    Let us have $\partial \Omega = \Gamma_N \cup \Gamma_D, \Gamma_N \cap \Gamma_D = \emptyset,$ where the parts of the boundary are those when Neumann/Dirichlet boudary conditions are prescribed.
The energy functional of the material is the functional
\[
	I\qty(\vb{y}) = \int_{\Omega}W\qty(\vb{x}, \fgrad\qty(\vb{x}))\dd{\vb{x}} - \int_{\Omega}\vb{f}\qty(\vb{x})\vdot \vb{y}\qty(\vb{x}) \dd{\vb{x}}-\int_{\Gamma_N}\vb{g}\qty(\vb{x}) \vdot \vb{y}\qty(\vb{x})\dd{S},
\]
where the first part corresponds to the stored energy and the remaining terms are the work done by the external loads.
\end{definition}

\begin{remark}
	If $\vb{y}$ is the minimizer of $I$, then $I\qty(t \vb*{\varphi}+\vb{y})\geq I\qty(\vb{y}), \forall t, \vb*{\varphi}.$ If we denote
    \[
	    a(t)\coloneq I\qty(t \vb*{\varphi}+\vb{y}),
    \]
    then it most hold
    \[
	    0 = a'(0) = \dv{t} \eval{\qty(\int_{\Omega}W\qty(\fgrad + t \grad \vb*{\varphi})\dd{\vb{x}}- \int_{\Omega}\vb{f}\qty(\vb{x})\vdot \qty(\vb{y}\qty(\vb{x})+t \vb*{\varphi}\qty(\vb{x}))\dd{\vb{x}} - \int_{\Gamma_N}\vb{g}\qty(\vb{x}) \vdot(\vb{y}\qty(\vb{x})+t \vb*{\varphi}\qty(\vb{x}))\dd{x})}_{t=0},
    \]

calculating the derivatives yields
\begin{align*}
	0 &= \int_{\Omega}\pdv{W\qty(\fgrad)}{\fgrad}: \grad \vb*{\varphi}\dd{\vb{x}} - \int_{\Omega}\vb{f}\vdot \vb*{\varphi}\dd{\vb{x}} - \int_{\Gamma_N}\vb{g}\vdot \vb*{\varphi}\dd{S} =\\
	  &= \int_{\Omega}\pdv{x_j}\qty(\pdv{W\qty(\fgrad)}{F_{ij}}\varphi_i)\dd{\vb{x}}-\int_{\Omega}\pdv{x_j}\qty(\pdv{W\qty(\fgrad)}{F_{ij}})\varphi_i\dd{\vb{x}}- \int_{\Omega}\vb{f}\vdot \vb*{\varphi}\dd{\vb{x}}-\int_{\Gamma_N}\vb{g}\vdot \vb*{\varphi}\dd{S}= \\
	  & = \int_{\Gamma_N}\pdv{W\qty(\fgrad)}{F_{ij}}\varphi_i n_j \dd{S}-\int_{\Omega}\pdv{x_j}\qty(\pdv{W\qty(\fgrad)}{F_{ij}})\varphi_i\dd{\vb{x}}-\int_{\Omega}\vb{f}\vdot \vb*{\varphi}\dd{\vb{x}} - \int_{\Gamma_N}\vb{g} \vdot \vb*{\varphi}\dd{S},
\end{align*}
so it must hold
\[
	-\pdv{x_j}\qty(\pdv{W\qty(\fgrad)}{F_{ij}})= f_i \ \text{in} \, \Omega, \pdv{W\qty(\fgrad)}{F_{ij}}n_j=g_i \, \text{on} \, \Gamma_N.
\]
This is exactly
\[
	-\divergence{\pkstress} = \vb{f} \, \text{in} \,\Omega, \pkstress \vb{n} = \vb{g}, \, \text{on} \, \Gamma_N.
\]

This implies that $\vb{y}$ minimizes energy $\Leftrightarrow \vb{y}$ is governed by the equations of classical mechanis.

\end{remark}

Are there some other qualities of $W$? It is natural to assume
\[
	W\qty(\identity) = 0 \Rightarrow W\qty(\tensorq{R})=0, \forall \tensorq{R}\in \, \text{SO(3)} \,
\]
and $W\qty(\fgrad)>0$ whenever $\fgrad \notin \, \text{SO(3)} \,$
This however implies $W$ is not convex! Assume
\[
	\tensorq{R}_1 = \begin{bmatrix} 1 & 0 & 0\\ 0 & -1 & 0\\ 0 & 0 & -1 \end{bmatrix}, \tensorq{R}_2 = \identity,
\]
then
\[
	W\qty(\frac{1}{4}\tensorq{R}_1+\frac{3}{4}\tensorq{R}_2)>\frac{1}{4}W\qty(\tensorq{R}_1)+\frac{3}{4}W\qty(\tensorq{R}_2)=0.
\]

\begin{example}[Minimizer does not exist]
Assume $J(u) = \int_{0}^1 \qty(1-\qty(u'(x))^{2})^{2}+u(x)^{2}\dd{x}, u \in \text{W}^{1,4}(0,1), u(0) = u(1) = 0,$ and find the minimum of $J$. First of all, $J>0$, so the minimum also. I can take $u_k$ such that $u'_k(x) = 1$ on $(0,1/2)$ and $u'_k(x) = -1$ on $(1/2,1)$. Then $J\qty(u_k)\to 0 \Rightarrow \inf J = 0$ but there is no minimizer.
\end{example}

Not everything is lost...

\begin{definition}[Polyconvexity, 1977 J.M. Ball]
	$W: \R^{3 \times 3}\to \R \cup \{\infty\}$ is polyconvex provided there exists convex and lower-semicontinuous function $h: \R^{19}\to \R \cup \{\infty\}$:
\[
	W\qty(\tensorq{A})=h\qty(\tensorq{A}, \cof \tensorq{A}, \det \tensorq{A}).
\]
\end{definition}

\begin{example}
	\begin{itemize}
		\item	If $W$ is convex and lower-semicontinuous then $W$ is polyconvex. 
		\item	$W\qty(\tensorq{A}) = \det \tensorq{A}$ is polyconvex but not convex.
	\end{itemize}
\end{example}

\begin{remark}[Weak convergence in $\LpSet{\Omega; \R^3}$]
	Let $1<p<\infty$ and $\{\vb{u_k}\} \subset \LpSet{\Omega; \R^3}.$ We say $\{\vb{u_k}\}$ converges weakly to $\vb{u}$ in $\LpSet{\Omega;\R^3}$ provided
	\[
		\int_{\Omega}\vb{u_k} \vdot \vb*{\varphi}\dd{\vb{x}} \to \int_{\Omega}\vb{u}\vdot \vb*{\varphi}\dd{\vb{x}}, \forall \vb*{\varphi} \in \LpSet[p']{\Omega;\R^3}.
	\]
\end{remark}

\begin{theorem}[Magic]
	Assume that $\vb{y}^k$ converges weakly to $\vb{y}$ in $\WkpSet[1][p]{\Omega;\R^3}, \Omega \subset \R^3 \in C^{0,1}, p>3.$ Then $\det \grad \vb{y}^k$ converges weakly to $\det \grad \vb{y} \, \text{in} \, \LpSet[\frac{p}{3}]{\Omega}.$ Moreover $\cof \grad \vb{y}^l$ converges weakly to $\cof \grad \vb{y} \, \text{in} \, \LpSet[\frac{p}{2}]{\Omega;\R^{3 \times 3}}.$
\end{theorem}

\begin{proof}
	Only in 2 dimensions. The determinant can be written as:
	\[
		\det \grad \vb{y} = \pdv{y_1}{x_1}\pdv{y_2}{x_2}-\pdv{y_1}{x_2}\pdv{y_2}{x_1} = \pdv{x_1}\qty(y_1 \pdv{y_2}{x_2})-\pdv{x_2}\qty(y_1 \pdv{y_2}{x_1}) = \divergence{\qty(y_1,-\pdv{y_2}{x_1})},
	\]
	so then
	\[
		\int_{\Omega}\det \grad \vb{y}^k \varphi \dd{x} = \int_{\Omega}\pdv{x_1} \qty(y^k_1 \pdv{y_2^k}{x_2})\varphi\dd{x} - \int_{\Omega}\pdv{x_2}\qty(y^k_1 \pdv{y^k_2}{x_1})\varphi\dd{x} = -\int_{\Omega}y_1^k \pdv{y_2^k}{x_2}\pdv{\varphi}{x_1} \dd{x} + \int_{\Omega}\pdv{\varphi}{x_2} y_1^k \pdv{y_2^k}{x_1}\dd{x},
	\]
	and the result follows from the embedding theorems and strong convergence (strong times weak gives weak convergence).
\end{proof}

\subsection{Rank-one convexity}
\label{sec:rank-one}


	Assume the following domain: $\Omega = (1,2) \cross (0,4 \pi) \cross (1,2)$ and the deformation $\vb{y}: \overline{\Omega} \to \R^3, \vb{y}\qty(x_1,x_2,x_3) = \qty(x_1 \cos x_2, x_1 \sin x_2, x_3), \fgrad  = \begin{bmatrix}
		\cos x_2 & -x_1 \sin x_2 & 0\\
		\sin x_2 & x_1 \cos x_2 & 0\\
		0 & 0 & 1
	\end{bmatrix}.$
	We can calculate $\detf = x_1 \cos^{2}x_2 + x_1 \sin^{2}x_2 = x_1.$ But even though the deformation has positive determinant, we still face self-penetration issues, i.e., \textit{$\vb{y}$ is not injective}.

	\begin{theorem}[Ciarlet-Nečas condition]
    Let $p>3$ and let $\detf > 0$ a.e. in $\Omega \subset \R^3, \vb{y} \in \WkpSet[1][p]{\Omega; \R^3}.$ If
    \[
	    \int_{\Omega}\detf\qty(\vb{x})\dd{\vb{x}} \leq \lambda\qty(\vb{y}\qty(\Omega))
    \]
    then $\vb{y}$ is injective almost everywhere in $\Omega$, i.e., $\exists \omega \subset \Omega: \lambda\qty(\omega) = 0, \vb{y}|_{\Omega / \omega}$ is injective.
\end{theorem}

Is the determinant condition of any use? Let us compute, assuming $\vb{y}= \vb{0} \, \text{on } \, \partial \Omega.$
\[
	\int_{\Omega}\detf\dd{\vb{x}} = \int_{\Omega}\pdv{x_1}\qty(y_1 \pdv{y_2}{x_{2}})-\pdv{x_2}\qty(y_1 \pdv{y_2}{x_1})\dd{\vb{x}} = \int_{\partial \Omega}y_1 \pdv{y_2}{x_2}n_1 - y_1 \pdv{y_2}{x_1} n_2\dd{S} \underbrace{\Rightarrow}_{y=0 \, \text{on} \, \partial \Omega} \int_{\Omega}\detf\qty(\vb{x})\dd{\vb{x}} = 0.
\]
This is powerful! Assume that $\vb{y}\qty(\vb{x}) = \fgrad \vb{x}$ on $\partial \Omega$, then
\[
	\int_{\Omega}\detf\qty(\vb{x})\dd{\vb{x}} = \lambda\qty(\Omega)\detf.
\]
Now let
\[
	I\qty(\vb{y}) = \int_{\Omega}\detf\qty(\vb{x})\dd{\vb{x}}, \vb{y}\in \WkpSet[1][p]{\Omega; \R^3}, \vb{y}\qty(\vb{x}) = \fgrad \vb{x}.
\]
Then $I$ is constant\footnote{All constant functionals are convex.} and it holds
\[
	I\qty(\vb{y}) = \lambda\qty(\Omega)\detf.
\]

\section{Linearized elasticity}
\label{sec:linearized_elasticity}

Recall the Right Cauchy-Green tensor: $\rcg = \transpose{\fgrad}\fgrad$. Using it, we can define

\begin{theorem}[Green-Lagrange strain tensor/Green-St.-Venaint strain tensor]
    Let $\rcg$ be the Right Cauchy-Green tensor. We define the Green-Lagrange strain tensor/Green-St.-Venaint strain tensor as
    \[
	    \gsv = \frac{1}{2}\qty(\rcg - \identity).
    \]
\end{theorem}
\begin{remark}
    The Green-St.-Venaint strain tensor can be rewritten as:
    \[
	    \gsv = \frac{1}{2}\qty(\transpose{\qty(\identity+\grad \vb{u})}\qty(\identity + \grad \vb{u})- \identity) = \frac{1}{2}\qty(\grad \vb{u}+\transpose{\qty(\grad \vb{u})})+ \frac{1}{2}\transpose{\qty(\grad \vb{u})}\grad \vb{u} = \tensorq{e}\qty(\vb{u}) + \frac{1}{2}\rcg\qty(\grad \vb{u}).
    \]
\end{remark}
For the stored energy density, we can write
\[
	W\qty(\fgrad) = W\qty(\tensorq{R}\fgrad) = \overline{W}\qty(\rcg\qty(\fgrad)) = \hat{W}\qty(\gsv\qty(\fgrad)).
\]
and also
\[
	W\qty(\fgrad) = \hat{W}\qty(\tensorq{e}\qty(\vb{u})+\rcg\qty(\grad \vb{u})).
\]
It is our assumption that
\[
	\hat{W}\qty(\tensorq{0}) = 0, \hat{W}\qty(\gsv)>0 \, \text{if} \, \gsv \neq \tensorq{0},
\]
and also that
\[
	\rcg\qty(\grad \vb{u}) = \vb{0}.
\]
Using Taylor expansion, we can write

\[
	\hat{W}\qty(\tensorq{e}\qty(\vb{u}))= \hat{W}\qty(\tensorq{0})+ \pdv{\hat{W}}{\tensorq{e}}\qty(\tensorq{0})\tensorq{e}\qty(\vb{u})+\frac{1}{2}\pdv[2]{\hat{W}}{\tensorq{e}}\qty(\tensorq{0})\tensorq{e}\qty(\vb{u}) \vb{e}\qty(u) + \, \text{h.o.t.} \,.
\]
Since $\hat{W}\qty(\tensorq{0}) = \pdv{\hat{W}}{\tensorq{0}}\qty(\tensorq{0}) = 0$ the above (formal) manipulation leads us to the definition
\begin{definition}[Tensor of elastic constants]
	\[
	\mathcal{C} = \pdv[2]{\hat{W}}{\tensorq{e}}\qty(\tensorq{0}), C_{ijkl} = \pdv[2]{\hat{W}}{e_{ij}}{e_{kl}}.
	\]
\end{definition}
\begin{remark}
    Since we assume $\hat{W}$ is smooth, we have some symmetries, and from the general 81 components of $C_{ijkl}$ only 21 are unique.
\end{remark}

With the notion of a tensor of elastic constants, we can then write the stored energy density as
\[
	w\qty(\tensorq{e}) = \frac{1}{2}(\mathcal{C}\tensorq{e}):\tensorq{e}.
\]
Following our definition $\pkstress = \pdv{\hat{W}}{\fgrad}$ we see
\[
	\tensorq{\sigma} = \pdv{w\qty(\tensorq{e})}{\tensorq{e}} = \tensorq{C}\tensorq{e}, \sigma_{ij} = C_{ijkl}e_{kl}.
\]
Is a useful notion of stress. It is denoted as the \textit{Cauchy stress}.
or in components
\[
	\sigma_{ij}=C_{ijkl}e_{kl}.
\]

\subsection{Equations}
\label{sec:linel_equations}

Rewritting the equations in the linearized elasticity setting we obtain the system
\begin{align*}
	-\divergence{\tensorq{\sigma}} = - \divergence{\qty(\mathcal{C}\tensorq{e})} &= \vb{f} \, \text{in} \, \Omega \\
	\tensorq{\sigma}\vb{n} &= \vb{g} \, \text{on} \, \Gamma_{N}, \\
	\vb{u} &= \vb{0} \, \text{on} \, \Gamma_{D}.
\end{align*}
The weak formulation can be obtained as
\[
	\int_{\Omega} \pdv{x_j}\qty(C_{ijkl}e_{kl})v_i \dd{\vb{x}} = \int_{\Omega}\vb{f}\vdot \vb{v}\dd{\vb{x}}, \forall \vb{v} \in \WkpSet[1][2]{\Omega;\R^3}, u=0 \, \text{on} \, \Gamma_D,
\]
so
\[
	\int_{\Omega}C_{ijkl}e_{kl} \pdv{v_i}{x_j}\dd{\vb{x}}-\int_{\partial \Omega}C_{ijkl}e_{kl}v_i n_j \dd{S} = \int_{\Omega}\vb{f}\vdot \vb{v}\dd{\vb{x}},
\]
which can be rewritten as
\[
	\underbrace{\int_{\Omega}\tensorq{C}\tensorq{e}\qty(\vb{u}) \vdot \tensorq{e}\qty(\vb{v})\dd{\vb{x}}}_{\coloneq B\qty(u,v)} = \underbrace{\int_{\Omega}\vb{f} \vdot \vb{v}\dd{\vb{x}} + \int_{\Gamma_N}\vb{g}\vdot \vb{v}\dd{S}}_{\coloneq L(v)},
\]
where we have denoted
\[
	\tensorq{e}\qty(\vb{v}) = \, \text{sym} \,\qty(\grad \vb{v}).
\]
We are looking for
\[
	u \in V = \{ u \in \WkpSet[1][2]{\Omega; \R^3}, \tr u = 0 \, \text{on} \,\Gamma_D\}: B\qty(u,v) = L(v) \forall v \in V,
\]
and to prove the existence, we will use the Lax-Milgram lemma. Show that
\begin{itemize}
	\item $L \in V^{*}$
	\item $B:V \times V \to \R$ is V-bounded and V-coercive
\end{itemize}
Realize that in order to show the properties, we would have to be able to control $\grad \vb{u}$ by $\, \text{sym} \,\qty(\grad \vb{u}).$ Is that even possible?

\begin{example}
	Let $u=0 \, \text{on} \, \partial \Omega.$ In particular, let us take $\vb{u} \in \mathcal{D}\qty(\Omega; \R^n).$ Then
	\[
		\exists C>0: \int_{\Omega}|\tensorq{e}\qty(\vb{u})|^{2}\dd{\vb{x}} \geq c \int_{\Omega}|\grad \vb{u}|^{2}\dd{\vb{x}}.
	\]
	Can this hold? Make a quick test: Take $\vb{u}$ such that $\tensorq{e}\qty(\vb{u}) = \tensorq{0}, $ so $\frac{1}{2}\qty(\pdv{u_i}{x_j}+\pdv{u_j}{x_i}) = 0$, so of course:
	\[
		\grad \vb{u} = - \transpose{\qty(\grad \vb{u})},
	\]
	and $\grad \vb{u}$ must have the form
	\[
		\grad \vb{u} = \begin{bmatrix}
			0 & \pdv{u_1}{x_2} \\
			\pdv{u_2}{x_1} & 0
		\end{bmatrix},
	\]
	where $\pdv{u_1}{x_1} = \pdv{u_2}{x_2} = 0,$ but since $\vb{u} = \vb{0}$ at the boundary, it also holds that $\vb{u} = \vb{0}$ in $\Omega$. Okay, so that not disprove the above inequality.

	Let us try something else (although unsure what this means):
	\begin{align*}
		\int_{\Omega}|\tensorq{e}(\vb{u})|^{2}\dd{x} &= \frac{1}{4}\int_{\Omega}\qty(\pdv{u_i}{x_j}+\pdv{u_j}{x_i})\qty(\pdv{u_i}{x_j}+\pdv{u_j}{x_i})\dd{x} \\
								 &= \frac{1}{4} \int_{\Omega}\qty(\pdv{u_i}{x_j})^{2} + 2\pdv{u_i}{x_j}\pdv{u_j}{x_i} + \qty(\pdv{u_j}{x_i})^{2} \dd{x} = \frac{1}{2}\int_{\Omega}\qty(\pdv{u_i}{x_j})^{2}+ \pdv{u_i}{x_j}\pdv{u_j}{x_i}\dd{x},
	\end{align*}
	where we used the symmetry property. Integrating by parts two times to obtain "$\partial_{i}u_i \partial_{j}u_j = \qty(\partial_{j}u_j)^{2}$ \footnote{Sign does not change as we integrate 2 times. Also, we have homogenous Dirichlet}. All in all
	\[
		\frac{1}{2}\int_{\Omega}\qty(\pdv{u_i}{x_j})^{2} + \qty(\pdv{u_i}{x_i})^{2}\dd{x} \geq 0.
	\]
\end{example}
\begin{theorem}[Korn's inequality]
Let $\Omega \subset \R^n$ be bounded Lipschitz domain ($\Omega \in C^{0,1}$). Then there exists $C>0$ such that $\forall \vb{u} \in \WkpSet[1][2]{\qty(\Omega;\R^n)}$ it holds
\[
\qty(\norm{\tensorq{e}\qty(\vb{u})}_{\LpSet[2]{(\Omega; \R^{n \times n)}}}^{2}+\norm{\vb{u}}_{\LpSet[2]{(\Omega; \R^n)}}^{2}) \geq c \norm{\vb{u}}_{\WkpSet[1][2]{(\Omega;\R^n)}}.
\]
\end{theorem}


\begin{definition}[Axial vectors]
	Let $\tensorq{A} = - \transpose{\tensorq{A}}, \tensorq{A} \in \R^{n \times n}.$ Then there is $b \in \R^n$ such that $\tensorq{A} \vb{v} = \vb{b} \cross \vb{v}, \forall \vb{v} \in \R^n.$ The vector $\vb{b}$ is called the axial vector of $\tensorq{A}.$
\end{definition}

\begin{remark}[$\R^n$]
    This truly holds in $\R^n$, not only in $\R^3$. We only have to replace $\cross$ by $\wedge$, the outter product.
\end{remark}

Assume that $\vb{u} \in C^2\qty(\Omega ; \R^3).$ Then
\[
	\pdv[2]{u_i}{x_j}{x_k} = \pdv{e_{ik}}{x_j}\qty(\vb{u}) + \pdv{e_{ij}}{x_k}\qty(\vb{u}) - \pdv{e_{jk}}{x_i}\qty(\vb{u}).
\]
If now $\tensorq{e}\qty(\vb{u}) = \tensorq{0}, $ then $\vb{u}$ is an affine function, because $\pdv[2]{u_i}{x_j}{x_k}, \forall i,j,k \in \{1, 2, 3\}.$ \footnote{Recall that $\Omega$ is simply connected.} It must thus hold
\[
	u_i(x) = a_i + b_{ij}x_j,
\]
and $\pdv{u_i}{x_j} = b_{ij} = - b_{ji},$ because $\tensorq{e}\qty(\vb{u}) = \tensorq{0},$ so it must be skew symmetric. The skew-symmetry also means it can be written
\[
	\vb{u}\qty(\vb{x}) = \vb{a} + \vb{d} \cross \vb{x}.
\]

If additionaly we assume that $\vb{u} = \vb{0}$ on some $\Gamma_D \subset \partial \Omega, \mathcal{H}\qty(\Gamma_{D})>0$ and $\vb{u}\qty(\vb{x}) = \vb{a}+ \vb{d} \cross \vb{x},$ then $\vb{u} = \vb{0}$ identically in $\Omega$
This moreover means that
\[
	\vb{u} \mapsto \norm{\tensorq{e}\qty(\vb{u})}_{\LpSet[2]{\qty(\Omega; \R^{n \times n})}}
\]
is a norm on
\[
	V = \{ \vb{w} \in \WkpSet[1][2]{\qty(\Omega; \R^3)}, \vb{w} = \vb{0} \, \text{on} \, \Gamma_D \}
\]
which is equivalent to the norm of $\WkpSet[1][2]{\qty(\Omega; \R^3)}.$

Coming back to our equation $B\qty(u,v) = L(v), \forall v \in V$, we have showed everything to use Lax-Milgram $\Rightarrow \exists ! u \in V.$ This also means the functional
\[
	J(\vb{v}) = \frac{1}{2} \int_{\Omega}\qty(\tensorq{C}\tensorq{e}\qty(\vb{v}) :\tensorq{e}\qty(\vb{v})-L\qty(\vb{v}))\dd{\vb{x}}, \forall v \in V.
\]
has an unique minimizer.

\subsection{Convex analysis}
\label{sec:convex_analysis}

We will deal with the analysis of the functions $f: \R^n \to \R \cup \{+ \infty \}, f$ is convex.

\begin{definition}[Epigraph of a set]
	The epigraph of a function $f$ is the set
	\[
		\, \text{epi} \, f = \{\qty(x,y): y \geq f(x) \}
	\]
\end{definition}

\begin{remark}
    With the notion of $\, \text{epi} \,f$ we can work with sets instead of functions. Moreover, it holds
    \begin{itemize}
	    \item $\, \text{epi} \,f$ is closed $\Leftrightarrow$ $f$ is lower-semicontinuous,
	    \item $f$ is convex $\Leftrightarrow\, \text{epi} \,f$ is convex
    \end{itemize}
\end{remark}
From one of the consequences of Hahn-Banach theorem (oddělovací věty), we obtain the existence of such $\xi \in \R^n$ (dependent of $x$) that for fixed $x$ it yields
\[
	f(z) \geq f(x) + \xi \vdot (z-x), \forall z \in \R^n.
\]
\textit{If $f$ is differentiable} at $x$, then
\[
	\xi = \grad f(x).
\]
But in general it does not have to be differentiable. This motivates the following definition

\begin{definition}[Subgradient, subdifferential]
	The function $\xi(x)$ such that
	\[
		f(z) \geq f(x) + \xi(x) \vdot (z-x), \forall z \in \R^n,
	\]
	is called the \textbf{subgradient} of $f$ at $x$.
	The set of all subgradients of $f$ at $x$ is called the \textbf{subdifferential} of $f$ at $x$ and it is denoted $\partial f (x).$

	Let $f: \R^{n \times m} \to \R \cup \{\infty\}, $ convex and lower semicontinuous \footnote{$f\qty(x) \leq \liminf_{k\to \infty}f(x_k), x_k \to x$}, $f \neq \infty$. The function $\xi\qty(\tensorq{X})$ such that
	\[
		f\qty(\tensorq{Y}) \geq f\qty(\tensorq{X})+ \xi\qty(\tensorq{X}) \vdot \qty(\tensorq{Y}- \tensorq{X}), \forall \tensorq{Y} \in \R^{n \times m},
	\]
	 is called the subgradient of $f$ at $\tensorq{X}$. The set of all subgradients of $f$ at $\tensorq{X}$ is called the subdifferential and denoted $\partial f\qty(\tensorq{X})$. 

\end{definition}


\begin{remark}
	\begin{itemize}
		\item If $\partial f\qty(\tensorq{X})$ is a singleton, then $\grad f\qty(\tensorq{X})$ exist.
		\item $\partial f\qty(\tensorq{X})$ is convex
		\item  $0 \in \partial f(x) \forall x \in \R^n$ is a condition for the minimizer.
	\end{itemize}
\end{remark}

\begin{definition}[Indicator function]
    Let $K \subset \R^{n \times m}$ be a closed convex nonempty set. The function $I_K\qty(\tensorq{X})$ given as
    \[
	    I_K\qty(\tensorq{X})=\begin{cases}
		    0, & \, \text{if} \, \tensorq{X} \in K \\
		    + \infty,& \, \text{otherwise} \,
	    \end{cases},
    \]
    is called the indicator function of $K$
\end{definition}

The indicator function is hepful for constraint minimization. If $f$ is reasonably (at least finitely valued on $K$) then it holds:

\[
	\min_K f = \min_{\R^{n \times m}}\qty(f + I_K).
\]

\begin{example}[Unit interval]

	Let $K = [0,1].$ What is $\partial I_K(x)?$ 

	If $x \in (0,1)$, then $I_K(x) = 0$ so the only $\xi$ such that $I_K(y) \geq 0 + \xi (y-x)$ holds is $\xi = 0.$

	If $x = 0, x=1$ then $\partial I_K(0) = (-\infty, 0], \partial I_K(1) = [0, \infty).$ This resembles a normal "vector", but in fact it is not a single vector and more a "cone" of vectors.
\end{example}

\begin{definition}[Normal cone to a set]
    Let $K$ be closed convex nonempty set. The subdifferential of the indicator function $I_K$ is called the normal cone to the set $K$ and it is denoted by $N_K$.
\end{definition}

\begin{example}
	Minimize $x^{2}$ on $[1,2]$. We are looking for
	\[
		\min_{[1,2]}x^{2} = \min_{\R}\qty(x^{2}+I_{[1,2]}\qty(x)).
	\]
	It must hold at the minimum
	\[
		0 \in \partial \qty(x^{2}+I_{[1,2]}\qty(x)) \Leftrightarrow - \partial I_{[1,2]}(x) \subset \partial x^{2} \Leftrightarrow \qty(x^{2})' \in -N_{[1,2]}(x)
	\]
\end{example}

\begin{example}
	Take a square $K = [0,1] \times [0,1] \subset \R^2.$ We know $K \in C^{0,1}$ so the outter normal exist at a.a. points on the boundary. The outter normal does not exist in the corners, but the normal cone does.
	\begin{tikzpicture}
		%\draw (0,0) square;
	\end{tikzpicture}
\end{example}

\begin{definition}[Fenchel (convex) conjugate | Legendre transformation]
Let $x^{*}$ be a slope i have chosen (it is given). I require
\[
	f(x) \geq x^{*} \vdot x - k, \forall x \in \R^{n \times m},
\]
which means $k \geq x^{*} \vdot x - f(x), \forall x\R^{n \times m},$ and so we can define
\[
	f^{*}\qty(x^{*}) \coloneq \sup_{x \in \R^{n \times m}}\qty(x^{*} \vdot x - f(x)).
\]
\end{definition}

\begin{remark}
	$f^{*}$ is always convex even if $f$ is not. But when $f$ is convex and lower-semicontinuous, then
	\[
		f^{**} = f, \, \text{(biconjugate)} \,.
	\]
\end{remark}

\begin{theorem}[Fenchel identity]
Let $x^{*} \in \partial f(x).$ Then

\[
	x^{*} \vdot x = f(x) + f^{*}(x^{*}).
\]
\begin{proof}
Let us assume that $x^{*} \in \partial f(x)$. Then it must hold
\[
	f(y) \geq f(x) + x^{*} \vdot \qty(y-x), \forall y,
\]
so
\[
	x^{*} \vdot x - f(x) \geq x^{*} \vdot y - f\qty(y),
\]
and taking the supremum over $y$ yields\footnote{The inequality becomes equality, as it can be reached by taking $y=x$.}
\[
	x^{*} \vdot x -f (x) = \sup_y \qty(x^{*} \vdot y - f(y)) = f^{*}\qty(x^{*}).
\]
We have thus obtained
\[
	x^{*} \vdot x = f(x) + f^{*}(x^{*}).
\]

\end{proof}
\end{theorem}

\begin{remark}[Minimization of $f \Leftrightarrow$ minimization of $f^{*}$]
	We see that it holds:
	\[
		x^{*} \in \partial f(x) \Leftrightarrow x \in \partial f^{*}\qty(x^{*}).
	\]
\end{remark}

\subsection{Problem of a man...}
\label{sec:problem}

Assume a person is pulling a box of weight $m$ of weight $m$ of weight $m$ of weight $m$ by a spring. If he is pulling just a little, the box does not move, only the spring is deformed - but in a reversible, elastic way.
To move the box, the man needs to pull at least with the force $\sigma_0 = mgc,$ where c is some friction coefficient. When he us pulling with force greater than $\sigma_0,$ the box is moving and does not require any extra force to be moved (the system to be deformed). The deformation can be decomposed as
\[
	\tensorq{e} = \tilde{\tensorq{e}} + \tensorq{p},
\]
where $\tensorq{e}$ is the total strain, $\tilde{\tensorq{e}}$ is the elastic strain and $\tensorq{p}$ is the plastic strain.

\subsection{von Mieses elatoplasticity}
\label{sec:von_mieses}

The elasticity part is described as
\[
	\begin{cases}
		- \divergence{\tensorq{\sigma}} = \vb{f}, &  \, \text{in the bulk} \,\\
		\tensorq{\sigma} \vb{n} = \vb{g}, & \, \text{on the boundary} \, \\
	\end{cases},
\]
with some constituive relation $\tensorq{\sigma} = \mathcal{C}\tilde{\tensorq{e}} = \mathcal{C}\qty(\tensorq{e}-\tensorq{p}).$
What about the plastic part? 
\[
	\begin{cases} 
		\dot{\tensorq{p}}(t) \in N_K\qty(\tensorq{\sigma}), & \\
		\tensorq{p}(0)=\tensorq{p}_0, &
	\end{cases}
\]
where $K$ is a convex closed subset such that $0 \in K$. This means that the plastic deformation is zero inside $K$, i.e. for some stresses. 

\begin{remark}
    Very often, the deformation is considered "incompressible", i.e.,
    \[
	    \det \fgrad = 1,
    \]
    which in linear case translates into
    \[
	    \tr \tensorq{\varepsilon} = 0.
    \]
\end{remark}

In most cases, the set $K$ is given as
\[
	K = \{ \tensorq{\sigma}:\varphi\qty(\tensorq{\sigma}) \leq 0\},
\]
where $\varphi$ is the \textbf{yield function.} The set
\[
	\{\tensorq{\sigma}| \varphi\qty(\tensorq{\sigma}) = 0\}
\]
is called the \textbf{yield surface.} Very often we have
\[
	\varphi\qty(\tensorq{\sigma}) = |\tensorq{\sigma}^D|-c_0,
\]
where $| \vdot |$ denotes the Frobenius norm and
\[
	\tensorq{\sigma}^D = \tensorq{\sigma} - \frac{1}{3}\qty(\tr \tensorq{\sigma})\identity,
\]
is the \textit{deviatoric part of the stress tensor}.

\subsubsection{Plastic evolution}
\label{sec:plastic_evolution}

From the previous we have
\[
	\dot{\tensorq{p}} = \begin{cases}
		\tensorq{0}, & \, \text{if} \, \varphi\qty(\tensorq{\sigma})<0, \\
		\frac{\lambda}{|\tensorq{\sigma}^D|}\tensorq{\sigma}^D, & \, \text{if} \, \varphi\qty(\tensorq{\sigma}) = 0, \lambda \geq 0
	\end{cases}.
\]
Also $\dot{\tensorq{p}} \in N_K\qty(\tensorq{\sigma}) = \partial I_K\qty(\tensorq{\sigma})$ so
\[
	\sigma \in \partial I_K^{*}\qty(\dot{\tensorq{p}}),
\]
where
\[
	I_K^{*}\qty(\dot{\tensorq{p}}) = \sup_{\tensorq{q}\in \R^{3 \times 3}}\qty(\dot{\tensorq{p}}: \tensorq{q}- I_K\qty(\tensorq{q})) = \sup_{\tensorq{q}\in K}\dot{\tensorq{p}}: \tensorq{q},
\]
is the Fenckel transformation of $I_K$, also called the \textbf{supporting function} of $\dot{\tensorq{p}}.$ We are able to rewrite the supremum to take the form\footnote{To utilize Cauchy-Schwarz later.}
\[
	I_K^{*}\qty(\dot{\tensorq{p}}) = \dot{\tensorq{p}} : \frac{c_0}{|\dot{\tensorq{p}}|}\dot{\tensorq{p}},
\]
if however the second term lies in $K$. Realize now that if $\tr \dot{\tensorq{p}} = 0$ then
\[
	I_K^{*}\qty(\dot{\tensorq{p}}) = c_0 |\dot{\tensorq{p}}|,
\]
and if $\tr \dot{\tensorq{p}} \neq 0$, then $I_K^{*}\qty(\dot{\tensorq{p}})= + \infty$. If we now define the \textbf{dissipation potential} $D$ as
\[
	D\qty(\dot{\tensorq{p}}) = \begin{cases}
		c_0 |\dot{\tensorq{p}}|, & \, \text{if} \, \tr \dot{\tensorq{p}} = 0 \\
		+ \infty, & \, \text{otherwise} \,
	\end{cases},
\]
we get the following condition
\[
	\tensorq{\sigma} \in \partial D\qty(\dot{\tensorq{p}}).
\]
Let us summarise a bit. For the stress tensor we have $\tensorq{\sigma} = \mathcal{C}\qty(\tensorq{e}-\tensorq{p}) \in D\qty(\dot{\tensorq{p}}).$ The general relation also yields $\tensorq{\sigma}= \pdv{w\qty(\tilde{\tensorq{e}})}{\tilde{\tensorq{e}}} = \pdv{w\qty(\tensorq{e}-\tensorq{p})}{\tilde{\tensorq{e}}}, $ where $w\qty(\tilde{\tensorq{e}}) = \frac{1}{2}C \tilde{\tensorq{e}}: \tilde{\tensorq{e}}$ is the free energy density. Using the chain rule we obtain the condition
\[
	\pdv{w\qty(\tensorq{e}-\tensorq{p})}{\tensorq{p}} \in \partial D\qty(\dot{\tensorq{p}}).
\]

In total, we are solving the following system 
\begin{equation*}
    \begin{cases}
	    0 \in \pdv{w\qty(\tensorq{e}-\tensorq{p})}{\tensorq{p}} + \partial D\qty(\dot{\tensorq{p}}), & \, \text{in} \, \Omega (\, \text{\textit{flow rule}}) \, \\
	    \tensorq{p}(0) = \tensorq{p}_0, & \, \text{in} \, \Omega \\
	    - \divergence{\qty(\mathcal{C}\qty(\tensorq{e}-\tensorq{p}))}= \vb{f}, &\, \text{in} \,\Omega \\
	    \, \text{boundary condiitons} \,, & \, \text{on} \,\partial \Omega
    \end{cases}.
\end{equation*}

How to solve the system?

\subsubsection{Discrete time setting}
\label{sec:discrete_time}

Let us take $t \in [0, T]$ and fix $\tau = \frac{T}{N}, N \in \N$ for some $N >> 1$. Assume that using some discrete scheme, we are able to calculate $\tensorq{p}$ at a certain time. Then we must solve 
\begin{equation*}
    \begin{cases}
	    0 \in \pdv{w\qty(\tensorq{e}-\tensorq{p})}{\tensorq{p}} + \partial D\qty(\dot{\frac{\tensorq{p}-\tensorq{p}_{k-1}}{\tau}}), & \, \text{in} \, \Omega \, \\
	    - \divergence{\qty(\mathcal{C}\qty(\tensorq{e}_k-\tensorq{p}_k))}= \vb{f}_k, &\, \text{in} \,\Omega \\
    \end{cases}.
\end{equation*}

Which are the E-L equations of the functional \footnote{We have guessed it.}
\[
	I\qty(\vb{u}, \tensorq{p}) = \int_{\Omega}w\qty(\tensorq{e}\qty(\vb{u})-\tensorq{p})\dd{x} + \tau \int_{\Omega}D\qty(\frac{1}{\tau}\qty(\tensorq{p}-\tensorq{p}_{k-1}))\dd{x} - \int_{\Omega}\vb{f}_k \vdot \vb{u}\dd{x} - \int_{\Gamma_N}\vb{g}_k \vdot u\dd{S}.
\]
Really, taking the variaton with respect to $\vb{u}$ gives us
\[
	- \divergence{\qty(\mathcal{C}\qty(\tensorq{e}_k -\tensorq{p}_k))} = \vb{f}_k,
\]
and the variation with respect to $\tensorq{p}$ gives us
\[
0 \in	- \tensorq{\sigma}+ \partial D\qty(\frac{1}{\tau}\qty(\tensorq{p}-\tensorq{p}_{k-1})).
\]

If we want to minimize this functional, \textit{i.e.}, solve the equations, it must hold \footnote{If not, we have no chance of minimizing it.} $D(\tensorq{q}) \neq +\infty$ (for $\tensorq{q}$ being the argument). From our assumptions on the dissipation potential this however implies.
\[
	D\qty(\tensorq{q}) = c_0 |\tensorq{q}|, \tr \tensorq{q} = 0,
\]
and we say the evolution is \textbf{rate-independent}. We see that $D$ is 1-homogenous:
\[
	D\qty(\alpha \tensorq{q}) = \alpha D\qty(\tensorq{q}).
\]
Rewritting the functional now yields:

\[
	I\qty(\vb{u}, \tensorq{p}) = \frac{1}{2} \int_{\Omega}\mathcal{C}\qty(\tensorq{e}\qty(\vb{u})-\tensorq{p}):\qty(\tensorq{e}\qty(\vb{u})-\tensorq{p})\dd{x} + \int_{\Omega}c_0 |\tensorq{p}-\tensorq{p}_{k-1}| \dd{x} - L_k\qty(\vb{u}), \tensorq{p}\qty(0) = \tensorq{p}_0,
\]
where $L_k\qty(\vb{u})$ is the loading (at the $k$-th time step.) The sought solution is the pair $(\vb{u}_k,\tensorq{p}_k)$ which satisfies

\[
	I\qty(\vb{u}_k, \tensorq{p}_k) = \min_{\vb{u}, \tensorq{p}} I\qty(\vb{u},\tensorq{p}).
\]


\subsection{Rheological models}
\label{sec:rheology}

\subsubsection{Dashpots}
\label{sec:dashpot}
Or \textit{tlumič} in Czech. The stress is assumed to take the form

\[
	\tensorq{\sigma} = \mathcal{D}\dot{\tensorq{e}}\qty(\grad \vb{u}), \sigma_{ij} = D_{ijkl}\dot{e}_{kl}\qty(\grad \vb{u}),
\]
where $\mathcal{D}$ is the \textbf{tensor of viscosity constants.} \footnote{People say viscosity stresses or viscous stress. This is used, but nonetheless it is wrong.}

\subsubsection{Kelvin-Voigt mateial}
\label{sec:kelvin-voigt}
The response of some materials can be modelled as a "parallel composition of a spring and a dashpot." Then, the total stress is
\[
	\tensorq{\sigma} = \tensorq{\sigma}_{p} + \tensorq{\sigma}_{e},
\]
that is the sum of the plastic and the elastic stresses. The strain is of course the same:

\[
	\tensorq{e} = \tensorq{e}_{p} = \tensorq{e}_{e}.
\]

The governing equations thus are

\begin{align*}
	- \divergence{\qty(\mathcal{C} \tensorq{e}\qty(\vb{u})+ \mathcal{D}\dot{\tensorq{e}}\qty(\vb{u}))} &= \vb{f}, \, \, \text{in} \,\Omega\\
	\qty(\mathcal{C}\tensorq{e}+ \mathcal{D} \dot{\tensorq{e}})\vb{n} &= \vb{0}, \, \, \text{on} \, \Gamma_{N}\\
	\vb{u} &= \vb{0}, \, \, \text{on} \,\Gamma_D \\
	\tensorq{e}\qty(t = 0) &= \tensorq{e}_0, \, \, \text{in} \, \Omega.
\end{align*}

Let us obtain the energy \textit{formally} balance. As usual, multiply the first equation by $\dot{\vb{u}}$ and integrate $\int_{\Omega}\dd{\vb{x}}.$

\[
	\int_{\Omega}- \divergence{\qty(\mathcal{C}\tensorq{e}+ \mathcal{D}\dot{\tensorq{e}})} \vdot \dot{\vb{u}}\dd{x} = \int_{\Omega}\vb{f} \vdot \dot{\vb{u}}\dd{x},
\]

using Gauss 

\[
	\int_{\Omega}\qty(\mathcal{C} \tensorq{e}+ \mathcal{D} \dot{\tensorq{e}}): \grad \dot{\vb{u}}\dd{x} = \int_{\Omega}\vb{f} \vdot \dot{\vb{u}}\dd{x} =\footnote{It holds $\dot{\tensorq{e}}\qty(\vb{u}) = \tensorq{e}\qty(\dot{\vb{u}}).$} \int_{\Omega}\mathcal{C}\tensorq{e}:\dot{\tensorq{e}}\dd{x} + \int_{\Omega}\mathcal{D}\dot{\tensorq{e}}:\dot{\tensorq{e}}\dd{x} = \int_{\Omega}\vb{f}\vdot \vb{u}\dd{x},
\]
and now we rewrite 
\[
	= \int_{\Omega}\dv{t} \qty(\frac{1}{2} \mathcal{C} \tensorq{e}\qty(\vb{u}):\tensorq{e}\qty(\vb{u}))\dd{x} + \int_{\Omega}\mathcal{D} \dot{\tensorq{e}}\qty(\vb{u}): \dot{\tensorq{e}}\qty(\vb{u})\dd{x} = \int_{\Omega}\vb{f} \vdot \dot{\vb{u}}\dd{x},
\]
and integrate in time:
\[
	\int_0^T\int_{\Omega}\dv{t} \qty(\frac{1}{2} \mathcal{C} \tensorq{e}\qty(\vb{u}):\tensorq{e}\qty(\vb{u}))\dd{x}\dd{t} + \int_0^T\int_{\Omega}\mathcal{D} \dot{\tensorq{e}}\qty(\vb{u}): \dot{\tensorq{e}}\qty(\vb{u})\dd{x}\dd{t} =\int_0^T \int_{\Omega}\vb{f} \vdot \dot{\vb{u}}\dd{x} \dd{t}.
\]
Remember that
\[
	w\qty(\tensorq{e}\qty(\vb{u})) = \frac{1}{2}\mathcal{C}\tensorq{e}\qty(\vb{u}):\tensorq{e}\qty(\vb{u}),
\]
so we have obtained

\[
	\int_{\Omega}w\qty(\tensorq{e}\qty(\vb{u}(T)))\dd{x} - \int_{\Omega}w\qty(\tensorq{e}\qty(\vb{u}\qty(0)))\dd{x} + \int_0^T \int_{\Omega}\mathcal{D} \dot{\tensorq{e}}\qty(\vb{u}):\dot{\tensorq{e}}\qty(\vb{u})\dd{x} \dd{t} = \int_0^T \int_{\Omega}\vb{f} \vdot \dot{\vb{u}}\dd{x} \dd{t}.
\]
\subsubsection{Maxwell material}
\label{sec:maxwell}
This is the case when we "put the spring and the dashpot in serial composition". The total stress is
\[
	\tensorq{\sigma} = \tensorq{\sigma}_p = \tensorq{\sigma}_e,
\]
and the total strain is
\[
	\varepsilon = \tensorq{e}_{p}+ \tensorq{e}_{e}.
\]


\subsection{Internal parameters}
\label{sec:internal_parameters}
A lot of materials can be described using some internal parameters $\tensorq{z}$ (scalars, vectos, tensors; we take the tensor case for generality); for example, plastic strain, fatique, damage, length of a crack, delamination.

The model 
\[
	\tensorq{\sigma} = \partial_{\dot{\tensorq{e}}} \zeta\qty(\dot{\tensorq{e}}, \dot{\tensorq{z}}) + \partial_{\tensorq{e}}w\qty(\tensorq{e}, \tensorq{z}),
\]
with the flow rule 
\[
	0 \in \partial_{\dot{\tensorq{z}}} \zeta\qty(\dot{\tensorq{e}}, \dot{\tensorq{z}}) + \partial_{\tensorq{z}}\varphi\qty(\tensorq{e},\tensorq{z}).
\]
is called the \textbf{generalized Kelvin-Voigt} model/material. From now on, we will be using $\varphi$ for the stored energy density. There is some analogy:
\begin{itemize}
	\item $\varphi$ is the stored energy density $=$ potential of stress
	\item $\zeta$ is the (pseudo)potential of dissipative forces.
\end{itemize}
To do anything, we need to obtain some energy balance, so test by $\dot{\vb{u}}.$
Investigate the terms:
\[
	\tensorq{\sigma} : \dot{\tensorq{e}} = \partial_{\dot{\tensorq{e}}}\zeta\qty(\dot{\tensorq{e}}, \dot{\tensorq{z}}): \dot{\tensorq{e}}+ \partial_{\tensorq{e}}\varphi\qty(\tensorq{e},\tensorq{z}):\dot{\tensorq{e}},
\]
realize now that from the flow rule it follows
\[
	\qty(\partial_{\dot{\tensorq{z}}}\zeta\qty(\dot{\tensorq{e}},\dot{z})+\partial_{\tensorq{z}}\varphi\qty(\tensorq{e},\tensorq{z})):\dot{\tensorq{z}} =0,
\]
so i can add it to the previous term and obtain
\[
	\tensorq{\sigma} : \dot{\tensorq{e}} = \partial_{\dot{\tensorq{e}}}\zeta\qty(\dot{\tensorq{e}}, \dot{\tensorq{z}}): \dot{\tensorq{e}}+ \partial_{\tensorq{e}}\varphi\qty(\tensorq{e},\tensorq{z}):\dot{\tensorq{e}} + \partial_{\dot{\tensorq{z}}}\zeta\qty(\dot{\tensorq{e}},\dot{\tensorq{z}}):\dot{\tensorq{z}}+\partial_{\tensorq{z}}\varphi\qty(\tensorq{e},\tensorq{z}):\dot{\tensorq{z}} =0,
\]
Realize now that we have obtained
\[
	\partial_{\tensorq{e}} \varphi\qty(\tensorq{e},\tensorq{z})\dot{\tensorq{e}} + \partial_{\tensorq{z}}\varphi\qty(\tensorq{e},\tensorq{z}) = \dv{t} \varphi\qty(\tensorq{e},\tensorq{z}),
\]
and denoting the quantity
\[
	\xi\qty(\dot{\tensorq{e}}, \dot{\tensorq{z}})\coloneq \partial_{\dot{\tensorq{e}}}\zeta\qty(\dot{\tensorq{e}}, \dot{\tensorq{z}}): \dot{\tensorq{e}} + \partial_{\dot{\tensorq{z}}}\zeta\qty(\dot{\tensorq{e}}, \dot{\tensorq{z}}):\dot{\tensorq{z}},
\]
as the \textit{rate of the dissipation} we obtain
\[
	\tensorq{\sigma}: \dot{\tensorq{e}} = \dv{t} \varphi\qty(\tensorq{e},\tensorq{z})+ \xi\qty(\dot{\tensorq{e}}, \dot{\tensorq{z}}).
\]

What are the properties of $\xi$? First of all, we require
\[
	\xi \geq 0.
\]
Assume $\zeta$ is a covex function:

\[
	\zeta\qty(0,0) \geq \zeta\qty(\dot{\tensorq{e}}, \dot{\tensorq{z}})+ \partial_{\dot{\tensorq{e}}} \zeta\qty(\dot{\tensorq{e}}, \dot{\tensorq{z}}):\qty(- \dot{\tensorq{e}})+ \partial_{\dot{\tensorq{z}}}\zeta\qty(\dot{\tensorq{e}}, \dot{\tensorq{z}}):(-\dot{\tensorq{z}}).
\]
Moreover, assume now $\zeta\qty(0,0) = 0.$ We have
\[
	\xi\qty(\dot{\tensorq{e}}, \dot{\tensorq{z}}) = \partial_{\dot{\tensorq{e}}}\zeta\qty(\dot{\tensorq{e}}, \dot{\tensorq{z}}): \dot{\tensorq{e}} + \partial_{\dot{\tensorq{z}}}\zeta\qty(\dot{\tensorq{e}}, \dot{\tensorq{z}}):\dot{\tensorq{z}} \geq \zeta\qty(\dot{\tensorq{e}}, \dot{\tensorq{z}}) \geq 0.
\]

Finally, the total power balance becomes


\[
	\int_{\Omega}\frac{1}{2}\dv{t} \rho |\dot{\vb{u}}|^{2}\dd{x} + \int_{\Omega}\dv{t} \varphi\qty(\tensorq{e}, z)\dd{x} + \int_{\Omega}\xi\qty(\dot{\tensorq{e}}, \dot{z})\dd{x} = \int_{\Omega}\vb{f} \vdot \dot{\vb{u}}\dd{x},
\]
and the total energy balance becomes

\[
\int_{\Omega}\frac{1}{2}\rho\qty(|\dot{\vb{u}}(T)|^{2}-|\dot{\vb{u}}(0)|^{2})\dd{x} + \int_{\Omega}(\varphi\qty(\tensorq{e}(T), z(T)) - \varphi\qty(\tensorq{e}(0), z(0)))\dd{x} + \int_0^T \int_{\Omega}\xi\qty(\dot{\tensorq{e}}, \dot{z})\dd{x} \dd{t} = \int_0^T \int_{\Omega}\vb{f} \vdot \dot{\vb{u}}\dd{x} \dd{t}.
\]


\section{Thermodynamics in the framework of GSM (generalized standard materials)}
\label{sec:thermo}
\textit{Having obtained some knowledge of thermodynamical quantites, we are ready to generalize the theory. We will see that the evolution of a specimen can be acquired by the knowledge of the stored energy density $\psi$ and the dissipation "potential" $\zeta$}


Denote
\[
	\psi = \psi\qty(\tensorq{e}, \tensorq{z}, \theta), \zeta = \zeta\qty(\dot{\tensorq{e}}, \dot{\tensorq{z}})
\]
to be \textit{the stored energy} and the \textit{dissipation potential}. Here $\theta >0 $ denotes the absolute thermodynamic temperature. Let us denote 

\[
\tensorq{\sigma}_{el} = \pdv{\psi}{\tensorq{e}}, \tensorq{\sigma}_{in} = \pdv{\psi}{\tensorq{z}}, s = - \pdv{\psi}{\theta},
\]
as the elastic and inelastic stress and the entropy density. Moreover, define
\[
	w\qty(\tensorq{e},\tensorq{z},\theta,s) = \psi\qty(\tensorq{e},\tensorq{z},\theta)+\theta s
\]
as the \textbf{internal energy density}. If we calculate the time derivative of the internal energy density we obtain:
\[
	\dot{w} = \pdv{t}\qty(\psi\qty(\tensorq{e},z,\theta)+\theta s) = \pdv{\psi}{\tensorq{e}}: \dot{\tensorq{e}} + \pdv{\psi}{\tensorq{z}}:\dot{\tensorq{z}} + \underbrace{\pdv{\psi}{\theta}\dot{\theta} + \dot{\theta} s}_{=-s \dot{\theta} + \dot{\theta} s = 0} + \theta \dot{s}.
\]
We \textit{postulate}:
\[
	\dot{w} = \tensorq{\sigma}_{el}: \dot{\tensorq{e}} + \tensorq{\sigma}_{in}:\dot{\tensorq{z}} + \xi\qty(\dot{\tensorq{e}}, \dot{\tensorq{z}}) - \divergence{\vb{j}},
\]
where $\vb{j}$ is the heat flux. From this postulate, we obtain

\begin{equation}
	\label{eq:dissipation}
	\xi\qty(\dot{\tensorq{e}}, \dot{\tensorq{z}}) = \theta \dot{s} + \divergence{\vb{j}}.
\end{equation}
A common modelling choice is the dependency

\[
	\vb{j} = \vb{j}\qty(\theta, \tensorq{e}, \tensorq{z}, \grad \theta) = - \tensorq{K}\qty(\tensorq{e}, \tensorq{z}, \theta)\grad \theta,
\]
known as the \textit{Fourier law}. Here
\[
	\tensorq{K} \in \{ \tensorq{A} \in \R^{3 \times 3}| \tensorq{A} >0\},
\]
is the \textit{matrix of heat flux coefficients.} This is a classical example of a constitutive law.


\begin{align*}
	\dv{t} \int_{\Omega}s\qty(t,\vb{x})\dd{\vb{x}} &= \int_{\Omega}\frac{1}{\theta}\qty(\xi-\divergence{\vb{j}})\dd{\vb{x}} = \int_{\Omega}\frac{\xi}{\theta}\dd{\vb{x}} + \int_{\Omega}\frac{\divergence{\qty(\tensorq{K} \grad \theta)}}{\theta}\dd{\vb{x}} = \\
						       & = \int_{\partial \Omega}\frac{\tensorq{K}\grad \theta}{\theta} \vdot \vb{n}\dd{S} - \int_{\Omega}\tensorq{K} \grad \theta \vdot \grad\qty(\frac{1}{\theta})\dd{\vb{x}}+\int_{\Omega}\frac{\xi}{\theta}\dd{\vb{x}} = \\
						       & = \int_{\Omega}\qty(\frac{\xi}{\theta}+ \frac{\tensorq{K}\grad \theta \vdot \grad \theta}{\theta^{2}})\dd{\vb{x}}-\int_{\partial \Omega}\frac{\vb{j}}{\theta} \vdot \vb{n}\dd{S}.
\end{align*}
This relation is known as the \textit{Clausius-Duhem inequality.}\footnote{Although inequalty, there appears only the equality sign "=". I do not actually know what that means.}

From the definition of $s$

\[
	s = - \pdv{\psi}{\theta}\qty(\theta, \tensorq{e}, \tensorq{z}),
\]
it follows
\[
	\dot{s} = -\pdv[2]{\psi}{\theta} \theta - \pdv[2]{\psi}{\theta}{\tensorq{e}}:\dot{\tensorq{e}} - \pdv[2]{\psi}{\theta}{\tensorq{z}}:\dot{\tensorq{z}},
\]
and so
\[
	\theta \dot{s} = \underbrace{- \pdv[2]{\psi}{\theta} \theta}_{\coloneq C_V} \dot{\theta} - \pdv[2]{\psi}{\theta}{\tensorq{e}}:\qty(\dot{\tensorq{e}} \theta) - \pdv[2]{\psi}{\theta}{\tensorq{z}}:\qty(\dot{\tensorq{z}}\theta) = C_V \dot{\theta} - \pdv[2]{\psi}{\theta}{\tensorq{e}}:\qty(\dot{\tensorq{e}}\theta)- \pdv[2]{\psi}{\theta}{\tensorq{z}}:\qty(\dot{\tensorq{z}}\theta),
\]
where we have identified
\[
	C_V = - \theta \pdv[2]{\psi}{\theta},
\]
as the \textit{heat capacity at the constant volume.} Coming back to \ref{eq:dissipation}, we read

\[
	C_V \dot{\theta} + \divergence{\vb{j}} = \xi\qty(\dot{\tensorq{e}}, \dot{\tensorq{z}}) + \theta \pdv[2]{\psi}{\theta}{\tensorq{e}}: \dot{\tensorq{e}} + \theta \pdv[2]{\psi}{\theta}{\tensorq{z}}: \dot{\tensorq{z}}.
\]
This is our \textit{heat equation,} the right hand side are the sources. We could identify the derivatives of the potential with lets say some derivative of $\tensorq{\sigma}_{el}$, but let us keep the "thermodynamics and mechanics separated."; although it does not really make sense. In total

\begin{align*}
	C_V \dot{\theta} - \divergence{\qty(\tensorq{K} \grad \theta)} & = \xi + \theta \pdv[2]{\psi}{\theta}{\tensorq{e}}:\dot{\tensorq{e}}+ \theta \pdv[2]{\psi}{\theta}{\tensorq{z}}:\dot{\tensorq{z}}, \\
	\rho \ddot{\vb{u}} - \divergence{\qty(\tensorq{\sigma}_{el}+ \tensorq{\sigma}_{in})} &= \vb{f}, \\
	0 &\in \partial_{\dot{\tensorq{z}}}\zeta\qty(\dot{\tensorq{e}}, \dot{\tensorq{z}})+ \partial_{\tensorq{z}}\psi\qty(\tensorq{e}, \tensorq{z}, \theta),
\end{align*}
plus of course some initial and boundary conditions.

\section{Summary}
\label{sec:summary}
\textit{At the end, the lecture is summarized.}


It began with deformation:
\[
	\vb{y}: \overline{\Omega} \to \R^{3}, \grad \vb{y} = \fgrad, \rcg = \transpose{\fgrad}\fgrad, \detf >0.
\]
and some quantities associated with these. A little excursion allowed as to define
\[
	W = W\qty(\grad \vb{y}) = W\qty(\fgrad),
\]
to be the stored energy density. Note that later on, we have called it $\psi$. 
Coming back to deformation, we have defined various stress measures:
\[
	\cstress, \pkstress = \cstress \cof \fgrad, \spkstress =\inverse{\fgrad} \pkstress.
\]

Wanting to show existence of solutions, we needeed the convexity of some functionals. A problem with rotations however meant we needed to lower our expectations and we had to discover polyconvexity and rank-1 convexity. This included \textit{e.g.} Legendre-Hadamard condition.

Realizing we are stuck in full theory, we began exploring linearized elasticity. To show existence, we refreshed the Korn's inequality. And because that all seemed easy, a question about time dependence has been asked: is everything truly stationary?

No, it is not; that lead us to von Mises elastoplasticity and to a class of materials, such as Kelvin-Voigt or Maxwell materials. Generalizing this framework and also including some internal variables, we have given the foundations of (the thermodynamics of)gave generalized standard materials: this was especially elegant, as from the Helmholtz free energy and the dissipation potential, we were able to derive evolution equations for the important thermodynamical quantites. This included some energy/power estimates, balances and the notion of entropy and its rate. 

\section{(Some) tutorials}
\label{sec:tutorials}

\subsection{Change of observer}
\label{sec:chobserver}

The requirement of material frame indifference yields:
\[
	W = \hat{W}\qty(\fgrad) = \hat{W}\qty(\tensorq{Q}\fgrad), \forall \tensorq{Q} \in \, \text{orth} \,.
\]

\subsection{Change of reference configuration}
\label{sec:chreference}
The requiremenent of material symmetry yields:
\[
	W= \hat{W}\qty(\fgrad)= \hat{W}\qty(\fgrad \tensorq{P}), \forall \tensorq{P} \in \mathcal{G},
\]
where $\mathcal{G}$ is the symmetry group of the material.

\subsection{Consequences of isotropic hyperelastic solid}
\label{sec:conshypelisosolid}

\begin{remark}[Groups unim, orth]
	The "biggest sensible" symmetry group is the unimodular group:
	\[
		\, \text{unim} \, = \{\tensorq{P}, \det \tensorq{P} = \pm 1\}.
	\]
	There exists another common group:
	\[
		\, \text{orth} \, \{ \tensorq{Q}, \tensorq{Q}\transpose{\tensorq{Q}} = \transpose{\tensorq{Q}} \tensorq{Q} = \identity\} \subset \, \text{unim} \,.
	\]
\end{remark}
We thus have $W = \hat{W}\qty(\fgrad) = \hat{W}\qty(\tensorq{Q}\fgrad) = \hat{W}\qty(\fgrad \tensorq{Q}), \forall \tensorq{Q} \in \, \text{orth} \,, \forall \fgrad.$

Use \textit{polar decomposition}: $\tensorq{F} = \tensorq{R}\tensorq{U} = \tensorq{V}\tensorq{R}, \tensorq{R} \in \, \text{orth} \,, \tensorq{U,V} \, \text{positively definite} \,, \tensorq{U} = \sqrt{\tensorq{C}}, \tensorq{V} = \sqrt{\tensorq{B}}.$ 

To make use of it, we first use m.f.i. and then isotropy and then first use isotropy and then m.f.i.
So from material frame indifference
\[
	W = \hat{F}\qty(\fgrad) = \hat{W}\qty(\tensorq{Q}\fgrad) = \hat{W}\qty(\transpose{\tensorq{R}} \tensorq{R} \tensorq{U}) = \hat{W}\qty(\tensorq{U}) = \overline{W}\qty(\rcg),
\]
where we have taken $\tensorq{Q} = \transpose{\tensorq{R}}.$ Note that this works universaly (without the need of isotropy), as it comes from the objectivity consideration.

From isotropy
\[
	W = \hat{W}\qty(\tensorq{F}) = \hat{W}\qty(\fgrad \tensorq{Q}) = \overline{W}\qty(\rcg) = \overline{W}\qty(\transpose{\qty(\fgrad \tensorq{Q})} \qty(\fgrad \tensorq{Q})) = \overline{W}\qty(\transpose{\tensorq{Q}} \transpose{\fgrad}\fgrad \tensorq{Q}) = \overline{W}\qty(\transpose{\tensorq{Q}}\rcg \tensorq{Q}), \forall Q \in \, \text{orth} \,, \forall \rcg \, \text{admissable} \,.
\]
Now backwards using the second polar decomposition:
\[
	W = \hat{W}\qty(\fgrad) = \hat{W}\qty(\fgrad \tensorq{Q}) = \hat{W}\qty(\tensorq{V}\tensorq{R}\tensorq{Q}) = \hat{W}\qty(\tensorq{V}\tensorq{R}\transpose{\tensorq{R}}) = \hat{W}\qty(\sqrt{\lcg}) = \tilde{W}\qty(\lcg),
\]

\[
	W = \tilde{W}\qty(\lcg) = \tilde{W}\qty(\tensorq{Q}\fgrad \transpose{\qty(\tensorq{Q}\fgrad)}) = \tilde{W}\qty(\tensorq{Q}\lcg \transpose{\tensorq{Q}}).
\]

So far, we have shown
\begin{align*}
	W\qty(t,\vb{X}) &= \overline{W}\qty(\rcg\qty(t,\vb{X}),\vb{X}) = \overline{W}\qty(\tensorq{Q}\rcg \transpose{\tensorq{Q}}), \\ W\qty(t,\vb{X}) &= \tilde{W}\qty(\lcg(t,\vb{X}), \vb{X}) = \tilde{W}\qty(\tensorq{Q} \lcg \transpose{\tensorq{Q}}),
\end{align*}

In HW, we will know
\[
	\pkstress = 2 \pdv{\hat{W}\qty(\fgrad)}{\fgrad}, T \indices{_{iJ}} = 2 \pdv{\hat{W}}{F \indices{_{iJ}}}
\]
and we can show
\[
	\pkstress = 2\pdv{\hat{W}\qty(\fgrad)}{\fgrad} = 2 \fgrad \pdv{\tilde{W}\qty(\lcg)}{\lcg}\fgrad, \cstress  = \dots, \tensorq{S} = 2 \pdv{\overline{W}\qty(\rcg)}{\rcg}.
\]

\begin{definition}[Isotropic functions]
	We say the functions $\hat{a}\qty(y_{\alpha}, \vb{y}_{\alpha}, \tensorq{Y}_{\alpha}), \vb{\hat{a}}\qty(y_{\alpha}, \vb{y}_{\alpha}, \tensorq{Y}_{\alpha}), \hat{\tensorq{A}}\qty(y_{\alpha},\vb{y}_{\alpha},\tensorq{Y}_{\alpha}), \alpha = 1, \dots, N$ are isotropic functions (of their respective arguments) if it holds
	\begin{align*}
		\hat{a}\qty(y_{\alpha}, \vb{y}_{\alpha}, \tensorq{Y}_{\alpha}) &= \hat{a}\qty(y_{\alpha}, \tensorq{Q}\vb{y}_{\alpha}, \tensorq{Q} \tensorq{Y}_{\alpha} \transpose{\tensorq{Q}}), \\
		\tensorq{Q} \vb{\hat{a}}\qty(y_{\alpha}, \vb{y}_{\alpha}, \tensorq{Y}_{\alpha}) &= \vb{\hat{a}}\qty(y_{\alpha}, \tensorq{Q}\vb{y}_{\alpha}, \tensorq{Q}\tensorq{Y}_{\alpha}\transpose{\tensorq{Q}}), \\
		\tensorq{Q}\hat{\tensorq{A}} \transpose{\tensorq{Q}} &= \hat{\tensorq{A}}\qty(y_{\alpha}, \tensorq{Q}\vb{y}_{\alpha}, \tensorq{Q}\tensorq{Y}_{\alpha}\transpose{\tensorq{Q}}), \\.
	\end{align*}
\end{definition}
So we see that $\overline{W}\qty(\rcg), \tilde{W}\qty(\lcg)$ are \textbf{scalar isotropic functions of 1 tensorial (symmetric) argument.}

\begin{theorem}[Representation theorem for scalar isotropic functions]
	Let $\psi = \hat{\psi}\qty(\tensorq{A}) = \hat{\psi}\qty(\tensorq{Q}\tensorq{A}\transpose{\tensorq{Q}})$ be a scalar isotropic function of a single symmetric tensorial variable. Then it must hold
	\[
		\hat{\psi}\qty(\tensorq{A}) \equiv \hat{\psi}\qty(\invI\qty(\tensorq{A}),\invII\qty(\tensorq{A}),\invIII\qty(\tensorq{A})),
	\]	
where
\begin{align*}
	\invI\qty(\tensorq{A}) &= \tr \tensorq{A},\\
	\invII\qty(\tensorq{A}) &= \frac{1}{2}\qty(\qty(\tr \tensorq{A})^{2}-\tr \tensorq{A}^{2}),\\
	\invIII\qty(\tensorq{A}) &= \det \tensorq{A},
\end{align*}
are the invariants of $\tensorq{A}$.
\end{theorem}

\begin{proof}
    $\det\qty(\tensorq{A}-\lambda \identity) = -\lambda^{3} + \lambda^{2} \invI - \lambda \invII + \invIII = p_{\lambda}\qty(\tensorq{A})$
    We will prove a different assertion:

    $\tensorq{A}, \tensorq{B}$ are symmetric with the same invariants $\Leftrightarrow$  $\exists \tensorq{Q}: \tensorq{A} = \tensorq{Q} \tensorq{B} \transpose{\tensorq{Q}}$ 
    $"\Leftarrow"$ is trivial, as then the matrices are similliar, so they have the same char. polynomial, so they have the same invariants.
    $\Rightarrow$ have same eigenvalues, so if i write the spectral decomposiiton, i can write
    \[
	    \tensorq{A} = \tensorq{Q} \tensorq{\Lambda} \transpose{\tensorq{Q}}, \tensorq{B} = \tensorq{Q} \tensorq{\Lambda}\transpose{\tensorq{R}} = \tensorq{R}\transpose{\tensorq{Q}} \tensorq{A} \tensorq{Q}\transpose{\tensorq{R}}.
    \]
    Now suppose that the function is not a function of the invariants: $\hat{\psi} \neq \tilde{\psi}\qty(\invI,\invII,\invIII).$ That means $\exists \tensorq{A}_,, \tensorq{A}_2 \, \text{such that} \, \invI\qty(\tensorq{A}_1) = \invI\qty(\tensorq{A}_2)$ and the same for the remaining invariants. Using the previous assertion, we have
    \[
	    \exists \tensorq{Q}: \tensorq{A}_1 = \tensorq{Q}\tensorq{A}_2 \transpose{\tensorq{Q}} \Rightarrow \hat{\psi}\qty(\tensorq{A}_1) = \hat{\psi}\qty(\tensorq{Q}\tensorq{A}_2 \transpose{\tensorq{Q}}) = \hat{\psi}\qty(\tensorq{A}_2), \, \text{but} \, \hat{\psi}\qty(\tensorq{A}_1) \neq \hat{\psi}\qty(\tensorq{A}_2).
    \]
\end{proof}
Since using polar decomposition it can be shown the invariants of $\lcg, \rcg$ are the same we recieve
\[
	W = \tilde{W}(\invI\qty(\lcg), \invII\qty(\lcg), \invIII\qty(\lcg)) = \overline{W}\qty(\invI\qty(\rcg), \invII\qty(\rcg), \invIII\qty(\rcg)).
\]

\subsection{Representation in terms of principial stresses}
\label{sec:representation_principial}
... in terms of the eigenvalues $\tensorq{U},\tensorq{V}.$ The invariants can be expressed as
\begin{align*}
	\invI &= \lambda_1+\lambda_2 + \lambda_3, \\
	\invII &= \lambda_1 \lambda_2 + \lambda_2 \lambda_3 + \lambda_1 \lambda_3,\\
	\invIII &= \lambda_1 \lambda_2 \lambda_3.
\end{align*}
Often in materials science the quantites can be expressed in these variables:

\begin{example}[Ogden materials]
	\[
		\hat{W}\qty(\lambda_1,\lambda_2,\lambda_3) = \sum_{k=1}^n \frac{\mu_k}{\alpha_k}\qty(\lambda_1^{\alpha_k}+\lambda_2^{\lambda_k}+\lambda_3^{\alpha_k}-3)
	\]
\end{example}
How to calculate e.g. $\pkstress$ in this representation?
\[
	\pkstress = 2 \pdv{W\qty(\invI, \invII, \invIII)}{\lcg}\fgrad = 2 \pdv{\hat{W}}{\lcg}\qty(\lambda_1\qty(\lcg), \lambda_2\qty(\lcg), \lambda_3\qty(\lcg)) 2 \pdv{\hat{W}}{\lambda_i}\pdv{\lambda_i\qty(\lcg)}{\lcg}\fgrad.
\]

What (in the hell) is $\pdv{\lambda_i\qty(\lcg)}{\lcg}$? \footnote{Recall the Daleckii-Krein theorem:}

\[
	\lcg(s) = \sum_{\alpha = 1}^3 \omega_{\alpha}(s) \vb{g}_{\alpha}(s) \otimes \vb{g}_{\alpha}(s), \forall s \in I
\]
where $I$ is some open interval and $\{\vb{g}_{\alpha}\}$ is an ON eigenbasis of $\lcg.$ Next, realize
\[
	\omega_1(s) = \vb{g}_1(s)\vdot \lcg(s) \vb{g}_1(s),
\]
and differentiate this:
\[
	\dv{\omega(s)}{s} = \dv{\vb{g}_1}{s}\vdot \lcg \vb{g}_1  + \vb{g}_1 \dv{\lcg}{s}\vb{g}_1+\vb{g}_1 \vdot \lcg \dv{\vb{g}}{s} = \frac{1}{2} + +0.
\]

\bibliographystyle{chicago} %\bibliography{}
\end{document}

%%% Local Variables: 
%%% mode: latex
%%% TeX-master: t
%%% End: 
