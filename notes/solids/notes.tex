\documentclass[reqno, a4paper]{article}
\usepackage{amsmath}
\usepackage{amssymb}
\usepackage{amsthm}

% PAGE DIMENSION

% BIBLIOGRAPHY
\usepackage{natbib}
\usepackage{bibentry} % inline refereces

% ENCODING, LANGUAGE
\usepackage[english]{babel}
\usepackage[utf8]{inputenc}

% GRAPHICS
\usepackage{subfig}
\usepackage{graphicx}
\usepackage{tikz}

% HYPERTEXT, SOURCE CODE SPECIALS
\usepackage[unicode]{hyperref}
\usepackage[active]{srcltx} % use TeX-souce-specials-mode

% SYMBOLS, FONTS
\usepackage{mathbbol}
\usepackage{bm} % sophisticated \boldsymbol
\usepackage{physics}
%\usepackage{stmaryrd}
\usepackage{MnSymbol} % \lsem, \rsem, tensor product :
\usepackage{gensymb}
\usepackage{eurosym}

% UNITS, TYPESETTING TENSORS
\usepackage{units}
\usepackage{tensor}
\usepackage{accents}

% COMPACT LIST ENVIRONMENT
\usepackage{enumitem}

% LINE NUMBERS
\usepackage{lineno}

\usepackage{multicol}

% SELECTIVELY INCLUDE/EXCLUDE PARTS OF TEXT
\usepackage{comment}

% FLOAT BARRIER
\usepackage{placeins}

%% User macros
\DeclareMathOperator{\sym}{sym}
\DeclareMathOperator{\asym}{asym}
\DeclareMathOperator{\cof}{sym}
\DeclareMathOperator{\laplace}{\bigtriangleup}
\newcommand{\tensorq}[1]{\ensuremath{\mathbb{#1}}}      % tensorial quantity
\newcommand{\identityM}{\ensuremath{\tensorq{I}}}
\newcommand{\fgrad}{\tensorq{F}}
\newcommand{\detf}{\det\, \fgrad}
\newcommand{\vgrad}{\tensorq{L}}
\newcommand{\symvgrad}{\tensorq{D}}
\newcommand{\asymvgrad}{\bbomega}
\newcommand{\cstress}{\tensorq{T}}
\newcommand{\pkstress}{\tensorq{T}_{\mathrm{R}}}
\newcommand{\spkstress}{\tensorq{S}_{\mathrm{R}}}
\newcommand{\rcg}{\tensorq{C}}
\newcommand{\lcg}{\tensorq{B}}
% \mdv[<species>]{<quantity>}:
% - \mdv                -> \dfrac{\mathrm{D}}{\mathrm{D}t}
% - \mdv{v}             -> \dfrac{\mathrm{D} v}{\mathrm{D}t}
% - \mdv[a]{v}          -> \dfrac{\mathrm{D}_{a} v}{\mathrm{D}t}
% - \mdv[a]             -> \dfrac{\mathrm{D}_{a}}{\mathrm{D}t}
\NewDocumentCommand{\mdv}{o g}{%
  \IfNoValueTF{#2}{% no braced arg -> operator form
    \IfValueTF{#1}
      {\dfrac{\mathrm{D}_{#1}}{\mathrm{D}t}}% species subscript on operator
      {\dfrac{\mathrm{D}}{\mathrm{D}t}}% plain operator
  }{% braced arg present -> numerator form
    \IfValueTF{#1}
      {\dfrac{\mathrm{D}_{#1} #2}{\mathrm{D}t}}% \mdv[a]{v}
      {\dfrac{\mathrm{D} #2}{\mathrm{D}t}}% \mdv{v}
  }%
}




\begin{document}

\title{Thermodynamics and mechanics of solids}

\date{\today}

\author{Kamil Belan}



\begin{comment}
\begin{abstract}
  % \input{article-abstract}
  Here comes the abstract.
\end{abstract}
\end{comment}

\maketitle

\tableofcontents
\section{Geometry}
\label{sec:geometry}

\subsection{Deformation}
\label{sec:deformation}

Suppose we are given an abstract body $\Omega \subset \R^d, d=2,3$. Choosing a particular state, we denote it the \textbf{reference configuration}. After the acting of forces, the body deforms into the \textbf{current, deformed configuration}. 

\begin{tikzpicture}
 \draw (0,0) .. controls (1,2) and (3,2) .. (3,0) 
                        .. controls (3,-2) and (1,-2) .. (0,-0.5)
                        .. controls (-1,-2) and (-2,-1) .. (-2,0)
                        .. controls (-2,1) and (-1,1) .. (0,0);
\node at (2,2) {$\Omega$};

 % Second blob (on the right)
\draw (7,0) .. controls (8,2) and (10,2) .. (11,0) 
                        .. controls (12,-2) and (6,-2) .. (7,-0.5)
                        .. controls (4,-5) and (1,0) .. (5,0)
                        .. controls (3,1) and (6,1) .. (7,0);

    % Label for the second blob
\node at (7,2) { $\vb{y}(\Omega)$};

    % Curved arrow connecting the two blobs
\draw[->, thick] (3,0) .. controls (4,1.5) and (4,1.5) .. (5,0);
\node at (4,1.5) {$\vb{y}:\overline{\Omega} \to \R^d$};
\end{tikzpicture}

The mapping that produces the deformation is called the \textbf{deformation}, denoted $\vb{y}$, i.e.
\[
	\vb{y}: \overline{\Omega} \to \R^d.
\]
Of large interest will be the \textbf{deformation gradient}
\[
	\fgrad(\vb{x}) = \grad{y}(\vb{x}), (\grad{y})_{ij}=\pdv{y^i}{x^j},
\]
on which we put some physically sound restrictions, such as$\det \fgrad> 0$. This means in particular that the determinant is nonzero, but also that preserves orientations of bases:
\[
(\vb{e}_1 \cross \vb{e}_2) \vdot \vb{e}_3 > 0 \Rightarrow(\fgrad\vb{e}_1 \cross \vb{e}_2) \vdot \fgrad \vb{e}_3 >0.
\]

\begin{example}
	Suppose the deformation is given as
	\[
		\vb{y}(\vb{x}) = \begin{bmatrix}-1 & 0 \\ 0 & 1\end{bmatrix} \vb{x},
	\]
i.e., $\fgrad=\begin{bmatrix}-1 & 0 \\ 0 & 1\ \end{bmatrix}, \detf = -1$. This is an example of a deformation that is \textit{forbidden}.

\begin{tikzpicture}
	\draw (-3,0) -- (3,0);
	\draw (0,0) -- (0,2);
	\draw (0,0) rectangle (1,1) node[above right]{$\Omega$};
	\draw[dashed] (0,0) rectangle (-1,1) node[above left] {$\vb{y}(\Omega)$};
\end{tikzpicture}
Imagine it is a sheet of paper in a plane - you cannot reflect it without lifting it from the plane.
\end{example}

\subsection{Displacement}
\label{sec:displacement}
Another useful way of describing the deformation is by using the \textbf{displacement vector} $\vb{u}$:
\[
	\vb{u}(\vb{x}) = \vb{y}(\vb{x}) - \vb{x},
\]
so taking the gradient gives
\[
	\grad{\vb{u}(\vb{x})} = \fgrad(\vb{x}) - \identity.
\]
\begin{remark}
It is interesting that we do not place any restrictions on the determinant of the displacement gradient.
\end{remark}

\subsection{Changes of measures}
\label{sec:changes}

We need to examine how the volume, area and lengths change under deformation. In what follows, for a set $\omega \subset \R^d$ in the reference configuration we denote $\omega^y \subset \R^d$ to be the deformed body in the current configuration, i.e.
\[
	\omega^y = \vb{y}(\omega).
\]

\subsubsection{Change of volume}
\label{sec:Change of volume}
Using the change of variable theorem we obtain

\begin{equation*}
	\lambda(\omega^y) = \int_{\omega^y} 1 \dd{\vb{x}^y} = \int_\omega \detf(\vb{x}) \dd{\vb{x}},
\end{equation*}
so we write $\dd{\vb{x}^y} = \detf \dd{\vb{x}}$. This motivates "our" definition of the determinant of the deformation gradient:

\begin{equation}
\label{eq:detfdefinition}
	\detf (\vb{x}) = \lim_{r\to 0} \frac{\lambda(B(\vb{x},r)}{\lambda(B(\vb{x},r))},
\end{equation}
where $B(\vb{x},r)$ is a (closed) ball centered at $\vb{x}$ of radius $r$.

\subsubsection{Change of lengths}
\label{sec:chlengths}
Suppose the line segment $\vb{x}+\Delta \vb{x}$ undergoes deformation. How does its length change?
Taylor expansion yields:
\begin{equation*}
	\vb{y}(\vb{x}+\vb{\Delta x})= \vb{y}(\vb{x}) + \fgrad(\vb{x})\Delta \vb{x} + \, \text{h.o.t.},
\end{equation*}
where h.o.t. stands for higher order terms. Using the expansion, we can write

\[
	\norm{\vb{y}(\vb{x}+\Delta \vb{x}) - \vb{y}(\vb{x})}^2 = \transpose{(\Delta \vb{x})} \transpose{\fgrad} \fgrad \Delta \vb{x} = 
	\transpose{(\Delta \vb{x})} \tensorq{C}(\vb{x}) \Delta \vb{x},
\]
where we have denoted
\[
	\tensorq{C}(\vb{x}) = \transpose{\fgrad}(\vb{x})\fgrad(\vb{x}),
\]
as the \textbf{Right Cauchy Green tensor}. 

\begin{example}
	Let the deformation $\vb{y}$ be given as $\vb{y}(\vb{x}) = \tensorq{R} \vb{x}+ \vb{v}, \vb{v} \in \R^d, \tensorq{R} \in \, \text{SO(d)} = \Big\{\tensorq{A} \in \R^{d \cross d}, \transpose{\tensorq{A}} \tensorq{A} = \tensorq{A} \transpose{\tensorq{A}} = \identity\Big\}$.
	Then \fgrad = \tensorq{R}, \rcg = \identity.
\end{example}


\subsubsection{Change of surfaces}
\label{sec:chsurfaces}
For $\tensorq{A} \in \R^{d \cross d}$ regular we define the \textbf{cofactor matrix} $\cof \tensorq{A}$ as
\[
	\cof \tensorq{A} = (\det \tensorq{A}) \transposei{\tensorq{A}},
\]
which is an interesting quantity whatsoever; we will use the following theorem

\begin{theorem}[Piola's identity]
	Let $\vb{y} \in \text{C}^2(\Omega;\R^d)$, then $\forall \vb{x}\in \Omega:$
	\[
		\divergence({\cof \grad{\vb{y}(\vb{x})}}) = \vb{0}.
	\]
\end{theorem}
For a regular matrix $\tensorq{A}$, we also have the identity
\begin{equation}
	\label{eqn:inv}
	\tensorq{A}^{-1} = \frac{1}{\det \tensorq{A}} (\transpose{\cof \tensorq{A})},
\end{equation}
What about the determinant of the cofactor? Clearly
\[
	\det \cof \tensorq{A} = (\det \tensorq{A})^d \det \transposei{\tensorq{A}} = (\det \tensorq{A})^{d-1},
\]
so we can also express equation \ref{eqn:inv} in a different way

\begin{equation}
    \label{eqn:invbetter}
    \tensorq{A}^{-1} = \frac{\transpose{(\cof \tensorq{A})}}{(\det \cof \tensorq{A})^{1/d-1}}.
\end{equation}
From geometry, recall the change of variables for surface integration:
\[
	\int_{\partial \omega^y}\vb{n}^y \dd{S}^y = \int_{\partial \omega}\cof \fgrad\vb{n} \dd{S},
\]
where $\vb{n}^y$ is the outward unit normal to the deformed boundary $\omega^y$. Informally, we write $\vb{n}^y \dd{S}^y = \cof \fgrad \vb{n} \dd{S}$. We can also explicitely express the normal to the deformed boundary as

\begin{equation}
    \label{eqn:deformed_normal}
    \vb{n}^y(\vb{x}^y)=\frac{\cof \fgrad (\vb{x}) \vb{n}(\vb{x})}{\norm{\cof \fgrad(\vb{x}) \vb{n}(\vb{x})}}, \vb{x} \in \partial \omega, \vb{y}(\vb{x}) \in \partial \omega^y.
\end{equation}

\subsection{Affine transformations}
\label{sec:affine_transf}
An example of deformation is the so called \textbf{affine transformation}.
\begin{example}
	Suppose the folowing deformation:
	\[
		\vb{y}(\vb{x}) = \tensorq{A} \vb{x} + \vb{v}, \tensorq{A}\in \R^{d \cross d}, \vb{v} \in \R^d, \detf > 0.
	\]
	Clearly then $\fgrad(\vb{x}) = \tensorq{A}.$
\end{example}
It is crucial to realize how $\fgrad, \transpose{\fgrad}, \transposei{\fgrad}$ work.

\begin{itemize}
	\item $\fgrad$ takes a vector $\vb{x}-\vb{0}$ from the \textit{reference configuration} and maps it to the vector $\fgrad \vb{x}- \fgrad \vb{0}$ in the \textit{current configuration}
	\item $\fgrad^{-1}$ takes the vector $\fgrad \vb{x} - \fgrad \vb{0}$ from the \textit{current configuration} and maps it to the vector $\vb{x}-\vb{0}$ from the \textit{reference configuration}
	\item $\transpose{\fgrad}$ is defined through: $\fgrad \vb{u} \vdot \vb{w} = \vb{u} \vdot \transpose{\fgrad}\vb{w}$, and since $\fgrad$ is defined on the reference configuration, $\transpose{\fgrad}$ must take something from the \textit{current configuration} and return something from the \textit{reference configuration}.
	\item $\transposei{\fgrad}$ consequently takes something from the \textit{referrence configuration} and maps it to something from the \textit{current configuration}.
\end{itemize}	
\begin{example}
	What when $\rcg = \identity$? Can we say something about $\fgrad$? Write $\rcg = \transpose{\fgrad}\fgrad = \identity$, so $\transpose{\fgrad} = \fgrad^{-1}, \det \fgrad > 0$. From this we have $\fgrad(\vb{x}) = \tensorq{R}(\vb{x}), \vb{x} \in \Omega$, where $\tensorq{R}$ is a rotation tensor. Investigate the cofactor of the deformation gradient:
	\[
		\cof \fgrad = \det \fgrad \transposei{\fgrad} = \cof \tensorq{R} = 1 \tensorq{R}(\vb{x}) = \fgrad (\vb{x}).
	\]
This implies $\cof \fgrad = \fgrad$. Recall Piola's identity:
\[
	\vb{0} = \divergence{\cof \fgrad} = \divergence{\fgrad (\vb{x})} = \laplacian \vb{y}(\vb{x}).
\]
We have the identity:
\begin{comment}
\begin{equation}
	\frac{1}{2}(\norm{\grad{\vb{y}})^{2} = \norm{\grad \grad \vb{y}}^{2} + \grad{\vb{y}} \vdot \grad \laplacian \vb{y},
\end{equation}
\end{comment}
and since the LHS is zero, we also have $\norm{\grad \grad \vb{y}} = 0 \Rightarrow \vb{y}(\vb{x}) = \tensorq{R}\vb{x}+\vb{v}$.
Let $\tensorq{R}$ be piecewise affine. Then $\tensorq{R}_1 (\identity - \vb{n} \otimes \vb{n}) = \tensorq{R}_2(\identity - \vb{n}\otimes \vb{n})$, so $\tensorq{R}_1 - \tensorq{R}_2 = (\tensorq{R}_1 = \tensorq{R}_2) = (\tensorq{R}_1 - \tensorq{R}_2) \vb{n} \otimes \vb{n} = \vb{a} \otimes \vb{b}$, but that is not possible for two rotations; the rank of the RHS is one, whereas the LHS is not.
\end{example}

\section{Forces}
\label{sec:forces}

\subsection{Forces in the deformed configuration}
\label{sec:forcesdeformed}

Recall $\vb{y}: \overline{\Omega} \to \overline{\Omega}^y$.We can define the \textbf{volume density of applied forces} $\vb{f}^y: \overline{\Omega}^y \to \R^3$ (in newtons per cubic meters, e.g. gravity).
The same on the boundary $\vb{g}^y : \Gamma^y_N \to \R^3$ (\textbf{surface density of applied forces} (in newtons per square meters = Pascals, e.g. hydrostatic pressure.)

\subsubsection{Cauchy stress tensor}
\label{sec:cstress}

\begin{lemma}[Stress principle of Euler and Cauchy]
	There exists a (Cauchy) stress vector function $\vb{t}^y : \overline{\Omega}^y \times \mathcal{S}^{d-1} \to \R^d $ with the following properties.
\begin{enumerate}
	\item If $\vb{x}^y \in \Gamma^y_N$, then $\vb{t}^y(\vb{x}^y, \vb{n}^y) = \vb{g}^y(\vb{x}^y),$ where $\vb{n}^y$ is the unit outer normal vector to $\partial \Omega^y$ at $\vb{x}^y$.
	\item $\forall \omega^y \subset \Omega^y$ it holds that $\int_{\omega^y} \vb{f}(\vb{x^y})\dd{\vb{x}^y} + \int_{\partial \omega^y} \vb{t}^y(\vb{x}^y, \vb{n}^y)\dd{S}^y = 0.$ (\textit{Balance of forces in static equilibrium.}
	\item $\forall \omega^y \subset \Omega^y$ it holds that $\int_{\omega^y} \vb{x}^y \cross \vb{f}^y(\vb{x}^y) \dd{\vb{x}^y} + \int_{\partial \omega^y} \vb{x}^y \cross \vb{t}^y(\vb{x}^y, \vb{n}^y) \dd{S}^y = \vb{0}$. (\textit{Balance of moment of forces in static equilibrium.})
\end{enumerate}
Euler says that the direct consequence of this is the existence of $\tensorq{T}^y(\vb{x}^y)$ such that

\begin{equation}
	\label{eqn:cauchy}
\vb{t}^y(\vb{x}^y,\vb{n}^y) = \tensorq{T}^y(\vb{x}^y) \vb{n}^y, 
\end{equation}
where the tensorial quantity $ \tensorq{T}$ is called the \textbf{Cauchy stress tensor}.
\end{lemma}

\subsubsection{Balance equations in the deformed configuration}
\label{sec:balance_equations_cauchy}
Classical physics gives us 2 fundamental relations: Newtons second law for momenta and for angular momenta. We examine these in the continuum mechanics setting.

From second property it follows:

\begin{equation}
	\int_{\omega^y} \vb{f}^y(\vb{x}^y) \dd{\vb{x}^y} = \int_{\partial \omega^y} \tensorq{T}^y(\vb{x}^y) \vb{n}^y \dd{S}^y = 0 = \int_{\omega^y} \divergence{\tensorq{T}^y}(\vb{x}^{y}) \dd{\vb{x}^{y}},
\end{equation}

so using the localization theorem we get

\begin{equation*}
	\label{eqn:balanceofforces}
	\vb{f}^y(\vb{x}^y) + \divergence{\vb{T}^y}(\vb{x}^y) = \vb{0}, \forall \vb{x}^y \in \Omega^y.
\end{equation*}
From the third property it follows

\begin{align*}
&	\int_{\omega^y} \vb{x}^y \cross \vb{f}^y(\vb{x}^y) \dd{\vb{x}^y} + \int_{\partial \omega^y} \vb{x}^y \cross \tensorq{T}^y(\vb{x}^y) \vb{n}^{y} \dd{S}^y = \\ &= \int_{\omega^y} \varepsilon_{ijk} x^y_j f^y_k \dd{\vb{x}^y} + \int_{\omega^y}\varepsilon_{ijk}x^y_j (T^y_{km} n^y_m) \dd{S}^y = \int_{\omega^y} \varepsilon_{ijk} x^y_j f^y_k \dd{\vb{x}^y}=\int_{\omega^y} \varepsilon_{ijk} \pdv{(x^y_j T^y_{km})}{x^y_m}\dd{\vb{x}^y} = \\& = \int_{\omega^y}\varepsilon_{ijk}x^y_j f^y_k \dd{\vb{x}^y} + \int_{\omega^y}\varepsilon_{ijk}x^y_j \pdv{T_km}{x^y_m}\dd{\vb{x}^y} + \int_{\omega^y}\varepsilon_{ijk}\delta_{jm}T^y_{km} \dd{\vb{x}^y} = \vb{0}.
\end{align*}
The last term implies

\begin{equation*}
	\int_{\omega^y} \varepsilon_{ijk}T^y_{kj} = 0,
\end{equation*}
and using the localization theorem, we obtain

\begin{equation}
	T^y_{ij} (\vb{x}^y) = T^y_{ji}(\vb{x}^y), \quad{i.e.} \tensorq{T}^y(\vb{x}^y) = \transpose{(\tensorq{T}^y(\vb{x}^y))}.
\end{equation}
The \textbf{Cauchy stress tensor is symmetric.}

\subsection{Forces in the undeformed configuration}
\label{sec:forces_undeformed}

We have obtained the equations in the deformed configuration. That is however unconvenient - we solve the equations to find the deformed configuration. This brings us to find a new way to write the equations - in the reference configuration. The construction is a bit synthetic, our intuition will be guided by the requirement to obtain similair equations as in the current configuration.

\subsubsection{Piola-Kirchhoff stresses}
\label{sec:pkstresses}

\begin{definition}[First Piola-Kirchhoff stress tensor]
Given the Cauchy stress tensor $\tensorq{T}^y(\vb{x^y})$, we define the \textbf{First Piola Kirchhoff stress tensor}
\[
	\tensorq{T}: \overline{\Omega} \to \R^{3 \times 3}, \tensorq{T}(\vb{x}) = \tensorq{T}^y(\vb{x}^y) \cof \fgrad (\vb{x}) = \det \fgrad (\vb{x}) \tensorq{T}^y(\vb{x}^y) \transposei{\fgrad}(\vb{x}).
\]
\end{definition}

\begin{definition}[Second Piola-Kirchhoff stress tensor]
The quantity
	\[
		\tensorq{S}(\vb{x}) = \fgrad^{-1} \tensorq{T}(\vb{x}) = \transpose{\tensorq{S}(\vb{x})},
	\]
	is called the \textbf{second Piola-Kirchhoff stress tensor.}
\end{definition}

\begin{remark}
	The first PK tensor $\tensorq{T}$ is \textit{not symmetric in general.}, but the second $\tensorq{S}(\vb{x}) = \fgrad^{-1} = \det \fgrad (\vb{x}) \fgrad^{-1} \tensorq{T}^y(\vb{x}^y) \transposei{\fgrad}(\vb{x})$ \textit{is}. Also, we see that not every matrix can serve as $\pkstress$; it must hold $\pkstress\qty(\vb{x}) \qty(\inverse{\cof \fgrad})$ is symmetric.
\end{remark}

\begin{remark}
We have the following identity (using Piola's identity):
\begin{equation}
	\divergence{\pkstress} \qty(\vb{x}) = \detf \qty(\vb{x}) \divergence{\cstress}\qty(\vb{x}^{y})^{y}.
\end{equation}
\end{remark}

\subsubsection{Balance equations in the deformed configuration}
\label{sec:balance_equations_piola}
Recall the balance of momentum

\begin{equation*}
    -\divergence{\qty(\cstress\qty(\vb{x}^{y}))}=\vb{f}^{y}\qty(\vb{x}^{y}) \, \text{in} \, \Omega^{y},
\end{equation*}
multiplying by $\detf >0$ yields

\begin{equation}
    \detf \divergence{\qty(\cstress\qty(\vb{x}^{y}))} = \detf \qty(\vb{x}) \vb{f}^{y}\qty(\vb{x}^{y}),
\end{equation}
which \textit{begs} for the definition

\begin{equation*}
    \vb{f}\qty(\vb{x})= \detf \qty(\vb{x}) \vb{f}^{y}\qty(\vb{y}\qty(\vb{x})),
\end{equation*}
as the force in the \textit{referential configuration}.

In total, the total acting body force on the body can be written as
\[
	\int_{\vb{y}\qty(\omega)}\vb{f}^{y}\qty(\vb{x}^{y})\dd{\vb{x}^{y}} = \int_{\omega}\vb{f}^{y}\qty(\vb{y}\qty(\vb{x}))\detf \qty(\vb{x})\dd{x} = \int_{\omega}\vb{f}\qty(\vb{x})\dd{\vb{x}}.
\]

We can do the same for the balance of surface forces; the total contact force acting on the body is
\begin{align*}
	\int_{\Gamma_N^{y}}\vb{g}^{y}\qty(\vb{x}^{y})\dd{S}^{y} &=\int_{\partial \omega^{y}}\vb{t}^{y}\qty(\vb{x}^{y},\vb{n}^{y})\dd{S}^{y} = \int_{\partial \vb{y}\qty(\omega)}\cstress\qty(\vb{x}^{y})\vb{n}^{y}\dd{S}^{y}=\\
								&=\int_{\partial \omega}\cstress(\vb{y}(\vb{x}))\cof \fgrad(t,\vb{x}) \vb{n}\dd{S} = \int_{\partial \omega}\pkstress\qty(\vb{x})\vb{n}\dd{S},
\end{align*}
so if we define
\[
	\vb{g}(\vb{x}) = \pkstress\qty(\vb{x})\vb{n}\qty(\vb{x}),
\]
as the contact force in the \textit{referential configuration}, we formally have a similiar expression.

\section{Elasticity}
\label{sec:elasticity}

\begin{definition}[Elasticity]
We say that a material is \textbf{elastic (or Cauchy elastic)} if there is a response function $\tilde{\tensorq{T}}^D:\Omega^{y} \cross \R^{3 \cross 3}_+ \to \R^{3 \cross 3}_{\, \text{sym} \,}$ such that
\[
	\cstress\qty(\vb{x}^{y})= \tilde{\tensorq{T}}^D\qty(\vb{x},\fgrad).
\]
The response function is also called the \textbf{constitutive law}.
\end{definition}

\begin{remark}
	If we know the material is elastic, we also have the information about the First Piola-Kirchhoff stress, as $\pkstress\qty(\vb{x}) = \cstress\qty(\vb{x}^{y})\cof \fgrad,$ so
	\begin{equation}
		\label{eq:response_undeformed}
		\pkstress\qty(\vb{x}) = \tilde{\tensorq{T}}^D\qty(\vb{x}, \fgrad) \cof \fgrad = \tilde{\tensorq{T}}\qty(\vb{x},\fgrad).
	\end{equation}
\end{remark}

\subsection{Frame invariance principle}
\label{sec:frame_invariance}
The frame invariance principle states:
\[
	\tilde{\tensorq{T}}^D\qty(\vb{x}, \tensorq{R} \vb{x}) = \tensorq{R}\tilde{\tensorq{T}}^D\qty(\vb{x},\fgrad)\transpose{\tensorq{R}}, \forall \tensorq{R}\in \, \text{SO(3)} \,, \forall \vb{x}\in \overline{\Omega},
\]
from which it follows ($\tilde{\tensorq{T}}$ is defined in \ref{eq:response_undeformed})
\[
	\tilde{\tensorq{T}}\qty(\vb{x}, \tensorq{R} \fgrad) = \det\qty(\tensorq{R} \fgrad) \tilde{\tensorq{T}}^D\qty(\vb{x}, \tensorq{R}\fgrad)\transposei{\qty(\tensorq{R} \fgrad)} = \det\qty(\tensorq{R} \tensorq{F}) \tensorq{R} \tilde{\tensorq{T}}^D(\vb{x},\fgrad) \transpose{\tensorq{R}} \tensorq{R} \transposei{\fgrad} = \detf \tensorq{R}\tilde{\tensorq{T}}^D\qty(\vb{x},\fgrad)\transposei{\fgrad}=\tensorq{R}\tilde{\tensorq{T}}\qty(\vb{x},\fgrad),
\]
thus
\[
	\tilde{\tensorq{T}}\qty(\vb{x}, \tensorq{R} \fgrad) = \tensorq{R}\tilde{\tensorq{T}}\qty(\vb{x},\fgrad), \, \text{i.e.} \, \transpose{\tensorq{R}}\tilde{\tensorq{T}}\qty(\vb{x},\tensorq{R}\tensorq{F}) = \tilde{\tensorq{T}}\qty(\vb{x},\fgrad), \forall \tensorq{R} \in \, \text{SO(3)} \,, \forall \fgrad \in \R^{3 \cross 3}_+.
\]

\subsection{Isotropic material}
\label{sec:isotropic_materials}
Recall $\cstress(\vb{x}^{y})=\tilde{\tensorq{T}}^D\qty(\vb{x},\fgrad), \vb{y}:\overline{\Omega} \to \Omega^y = \vb{y}(\Omega).$ Take $\vb{x}_0 \in \overline{\Omega}$ general but fixed, take $\vb{v}\qty(\vb{z})=\vb{x}_0 + \transpose{R}\qty(\vb{z}-\vb{x}_0)$ for some $\tensorq{R} \in \, \text{SO(3)} \,$ and define a \textit{new deformation}
\[
	\tilde{\vb{y}}=\vb{y}\circ \inverse{\vb{v}}: \vb{v}\qty(\overline{\Omega}) \to \vb{y}\qty(\overline{\Omega}), \tilde{\vb{y}}\qty(\tilde{\vb{x}})=\vb{y}\qty(\vb{x}_0+\tensorq{R}\qty(\tilde{\vb{x}}-\vb{x}_0)).
\]
This implies
\[
	\vb{x}_0^{y} = \vb{x}_0^{\tilde{y}}, \cstress\qty(\vb{x}_0^{y})=\tilde{\tensorq{T}}^D\qty(\vb{x}_0, \fgrad)= \tensorq{T}^{\tilde{y}}(\vb{x}^{\tilde{y}}_0) = \tilde{\tensorq{T}}^D\qty(\vb{x}_0, \tilde{\fgrad}\qty(\vb{x}_0)) = \tilde{\tensorq{T}}^D\qty(\vb{x}_0, \fgrad\qty(\vb{x}_0) \tensorq{R}).
\]

\begin{definition}[Isotropic material]
	We cal the material \textbf{isotropic} if it holds
	\[
		\tilde{\tensorq{T}}^D\qty(\vb{x},\fgrad) = \tilde{\tensorq{T}}^D\qty(\vb{x},\fgrad \tensorq{R}), \forall \tensorq{R} \in \, \text{SO(3)} \,, \forall \fgrad \in \R^{3 \cross 3}_+.
	\]
\end{definition}
\begin{remark}
	For the first Piola-Kirchhoff we obtain: $\tensorq{T}^D\qty(\vb{x},\fgrad \tensorq{R}) = \tensorq{T}^D\qty(\vb{x},\fgrad) \tensorq{R}, $ which means
	\[
		\tensorq{T}^D\qty(\vb{x}, \tensorq{Q}\fgrad \tensorq{R}) = \tensorq{Q}\tilde{\tensorq{T}}^D \tensorq{R}, \forall \tensorq{R}, \tensorq{Q} \in \, \text{SO(3)} \,, \forall \fgrad \in \R^{3 \cross 3}_+.
	\]
\end{remark}

\subsection{Hyperelastic materials}
\label{sec:hyperelasticity}
\begin{definition}
	We say that a material is hyperelastic if there is a function $W:\overline{\Omega} \cross \R^{3 \cross 3}_+ \to \R$ such that
	\[
		\pkstress\qty(\vb{x}) = \tilde{\tensorq{T}}\qty(\vb{x}, \fgrad) = \pdv{W\qty(\vb{x},\fgrad)}{\fgrad}, \fgrad = \grad \vb{y}(\vb{x}).
	\]

The function $W, [W] = \frac{J}{m^3} = \frac{N m}{m^3} = \text{Pa}$ is called \textbf{stored energy density.}
\end{definition}
	
\begin{remark}
    Evidently, $W$ has a potential.
\end{remark}

\subsection{Properties of W}
\label{sec:propetiesW}
It is physical to assume
\begin{enumerate}
	\item $W\geq 0$ (energy is nonnegative)
	\item $W\qty(\vb{x},\fgrad) = W\qty(\vb{x},\tensorq{R} \fgrad), \forall \tensorq{R} \in \, \text{SO(3)} \,, \forall \vb{x} \in \overline{\Omega}, \forall \fgrad \in \R^{3 \cross 3}_+.$ (energy does not change under rotations \footnote{If this was not true, you could create infinite energy by just spinning a rubber.}
	\item $W\qty(\vb{x}, \tilde{\tensorq{R}} \tensorq{U}) = W\qty(\vb{x}, \tensorq{U}), \tensorq{U}=\sqrt{\rcg}.$ (matrices are from the polar decomposition)
	\item $W\qty(\vb{x},\fgrad) \to \infty \, \text{if} \, \detf \to 0_+$ (it takes infinite energy to deform the body to a point)
	\item $W\qty(\vb{x},\fgrad) \geq \alpha\qty(\norm{\fgrad}^p + \norm{\cof \fgrad}^q + \qty(\detf)^r)-d, \forall \alpha > 0, \forall p,q,r \geq 1, \forall d \in \R, \forall \vb{x} \in \overline{\Omega}, \forall \fgrad \in \R^{3 \cross 3}_+.$
\end{enumerate}

\begin{definition}[Natural state of the body]
    The natural state of the body is the state in which
    \begin{equation}
	    \label{eq:natural}
	    W(\vb{x},\fgrad) = 0 \wedge \pdv{W}{\fgrad}\qty(\vb{x},\fgrad) = \tensorq{0}.
    \end{equation}
\end{definition}

\begin{remark}[Unnatural vegetable]
    Not all materials (bodies) have its natural states. Zum Beispiel, carrot does not have a natural state.
\end{remark}

From the previous work, we can write $\transpose{\tensorq{R}} \pdv{W\qty(\vb{x}, \tensorq{R} \fgrad)}{\fgrad} = \pdv{W\qty(\vb{x}, \fgrad)}{\fgrad},$ and for brevity denote $W\qty(\vb{x}, \tensorq{R}\fgrad) = W_R(\vb{x}, \fgrad).$ Next, we \textit{suppose we can Taylor expand}:
\begin{align*}
	W_R\qty(\vb{x}, \fgrad + \tilde{\fgrad}) &= W\qty(\vb{x}, \tensorq{R}\fgrad + \tensorq{R}\tilde{\fgrad}) = W\qty(\vb{x}, \tensorq{R} \fgrad) + \pdv{W\qty(\vb{x},\tensorq{R} \fgrad)}{\fgrad}:\qty(\tensorq{R} \tilde{\fgrad}) + \, \text{h.o.t.} \, \\&= W_R\qty(\vb{x}, \fgrad)+ \transpose{\tensorq{R}} \pdv{W\qty(\vb{x}, \tensorq{R}\fgrad)}{\fgrad}:\tilde{\fgrad} + \, \text{h.o.t.} \,.
\end{align*}
Moreover
\[
	W_R\qty(\vb{x},\fgrad+\tilde{\fgrad}) = W_R\qty(\vb{x},\fgrad)+ \pdv{W_R\qty(\vb{x},\fgrad)}{\fgrad}: \tilde{\fgrad}.
\]

Altogether
\[
	\pdv{\fgrad}\qty(W_R\qty(\vb{x},\fgrad)-W_R\qty(\vb{x},\fgrad)) = \tensorq{0},
\]
from which it follows \footnote{The set of matrices with positive determinant is connected.}
\[
	W\qty(\vb{x},\tensorq{R} \fgrad) = W\qty(\vb{x},\fgrad) + k\qty(\tensorq{R}).
\]
Take $\fgrad = \identity,$ then
\[
	W\qty(\vb{x}, \tensorq{R}^{2}) = W\qty(\vb{x},\tensorq{R}\tensorq{R}) = W\qty(\vb{x},\tensorq{R})+k\qty(\tensorq{R}) = W\qty(\vb{x},\identity)+2 k\qty(\tensorq{R}),
\]
 so
\[
	W\qty(\vb{x},\tensorq{R}^n)=W\qty(\vb{x},\identity)+nk\qty(\tensorq{R}).
\]
Since the set of rotations is closed and bounded, it is compact, so there exists a convergent subsequence of $\{\tensorq{R}^n\}$. Moreover, we assume $W$ to be continuous (we took the derivative...), so $\lim_{n \to \infty}W\qty(\vb{x},\tensorq{R}^n)$ exists and from the properties of $W$ we get it is finite. But then $k\qty(\tensorq{R}) = 0$, as otherwise $nk\qty(\tensorq{R})\to \infty.$ All in all, we have shown
\begin{equation}
    W\qty(\vb{x}, \tensorq{R}\fgrad) = W\qty(\vb{x},\fgrad).
\end{equation}

\begin{definition}[Energy functional]
    Let us have $\partial \Omega = \Gamma_N \cup \Gamma_D, \Gamma_N \cap \Gamma_D = \emptyset,$ where the parts of the boundary are those when Neumann/Dirichlet boudary conditions are prescribed.
The energy functional of the material is the functional
\[
	I\qty(\vb{y}) = \int_{\Omega}W\qty(\vb{x}, \fgrad\qty(\vb{x}))\dd{\vb{x}} - \int_{\Omega}\vb{f}\qty(\vb{x})\vdot \vb{y}\qty(\vb{x}) \dd{\vb{x}}-\int_{\Gamma_N}\vb{g}\qty(\vb{x}) \vdot \vb{y}\qty(\vb{x})\dd{S},
\]
where the first part corresponds to the stored energy and the remaining terms are the work done by the external loads.
\end{definition}

\begin{remark}
	If $\vb{y}$ is the minimizer of $I$, then $I\qty(t \vb*{\varphi}+\vb{y})\geq I\qty(\vb{y}), \forall t, \vb*{\varphi}.$ If we denote
    \[
	    a(t)\coloneq I\qty(t \vb*{\varphi}+\vb{y}),
    \]
    then it most hold
    \[
	    0 = a'(0) = \dv{t} \eval{\qty(\int_{\Omega}W\qty(\fgrad + t \grad \vb*{\varphi})\dd{\vb{x}}- \int_{\Omega}\vb{f}\qty(\vb{x})\vdot \qty(\vb{y}\qty(\vb{x})+t \vb*{\varphi}\qty(\vb{x}))\dd{\vb{x}} - \int_{\Gamma_N}\vb{g}\qty(\vb{x}) \vdot(\vb{y}\qty(\vb{x})+t \vb*{\varphi}\qty(\vb{x}))\dd{x})}_{t=0},
    \]

calculating the derivatives yields
\begin{align*}
	0 &= \int_{\Omega}\pdv{W\qty(\fgrad)}{\fgrad}: \grad \vb*{\varphi}\dd{\vb{x}} - \int_{\Omega}\vb{f}\vdot \vb*{\varphi}\dd{\vb{x}} - \int_{\Gamma_N}\vb{g}\vdot \vb*{\varphi}\dd{S} =\\
	  &= \int_{\Omega}\pdv{x_j}\qty(\pdv{W\qty(\fgrad)}{F_{ij}}\varphi_i)\dd{\vb{x}}-\int_{\Omega}\pdv{x_j}\qty(\pdv{W\qty(\fgrad)}{F_{ij}})\varphi_i\dd{\vb{x}}- \int_{\Omega}\vb{f}\vdot \vb*{\varphi}\dd{\vb{x}}-\int_{\Gamma_N}\vb{g}\vdot \vb*{\varphi}\dd{S}= \\
	  & = \int_{\Gamma_N}\pdv{W\qty(\fgrad)}{F_{ij}}\varphi_i n_j \dd{S}-\int_{\Omega}\pdv{x_j}\qty(\pdv{W\qty(\fgrad)}{F_{ij}})\varphi_i\dd{\vb{x}}-\int_{\Omega}\vb{f}\vdot \vb*{\varphi}\dd{\vb{x}} - \int_{\Gamma_N}\vb{g} \vdot \vb*{\varphi}\dd{S},
\end{align*}
so it must hold
\[
	-\pdv{x_j}\qty(\pdv{W\qty(\fgrad)}{F_{ij}})= f_i \ \text{in} \, \Omega, \pdv{W\qty(\fgrad)}{F_{ij}}n_j=g_i \, \text{on} \, \Gamma_N.
\]
This is exactly
\[
	-\divergence{\pkstress} = \vb{f} \, \text{in} \,\Omega, \pkstress \vb{n} = \vb{g}, \, \text{on} \, \Gamma_N.
\]

This implies that $\vb{y}$ minimizes energy $\Leftrightarrow \vb{y}$ is governed by the equations of classical mechanis.

\end{remark}

Are there some other qualities of $W$? It is natural to assume
\[
	W\qty(\identity) = 0 \Rightarrow W\qty(\tensorq{R})=0, \forall \tensorq{R}\in \, \text{SO(3)} \,
\]
and $W\qty(\fgrad)>0$ whenever $\fgrad \notin \, \text{SO(3)} \,$
This however implies $W$ is not convex! Assume
\[
	\tensorq{R}_1 = \begin{bmatrix} 1 & 0 & 0\\ 0 & -1 & 0\\ 0 & 0 & -1 \end{bmatrix}, \tensorq{R}_2 = \identity,
\]
then
\[
	W\qty(\frac{1}{4}\tensorq{R}_1+\frac{3}{4}\tensorq{R}_2)>\frac{1}{4}W\qty(\tensorq{R}_1)+\frac{3}{4}W\qty(\tensorq{R}_2)=0.
\]

\begin{example}[Minimizer does not exist]
Assume $J(u) = \int_{0}^1 \qty(1-\qty(u'(x))^{2})^{2}+u(x)^{2}\dd{x}, u \in \text{W}^{1,4}(0,1), u(0) = u(1) = 0,$ and find the minimum of $J$. First of all, $J>0$, so the minimum also. I can take $u_k$ such that $u'_k(x) = 1$ on $(0,1/2)$ and $u'_k(x) = -1$ on $(1/2,1)$. Then $J\qty(u_k)\to 0 \Rightarrow \inf J = 0$ but there is no minimizer.
\end{example}

Not everything is lost...

\begin{definition}[Polyconvexity, 1977 J.M. Ball]
	$W: \R^{3 \times 3}\to \R \cup \{\infty\}$ is polyconvex provided there exists convex and lower-semicontinuous function $h: \R^{19}\to \R \cup \{\infty\}$:
\[
	W\qty(\tensorq{A})=h\qty(\tensorq{A}, \cof \tensorq{A}, \det \tensorq{A}).
\]
\end{definition}

\begin{example}
	\begin{itemize}
		\item	If $W$ is convex and lower-semicontinuous then $W$ is polyconvex. 
		\item	$W\qty(\tensorq{A}) = \det \tensorq{A}$ is polyconvex but not convex.
	\end{itemize}
\end{example}

\begin{remark}[Weak convergence in $\LpSet{\Omega; \R^3}$]
	Let $1<p<\infty$ and $\{\vb{u_k}\} \subset \LpSet{\Omega; \R^3}.$ We say $\{\vb{u_k}\}$ converges weakly to $\vb{u}$ in $\LpSet{\Omega;\R^3}$ provided
	\[
		\int_{\Omega}\vb{u_k} \vdot \vb*{\varphi}\dd{\vb{x}} \to \int_{\Omega}\vb{u}\vdot \vb*{\varphi}\dd{\vb{x}}, \forall \vb*{\varphi} \in \LpSet[p']{\Omega;\R^3}.
	\]
\end{remark}

\begin{theorem}[Magic]
	Assume that $\vb{y}^k$ converges weakly to $\vb{y}$ in $\WkpSet[1][p]{\Omega;\R^3}, \Omega \subset \R^3 \in C^{0,1}, p>3.$ Then $\det \grad \vb{y}^k$ converges weakly to $\det \grad \vb{y} \, \text{in} \, \LpSet[\frac{p}{3}]{\Omega}.$ Moreover $\cof \grad \vb{y}^l$ converges weakly to $\cof \grad \vb{y} \, \text{in} \, \LpSet[\frac{p}{2}]{\Omega;\R^{3 \times 3}}.$
\end{theorem}

\begin{proof}
	Only in 2 dimensions. The determinant can be written as:
	\[
		\det \grad \vb{y} = \pdv{y_1}{x_1}\pdv{y_2}{x_2}-\pdv{y_1}{x_2}\pdv{y_2}{x_1} = \pdv{x_1}\qty(y_1 \pdv{y_2}{x_2})-\pdv{x_2}\qty(y_1 \pdv{y_2}{x_1}) = \divergence{\qty(y_1,-\pdv{y_2}{x_1})},
	\]
	so then
	\[
		\int_{\Omega}\det \grad \vb{y}^k \varphi \dd{x} = \int_{\Omega}\pdv{x_1} \qty(y^k_1 \pdv{y_2^k}{x_2})\varphi\dd{x} - \int_{\Omega}\pdv{x_2}\qty(y^k_1 \pdv{y^k_2}{x_1})\varphi\dd{x} = -\int_{\Omega}y_1^k \pdv{y_2^k}{x_2}\pdv{\varphi}{x_1} \dd{x} + \int_{\Omega}\pdv{\varphi}{x_2} y_1^k \pdv{y_2^k}{x_1}\dd{x},
	\]
	and the result follows from the embedding theorems and strong convergence (strong times weak gives weak convergence).
\end{proof}

\subsection{Rank-one convexity}
\label{sec:rank-one}


	Assume the following domain: $\Omega = (1,2) \cross (0,4 \pi) \cross (1,2)$ and the deformation $\vb{y}: \overline{\Omega} \to \R^3, \vb{y}\qty(x_1,x_2,x_3) = \qty(x_1 \cos x_2, x_1 \sin x_2, x_3), \fgrad  = \begin{bmatrix}
		\cos x_2 & -x_1 \sin x_2 & 0\\
		\sin x_2 & x_1 \cos x_2 & 0\\
		0 & 0 & 1
	\end{bmatrix}.$
	We can calculate $\detf = x_1 \cos^{2}x_2 + x_1 \sin^{2}x_2 = x_1.$ But even though the deformation has positive determinant, we still face self-penetration issues, i.e., \textit{$\vb{y}$ is not injective}.

	\begin{theorem}[Ciarlet-Nečas condition]
    Let $p>3$ and let $\detf > 0$ a.e. in $\Omega \subset \R^3, \vb{y} \in \WkpSet[1][p]{\Omega; \R^3}.$ If
    \[
	    \int_{\Omega}\detf\qty(\vb{x})\dd{\vb{x}} \leq \lambda\qty(\vb{y}\qty(\Omega))
    \]
    then $\vb{y}$ is injective almost everywhere in $\Omega$, i.e., $\exists \omega \subset \Omega: \lambda\qty(\omega) = 0, \vb{y}|_{\Omega / \omega}$ is injective.
\end{theorem}

Is the determinant condition of any use? Let us compute, assuming $\vb{y}= \vb{0} \, \text{on } \, \partial \Omega.$
\[
	\int_{\Omega}\detf\dd{\vb{x}} = \int_{\Omega}\pdv{x_1}\qty(y_1 \pdv{y_2}{x_{2}})-\pdv{x_2}\qty(y_1 \pdv{y_2}{x_1})\dd{\vb{x}} = \int_{\partial \Omega}y_1 \pdv{y_2}{x_2}n_1 - y_1 \pdv{y_2}{x_1} n_2\dd{S} \underbrace{\Rightarrow}_{y=0 \, \text{on} \, \partial \Omega} \int_{\Omega}\detf\qty(\vb{x})\dd{\vb{x}} = 0.
\]
This is powerful! Assume that $\vb{y}\qty(\vb{x}) = \fgrad \vb{x}$ on $\partial \Omega$, then
\[
	\int_{\Omega}\detf\qty(\vb{x})\dd{\vb{x}} = \lambda\qty(\Omega)\detf.
\]
Now let
\[
	I\qty(\vb{y}) = \int_{\Omega}\detf\qty(\vb{x})\dd{\vb{x}}, \vb{y}\in \WkpSet[1][p]{\Omega; \R^3}, \vb{y}\qty(\vb{x}) = \fgrad \vb{x}.
\]
Then $I$ is constant\footnote{All constant functionals are convex.} and it holds
\[
	I\qty(\vb{y}) = \lambda\qty(\Omega)\detf.
\]

\section{Linearized elasticity}
\label{sec:linearized_elasticity}

Recall the Right Cauchy-Green tensor: $\rcg = \transpose{\fgrad}\fgrad$. Using it, we can define

\begin{theorem}[Green-Lagrange strain tensor/Green-St.-Venaint strain tensor]
    Let $\rcg$ be the Right Cauchy-Green tensor. We define the Green-Lagrange strain tensor/Green-St.-Venaint strain tensor as
    \[
	    \gsv = \frac{1}{2}\qty(\rcg - \identity).
    \]
\end{theorem}
\begin{remark}
    The Green-St.-Venaint strain tensor can be rewritten as:
    \[
	    \gsv = \frac{1}{2}\qty(\transpose{\qty(\identity+\grad \vb{u})}\qty(\identity + \grad \vb{u})- \identity) = \frac{1}{2}\qty(\grad \vb{u}+\transpose{\qty(\grad \vb{u})})+ \frac{1}{2}\transpose{\qty(\grad \vb{u})}\grad \vb{u} = \tensorq{e}\qty(\vb{u}) + \frac{1}{2}\rcg\qty(\grad \vb{u}).
    \]
\end{remark}
For the stored energy density, we can write
\[
	W\qty(\fgrad) = W\qty(\tensorq{R}\fgrad) = \overline{W}\qty(\rcg\qty(\fgrad)) = \hat{W}\qty(\gsv\qty(\fgrad)).
\]
and also
\[
	W\qty(\fgrad) = \hat{W}\qty(\tensorq{e}\qty(\vb{u})+\rcg\qty(\grad \vb{u})).
\]
It is our assumption that
\[
	\hat{W}\qty(\tensorq{0}) = 0, \hat{W}\qty(\gsv)>0 \, \text{if} \, \gsv \neq \tensorq{0},
\]
and also that
\[
	\rcg\qty(\grad \vb{u}) = \vb{0}.
\]
Using Taylor expansion, we can write

\[
	\hat{W}\qty(\tensorq{e}\qty(\vb{u}))= \hat{W}\qty(\tensorq{0})+ \pdv{\hat{W}}{\tensorq{e}}\qty(\tensorq{0})\tensorq{e}\qty(\vb{u})+\frac{1}{2}\pdv[2]{\hat{W}}{\tensorq{e}}\qty(\tensorq{0})\tensorq{e}\qty(\vb{u}) \vb{e}\qty(u) + \, \text{h.o.t.} \,.
\]
Since $\hat{W}\qty(\tensorq{0}) = \pdv{\hat{W}}{\tensorq{0}}\qty(\tensorq{0}) = 0$ the above (formal) manipulation leads us to the definition
\begin{definition}[Tensor of elastic constants]
	\[
		\tensorq{C} = \pdv[2]{\hat{W}}{\tensorq{e}}\qty(\tensorq{0}), C_{ijkl} = \pdv[2]{\hat{W}}{e_{ij}e_{kl}}.
	\]
\end{definition}
\begin{remark}
    Since we assume $\hat{W}$ is smooth, we have some symmetries, and from the general 81 components of $C_{ijkl}$ only 21 are unique.
\end{remark}

With the notion of a tensor of elastic constants, we can then write the stored energy density as
\[
	w\qty(\tensorq{e}) = \frac{1}{2}(\tensorq{C}\tensorq{e}):\tensorq{e}.
\]
Following our definition $\pkstress = \pdv{\hat{W}}{\fgrad}$ we see
\[
	\tensorq{\sigma} = \pdv{w\qty(\tensorq{e})}{\tensorq{e}} = \tensorq{C}\tensorq{e}, \sigma_{ij} = C_{ijkl}e_{kl}.
\]
Is a useful notion of stress. It is denoted as the \textit{Cauchy stress}.
or in components
\[
	\sigma_{ij}=C_{ijkl}e_{kl}.
\]

\subsection{Equations}
\label{sec:linel_equations}

Rewritting the equations in the linearized elasticity setting we obtain the system
\begin{align*}
	-\divergence{\tensorq{\sigma}} = - \divergence{\qty(\tensorq{C}\tensorq{e})} &= \vb{f} \, \text{in} \, \Omega \\
	\tensorq{\sigma}\vb{n} &= \vb{g} \, \text{on} \, \Gamma_{N}, \\
	\vb{u} &= \vb{0} \, \text{on} \, \Gamma_{D}.
\end{align*}
The weak formulation can be obtained as
\[
	\int_{\Omega} \pdv{x_j}\qty(C_{ijkl}e_{kl})v_i \dd{\vb{x}} = \int_{\Omega}\vb{f}\vdot \vb{v}\dd{\vb{x}}, \forall \vb{v} \in \WkpSet[1][2]{\Omega;\R^3}, u=0 \, \text{on} \, \Gamma_D,
\]
so
\[
	\int_{\Omega}C_{ijkl}e_{kl} \pdv{v_i}{x_j}\dd{\vb{x}}-\int_{\partial \Omega}C_{ijkl}e_{kl}v_i n_j \dd{S} = \int_{\Omega}\vb{f}\vdot \vb{v}\dd{\vb{x}},
\]
which can be rewritten as
\[
	\underbrace{\int_{\Omega}\tensorq{C}\tensorq{e}\qty(\vb{u}) \vdot \tensorq{e}\qty(\vb{v})\dd{\vb{x}}}_{\coloneq B\qty(u,v)} = \underbrace{\int_{\Omega}\vb{f} \vdot \vb{v}\dd{\vb{x}} + \int_{\Gamma_N}\vb{g}\vdot \vb{v}\dd{S}}_{\coloneq L(v)},
\]
where we have denoted
\[
	\tensorq{e}\qty(\vb{v}) = \, \text{sym} \,\qty(\grad \vb{v}).
\]
We are looking for
\[
	u \in V = \{ u \in \WkpSet[1][2]{\Omega; \R^3}, \tr u = 0 \, \text{on} \,\Gamma_D\}: B\qty(u,v) = L(v) \forall v \in V,
\]
and to prove the existence, we will use the Lax-Milgram lemma. Show that
\begin{itemize}
	\item $L \in V^{*}$
	\item $B:V \times V \to \R$ is V-bounded and V-coercive
\end{itemize}
Realize that in order to show the properties, we would have to be able to control $\grad \vb{u}$ by $\, \text{sym} \,\qty(\grad \vb{u}).$ Is that even possible?

\begin{example}
	Let $u=0 \, \text{on} \, \partial \Omega.$ Then
	\[
		\exists C>0: \int_{\Omega}|\tensorq{e}\qty(\vb{u})|^{2}\dd{\vb{x}} \geq c \int_{\Omega}|\grad \vb{u}|^{2}\dd{\vb{x}}.
	\]
\end{example}
\begin{theorem}[Korn's inequality]

\end{theorem}
\section{Tutorials}
\label{sec:tutorials}


\subsection{Change of observer}
\label{sec:chobserver}

The requirement of material frame indifference yields:
\[
	W = \hat{W}\qty(\fgrad) = \hat{W}\qty(\tensorq{Q}\fgrad), \forall \tensorq{Q} \in \, \text{orth} \,.
\]

\subsection{Change of reference configuration}
\label{sec:chreference}
The requiremenent of material symmetry yields:
\[
	W= \hat{W}\qty(\fgrad)= \hat{W}\qty(\fgrad \tensorq{P}), \forall \tensorq{P} \in \mathcal{G},
\]
where $\mathcal{G}$ is the symmetry group of the material.

\subsection{Consequences of isotropic hyperelastic solid}
\label{sec:conshypelisosolid}

\begin{remark}[Groups unim, orth]
	The "biggest sensible" symmetry group is the unimodular group:
	\[
		\, \text{unim} \, = \{\tensorq{P}, \det \tensorq{P} = \pm 1\}.
	\]
	There exists another common group:
	\[
		\, \text{orth} \, \{ \tensorq{Q}, \tensorq{Q}\transpose{\tensorq{Q}} = \transpose{\tensorq{Q}} \tensorq{Q} = \identity\} \subset \, \text{unim} \,.
	\]
\end{remark}
We thus have $W = \hat{W}\qty(\fgrad) = \hat{W}\qty(\tensorq{Q}\fgrad) = \hat{W}\qty(\fgrad \tensorq{Q}), \forall \tensorq{Q} \in \, \text{orth} \,, \forall \fgrad.$

Use \textit{polar decomposition}: $\tensorq{F} = \tensorq{R}\tensorq{U} = \tensorq{V}\tensorq{R}, \tensorq{R} \in \, \text{orth} \,, \tensorq{U,V} \, \text{positively definite} \,, \tensorq{U} = \sqrt{\tensorq{C}}, \tensorq{V} = \sqrt{\tensorq{B}}.$ 

To make use of it, we first use m.f.i. and then isotropy and then first use isotropy and then m.f.i.
So from material frame indifference
\[
	W = \hat{F}\qty(\fgrad) = \hat{W}\qty(\tensorq{Q}\fgrad) = \hat{W}\qty(\transpose{\tensorq{R}} \tensorq{R} \tensorq{U}) = \hat{W}\qty(\tensorq{U}) = \overline{W}\qty(\rcg),
\]
where we have taken $\tensorq{Q} = \transpose{\tensorq{R}}.$ Note that this works universaly (without the need of isotropy), as it comes from the objectivity consideration.

From isotropy
\[
	W = \hat{W}\qty(\tensorq{F}) = \hat{W}\qty(\fgrad \tensorq{Q}) = \overline{W}\qty(\rcg) = \overline{W}\qty(\transpose{\qty(\fgrad \tensorq{Q})} \qty(\fgrad \tensorq{Q})) = \overline{W}\qty(\transpose{\tensorq{Q}} \transpose{\fgrad}\fgrad \tensorq{Q}) = \overline{W}\qty(\transpose{\tensorq{Q}}\rcg \tensorq{Q}), \forall Q \in \, \text{orth} \,, \forall \rcg \, \text{admissable} \,.
\]
Now backwards using the second polar decomposition:
\[
	W = \hat{W}\qty(\fgrad) = \hat{W}\qty(\fgrad \tensorq{Q}) = \hat{W}\qty(\tensorq{V}\tensorq{R}\tensorq{Q}) = \hat{W}\qty(\tensorq{V}\tensorq{R}\transpose{\tensorq{R}}) = \hat{W}\qty(\sqrt{\lcg}) = \tilde{W}\qty(\lcg),
\]

\[
	W = \tilde{W}\qty(\lcg) = \tilde{W}\qty(\tensorq{Q}\fgrad \transpose{\qty(\tensorq{Q}\fgrad)}) = \tilde{W}\qty(\tensorq{Q}\lcg \transpose{\tensorq{Q}}).
\]

So far, we have shown
\begin{align*}
	W\qty(t,\vb{X}) &= \overline{W}\qty(\rcg\qty(t,\vb{X}),\vb{X}) = \overline{W}\qty(\tensorq{Q}\rcg \transpose{\tensorq{Q}}), \\ W\qty(t,\vb{X}) &= \tilde{W}\qty(\lcg(t,\vb{X}), \vb{X}) = \tilde{W}\qty(\tensorq{Q} \lcg \transpose{\tensorq{Q}}),
\end{align*}

In HW, we will know
\[
	\pkstress = 2 \pdv{\hat{W}\qty(\fgrad)}{\fgrad}, T \indices{_{iJ}} = 2 \pdv{\hat{W}}{F \indices{_{iJ}}}
\]
and we can show
\[
	\pkstress = 2\pdv{\hat{W}\qty(\fgrad)}{\fgrad} = 2 \fgrad \pdv{\tilde{W}\qty(\lcg)}{\lcg}\fgrad, \cstress  = \dots, \tensorq{S} = 2 \pdv{\overline{W}\qty(\rcg)}{\rcg}.
\]

\begin{definition}[Isotropic functions]
	We say the functions $\hat{a}\qty(y_{\alpha}, \vb{y}_{\alpha}, \tensorq{Y}_{\alpha}), \vb{\hat{a}}\qty(y_{\alpha}, \vb{y}_{\alpha}, \tensorq{Y}_{\alpha}), \hat{\tensorq{A}}\qty(y_{\alpha},\vb{y}_{\alpha},\tensorq{Y}_{\alpha}), \alpha = 1, \dots, N$ are isotropic functions (of their respective arguments) if it holds
	\begin{align*}
		\hat{a}\qty(y_{\alpha}, \vb{y}_{\alpha}, \tensorq{Y}_{\alpha}) &= \hat{a}\qty(y_{\alpha}, \tensorq{Q}\vb{y}_{\alpha}, \tensorq{Q} \tensorq{Y}_{\alpha} \transpose{\tensorq{Q}}), \\
		\tensorq{Q} \vb{\hat{a}}\qty(y_{\alpha}, \vb{y}_{\alpha}, \tensorq{Y}_{\alpha}) &= \vb{\hat{a}}\qty(y_{\alpha}, \tensorq{Q}\vb{y}_{\alpha}, \tensorq{Q}\tensorq{Y}_{\alpha}\transpose{\tensorq{Q}}), \\
		\tensorq{Q}\hat{\tensorq{A}} \transpose{\tensorq{Q}} &= \hat{\tensorq{A}}\qty(y_{\alpha}, \tensorq{Q}\vb{y}_{\alpha}, \tensorq{Q}\tensorq{Y}_{\alpha}\transpose{\tensorq{Q}}), \\.
	\end{align*}
\end{definition}
So we see that $\overline{W}\qty(\rcg), \tilde{W}\qty(\lcg)$ are \textbf{scalar isotropic functions of 1 tensorial (symmetric) argument.}

\begin{theorem}[Representation theorem for scalar isotropic functions]
	Let $\psi = \hat{\psi}\qty(\tensorq{A}) = \hat{\psi}\qty(\tensorq{Q}\tensorq{A}\transpose{\tensorq{Q}})$ be a scalar isotropic function of a single symmetric tensorial variable. Then it must hold
	\[
		\hat{\psi}\qty(\tensorq{A}) \equiv \hat{\psi}\qty(\invI\qty(\tensorq{A}),\invII\qty(\tensorq{A}),\invIII\qty(\tensorq{A})),
	\]	
where
\begin{align*}
	\invI\qty(\tensorq{A}) &= \tr \tensorq{A},\\
	\invII\qty(\tensorq{A}) &= \frac{1}{2}\qty(\qty(\tr \tensorq{A})^{2}-\tr \tensorq{A}^{2}),\\
	\invIII\qty(\tensorq{A}) &= \det \tensorq{A},
\end{align*}
are the invariants of $\tensorq{A}$.
\end{theorem}

\begin{proof}
    $\det\qty(\tensorq{A}-\lambda \identity) = -\lambda^{3} + \lambda^{2} \invI - \lambda \invII + \invIII = p_{\lambda}\qty(\tensorq{A})$
    We will prove a different assertion:

    $\tensorq{A}, \tensorq{B}$ are symmetric with the same invariants $\Leftrightarrow$  $\exists \tensorq{Q}: \tensorq{A} = \tensorq{Q} \tensorq{B} \transpose{\tensorq{Q}}$ 
    $"\Leftarrow"$ is trivial, as then the matrices are similliar, so they have the same char. polynomial, so they have the same invariants.
    $\Rightarrow$ have same eigenvalues, so if i write the spectral decomposiiton, i can write
    \[
	    \tensorq{A} = \tensorq{Q} \tensorq{\Lambda} \transpose{\tensorq{Q}}, \tensorq{B} = \tensorq{Q} \tensorq{\Lambda}\transpose{\tensorq{R}} = \tensorq{R}\transpose{\tensorq{Q}} \tensorq{A} \tensorq{Q}\transpose{\tensorq{R}}.
    \]
    Now suppose that the function is not a function of the invariants: $\hat{\psi} \neq \tilde{\psi}\qty(\invI,\invII,\invIII).$ That means $\exists \tensorq{A}_,, \tensorq{A}_2 \, \text{such that} \, \invI\qty(\tensorq{A}_1) = \invI\qty(\tensorq{A}_2)$ and the same for the remaining invariants. Using the previous assertion, we have
    \[
	    \exists \tensorq{Q}: \tensorq{A}_1 = \tensorq{Q}\tensorq{A}_2 \transpose{\tensorq{Q}} \Rightarrow \hat{\psi}\qty(\tensorq{A}_1) = \hat{\psi}\qty(\tensorq{Q}\tensorq{A}_2 \transpose{\tensorq{Q}}) = \hat{\psi}\qty(\tensorq{A}_2), \, \text{but} \, \hat{\psi}\qty(\tensorq{A}_1) \neq \hat{\psi}\qty(\tensorq{A}_2).
    \]
\end{proof}
Since using polar decomposition it can be shown the invariants of $\lcg, \rcg$ are the same we recieve
\[
	W = \tilde{W}(\invI\qty(\lcg), \invII\qty(\lcg), \invIII\qty(\lcg)) = \overline{W}\qty(\invI\qty(\rcg), \invII\qty(\rcg), \invIII\qty(\rcg)).
\]

\subsection{Representation in terms of principial stresses}
\label{sec:representation_principial}
... in terms of the eigenvalues $\tensorq{U},\tensorq{V}.$ The invariants can be expressed as
\begin{align*}
	\invI &= \lambda_1+\lambda_2 + \lambda_3, \\
	\invII &= \lambda_1 \lambda_2 + \lambda_2 \lambda_3 + \lambda_1 \lambda_3,\\
	\invIII &= \lambda_1 \lambda_2 \lambda_3.
\end{align*}
Often in materials science the quantites can be expressed in these variables:

\begin{example}[Ogden materials]
	\[
		\hat{W}\qty(\lambda_1,\lambda_2,\lambda_3) = \sum_{k=1}^n \frac{\mu_k}{\alpha_k}\qty(\lambda_1^{\alpha_k}+\lambda_2^{\lambda_k}+\lambda_3^{\alpha_k}-3)
	\]
\end{example}
How to calculate e.g. $\pkstress$ in this representation?
\[
	\pkstress = 2 \pdv{W\qty(\invI, \invII, \invIII)}{\lcg}\fgrad = 2 \pdv{\hat{W}}{\lcg}\qty(\lambda_1\qty(\lcg), \lambda_2\qty(\lcg), \lambda_3\qty(\lcg)) 2 \pdv{\hat{W}}{\lambda_i}\pdv{\lambda_i\qty(\lcg)}{\lcg}\fgrad.
\]

What (in the hell) is $\pdv{\lambda_i\qty(\lcg)}{\lcg}$? \footnote{Recall the Daleckii-Krein theorem:}

\[
	\lcg(s) = \sum_{\alpha = 1}^3 \omega_{\alpha}(s) \vb{g}_{\alpha}(s) \otimes \vb{g}_{\alpha}(s), \forall s \in I
\]
where $I$ is some open interval and $\{\vb{g}_{\alpha}\}$ is an ON eigenbasis of $\lcg.$ Next, realize
\[
	\omega_1(s) = \vb{g}_1(s)\vdot \lcg(s) \vb{g}_1(s),
\]
and differentiate this:
\[
	\dv{\omega(s)}{s} = \dv{\vb{g}_1}{s}\vdot \lcg \vb{g}_1  + \vb{g}_1 \dv{\lcg}{s}\vb{g}_1+\vb{g}_1 \vdot \lcg \dv{\vb{g}}{s} = \frac{1}{2} + +0.
\]

\bibliographystyle{chicago} %\bibliography{}
\end{document}

%%% Local Variables: 
%%% mode: latex
%%% TeX-master: t
%%% End: 
